
\chapter{Orthographical conclusion}
\label{cha:orth-concl}

\todo[inline]{something about  MDS and how categories may be inferred}

\todo[inline]{summary of what orthographical developments may be discerned, written as a handbook for someone wanting to date/localise texts.}

\section{Consequences for other ideas about Middle Welsh}
\label{sec:cons-other-ideas}
Knowledge of when orthographical lenition of voiceless stops was introduced in Welsh, and knowledge of how scribes would emend older texts after this introduction makes it necessary to reconsider what we think we know of Old Welsh and Middle Welsh grammar. I give an example below.

\Textcite[2]{schrijver_free_2010}, citing \textcite[18, 179]{evans_grammar_1964} and \textcite[193n]{morgan_y_1952}, states:
`Normally, if a plural subject noun follows the verb, the verb is in the 3sg [\dots]. In early poetry, a plural verb may precede a plural subject, in which case the subject undergoes lenition.' Evidence for this archaic rule comes from the sentences below:
\begin{mwl}
  \mwc[ex:schrsubplv1]{CLlH 23.5a}{yn Aber Cuawc yt ganant gogeu}{In Aber Cuawg cuckoos sing}
  \mwc[ex:schrsubplv2]{AP 5.141}{ymgetwynt Gymry}{the Welsh will see to it}
\end{mwl}

The Canu Llywarch Hen poems are thought to have been composed in Old Welsh, even though they survive in Middle Welsh orthography and in fourteenth century manuscripts, and the Book of Taliesin is shown to have added orthographic lenition of \mw[]{c}\todo{I must refer to this if I describe it elsewhere}. Having established that  lenition of voiceless stops was represented in the thirteenth century at the earliest, and having established that fourteenth-century scribes would add orthographical lenition of voiceless stops, we must amend Schrijver's (and Evans' and Morgan's) statement that nominal subjects of plural verbs undergo lenition in early poetry.

The fact of the matter is that lenition was probably added by the fourteenth-century manuscript scribes, rather than the original composers of the poem. It is indeed a feature of early poetry that plural verbs may have nominal subjects, but the lenition following it is not a feature of the early poetry itself, but rather a feature of the fourteenth-century manuscripts.

This means that those seeking to prove early poetry indeed had lenition of nominal subjects following plural verbs would either need to find a lenited nominal subject that has an initial consonant other than \mw[]{p, t, c}, or they would need to demonstrate that the orthographic lenition of these consonants indeed precedes the fourteenth century. In fact, \textcite[65--66]{van_development14} does not find subject lenition following plural verbs, but finds instances of non-lenition instead, including one case other than \mw[]{p, t, c}:
\mwcc[ex:vansluissubjlen]{RBH 641.38}{A gwrhau a \al{orugant gwyr} y iarllaeth y owein.}{And the noblemen did fealty to Owain’s earldom.}
This example comes from the Red Book of Hergest, dating from around 1400, and the tale itself is much younger than the examples given by Schrijver. This curious mix of lenition and non-lenition remains an elusive puzzle, but all the pieces in this puzzle discovered so far belong to the Middle Welsh period, not the Old Welsh period. Solving this puzzle would improve our knowledge of fourteenth-century Welsh, but it would not directly shed light upon the grammar of Old Welsh.

\section{Salesbury}
\label{sec:salesbury}

Fisher says:
\tqt{Prefaced to the Prayer Book of 1567 is given his promised `Explanation of certaine wordes being quareled withall', from which we take the following:

  `Vy-Dew for vynnuw, or vynyw, wherein D is now reteined, euen for the more significatiue expressing of the grace of the woord.

  Vy-popul for vymhobl, to saue the word the les maimed.
  
  Vy-troet for vynrhoed that the signification may be more apparent to the straunge Reader'.

  It is clear that he wanted the reader --- the silent reader particularly --- to see what the original form of the word was, not only as regards the initial letter, but also in compounds, as we find. This method, he, of course, applied to the written word only, that it might be self-explanatory, and he never meant that any one should read aloud the text actually as written. Rather, he expected the reader to be himself able to introduce the proper mutations as i the spoken language, which any intelligent Welshman could and would do, and so read correctly without the imputed \textit{llediaith} and \textit{anghyfiaith}. By thus writing the words so as to indicate their supposed etymology, he also though that Welsh people of both North and South Wales would be the better able to understand them. In fact, he concerned himself with written Welsh only, and wanted to preserve the words `the les maimed' and `uncorrupt', without the mutations, regardless of the fact that the language as a living organism must grow and change with time, and in accordance with its own genius.
}{Fis_Kynniver31}{xxxvii--xxxviii}

\tqt{He treats compounds and words that are not compounds in the same manner. After all he was simply following the earlier orthography; \eg [\dots] \textit{ympren croc, yn tuy vyntat, vygkorffi, amperffaith}.}{Fis_Kynniver31}{xxxix}

Professor Richards' review:
\tqt{In 1567, evidently in answer to his critics, he demanded, ``who \dots\ dyd ever wryte euery words as he sounded it?''. Another innovation was equally infelicitous. A stranger learning Welsh finds that the consonantal changes at the beginning of a word constitute his chief difficulty. Thus the word for father is written \textit{tad, dad, thad,} or \textit{nhad} according to its position in a sentence. Salesbury in every case used the radical or simplest form of such words. To judge from a sentence in the Dedication prexied to his Dictionary his object was to render it easier for readers to look up unusual words. The result was disastrous . His language appear clumsy and ungrammatical and the Welsh people would have none of it. Partly for this reason, and partly because it was superseded by Biship Richard Davies's translation of the whole Prayer Book in 1567, Kynnifer Llith a Ban was but little used.}{Ric_Welsh31}{}



%%% Local Variables:
%%% coding: utf-8
%%% mode: latex
%%% TeX-master: "../main"
%%% End:
