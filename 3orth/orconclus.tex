
\chapter{Orthographical conclusion}
\label{cha:orth-concl}

\todo[inline]{something about  MDS and how categories may be inferred}

\todo[inline]{summary of what orthographical developments may be discerned, written as a handbook for someone wanting to date/localise texts.}

\section{Consequences for other ideas about Middle Welsh}
\label{sec:cons-other-ideas}
Knowledge of when orthographical lenition of voiceless stops was introduced in Welsh, and knowledge of how scribes would emend older texts after this introduction makes it necessary to reconsider what we think we know of Old Welsh and Middle Welsh grammar. I give an example below.

\Textcite[2]{schrijver_free_2010}, citing \textcite[18, 179]{evans_grammar_1964} and \textcite[193n]{morgan_y_1952}, states:
`Normally, if a plural subject noun follows the verb, the verb is in the 3sg [\dots]. In early poetry, a plural verb may precede a plural subject, in which case the subject undergoes lenition.' Evidence for this archaic rule comes from the sentences below:
\begin{mwl}
  \mwc[ex:schrsubplv1]{CLlH 23.5a}{yn Aber Cuawc yt ganant gogeu}{In Aber Cuawg cuckoos sing}
  \mwc[ex:schrsubplv2]{AP 5.141}{ymgetwynt Gymry}{the Welsh will see to it}
\end{mwl}

The Canu Llywarch Hen poems are thought to have been composed in Old Welsh, even though they survive in Middle Welsh orthography and in fourteenth century manuscripts, and the Book of Taliesin is shown to have added orthographic lenition of \mw[]{c}\todo{I must refer to this if I describe it elsewhere}. Having established that  lenition of voiceless stops was represented in the thirteenth century at the earliest, and having established that fourteenth-century scribes would add orthographical lenition of voiceless stops, we must amend Schrijver's (and Evans' and Morgan's) statement that nominal subjects of plural verbs undergo lenition in early poetry.

The fact of the matter is that lenition was probably added by the fourteenth-century manuscript scribes, rather than the original composers of the poem. It is indeed a feature of early poetry that plural verbs may have nominal subjects, but the lenition following it is not a feature of the early poetry itself, but rather a feature of the fourteenth-century manuscripts.

This means that those seeking to prove early poetry indeed had lenition of nominal subjects following plural verbs would either need to find a lenited nominal subject that has an initial consonant other than \mw[]{p, t, c}, or they would need to demonstrate that the orthographic lenition of these consonants indeed precedes the fourteenth century. In fact, \textcite[65--66]{van_development14} does not find subject lenition following plural verbs, but finds instances of non-lenition instead, including one case other than \mw[]{p, t, c}:
\mwcc[ex:vansluissubjlen]{RBH 641.38}{A gwrhau a \al{orugant gwyr} y iarllaeth y owein.}{And the noblemen did fealty to Owain’s earldom.}
This example comes from the Red Book of Hergest, dating from around 1400, and the tale itself is much younger than the examples given by Schrijver. This curious mix of lenition and non-lenition remains an elusive puzzle, but all the pieces in this puzzle discovered so far belong to the Middle Welsh period, not the Old Welsh period. Solving this puzzle would improve our knowledge of fourteenth-century Welsh, but it would not directly shed light upon the grammar of Old Welsh.


%%% Local Variables:
%%% coding: utf-8
%%% mode: latex
%%% TeX-master: "../main"
%%% End:
