\section{On lenition}
\label{sec:lenition}
If we are to understand the contexts within which a difference between lenited voiceless stops and radical voiced stops may exist,
we must understand what the difference between a lenited and a radical consonant is in the first place. 
Again, this matter would seem obvious to educated speakers of Welsh, 
but I shall argue that there are different types of lenition, 
and that not everything that looks like lenition, behaves like it.

Lenition is the result of a regular sound law that occurred when a lenitable consonant was found following a vowel and before either another vowel or a sonorant~\autocite[96]{mccone_towards_1996}. 
As long as these vowels and sonorants were still around, these lenited sounds were predictable based on their phonological environment, so they were allophones of their radical counterpart. 
However, following apocope --- the loss of final syllables --- this environment no longer existed, and the radical or lenited pronunciation of a consonant could no longer be retrieved from context, so the distinction between radical and lenited consonant became phonemic.

\subsection{Morphophonemic lenition}
\label{sec:morph-lenit}
Typically, when lenition operates in \gls{mw}, it creates a variant inflection of a single lexeme. 
For example, \gmow[father]{tad} is a lexeme with a form that is inflected for lenition: \mow{dad}. 
Substituting one form for another has consequences for the morphosyntax of a phrase, and may either change its meaning or make it ungrammatical.
In other words: lenition itself is grammaticalized, as the lenited and unlenited variant morphemes stand in non-free variation.
Morphophonemic lenition may be subdivided into two categories: contact lenition and free lenition.
\subsubsection{Contact lenition}
\label{sec:contact-lenition}
Contact lenition is every instance where lenition is applied to a word because it is a property of the immediately preceding word to cause lenition. 

A typical example of contact lenition is lenition following various prepositions or pronouns. 
When a preposition or pronoun causing contact lenition is used, failing to lenite the following word may either change the meaning of a phrase, or make it ungrammatical, \eg failing to lenite \mow{ferch} in \mow[his daughter]{ei ferch} would change the meaning to `her daughter', and failing to lenite following the definite article in \mow[the daughter]{y ferch} would make the phrase ungrammatical, because  gender concord between the article the noun would be lost.

Contact lenition is by far the most common type of lenition in \gls{mw}. Virtually all instances of contact lenition are a direct consequence of apocope, and the ensuing phonemicization of phonetic lenition. Contact lenition started out as a functionless morphophonemic property of a morpheme, but this morpheme could acquire function if it contrasted with a homophone causing a different mutation~\autocite[1]{schrijver_free_2010}.
\subsubsection{Free lenition}
\label{sec:free-lenition}
`Free lenition' refers to any type of lenition `where neither synchronic contact with a preceding morpheme nor former phonetic context can account for its occurrence'~\autocite[1]{schrijver_free_2010}. Unlike contact lenition, free lenition always has a function. And unlike contact lenition, free lenition is not a direct result of apocope. 

An overview of different types of free lenition is given by \textcite{schrijver_free_2010}. An example is lenition of a noun in apposition to another element in the phrase, \eg Example~\ref{pryderiuab}.
\mwcc[pryderiuab]{\gls{wbr} 36.17}{prẏderi	uab pỽẏll}{Pryderi, son of Pwyll}
\Textcite{schrijver_free_2010} frames most instances of free lenition within as `lenition of apposition', where he employs a wide definition of apposition: `if two sememes have the same referent, the second is an apposition of the first'~\autocite[3]{schrijver_free_2010}, and where he gives the following rule for most instances of \gls{mw} free lenition: `If two sememes that belong to the same clause have the same referent, one, usually the second, is lenited'~\autocite[3]{schrijver_free_2010}.

\subsection{Petrification}
\label{sec:petrification}
Some \gls{mw} words may be lenited as a property of these words themselves, rather than as a result its syntactic role or the word immediately preceding.
The lenited form in such cases may be the only form left within a speaker's grammar.
Or, a radical and a lenited form may exist side by side, but are interchangeable in any phrase.
In other words, lenition has been degrammaticalized in these cases, and wherever a lenited and unlenited form stand side by side, they stand in free variation.

The resulting words may sometimes be called `lenited' in a diachronic sense, but never in a synchronic sense, since words whose lenition is petrified do not alternate with unlenited counterparts under the morphophonemic conditions otherwise governing lenition. Petrified lenited forms may emerge as the result of two distinct processes: clitic reduction and reanalysis of morphophonemic reduction.

\subsubsection{Clitic reduction}
\label{sec:clitic-reduction}
`Clitic reduction' refers to the reduction of the phonological load of an unstressed word%
\footnote{Conjugated and bare forms of \mw[with]{can} exemplify how stress influences whether a word is reduced: stressed conjugated forms of this preposition still write their radical initial later than unstressed simple forms. Later lenition of conjugated variants may be due to analogy with the simple form~\autocite[54]{jongeleen_lenition_2016}.}. 
The voicing, fricativization, or deletion of a consonant are some ways by which a word's phonological load may be reduced.
The result of such processes may look superficially identical to lenition.
A clitic may then become petrified in its reduced form if it becomes the unmarked or only form possible. 

This type of reduction is similar to free lenition in that it is not the result of the phonemicisation of phonetic lenition which occurred with apocope.
More importantly, however, this type of lenition differs from both contact lenition and free lenition in that the radical and the mutated form stand in free variation.
That is, one may use \mow[with]{gan, can} interchangeably without causing any change in morphology, syntax or semantics. 
This means that substituting the radical for the reduced form, or vice versa, never makes a grammatical phrase ungrammatical. 

Over time, the reduced form may become the unmarked form or even the only form left. 
Such a reduced form may then itself become subject to clitic reduction. 
An example of this is the \gls{mw} preposition \mw[to]{y}. 
This word is the reduced form of \mw{dy} /ði/, itself a reduced form of \gls{ow} \mw{di} /di/\footnote{Cf.\ \gmob[to]{da}.}. 
\todo{Refer to JTK's GOW paragraph on irregular development of vowels in clitics when it is finished}
The development of \mw[to]{y} demonstrates that this clitic reduction does not necessarily follow established sound laws. 
There is no general pattern of /d/ turning into /ð/ in Welsh outside of the environment of lenition, and there is certainly no established pattern of /ð/ disappearinɡ completely.

Irish clitics witnessed a similar development.  
\goi[to]{co} and \goi[over]{tar} were pronounced with initial /g/ and /d/, despite being voiceless etymologically%
\footnote{
  An \gls{oir} sound law has been formulated for this voicing: `a voiceless dental stop or fricative on the word boundary was regularly voiced in contact with an unstressed vowel, but otherwise remained unvoiced'~\autocite[42]{mccone_final_1981}. 
  \Textcite[43]{jongeleen_lenition_2016} proposes a similar process for velars.
  A similar sound law may perhaps be formulated for the reduction of \gls{mw} clitics, but must take into account that reduced and unreduced forms of \eg \gmw{can, gan} exist side by side.}.
\Gls{oir} lenition of /c/ and /t/ would give /θ/ and /x/, proving that lenition did not cause clitic reduction.
As a result, lenited \goi{**cho, thar} are not found, even though their conjugated variants, and therefore stressed variants, do show variation between lenited and unlenited variants~\autocite[43]{jongeleen_lenition_2016}.

A potentially relevant phonological development is discussed by \textcite[16--17]{schrijver_studies_1995}.
Within the debate of where the stress historically fell in \gls{pbr},  \textcite{thurneysen_zur_1883} argued that there was a period in \gls{pbr} when the stress fell on the word-initial syllable.
This argument was based on comparison of the pretonic particle \gpc{*tu} in the noun \gmw[lord]{ty-wyssawc}, and in the verb \gmow[I say]{dy-weddaf}.
These two words show that the same particle was weakened when it served as an unstressed preverbal particle, but was not weakened in nouns.
\Textcite{thurneysen_zur_1883} argued (and \textcite{schrijver_studies_1995} agrees) that \gls{pbr} had stress on the first syllable, and that any particles preceding this syllable had their consonants  reduced from \pc{*t} to \mw{d}.
The question on which syllable stress fell is irrelevant at present, but the outcome of this discussion implies a  sound law where initial unstressed voiceless stops were voiced.
This implied sound law is reminiscent of the type of reduction we seen from \mw{can} to \mw{gan}, although the precise sound law operating here is never discussed independently of the British accent.
I suggest that the reduction of \pc{*tu} to \mw{dy} in \mw{dy-weddaf} may in fact be considered a form of clitic reduction, and that this development was distinct from lenition\footnote{%
  \Textcite[125]{schrijver_studies_1995} also implies elsewhere that heconsiders clitic reduction a form of lenition: `However, the act that the preposition always occurs in the lenited form, viz.\ Co.\ \textit{war}, MW \textit{ar} < \textit{*war} rather suggests that it was unstressed. [\dots] Since, as we have seen, the preposition was always lenited[\dots].'}.


Comparative evidence thus demonstrates that it is a misnomer to refer to clitic reduction as lenition.
Diachronically, the processes of the phonemicisation of lenition with apocope, and reduction of clitics developed differently: the former did so according to sound laws while the latter did so haphazardly.
Synchronically, petrified clitic reduction and lenition also bear little relation to one another: lenition is a morphophonological process, meaning the difference between a radical and a lenited form is non-trivial, while a reduced clitic stands in free variation with its non-reduced counterpart, if the radical counterpart exists at all. 
On the basis of these considerations, we may conclude that clitic reduction is a qualitatively different process from morphophonemic lenition. 

\subsubsection{Reanalysis of morphophonemic lenition}
\label{sec:rean-morph-lenit}
In some cases, lenition is found to apply to a word so frequently that the word gets reanalysed. 
This means that what was previously considered the lenited variant of a word is now considered the radical. 
An example of this development is found in several \gls{mow} dialects in frequent words such as \mow[bridge]{pont}. 
This feminine word is lenited to \mow[the bridge]{y bont} following the article. 
This setup may become petrified, and speakers may again lenite the resulting word to \mow[the bridge]{y font}.
Free lenition is frequently petrified, \eg in words typically or only used as adverbs: \mow[yesterday]{ddoe}, \mow[before]{gynt}.


In the first page of the \gls{bbch}, we find what we may presume to be an instance of this type of petrification: \mw{garauuys} and \mw[Lent]{ga/rauuys} in ll.~9, 9--10, respectively\footnote{See Example~\ref{textpeniarthpone}, on \pref{textpeniarthpone}}. 
Etymologically, \mw{grawys} is a feminine word, which comes from \glat{quadragēsima}, and both instances follow the article. 
It differs from morphophonemic lenition in that the latter type of lenition is not represented for \mw{c} in this text. 
Moreover, there are no attested instances of unpetrified *\mw{C(a)rawys} in Welsh, and it is reanalysed as a masculine noun~\autocite[Grawys, Garawys]{bevan_geiriadur_2014}. 


Diachronically, petrification of morphophonemic lenition differs from petrification of clitic reduction. 
In the former case, we are dealing with the result of reanalysis of lenition, while in the latter case we are dealing with the petrification of a shift towards a form with lesser phonological load. 
Synchronically, however, both types of petrification are similar in that in neither case the radical and the mutated form stand in morphophonemic contrast to each other. 

\subsubsection{Their behaviour}
\label{sec:their-behaviour}
Both types of petrification behave similarly to each other, and differently from morphophonemic lenition. 
In both cases, \gls{D} descending from \gls{T} behaves like \xD\ rather than \lT.

Phonological evidence from the cynghanedd implies that petrified voiced stops descending from voiceless stops alliterate with \xD\ rather than \lT. An example of this is found in Example~\ref{kymrawtdreic} on \pref{kymrawtdreic}. 

Orthographical evidence shows that reduced/lenited voiceless stops were represented with \graph{b, d, g} from an earlier date onwards when they were petrified than when they were not petrified. 
In other words: a text that does not regularly represent lenition of voiceless stops may still write petrified forms with \graph{b, d, g}. 
An example of a manuscript shown such a pattern is \gls{ll1}, which consistently writes \mw[together]{y gyt}, with \mw{g-}, because it is petrified as a close compound of \mw[to + union]{i + cyd}, even though this manuscript barely represents morphophonemic lenition of words with \mw{c-} as their initial consonant.

\Textcite[52]{jongeleen_lenition_2016} notes that conjugated prepositions of \mw{tros, trwy} are found with initial \mw{t-}  in thirteenth-century manuscripts, while later manuscripts employ spellings with \mw{d-}.
\Textcite{sims-williams_variation_2013} differentiates variant spellings of \mw[with me, with you]{kennyf, kennyt}, which may be spelled with initial \mw{k-/c-}, or with \mw{g-}.
He similarly notes that `\textit{k-/c-} seems to indicate a pre-fourteenth century date whereas \textit{g-} is neutral, being found at all periods of Middle Welsh'~\autocite[24]{sims-williams_variation_2013}. \Textcite[55]{jongeleen_lenition_2016} gives a relative chronology of these petrification, noting that `[t]he simple preposition \textit{can} is the first to transition to its lenited form \textit{gan}, followed by \textit{tros} and finally \textit{trwy}'.

\todo[inline]{My editorial policy is to include petrified lenition and where lenition happens within compounds like `i gyd', but to mark them as research exceptions. However, I do not do this with personal pronouns such as \mw{mi/ỽi}}


%%% Local Variables:
%%% mode: latex
%%% TeX-master: "../main"
%%% End:
