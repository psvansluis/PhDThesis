\chapter{The nature of words and lenition}
\label{cha:some-phon-issu}
I argue that a distinction between \lT\ and \xD\ was only maintained word-initially, and only as a result of the morphophonological process of lenition. These two conditions, word-initialness and being part of grammatical lenition, may seem trivial to any educated speaker of Welsh, but speaker intuitions alone are insufficient on both accounts for making sense of the patterns found in both the phonological part and the orthographical part of my thesis. This chapter will treat both issues.

\section{Word-initial contrasts and what a word is}
\label{sec:excl-word-init}
In Part~\ref{part:phonology-phonetics}, it was confirmed that the phonological difference between \lT\ and \xD\ was only maintained word-initially. These consonant series must have merged word-medially and word-finally soon after apocope, and possibly even before then. Old Welsh evidence attesting to this merger is discussed in Section~\ref{bdgwithptc}. This finding is backed up by Cornish and Breton: voiced geminates similarly merged with lenited voiceless stops in Cornish and Breton, indicating that this merger most likely occurred as early as the Common Brittonic period~\autocite[31]{schrijver_old_2011}.


So why did \lT\ merge with \xD? And why did it only happen word-medially and word-finally? Martinet convincingly argues how and why this happened in Romance languages, and notes that his arguments are equally applicable to Brittonic. The explanation for this early merger lies in the rarity of voiced geminates both in Brittonic and in Greek, Germanic, and Latin: 

\tqt{Geminated voiced stops [in Brittonic] probably existed, but hardly at other points than at morphemic junctures. The situation must have been very much like the one which prevailed in Latin, where the gemination of surds was common both at the juncture of morphemes (as in \textit{at-tingo, ap-pello}) and elsewhere (as in \textit{bucca, puppa, mitto}), whereas geminated voiced stops were most exceptional except at morpheme juncture (as in \textit{ag-ger, ab-brevio, red-do}). Even here there was a tendency to eliminate them as soon as the feeling for composition became blurred; cf.\ \textit{cred\=o} as opposed to Skt.\ \textit{\c{c}rad-dadh\=ami}, and later \textit{redd\=o > rendo} with dissimilation, probably suggested by \textit{pre(he)nd\=o}; cf.\ also Ital.\ \textit{argine} `dam levee' from \textit{agger} (once \textit{arger}). A very similar situation must have prevailed in classical Greek, where \textit{-ππ-, -ττ-, -κκ-, -πφ-, -τθ-, -κχ-} are frequent (both as the reflexes of normal sound shifts and in hypocoristic formation as a result of some expressive process) but where \textit{-ββ-, -δδ-, -γγ-} are so exceptional nothing prevented the use of \textit{-γγ-} for [ŋg]. A tendency to unvoice geminates must have existed in the older stages of Germanic, at least in cases of expressive gemination, as is shown for instance by the geminated surd of OE \textit{liccian}, OHG \textit{lecch\=on}, as opposed to Goth.\ \textit{bilaigon} (with regular \textit{-g-} from *\textit{-\^gh-}; cf.\ Gk.\ \textit{λείχω}, OIr.\ \textit{ligim}, etc.).}
{martinet_celtic_1952}{198}

The consequence of this development is that \lT\ and \xD\ were only distinguished in those environments where they both appeared with a somewhat comparable frequency: word-initially. In other positions, voiced geminates were so rare that the functional value of a three-way stop distinction would have been minimal, so they merged with lenited voiceless stops. The consequence is that a word like \gpc{*ad-beros} > \pc{*abberos} > \gmw[estuary]{aber} was spelled as \mw{aper} in \gls{ow}, because the word would have merged with a hypothetical \pc{**aperos} by this stage, and because \gls{ow} would orthographically represent any sound existing in a lenited form using its radical counterpart.

\section{The indeterminacy of word separation}
\label{sec:indet-word-separ}
Having established that \lT/\xD\ were only distinguished word-initially, the question arises what this entails exactly. More precisely, this constraint raises the question `what is a word?'.
To a non-linguist, this issue may seem trivial, but linguists have been unable to agree on a cross-linguistic definition of `word'.
\Textcite[28]{haspelmath_indeterminacy_2011} even argues that the whole search for a definition of the concept `word' is because we are biased by writing habits, and he remarks the following: 
% \tqt{Joseph Vendryes remarked that modern linguistics was in a crisis, and that linguists were not even in agreement on what a word is, one of the fundamental concepts of their object of studies.}{haspelmath_indeterminacy_2011}{28}
\tqt{Linguists have no good basis for identifying words across languages, and hence no good basis for a general distinction between syntax and morphology as parts of the language system.}{haspelmath_indeterminacy_2011}{24}
What this means is that any formulation of a morphological or syntactical rule may not use the term `word', because it is not a meaningful concept cross-linguistically. 
A quick look at the history of Welsh and a comparison with other Brittonnic languages already reveals inconsistencies in where word boundaries are considered to exist.
A case in point is \gmow[by, against]{erbyn}, a single-word preposition.
This word is reconstructed as \gpbr[before head/end]{ari penn\=u} \autocite[258]{schrijver_studies_1995}, a preposition followed by a noun.
This combination was later grammaticalised into a compound preposition and finally as a simple preposition.
In \gls{mco} a trace of the original two-word syntax remains, as is borne out by infixed possessive pronouns in \mco[against you, him]{er-dhe-byn, er-y-byn}, \etc~\autocite[120]{koch_neo-brittonic_1989}.

Disambiguating word-internal consonant changes from word-initial mutations cross-linguistically is  a problem \textcite{iosad_right_2010} also ran into when presenting an overview of all languages with initial consonant mutations.
He refrains `from considering most cases of consonant alterations involving the left edges of units smaller than the word', as these alternations do not usually target first consonants specifically, and notes that `the majority of these cases do not involve any morphosyntactic information'~\autocite[108]{iosad_right_2010}.
He then acknowledges immediately that his definition leaves some leeway, and raises the question of what it means to be initial in a word.
He then opts for `an intuitive notion of ``word'' as the actual instantiation of a lexical item, without committing to a particular stance'~\autocite[109]{iosad_right_2010}.
In the end, he gives two useful criteria for diagnosing a word-initial mutation: the morphophonemic process uniquely appears at the beginning of a word, and it is used to convey morphosyntactic information.

This thesis argues that there were plenty of instances where \lT/\xD\ were kept apart, and also plenty of instances where they were not. 
So how do we formulate the limit of the applicability of this phonological distinction?
Such a formulation must be borne out by the evidence. Based on orthographical evidence, \gls{mw} word segmentation agrees largely, but not completely, with where \gls{mow} orthography puts spaces between words.
Still, exceptions exist, \eg \gmow[together]{i gyd} is written as  one word in \gmw{ygyt}.

Crucially, morphemes and lexemes do not exist in isolation. 
Phonemes of the \lT -type only stand in phonemic contrast with \xD\ because a morpheme containing \lT\ coexists with one containing \xT, so the existence of a phrase such as \gmw[his father]{y dad} implies the existence of the lexeme \mw{tad}.
Conversely, the existence of \mw{tad} allows for the phonological interpretation of a phrase containing \lT\ to be maintained as such, so it is due to \mw{tad} that \mw{y dad} may be interpreted as /i \gls{l}tad/ rather than /i \gls{x}dad/.
A similar morphological relationship between phonemes may not be posited for word-medial and word-final consonants, \eg the existence of \gmw{tad} does not imply the existence of \gmw{**tat} or \mw{**tath}.
Thus, the link maintaining this phonological distinction between different variants of a lexeme does not exist word-medially and word-finally in the same way it does word-initially.

In the case of compounds such as \mw{i gyd} `together', one must keep in mind that they constitute a lexeme of their own. This means that the existence of the word \mw[joining]{cyd} no longer influences the phonological structure of \mw{i gyd} in the same way as \mw{tad} influences \mw{ei dad}.  Thus, a speaker of \gls{mw} could no longer  interpret the phrase \mw{i gyd} as /i \gls{l}kɨd/ rather than /i \gls{x}gɨd/. Instead, a speaker of \gls{mw} would interpret the phrase as one word: /iˈgɨd/, and the opposition between \lT\ and \xD\ did not exist word-internally. In other words: there is no mutation on a synchronic level, because the petrified mutation does not carry morphophonemic information.

\section{On lenition}
\label{sec:lenition}
If we are to understand the contexts within which a difference between lenited voiceless stops and radical voiced stops may exist,
we must understand what the difference between a lenited and a radical consonant is in the first place. 
Again, this matter would seem obvious to educated speakers of Welsh, 
but I shall argue that there are different types of lenition, 
and that not everything that looks like lenition, behaves like it.

Lenition is the result of a sound law that occurred when a lenitable consonant was found following a vowel and before either a vowel or a sonorant~\autocite[96]{mccone_towards_1996}. 
As long as these vowels and sonorants were still around, these lenited sounds were predictable based on their phonological environment, \ie they were allophones of their radical counterpart. 
However, following apocope --- the loss of final syllables --- this environment no longer existed, and the radical or lenited pronunciation of a consonant was no longer automatically conditioned by its phonological context, so the distinction between radical and lenited consonant became phonemic.

\subsection{Morphophonemic lenition}
\label{sec:morph-lenit}
Typically, when lenition operates in \gls{mw}, it creates a variant inflection of a single lexeme. 
For example, \gmow[father]{tad} is a lexeme with a form that is inflected for lenition: \mow{dad}. 
Substituting one form for another has consequences for the morphosyntax of a phrase, and may either change its meaning or make it ungrammatical.
In other words: lenition itself is grammaticalised, as the lenited and unlenited variant morphemes stand in non-free variation.
Morphophonemic lenition may be subdivided into two categories: contact lenition and free lenition.
\subsubsection{Contact lenition}
\label{sec:contact-lenition}
Contact lenition is every instance where lenition is applied to a word because it is a property of the immediately preceding word to cause lenition. 

A typical example of contact lenition is lenition following various prepositions or pronouns. 
When a preposition or pronoun causing contact lenition is used, failing to lenite the following word may either change the meaning of a phrase, or make it ungrammatical, \eg failing to lenite \mow{ferch} in \mow[his daughter]{ei ferch} would change the meaning to `her daughter', and failing to lenite following the definite article in \mow[the daughter]{y ferch} would make the phrase ungrammatical, because  gender concord between the article and the noun would be lost.

Contact lenition is by far the most common type of lenition in \gls{mw}. Virtually all instances of contact lenition are a direct consequence of apocope, and the ensuing phonemicization of phonetic lenition. Contact lenition started out as a functionless morphophonemic property of a morpheme,  but this morpheme could acquire function if it contrasted with a homophone causing a different mutation~\autocite[1]{schrijver_free_2010}. This is exactly what we see with the different meanings of \mow[his, her]{ei}, where the pronouns themselves are homophonous, but \mow[]{ei} causes lenition when it means `his', but spirantisation when it means `her'.

\subsubsection{Free lenition}
\label{sec:free-lenition}
`Free lenition' refers to any type of lenition `where neither synchronic contact with a preceding morpheme nor former phonetic context can account for its occurrence'~\autocite[1]{schrijver_free_2010}. Unlike contact lenition, free lenition always has a function, and free lenition is not a direct result of apocope. 

An overview of different types of free lenition is given by \textcite{schrijver_free_2010}. An example is lenition of a noun in apposition to another element in the phrase, \eg Example~\ref{pryderiuab}.
\mwcc[pryderiuab]{\gls{wbr} 36.17}{prẏderi	\al{uab} pỽẏll}{Pryderi, son of Pwyll}
Other examples are lenition of the object following the verb and lenition of nouns following a preposed adjective.
\Textcite{schrijver_free_2010} frames most instances of free lenition  as `lenition of apposition', where he employs a wide definition of apposition: `if two sememes have the same referent, the second is an apposition of the first'~\autocite[3]{schrijver_free_2010}, and where he gives the following rule for most instances of \gls{mw} free lenition: `If two sememes that belong to the same clause have the same referent, one, usually the second, is lenited'~\autocite[3]{schrijver_free_2010}.

\subsection{Petrification}
\label{sec:petrification}
Some \gls{mw} words may be lenited as a property of these words themselves, rather than as a result of its syntactic role or the word immediately preceding.
The lenited form in such cases may be the only form left within a speaker's grammar.
Alternatively, a radical and a lenited form may exist side by side, but are interchangeable in any phrase. In such a case, lenition no longer served as a morpheme, and wherever a lenited and unlenited form stand side by side, they stand in free variation \eg \mw[over]{dros, tros}.

The resulting words may sometimes be called `lenited' in a diachronic sense, but never in a synchronic sense, since words of which lenition is petrified do not alternate with unlenited counterparts under the morphophonemic conditions otherwise governing lenition. Petrified lenited forms may emerge as the result of two distinct processes: clitic reduction and reanalysis of morphophonemic reduction.

\subsubsection{Clitic reduction}
\label{sec:clitic-reduction}
`Clitic reduction' refers to the reduction of the phonological load of an unstressed word%
\footnote{Conjugated and bare forms of \mw[with]{can} exemplify how stress influences whether a word is reduced: stressed conjugated forms of this preposition still write their radical initial later than unstressed simple forms. Later lenition of conjugated variants may be due to analogy with the simple form~\autocite[54]{jongeleen_lenition_2016}.}. 
The voicing, fricativisation, or deletion of a consonant are some ways by which a word's phonological load may be reduced.
The result of such processes may look superficially identical to lenition.
A clitic may then become petrified in its reduced form if it becomes the unmarked or only form possible. 

This type of reduction is similar to free lenition in that it is not the result of the phonemicisation of phonetic lenition which occurred with apocope.
More importantly, however, this type of lenition differs from both contact lenition and free lenition in that the radical and the mutated form stand in free variation.
That is, one may use \mow[with]{gan, can} interchangeably without causing any change in morphology, syntax or semantics. 
This means that substituting the radical for the reduced form, or vice versa, never makes a grammatical phrase ungrammatical~\autocite[451--453]{morgan_y_1952}. 

Over time, the reduced form may become the unmarked form or even the only form left. 
Such a reduced form may then itself become subject to clitic reduction. 
An example of this is the \gls{mw} preposition \mw[to]{y}. 
This word is the reduced form of \mw{dy} /ðɨ/, itself a reduced form of \gls{ow} \mw{di} /dɨ/\footnote{Cf.\ \gmob[to]{da}.}. 
\todo{Refer to JTK's GOW paragraph on irregular development of vowels in clitics when it is finished}
The development of \mw[to]{y} demonstrates that this clitic reduction does not necessarily follow established sound laws. 
There is no general pattern of /d/ turning into /ð/ in Welsh outside of the environment of lenition, and there is certainly no established pattern of /ð/ disappearinɡ completely.

Irish clitics witnessed a similar development.  
\goi[to]{co} and \goi[over]{tar} were pronounced with initial /g/ and /d/, despite being voiceless etymologically%
\footnote{
  An \gls{oir} sound law has been formulated for this voicing: `a voiceless dental stop or fricative on the word boundary was regularly voiced in contact with an unstressed vowel, but otherwise remained unvoiced'~\autocite[42]{mccone_final_1981}. 
  \Textcite[43]{jongeleen_lenition_2016} proposes a similar process for velars.
}.
\Gls{oir} lenition of /c/ and /t/ would give /θ/ and /x/, proving that lenition did not cause clitic reduction.
As a result, lenited \goi{**cho, thar} are not found, even though their conjugated counterparts, and therefore stressed counterparts, do show variation between lenited and unlenited variants~\autocite[43]{jongeleen_lenition_2016}.

A potentially relevant phonological development is discussed by \textcite[16--17]{schrijver_studies_1995}.
Within the debate of where the stress historically fell in \gls{pbr},  \textcite{thurneysen_zur_1883} argued that there was a period in \gls{pbr} when the stress fell on the word-initial syllable.
This argument was based on comparison of the pretonic particle \gpc{*tu} in the noun \gmw[lord]{ty-wyssawc}, and in the verb \gmow[I say]{dy-weddaf}.
These two words show that a homophonous particle was weakened when it served as an unstressed preverbal particle, but was not weakened in nouns.
\Textcite{thurneysen_zur_1883} argued (and \textcite{schrijver_studies_1995} agrees) that \gls{pbr} had stress on the first syllable, and that any particles preceding this syllable had their consonants  reduced from \pc{*t} to \mw{d}.
The question on which syllable stress fell is irrelevant at present, but the outcome of this discussion implies a  sound law where initial unstressed voiceless stops were voiced.
This implied sound law is reminiscent of the type of reduction we seen from \mw{can} to \mw{gan}, although the precise sound law operating here is never discussed independently of the British accent.
I suggest that the reduction of \pc{*tu} to \mw{dy} in \mw{dy-weddaf} may in fact be considered a form of clitic reduction, and that this development was distinct from lenition\footnote{%
  \Textcite[125]{schrijver_studies_1995} also implies elsewhere that he considers clitic reduction a form of lenition: `However, the fact that the preposition always occurs in the lenited form, viz.\ Co.\ \textit{war}, MW \textit{ar} < \textit{*war} rather suggests that it was unstressed. [\dots] Since, as we have seen, the preposition was always lenited[\dots].'}.

Clitic reduction may hardly be called a `sound law' in the Neogrammarian sense of the word, because such a sound law would have to take into account that reduced and unreduced forms of the same word  exist side by side, \eg \gmow{tros, dros}\footnote{%
Except, of course, if we considered the stressed and unstressed proto-forms of \mow{tros, dros} unrelated lexemes.
}.
The existence of such doublets is better paralleled by English contracted forms existing together with their fully expanded form, \eg \emph{do not, don't}.


Comparative evidence thus demonstrates that it is a misnomer to refer to clitic reduction as lenition.
Diachronically, the processes of the phonemicisation of lenition with apocope, and reduction of clitics developed differently: the former did so according to sound laws while the latter did so haphazardly.
Synchronically, petrified clitic reduction and lenition also bear little relation to one another: lenition is a morphophonological process, meaning the difference between a radical and a lenited form is non-trivial, while a reduced clitic stands in free variation with its non-reduced counterpart, if the radical counterpart exists at all. 
On the basis of these considerations, we may conclude that clitic reduction is a qualitatively different process from morphophonemic lenition. 

\subsubsection{Reanalysis of morphophonemic lenition}
\label{sec:rean-morph-lenit}
In some cases, lenition is found to apply to a word so frequently that the word gets reanalysed. 
This means that what was previously considered the lenited variant of a word is now considered the radical. 
% An example of this development is found in several \gls{mow} dialects in frequent words such as \mow[bridge]{pont}. 
% This feminine word is lenited to \mow[the bridge]{y bont} following the article. 
% This syntax may become petrified, and speakers may again lenite the resulting word to \mow[the bridge]{y font}.
Free lenition is frequently petrified, \eg in words typically or only used as adverbs: \mow[yesterday]{ddoe}, \mow[before]{gynt}. In \gls{mow} the unlenited forms of these words are virtually unknown.



In the first page of the Black Book of Chirk (\gls{sA}), an instance of this type of petrification is found: \mw{garauuys} and \mw[Lent]{ga/rauuys} in ll.~9, 9--10, respectively. 
Etymologically, \mw{grawys} is a feminine word, which comes from \glat{quadragēsima}, and both instances follow the article. 
It differs from morphophonemic lenition in that the latter type of lenition is not represented for \mw{c} in this text. 
Moreover, there are no attested instances of unpetrified *\mw{C(a)rawys} in Welsh, and it is reanalysed as a masculine noun~\autocite[Grawys, Garawys]{bevan_geiriadur_2014}. 

\Textcite[448-451]{morgan_y_1952} lists a few examples of petrified lenition. One of these is \mow[much]{fawr}, which provides another case study of reanalysed morphophonemic lenition. This word is often lenited in contexts where it denotes the degree of a negative: \eg \mow[where they did not receive much earth]{lle ni chawsant \al{fawr} ddaiar}, or following \oes: \mow[in whom there is not much consolation]{nad oes \al{fawr} gyssur ynddo}. Contexts like these have caused \mow[]{fawr} to be reanalysed as meaning `not much', and now the lenited form may be used even where there is no grammatical reason to expect lenition, \eg \mow[without understanding much of the lives of angels]{heb ddeall \al{fawr} o fywyd angelion} or \mow[that I do not now mention him much]{nad wyf i yr awron yn \al{fawr} sôn am dano}. Other examples mentioned include \mow[poor]{druan} in \mow[]{Dafydd druan} and \mow[I presume]{debygaf} in \mow[more, I presume, than in any other place in the world]{mwy, \al{dybygaf} i, nac yn lle arall o'r byt}.

Diachronically, petrification of morphophonemic lenition differs from petrification of clitic reduction. 
In the former case, we are dealing with the result of reanalysis of lenition, while in the latter case we are dealing with the petrification of a shift towards a form with lesser phonological load. 
Synchronically, however, both types of petrification are similar in that in neither case the radical and the mutated form stand in morphophonemic contrast to each other. 

\subsubsection{Their behaviour}
\label{sec:their-behaviour}
Both types of petrification behave similarly to each other, and differently from morphophonemic lenition. 
In both cases, \gls{D} descending from \gls{T} behaves like \xD\ rather than \lT. Phonological evidence from the cynghanedd implies that petrified voiced stops descending from voiceless stops alliterate with \xD\ rather than \lT\footnote{See Example~\ref{kymrawtdreic} on \pref{kymrawtdreic}.}.

Orthographical evidence shows that reduced/lenited voiceless stops were represented with \graph{b, d, g} from an earlier date onwards when they were petrified than when they were not petrified. 
In other words: a text that does not regularly represent lenition of voiceless stops may still write petrified forms with \graph{b, d, g}. 
An example of a manuscript shown such a pattern is \gls{ll1}, which consistently writes \mw[together]{y gyt}, with \mw{g-}, because it is petrified as a close compound of \mw[to + union]{i + cyd}, even though this manuscript barely represents morphophonemic lenition of words with \mw{c-} as their initial consonant.

\Textcite[52]{jongeleen_lenition_2016} notes that conjugated prepositions of \mw{tros, trwy} are found with initial \mw{t-}  in thirteenth-century manuscripts, while later manuscripts employ spellings with \mw{d-}.
\Textcite{sims-williams_variation_2013} differentiates variant spellings of \mw[with me, with you]{kennyf, kennyt}, which may be spelled with initial \mw{k-/c-}, or with \mw{g-}.
He similarly notes that `\textit{k-/c-} seems to indicate a pre-fourteenth century date whereas \textit{g-} is neutral, being found at all periods of Middle Welsh'~\autocite[24]{sims-williams_variation_2013}. \Textcite[55]{jongeleen_lenition_2016} gives a relative chronology of these petrifications, noting that `[t]he simple preposition \textit{can} is the first to transition to its lenited form \textit{gan}, followed by \textit{tros} and finally \textit{trwy}'.

My editorial policy is to include petrified lenition and where lenition happens within compounds like \mw{i gyd}, but to mark them as research exceptions. However, I do not do this with personal pronouns such as \mw{mi/ỽi}\footnote{The reason for this is that I only understood that expressions such sa \mw[]{i gyd} had to be marked research exceptions after seeing how they behaved. With personal pronouns, it is more obvious that their lenition is non-grammatical.}.

\section{Spirantisation and non-word-initial mergers}
\label{sec:spirantization}

Comparing and contrasting lenition with spirantisation offers some insight into the difficulties around the concept of `word'. I argue that spirantisation only happened non-word-initially.
Spirantisation is similar to the first merger of \lT/\xD\ in this regard, and its mechanics are therefore of interest to this thesis.
The fact that \gls{mow} still has spirantisation as a morphophonemic mutation is because the identification of word boundaries at the time of its phonemicisation differs from \gls{mow} orthography.

As noted in Section~\ref{sec:indet-word-separ}, it is impossible to consistently identify word boundaries across languages.
This also means that word boundaries may not be consistent within different stages of a single language.
The consequence of this point is that one may not give different rules for word-internal spirantisation and for word-external spirantisation without also defining a `word'.
An overview of phonetic contexts in which spirantisation occurred is given by \textcite[2--3]{schrijver_spirantization_1999}, and he separately lists external sandhi causing spirantisation and internal sandhi.
These lists largely overlap.
Word-external sandhi is the most problematic diachronically, because this is where we see the greatest differences between Welsh and Breton.


\Textcite[\S 24]{evans_grammar_1964} lists the following environments as causing spirantisation in  Middle Welsh:
\begin{itemize}
\item possessive pronoun \mw[her]{y}, and its infixed variant \mw[her]{'w};
\item numerals \mw[3]{tri}, \mw[6]{chwech};
\item preposition \mw[over]{tra}, and adverb \mw[very]{tra};
\item conjunction \mw[and, with]{a}, comparative conjunction \mw[than]{no}, and conjunction \mw[neither, nor]{na};
\item negative particles \mw{ny, na}, affirmative particles \mw{neu, ry}, and verbal prefixes \mw{go-, di-, dy-}, \etc;
\item conjunction \mw[if]{o};
\item \mw[where?]{cw}.
\end{itemize}
The same environments also cause spirantisation in Breton for the most part, \eg \mob[her father]{he zad} from \mob[father]{tad}.
Additionally,  spirantisation is found in \gmob[Easter Sunday]{Sul Fask}, and other close compounds containing a single fixed expression.
Also, \gmob{k} is spirantized after the article in some contexts: \mob[the dog]{ar c'hi}, and \gmob[my]{mo} also causes spirantisation.
Finally, place-names in both Wales and Brittany have undergone spirantisation \eg \mow{Ma-thafarn} in Wales and \mob{Poher}, consisting of \mob{pou} < \glat{pagus} + \ob{kaer} in Brittany.

None of these environments bar fixed expressions have spirantisation following a word in an open word class \ie nouns, verbs, or adverbs.
The fixed expressions of the type of \mob{Sul Fask} may be considered a single word for morphosyntactic purposes, \cf Subsection~\ref{sec:petrification}.
The spirant mutation is quite unlike lenition in this respect, considering how the latter may occur \eg following a feminine noun, or in a compound of two nouns. None of the environments above (except perhaps the numerals \mw[3]{tri} and \mw[6]{chwech}) may truly be considered words in the sense that they may be used as an independent phrase.
These elements are also typically unstressed, and may therefore be considered an affix to a stressed word.

A separate approach of internal and external spirantisation is therefore better abandonded. An alternative view is given by \textcite[126--129]{koch_neo-brittonic_1989}, who does note that `a word boundary is not a phonetic concept'.
He then appeals to prosody: two stressed words which became one single stressed word, such as place name \mw{Mathafarn}, may show spirantisation.
Spirantisation also occurs with two semantic words, but with only one word accent, \eg \mw[her house]{y ˈthy}.

The fact that we nowadays write a space between conjunctions, articles, and adverbs, \etc and the following word is not grounded linguistically, and the conventions for writing spaces may differ in other languages, \eg \glat[and men]{virum-que}, Danish \textit{stol-en} `the chair'.
Moreover, verbal prefixes in words such as \mw[take care]{go-chel} are not considered separate words even in Welsh.
If  these elements causing spirantisation may not be considered independent words, then spirantisation may not be considered a process involving two separate words.
Rather, it may be considered a specific type of internal sandhi where a word takes one of these prefixes.

Most pre-apocope masculine  nouns and adjectives ended in an \pbr{*s}.
We know \gpbr{*s} caused spirantisation within a prosodic unit \eg \gpbr{esjās tatos} > \gmow[her father]{ei thad}, yet masculine nouns followed by another prosodic unit never cause spirantisation.
This lack of spirantisation after nouns and adjectives is easily explained if we consider these elements independent words for the purpose of spirantisation.
Considering that they are typically stressed and make sense as an independent phrase, the difference between nouns and adjectives on the one hand, and the environments causing spirantisation on the other hand, lack of spirantisation in the former group can hardly be a coincidence.

These considerations show that the phonological development of spirantisation has a lot in common with the merger of \xD/\lT: both occurred at non-phrase initial position, and the combined result of this merger and spirantisation entailed the loss of phonemic length in stops non-word-initially.
A final point of similarity is that both innovations occurred in the \gls{pbr} period, after the the Brittonic-Goidelic split (Goidelic has neither development), but before the split of \gls{pbr} into Welsh, Cornish, and Breton.

\section{Definitions for words and lenition}
% Perhaps we may turn the question of word segmentation around, and we may answer the question `what is a word?' on the basis of where \lT\ and \xD\ were kept apart. 

Both the problems of word segmentation and lenition --- seemingly very different problems --- lead to the same key in answering their respective issues: 
\begin{itemize}
    \item `Morphophonemic lenition' in \gls{mw} is where both \gls{x}\gls{C}  and \gls{l}\gls{C} are possible initial phonemes of a given morpheme, and where  substitution of either variant phoneme for the other has consequences for the grammaticality or the semantics of a phrase.
    \item A `word' in \gls{mw} is any lexeme which has both \gls{x}\gls{C} and \gls{l}\gls{C} as possible initial phonemes for variant morphemes, and where  substitution of one variant phoneme for another has consequences for a phrase's grammaticality or semantics.
\end{itemize}
This definition of `word' is not a satisfying definition beyond the scope of this thesis, because a strict reading would invalidate any lexeme starting with \mw{s, h, ch, n} or a vowel could not possibly considered a word, because they are not lenitable. It is, however, satisfying as a criterion when deciding what to include in the databases of mutations given in Appendix~\ref{cha:database-lenition} and Appendix~\ref{cha:datab-lenit-mwbuch}.


%%% Local Variables:
%%% mode: latex
%%% TeX-master: "../main"
%%% End:
