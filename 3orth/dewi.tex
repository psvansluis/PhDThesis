\chapter{The stemmatics of \mw{Buchedd Dewi}}
\label{cha:stemm-mwbuch-dewi}
The Welsh life of Saint David, known in Welsh as \mw{Buchedd Dewi} survives in several manuscripts from the fourteenth century onwards. 

Brynley Roberts states that the Latin \lat{Vita Davidis}, a late eleventh-century composition, was translated only in the late thirteenth century. He does not discuss at length why he dates the translation to this period, but he notes that the Welsh version derived from a later edition and omits certain practices and doctrinal elements.

I argue that the dating of this common exemplar may be done on linguistic grounds using my methodology of counting occurrences of lenition of voiceless stops. 

Different stemmata reconstructable with my method:

\begin{figure}[h]
  \centering
  \begin{forest}
    [μ < 1250
    [X]
    [Y]]
  \end{forest}
  \begin{forest}
    [μ < 1250
    [ν > 1300
    [X]
    [Y]]]
 \end{forest}
  \begin{forest}
    [μ > 1300
    [X]
    [Y]]
  \end{forest}
  \caption{Reconstructable stemmata based on analysis of lenition}
  \label{fig:possiblestemmata}
\end{figure}

%%% Local Variables:
%%% mode: latex
%%% TeX-master: "../main"
%%% End:
