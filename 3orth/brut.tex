\chapter{Independent compositions of \mw{Brut y Brenhinedd}}
This chapter analyses the orthography of lenition in Geoffrey of Monmouth's \lat{Historia Regum Brittanniae}, a Latin-language historical prose text dating from the twelfth century.
Translations of this text into Welsh date from the thirteenth onwards.
In total, there are some sixty manuscripts containing some sort of translation or rendition of this history, and they may be divided into six individual classes.
Each of these classes are independent translations of the Latin text; some of them may be considered faithful translations, and others handle the source material more freely~\autocite[xxiv-xxxi]{roberts_brut_1971}.

The existence of several independent translations or reworkings of a single text made over much of the Middle Welsh period is valuable in studying linguistic or orthographical developments.
These texts are valuable because their manuscript witnesses are roughly contemporaneous with the composition of their translation.
Additionally, the independence of these translations means that each version makes for a reliable snapshot of the language and orthography of its time and place.
At the same time, the fact that all \gls{mw} texts are translations of the same text in Latin offers a controlled research environment:  any difference in the language of the translations cannot be due to differing subject matter.
As such, differences in the Welsh translations are controlled for the variable of subject matter.
The independence of these translations and the concurring independence of orthographies contrasts with the law texts discussed in another chapter\todo{Refer!}, as those law texts tend to be Welsh-language copies of Welsh-language texts, carrying over some old orthography.

The fact that we deal with translations of a Latin text means that any occasions where lenition may give rise to ambiguities may be resolved by comparing the Welsh with the Latin.
However, medieval translations are not necessarily faithful to the original:
\tqt{\begin{welsh}
  Nid yw'r cyfieithwyr hyn, fwy na chyfieithwyr cyffredin y cyfnod canol, yn gaeth i'r testun Lladin. [\dots] Newidir rhai manylion neu ychwanegu rhai brawddegau er mwyn cysylltu'r hanes â ffynonellau eraill.%
\end{welsh}%
\footnote{These translators are not slavish to the Latin text any more than typical translators of the medieval  period. [\dots]
  Some details are changed, or some sentences are added in order to connect the history with other sources.}}{bowen_testunau_1974}{290}
Nevertheless, the subject matter of each text --- no matter how many details are altered --- remains historiography, so wildly differing rates of lenition found in each translation cannot possibly be due to differing subject matter found in each manuscript.

A point of caution when dealing with translations is that the language of the original Latin seeps through into the grammar of the Welsh.
The result of this is that some constructions are more frequent than would be expected in a Welsh prose text:
\tqt{
  \begin{welsh}
    Gadawodd y Lladin ei hôl ar arddull y cyfieithiadau, yn yr ansoddeiriau lluosog, a'u safle o
flaen yr enw, yng nghytundeb berf â'i goddrych, yn y gystrawen berthynol, yn y defnydd o enwau
haniaethol, ac o ansoddeiriau berfol i gyfleu rhangymeriad gorffennol, yn y trosi llythrennol o
elfennau gair cyfansawdd, yn y cystrawennau trwsgl a'r colli gafael ar rediad brawddeg wrth geisio
llunio brawddegau cymhleth yn hytrach na brawddegau cydradd, byrion y testunau brodorol.%
  \end{welsh}%
\footnote{The Latin left its mark on the form of the translations, in the plural adjectives, and their place before the noun, agreeing with the verb and its subject, in the relative clause, in the use of abstract nouns, and of verbal adjectives to convey past participles, in literally translating elements of compound words, in the clumsy sentences and in losing the grip on the flow of a sentence from trying to fashion complex sentences instead of the coordinate, short sentences of the original texts.}
}{bowen_testunau_1974}{289}
The most important point here is the placement of adjectives before the noun they qualify.
Wherever this happens, the element following the preposed adjective is lenited.
The use of these adjectives is possible in Welsh, but is highly stylized.
Because translating from Latin brings preposed adjectives out more frequently in Welsh they may have been misunderstood occasionally, as they caused some erratic behaviour, as found in Example~\ref{ex:wychyrcalet}.
There is also a more obvious feature showing these texts were translated from Latin: Latin personal names are found and code switching to Latin also occurs.  

All in all, these translations tend to be slavish in terms of form compared to present-day translations, but freely deviate from the Latin original in terms of content. 





\section{Versions used}
I have taken four different manuscript witnesses of \mw{Brut y Brenhinedd} from four classes for analysis.
Three of these are thirteenth-century manuscripts, and the fourth is from the fourteenth century.
\Textcite{bowen_testunau_1974} discusses these thirteenth-century translations, and notes that they are all similar in terms of how well they are translated.  
\tqt{
  \begin{welsh}
    Fe'i  cyfieithwyd [yr \textit{Historia}] deirgwaith yn  y drydedd ganrif ar ddeg, ac ymddengys fod dau o'r fersiynau hyn, sef hwnnw sydd yn Peniarth 44, a'r un sydd yn Llansteffan 1, yn perthyn i flynyddoedd cynnar y ganrif.
    Dichon fod y trydydd, fersiwn Brut Dingestow, ychydig yn ddiweddarach.
    Y mae iaith y tri fersiwn yn debyg, a chyffelyb hefyd yw safonau'r cyfieithwyr a'u hagwedd at y testun Lladin.%
  \end{welsh}%
\footnote{The [Historia] was translated thrice in the thirteenth century, and it appears that two of these versions, \ie the one that is in Peniarth 44, and the one that is in Llansteffan 1, belong to the early years of the century. It seems that the third, the version of Brut Dingestow, is a bit later. The language of the three versions is similar, and the standards of the translators and their attitude towards the Latin texts are also similar.}
}{bowen_testunau_1974}{288--289}
\Textcite[xxix]{roberts_brut_1971} notes that all threee translations are `close renderings of the latin', and that all translations are independent.
This makes all of these three  manuscripts good independent snapshots of thirteenth-century orthography.


One of the earliest manuscripts containing the \mw{Brut} is \gls{ll1}, and
is dated in the second quarter of the thirteenth century.  A part of
the manuscript contains another version, namely a fragment
of \gls{p44}. The scribe was not a translator, as may be judged from
some mistakes and omissions~\autocite[xxxvii]{roberts_brut_1971}.
Lenition on pages 33 to 38 was analysed.

The \gls{ll1} manuscript forms the basis of an edition
by \textcite{roberts_brut_1971}.  He notes the following on the
orthography of stops and on lenition, respectively:
\tqt{Initially [b, d, g] are always denoted by \textit{b, d, g} ;
  medially they are usually represented by \textit{b, d, g}, with some
  examples of \textit{-p-, -t-, -k-}. Finally, [d] is always
  represented by \textit{t}, [g] by \textit{c} with the exception
  of \textit{og}, but final [b] is represented sometimes
  by \textit{p}, and sometimes by \textit{b, pob, pab, escyb}.  The
  unvoiced stops [p, t, k] occur initially and are written \textit{p,
    t, c/k}. \textit{c-} does not occur often and the scribe prefers
  to use \textit{k-}.  The convention of using \textit{k}
  with \textit{y, i,} or \textit{e} and \textit{c} with other vowels
  and consonants [\dots] does not seem to have been followed and the
  same word may appear with initial \textit{c-} or \textit{k-}.
  As \textit{b, d, g} medially denote [b, d, g] the corresponding
  unvoiced stops can be written \textit{p, t, tt, k, kc}
}{roberts_brut_1971}{xli}
\tqt{ Lenition of initial consonants is not always shown in the text
  but the scribe almost invariably denotes the spirant and nasal
  mutations }{roberts_brut_1971}{xlii}
Neither of these quotes on the language of \gls{ll1} find any pattern in the writing of lenition, or note anything specific about voiceless stops.
These quotes demonstrate that even editors of texts clearly showing a pattern as strong as \gls{ll1} have not yet broken down representation of lenition by consonant type.

The \gls{p44} translation of the \mw{Brut} is roughly as old as the \gls{ll1} one.
\Textcite{lewis_brut_1942} \todo{page?} states that \gls{p44} and \gls{ll1} are based on a common original translation, undoubtedly because the \gls{ll1} manuscript contains a fragment of the \gls{p44} translation, but according to \textcite[xliii--xliv]{roberts_astudiaeth_1969}  they are based on separate translations.
Lenition was analysed for pages 23 to 28.

\Gls{bd} is found in the following manuscript: \gls{nlw} MS 5266.
It is dated to the end of the thirteenth century~\autocite[xliii]{roberts_astudiaeth_1969}.
An edition of this text is prepared by \textcite{lewis_brut_1942}.
His introduction does not mention the orthography of lenition.
Lenition was analysed for pages 26 to 36.

\Textcite{bowen_testunau_1974} compares the methods of translation between these three versions.
He notes that the \gls{ll1} translation is the most faithful one in both style and contents.
The \gls{p44} translation is faithful the Latin on the level of the individual sentence, but shortens the text quite a bit by leaving out some parts.
The translation found in \gls{bd} is more free in terms of style than the \gls{ll1} and \gls{p44} ones, as he shortens the text by summarizing rather than selecting parts of the original.
\tqt{
    \begin{welsh}
Er mor debyg yw ymateb y tri chyfieithydd i'r Historia, y mae eu dull o gyfieithu'n amrywio. Gŵr
cydwybodol, gofalus oedd cyfieithydd fersiwn Llansteffan 1. Ychydig iawn a hepgorodd o'r testun
Lladin ond ceisiodd ei drosi frawddeg wrth frawddeg, hyd yn oed yn y manylion. Ef, yn sicr, a
luniodd y cyfieithiad mwyaf ffyddlon, er bod ei arddull braidd yn drwsgl. Yr oedd ei
destun Lladin ef hefyd yn wahanol mewn mannau i'r un a ddefnyddiwyd gan y ddau gyfieithydd
arall. Diddordeb cyfieithydd Peniarth 44 oedd rhediad yr hanes ei hun. Y mae ganddo arddull
weddol naturiol, uniongyrchol ac adroddiadol. Gall fywiogi'r hanes ond ei nodwedd gyffredin yw ei
fod yn ei gwtogi fel yr â rhagddo, trwy dorri allan frawddegau ac adrannau cyfain, nes y ceir ganddo,
erbyn y diwedd, grynodeb go chwyrn o'r Lladin. Llwyddodd cyfieithydd Brut Dingestow yntau i
gwtogi'r hanes, ond ceisiodd ef dalfyrru'n fwy deallus, nid trwy dorri adrannau allan, ond trwy
grynhoi wrth gyfieithu, a thrwy aralleirio, gan gadw'r cyfan o'r hanes, a hynny mewn Cymraeg digon
llyfn at ei gilydd, gydag adleisiau o'r arddull draddodiadol.%
\end{welsh}%
\footnote{However similar the response of the three translators is to the Historia, their method of translation varies. The translator of Llansteffan 1 was a conscientious, careful man. He left out very little of the Latin text, and tried to translate it sentence by sentence, even down to the details. He, surely, composed the most faithful translation, although his style is quite awkward. His Latin text was also different in some places from the one used by the other two translators. The translator of Peniarth 44 was interested in the flow of the history itself. He has a fairly natural, direct, and narrative style. He may enliven the history, but his general trait is to abbreviate what comes before him, by cutting out whole sentences and parts, so that what is left of him by now is quite a vigorous summary of the Latin. The translator of Brut Dingestow managed to abbreviate the history, but he tried to condense in a more understanding way, not by cutting out sections, but by summarizing while translating, and by rewording, while keeping the whole of the history, and this in an altogether fairly smooth Welsh, with echoes of the traditional style.}
}{bowen_testunau_1974}{292--293}


\Gls{bcc} is the latest manuscript under consideration.
It contains the only translation from after the thirteenth century among the texts analysed in this chapter.
It dates from the mid-fourteenth century according to~\textcite[xlv]{roberts_astudiaeth_1969}, and from the second quarter of the same century according to \textcite[xviii]{jones_brenhinedd_1971}.
The scribe of this \mw{Brut y Brenhinedd} translation is identified as scribe X89.
Lenition on folios 16r to 19v were taken for analysis.

\Textcite{bowen_testunau_1974} notes that the \gls{bcc} translation departs from the original in more ways than just summarizing, as it also incorporates other sources of history:
\tqt{
  \begin{welsh}
Cyfieithiad newydd yw Cotton Cleopatra lle y mae'r gwaith o gymathu â ffynonellau
eraill yn llawer mwy amlwg. Y mae'n fyrrach na'r Historia gan fod y cyfieithydd wedi torri'r areithiau
a llawer o'r disgrifiadau o'r brwydrau, ac wedi hepgor llawer o'r manylion. Ond o'r tu arall, y mae
yma lawer o ychwanegu sy'n gwneud y fersiwn hwn yn fwy storïol na'r un arall.%
\end{welsh}%
\footnote{Cotton Cleopatra is a new translation, where the incorporation of other sources is much more obvious. It is shorter than the Historia because the translator cut out the speeches and many of the descriptions of the battles, after removing many of the details. But on the other hand, there are many additions that make this version more storylike than the other one.}
}{bowen_testunau_1974}{293}


\section{Method}
\label{sec:method}
From each manuscript, I have collected instances of orthographical
lenition as well as instances of where orthographical lenition would
be expected, but is not found. The sample used comprises roughly the
story of Llŷr and his daughters.  Some 250 data points are taken
from each manuscript.

\Gls{petr}, both the type resulting from grammatical lenition and from clitic reduction have been collected for prepositions
and (historically) nouns, but not for personal pronouns such
as \mw[I]{mi/fi}. The reason why these forms are included --- albeit
as research exceptions --- is that their orthography also changed over the
thirteenth century\footnote{See \ref{sec:petrification}.}. Including them allows
us to gauge when this happened relatively to \gls{morphophonlen}.
Including them also allows us to understand how easy it is to miss how \gls{morphophonlen} of voiceless stops was barely ever represented in the earliest translations, because \gls{petr} was already common in the same period.

It is not always clear what types of lenition applied in different
stages of \gls{mw}. Particularly some types of \gls{freelen},
such as object lenition and NP lenition tend to be used only irregularly,
or only in later texts~\autocite{van_sluis_development_2014}, but
with no distinction for consonant type. Evidence for the existence of
lenition differs between texts here, precluding a catch-all policy for
the inclusion of such instances. I generally look for multiple positive
instances attesting to the lenition of objects and nominal predicates.
Consequently, I do not consider non-lenition of \mw[dead]{marỽ} in
Example~\ref{ex:bvmarw} to be relevant for the orthography of lenition,
because NP lenition is only found once elsewhere in this corpus.
As it happens, it is found in the same sentence.
This implies that NP lenition was grammatically optional.
\mwcc[ex:bvmarw]{\gls{ll1} 38.9--10}
{en e tryded wlwydyn gwedy henny e \al{bỽ ỽarỽ} llyr.\ ac e \al{bỽ marỽ} aganyppỽs brenyn ffreync}{In the third year after this Llŷr died, and Aganippus, king of France, died.}


\section{An overview of the results}
\label{sec:comparison-versions}
This section contains one sentence found in all the manuscripts in some form, in order to demonstrate how the orthography of lenition developed around the end of the thirteenth century.
Roughly speaking, Examples~\ref{ll1llyr} and \ref{p44llyr} from the mid-thirteenth century show that lenited voiceless stops were not written.
Example \ref{bdllyr} from the late thirteenth century shows lenition of \mw{g}, but not of \mw{t}, and thus forms an intermediate stage before voiceless stops were all written lenited.
The final stage of full lenition is found in Example~\ref{bccllyr}.

\begin{mwl}
  \mwc[ll1llyr]{\acrshort{ll1} 34.16--18}{Ac ena hep ỽn gohyr
    o \al{k}yghor y \al{w}yrda ef ar rodes e dwy \al{ỽ}erchet hynaf
    ydaỽ yr deỽ \al{t}ewyssaỽc. nyt amgen tewyssaỽc kernyw ac ỽn
    gogled}{And then without any delay of counsel of his noblemen, he
    gave his two eldest daughters to the two princes no other than the
    prince of Cornwall and one of [the] North.}
  \mwc[p44llyr]{\acrshort{p44} 24.21--23}{Ac ena ny \al{b}ỽ ỽn gohyr
    o \al{k}yghor y \al{w}yrda er rodes ef e dwy \al{ỽ}erchet hynaf y
    deỽ \al{t}ewyssaỽc nyt amgen a thewyssaỽc kernyw. a thewyssaỽc e
    gogled.}{And then there was no delay of counsel of his noblemen
    that he gave his two eldest daughters to two princes no other than
    the prince of Cornwall and the prince of the North.}
  \mwc[bdllyr]{\acrshort{bd} 31.6--7}{A hep \al{o}hir
    o \al{g}yt\al{g}yghor y \al{w}yrda y rodes ef y dỽy \al{u}erched
    hynhaf ydaỽ y \al{t}ywyssogyon yr alban. a chernyỽ. }{And without
    delay of counsel of his noblemen, he gave his two eldest daughters
    to the princes of Scotland and Cornwall.}
  \mwc[bccllyr]{\acrshort{bcc} 17r.21--24}{Ac yna yn diohir y rodes ef
    y dwy \al{v}erchet hynaf y deu \al{d}ywyssawc nyt amgen tywyssawc
    kernyw ar hwnn yr alban}{And then without delay he gave his two
    eldest daughters to two princes no other than the prince of
    Cornwall and the one of Scotland.}
\end{mwl}

Table~\ref{tab:perlenbrut} confirms these impressions: only a minority of lenited \mw{p, t} is written until \gls{bcc}, while a minority of lenited \mw{c} is only written until \gls{bd}
\footnote{This result shows that the behaviour of Scribe X86 is not unique to this scribe\todo[inline]{Refer to chapter}.}.

Comparison between Table~\ref{tab:perlenbrut} and Table~\ref{tab:perlenbrutex} shows that most early instances of lenited \mw{c} written with \mw{g} are in fact instances of \gls{petr}. 


\begin{table}[h]
  \centering
  \begin{tabular}{lddddddd}
    \toprule
    \tch{Manuscript} & \tch{\mw{b}} & \tch{\mw{g}} & \tch{\mw{ll}} & \tch{\mw{m}} & \tch{\mw{p}} & \tch{\mw{t}} & \tch{\mw{c}} \\
    \midrule
    \acrshort{ll1} & 82.1 & 72.1 & 83.3 & 98.2 & 0.0 & 9.1 & 35.8 \\
    \acrshort{p44} & 82.5 & 36.1 & 78.6 & 91.7 & 8.3 & 13.3 & 29.1 \\
    \acrshort{bd} & 89.7 & 98.4 & 87.5 & 100.0 & 22.2 & 33.3 & 92.2 \\
    \acrshort{bcc} & 67.9 & 98.5 & 85.7 & 100.0 & 95.8 & 83.3 & 86.2 \\
    \bottomrule
  \end{tabular}%
  \caption{Lenition including research exceptions represented in the \mw{Brut}, in percentages.}
  \label{tab:perlenbrut}
\end{table}

\begin{table}[h]
  \centering
  \begin{tabular}{lddddddd}
    \toprule
    \tch{Manuscript} & \tch{\mw{b}} & \tch{\mw{g}} & \tch{\mw{ll}} & \tch{\mw{m}} & \tch{\mw{p}} & \tch{\mw{t}} & \tch{\mw{c}} \\
    \midrule
\acrshort{ll1} & 82.1 & 72.1 & 83.3 & 98.1 & 0.0 & 0.0 & 8.7 \\
\acrshort{p44} & 82.5 & 36.1 & 78.6 & 91.4 & 8.3 & 0.0 & 7.3 \\
\acrshort{bd} & 89.7 & 98.4 & 87.5 & 100.0 & 22.2 & 0.0 & 88.0 \\
\acrshort{bcc} & 67.9 & 98.5 & 85.7 & 100.0 & 95.8 & 80.0 & 78.6 \\
    \bottomrule
  \end{tabular}%
  \caption{Lenition excluding research exceptions represented in the \mw{Brut}, in percentages.}
  \label{tab:perlenbrutex}
\end{table}

 


\section{Lenition of voiceless stops}
\label{sec:lenit-voic-stops}

Table~\ref{tab:perlenbrutex} shows how rarely morphophonemic lenition of \mw{p, t, c} is represented in the earlier translations of the \mw{Brut}. The orthographical development of these consonants will be discussed in this section.

\subsection{Lenition in \acrshort{ll1} and \acrshort{p44}}
\label{sec:lenit-acrsh-acrsh}


The rule is clear in \gls{ll1} and \gls{p44}: lenition of voiceless stops is not written.
Given this rule, exceptions, constituting instances of representation of lenited voiceless stops, need to be accounted for.
These exceptions are found in Table~\ref{tab:ltrepll1p44}.
As can be seen in this table, both the form and meaning of the words themselves and their reason for being lenited differs, so accounting for them must be done on a case-to-case basis, and a satisfying account may not always be found.

\begin{table}[h]
  \centering
  \begin{tabular}{lddwql}
    \toprule
    \tch{Source} & \tch{Page} & \tch{Line} & \tch{Word} & \tch{Translation} & \tch{Reason lenition} \\
    \midrule
    \acrshort{ll1} & 34 & 1 & gellweyr & jest & \mw{trwy} \\
    \acrshort{ll1} & 35 & 8 & gytdỽundep & agreement & \mw{o} \\
    \acrshort{ll1} & 36 & 4 & gof & memory & \mw{ar} \\
    \acrshort{ll1} & 37 & 17 & glaf & sick & \mw{en} \\
    \acrshort{p44} & 25 & 10 & gewylyd & shame & \mw{en} \\
    \acrshort{p44} & 26 & 2 & gynt & before & adv phrase \\
    \acrshort{p44} & 26 & 9 & bryt & moment & \mw{pa} \\
    \acrshort{p44} & 27 & 6 & glaf & sick & \mw{en} \\
    \bottomrule
  \end{tabular}%
  \caption{Instances of \lT\ represented in \acrshort{ll1} and \acrshort{p44}.}
  \label{tab:ltrepll1p44}
\end{table}

One word that may be accounted for, however, is \mw[before]{gynt} (\gls{p44} 26.2).
It stands out in that it is the only lenited adverbial phrase in this source.
Moreover, lenition of adverbial phrases of time is frequently petrified, \eg \gmow[yesterday]{ddoe}.
The writing of \mw{gynt} with \mw{g} may similarly point to \gls{petr}, rather than \gls{morphophonlen}.
If so, this word constitutes a research exception.

The case of \mw[what moment?]{pa bryt} may be explained in a similar manner.
Although lenition following \mw{pa} is morphophonemic and still grammatical, it occurs frequently in combination with \mw{bryt}, and orthographical lenition may thus be reminiscent of the \gls{petr} we find in research exception \mw[together]{y gyt}.
Moreover, the semantics of the phrase obviously have a temporal dimension similar to \mw{gynt}.
The phrase \mw{pa bryt} is also found with exceptional lenition in \gls{bd}, as can be seen in Table~\ref{tab:replenpbd}.

After accounting for these two examples, we are still left with six examples of orthographically represented \gls{morphophonlen}.
These examples are few and far between, but nevertheless essential, because they show that lenition of voiceless stops \emph{could} be represented, it was just not widely practiced.
This means that writing \graph{b, d, g} for \lT\ had already been invented by the mid-thirteenth century.
It had just not been adopted consistently.


\subsection{Lenition in \acrshort{bd} }
\label{sec:lenition-acrshortbd-}
In \gls{bd}, the rule is that lenition is not represented for \mw{p, t}, and it is represented for \mw{c}.
This rule is exceptionless for \mw{t}, but both \mw{p} and \mw{c} have some exceptions to this rule.
For \mw{p}, these exceptions constitute instances where lenition is represented, and for \mw{c} these exceptions constitute instances where lenition is not represented.
Table~\ref{tab:replenpbd} shows the two instances of orthographically represented lenited \mw{p}, and Table~\ref{tab:nonlencbd} shows every instance where lenited \mw{c} is not orthographically represented.

\begin{table}[h]
  \centering
  \begin{tabular}{addwql}
    \toprule
    \tch{Source} & \tch{Page} & \tch{Line} & \tch{Word} & \tch{Translation} & \tch{Reason lenition} \\
    \midrule
    bd & 34 & 4 & bryt & moment & \mw{pa} \\
    bd & 36 & 2 & baraỽt & ready & \mw{yn} \\
    \bottomrule
  \end{tabular}
  \caption{Representation of lenited \mw{p} in \acrshort{bd}}
  \label{tab:replenpbd}
\end{table}

\begin{table}[h]
  \centering
  \begin{tabular}{addwql}
    \toprule
    \tch{Source} & \tch{Page} & \tch{Line} & \tch{Word} & \tch{Translation} & \tch{Reason lenition} \\
    \midrule
    bd & 29 & 10 & caerussalem & (place name) & fem noun \\
    bd & 30 & 14 & kyuoeth & kingdom & \mw{y} ‘his' \\
    bd & 31 & 8 & kyuoeth & kingdom & \mw{y} ‘his' \\
    bd & 31 & 9 & kyuoeth & kingdom & \mw{y} ‘his' \\
    bd & 33 & 2 & caffei & received & \mw{na} \\
    bd & 35 & 1 & keissyaỽ & seek & \mw{y} ‘to' \\
    \bottomrule
  \end{tabular}%
  \caption{Non-representation of lenited \mw{c} in \acrshort{bd}}
  \label{tab:nonlencbd}
\end{table}

\subsection{Lenition in \acrshort{bcc}}
\label{sec:lenition-acrshortbcc}


Lenition of voiceless stops is as a rule represented in \gls{bcc}, so instances where lenition is not shown are the ones that need to be accounted for.
Table~\ref{tab:ltnotrepbcc} shows these instances of non-represented lenition.
Various reasons for lenition appear multiple times in this table.



\begin{table}[h]
  \centering
  \begin{tabular}{lddwql}
    \toprule
    \tch{Source} & \tch{Page} & \tch{Line} & \tch{Word} & \tch{Translation} & \tch{Reason lenition} \\
    \midrule
    \gls{bcc} & 16r & 29 & prosessio & procession & \mw{y} ‘to' \\
    \gls{bcc} & 16v & 6 & keluydodeu & arts & prep adj \\
    \gls{bcc} & 17r & 14 & kereis & I loved & \mw{th} \\
    \gls{bcc} & 17r & 15 & caraf & I love & \mw{th} \\
    \gls{bcc} & 17r & 16 & kerir & is loved & \mw{th} \\
    \gls{bcc} & 17v & 18 & tywyssauc & prince & apposition \\
    \gls{bcc} & 17v & 29 & tywyssawc & prince & apposition \\
    \gls{bcc} & 18r & 12 & trugarhae & mercy & \mw{y} ‘his' \\
    \gls{bcc} & 18r & 25 & Cordeilla & (personal name) & \mw{-ei} \\
    \gls{bcc} & 19r & 10 & Cordeilla & (personal name) & \mw{-ei} \\
    \gls{bcc} & 19v & 4 & tywyssawc & prince & apposition \\
    \gls{bcc} & 19v & 4 & tywyssawc & prince & apposition \\
    \gls{bcc} & 19v & 5 & calet & hard & prep adj \\
    \gls{bcc} & 19v & 13 & cordeilla & (personal name) & \mw{y} ‘to' \\
    \gls{bcc} & 19v & 25 & creftwyr & craftsmen & prep adj \\
    \bottomrule
  \end{tabular}%
  \caption{Instances of \lT\ not represented in \acrshort{bcc}.}
  \label{tab:ltnotrepbcc}
\end{table}

Within \gls{bcc}, \mw[prince]{tywyssawc} is found four times in apposition to a personal name.
A lenitable noun in apposition to a personal name is found eight times within this source, and is lenited only once in \mw[prophet]{broffwid} (\gls{bcc} 16r.26).
Lenition of nouns in apposition is not shown for other types of consonants either, as the word \mw[king]{brenhin} is found in unlenited form in this position also.

The word \mw{prosessio} is a late, learned loanword from \glat[procession]{prōcessiō}.
Writing \glat{c} in this word with \mw{s} confirms a medieval pronunciation.
Phonologically, the structure of the word does not look Middle Welsh, as /au/ had not yet turned into /o/ in the final syllable.
This means that \mw{prosessio} constitutes an instance of code switching, or a recent loan.
Recent and transparent loanwords are not typically mutated in \gls{mow}, nor are instances of code switching.
The same process was at play in this word.

Personal name \mw{Cordeilla} is found unlenited where lenition would be expected three times.
The fact that we are dealing with a personal name here may be of influence, as \mw{llyr} is found unlenited following \mw[fort]{caer} twice.
Furthermore, personal names are not as a rule lenited in \gls{mow}.
These instances of \mw{llyr} and \mw{Cordeilla} may be early examples of this \gls{mow} rule.
Additionally, \mw{Cordeilla} is a foreign name for the Welsh, and it may thus not have been lenited similarly to \mw{prosessio}.

Infixed object pronoun \mw{'th} should cause lenition. It does not do so here.
Lenition in this context may be problematic diachronically \todo{How exactly is it problematic?}

Several words following a preposed adjective fail to show lenition, although these instances are outnumbered by preposed adjectives shown to cause lenition.
Lenition following preposed adjectives is a type of free lenition, because any adjective principally causes lenition when it is used as a preposed adjective.
These same adjectives do not principally cause lenition when they are in their usual postnominal position.
This means that the correct application of lenition hinges on the morphosyntactic relationship between two elements in a clause rather than by simply checking whether the immediately preceding morpheme causes lenition.
This morphosyntactic relationship may not always be clear, either to us or the scribes and translators of the \gls{mw} text.
Preposed adjectives are more frequent in this text than in \gls{mw} in general, because it is a translation from Latin.
This further increases the likelihood of grammatical slip-ups in the application of preposed adjectives.
An example where I found it difficult to establish the exact relationship involving a supposedly preposed adjective is Example~\ref{ex:wychyrcalet}:
\mwcc[ex:wychyrcalet]{\gls{bcc} 19v.3--6}{ac yn ev herbyn wynt y doeth Maglawn tywyssawc yr alban. a henwyn tywyssawc kernyw ac ev holl allu. ac ymlad yn \al{wychyr calet} ac wynt.}{And against them came Maglawn, prince of Scotland, and Henwyn, prince of Cornwall, and their whole capability, and they fought \al{violently hard} against them.}
What exactly is the grammatical relationship between \mw[violent]{wychyr} and \mw[hard]{caled}?
Is \mw{wychyr} a preposed adjective, translating to the translation given, or does \mw{caled} modify \mw{wychyr}?
In the latter case, `toughly violent' would be a more suitable translation. 
Judgments like these may not be made consistently, either by the translators and copyists who produced this work, or by me when deciding where lenition was not represented\footnote{The Latin original is of little help either, as it is much terser than the Welsh: \lat[and they two went to Lloegr, llvr and kordaila his daughter and that host, to fight with his two sons-in-law, and got the victory over them.]{Quo facto duxit secum leir filiam \& collectam multitudinem. pugnauitque cum generis \& triumpho potitus.}~\autocite[269]{griscom_historia_1929}}.
This inability to make consistent judgments in writing and editing a text may make for some seeming inconsistencies in an otherwise consistent pattern of mutations.

Many exceptions to representation of lenition in \gls{bcc} stem from the difficulties involved in translating a Latin text.
After accounting for these, lenition seems to have been applied with great regularity, and remaining exceptions seem no more numerous than what is found in a \gls{mow} text of similar length, barring only the most carefully copy-edited ones.
As such, \gls{bcc} demonstrates that lenition of voiceless stops was wholly represented by the mid-fourteenth century.

\section{Lenited \mw{g}}
\label{sec:lenited-mwg}
The only number in Table~\ref{tab:perlenbrut} or Table~\ref{tab:perlenbrutex} not referring to a voiceless stop and dipping below the fifty per cent mark is that of the representation of lenited \mw{g} in \gls{p44}.
This begs the question why \mw{g} in \gls{p44} --- and to a lesser
extent \gls{ll1} --- is lenited so infrequently.  In \gls{ll1}
lenited \mw{g} is written in 49 out of 68 instances; in \gls{p44} it
is written in  only 26 out of 72 instances.  The element causing lenition
does not seem to be relevant, \eg verbal particle \mw{a} regularly
causes lenition to \mw[did]{oruc}, but not to \mw[did]{gwnaeth}.

The phonology of the lenited word does seem to play a role: when
initial \mw{g} is not followed by \mw{w}, it disappears in 48 out of
52 of such instances. The remaining four comprise the following
instances: \mw[they rested]{gorffowyssassant} (\gls{ll1} 38.25),
\mw[rested]{gor/ffowyssỽs} (\gls{p44} 23.9--10), \mw[gain]{gorescyn}
(\gls{p44} 27.15), and \mw[glorious]{gogonedỽs} (\gls{p44}
28.23). These words all used to start with \mw{g\cw}, but  \mw{\cw}
preceding \mw{o} has disappeared by the end of the \gls{ow} period. However, such words
starting with \mw{go} do represent lenition elsewhere, such as in the
plentiful instances of \mw[did]{oruc}.

The remaining 88 instances are all words starting with \mw{(g)w}. 27 of these show lenition.
Table~\ref{tab:gwphon} shows that the phonological structure of these words plays an important role in dictating whether lenition is written.
If the \mw{w} following lenited \mw{g} is vocalic, lenition is usually shown.
An example of such a word showing lenition is \mw[husband]{wr} (\gls{ll1} 34.13).
Similarly, lenition is usually shown if the quality of the following \mw{\cw} is consonantal, if this consonant is in turn followed by a vowel, \eg \mw[wear]{wyscaỽ} (\gls{p44} 27.7).
However, if the \mw{g} is followed by consonantal \mw{\cw}, and then followed by yet another consonant, lenition is usually not written, \eg \mw[make]{gwneỽthỽr} (\gls{p44} 27.28).

\begin{table}[h]
  \centering
  \begin{tabular}{ldd}
    \toprule
    & /g\cw{}C/ & /g\cw{}V, gu/ \\
    \midrule
    {\graph{w}} & 3 & 24 \\
    {\graph{gw}} & 48 & 13 \\
    \bottomrule
  \end{tabular}%
  \caption{Lenition of \mw{gw} divided by phonological structure of the word.}
  \label{tab:gwphon}
\end{table}

Not representing \mw{g} served a purpose: it served to distinguish consonantal \mw{w} from its syllabic counterpart if it followed \mw{g} and preceded a consonant.
Maintaining  radical \mw{g} in the face of lenition served to indicate that the following \mw{\cw} before another consonant was consonantal, so that no reader of a phrase like \mw[his wife]{ẏ wreic} would ever be fooled into saying /i ur\dots/ before reading the rest of the phrase and having to correct himself to /i \cw raɪɡ/.

\section{Conclusion}
\label{sec:conclusion-brut}
Lenition of voiceless stops was generally not written in the early thirteenth century, but it was not unfamiliarity with the concept of using \graph{b, d, g} for historical \mw{p, t, c} that kept them from writing lenition as such.
After all, these letters were used for \gls{petr}, and in exceptional cases even for \gls{morphophonlen}.
In the late thirteenth century, lenited \mw{c} came to be written as \mw{g}, but a similar shift did not occur for \mw{p} and \mw{t}.
These  departures from the thirteenth-century norm give us one key insight: lenition could be represented by this period, it had just not become norm yet.

The case of lenited \mw{g} shows how the question of whether and how to write lenition may interplay with seemingly unrelated conundrums. 
A scribe who wished to ensure that his consonantal \mw{\cw}'s were not mistaken for vocalic \mw{w}'s had to weigh this wish against his wish to write lenition.
The former wish took precedence for the scribes of \gls{ll1} and \gls{p44}, and their priorities make sense in an orthographical tradition where the writing of lenition was limited to only several consonants, and where ability to read out loud would have been held in high esteem.

So what was the conundrum for voiceless stops?
What wish had to be weighed against the wish to represent lenition of these consonants?
The obvious answer to this question is that \lT≠\xD.
If lenited voiceless stops had not yet merged with radical voiced stops word-initially, then confusing the two when speaking might have been just as embarassing as mixing up your vocalic and consonantal \mw{w}'s. 
According to this line of reasoning, the phonological distinction between \lT\ and \xD\ argued for in Part~\ref{part:phonology-phonetics} must have survived until the middle of the thirteenth century.

%%% Local Variables:
%%% coding: utf-8
%%% mode: latex
%%% TeX-master: "../main"
%%% End:
