\chapter{Independent compositions of \mw{Brut y Brenhinedd}}
This chapter analyses the orthography of lenition in Geoffrey of Monmouth's \lat{Historia Regum Brittanniae}, a Latin-language historical prose text which has been translated into Welsh several times. Translations of this eleventh-century text date from the thirteenth onwards. In total, there are some sixty manuscripts containing some sort of translation or rendition of the \lat{Historia Regum}, and they may be divided into six individual classes. Each of these classes are individual renditions of the Latin text; some of them may be considered independent translations, and others handle the source material more freely~\autocite[xxiv-xxxi]{roberts_brut_1971}.

The existence of several independent translations or reworkings of a single text made over much of the Middle Welsh period is valuable in studying linguistic or orthographical developments. These texts are valuable because their manuscript witnesses are often contemporaneous with the composition of their translation. Additionally, the independence of these translations from each other means that each version makes for a reliable snapshot of the language and orthography of its time and place. Furthermore, the fact that we deal with translations of a Latin text means that any occasions where the putative existence of lenition may give rise to ambiguities may be resolved by comparing the Welsh with the Latin. 

The independence of these translations and the concurring independence of orthographies may be compared and contrasted with the law texts discussed in another chapter\todo{Refer!}. 

\section{Versions used}
Here, I discuss the manuscript witnesses of \mw{Brut y Brenhinedd} chosen for analysis. \Gls{ll1} and \gls{p44} are some of the earliest manuscripts, and they constitute independent translations from the Latin:
\tqt{Dingestow, Peniarth 44, and Llanstephan 1, all belong to the thirteenth centuray and all are close renderings of the Latin. They represent three independent translations, although [\dots] the Dingestow and Llanstephan versions have influenced each other.}{roberts_brut_1971}{xxix}
This makes these two manuscripts good independent snapshots of thirteenth-century orthography.

\subsection{Llanstephan 1}
The earliest manuscript containing the \mw{Brut} is \gls{ll1}, which is dated in the second quarter of the thirteenth century. 
A part of the manuscript contains another version, namely a fragment of \gls{p44}. The scribe was not a translator, as may be judged from some mistakes and omissions~\autocite[xxxvii]{roberts_brut_1971}. 

This manuscript forms the basis of an edition by \textcite{roberts_brut_1971}. 
He notes the following on the orthography of stops and on lenition, respectively:
\tqt{Initially [b, d, g] are always deboted by \textit{b, d, g} ; medially they are usually represented by \textit{b, d, g}, with some examples of \textit{-p-, -t-, -k-}. Finally, [d] is always represented by \textit{t}, [g] by \textit{c} with the exception of \textit{og}, but final [b] is represented sometimes by \textit{p}, and sometimes by \textit{b, pob, pab, escyb}. The unvoiced stops [p, t, k] occur initially and are written \textit{p, t, c/k}. \textit{c-} does not occur often  and the scribe prefers to use \textit{k-}. The convention of using \textit{k} with \textit{y, i,} or \textit{e} and \textit{c} with other vowels and consonants [\dots] does not seem to have been followed and the same word may appear with initial \textit{c-} or \textit{k-}. As \textit{b, d, g} medially denote [b, d, g] the corresponding unvoiced stops can be written \textit{p, t, tt, k, kc}
}{roberts_brut_1971}{xli}
\tqt{Lenition of initial consonants is not always shown in the text but the scribe almost invariably denotes the spirant and nasal mutations
}{roberts_brut_1971}{xlii}
It is obvious from these quotes that Roberts fails to problematise the erratic representation of lenition. 
Not only does he fail to account for the distribution of represented and unrepresented lenitions, but he also fails to see that there is a topic to be studied at all.

The \todo{Some notes on the geography of its composition, if possible.}

\section{Peniarth 44}
