\chapter{Independent compositions of \mw{Brut y Brenhinedd}}
This chapter analyses the orthography of lenition in Geoffrey of Monmouth's \lat{Historia Regum Brittanniae}, a Latin-language historical prose text which has been translated into Welsh several times.
Translations of this eleventh-century text date from the thirteenth onwards.
In total, there are some sixty manuscripts containing some sort of translation or rendition of the \lat{Historia Regum}, and they may be divided into six individual classes.
Each of these classes are individual renditions of the Latin text; some of them may be considered independent translations, and others handle the source material more freely~\autocite[xxiv-xxxi]{roberts_brut_1971}.

The existence of several independent translations or reworkings of a single text made over much of the Middle Welsh period is valuable in studying linguistic or orthographical developments.
These texts are valuable because their manuscript witnesses are often contemporaneous with the composition of their translation. Additionally, the independence of these translations from each other means that each version makes for a reliable snapshot of the language and orthography of its time and place.
Furthermore, the fact that we deal with translations of a Latin text means that any occasions where the putative existence of lenition may give rise to ambiguities may be resolved by comparing the Welsh with the Latin. 

The independence of these translations and the concurring independence of orthographies may be compared and contrasted with the law texts discussed in another chapter\todo{Refer!}.



\section{Method}
\label{sec:method}
From each manuscript, I have collected instances of orthographical
lenition as well as instances of where orthographical lenition would
be expected, but is not found. The sample used comprises roughly the
story of Ll\^yr and his daughters.  Some 250 data points are taken
from each manuscript.

\Gls{petr}, both the type resulting from grammatical lenition and from clitic reduction have been collected for prepositions
and (historically) nouns, but not for personal pronouns such
as \mw[I]{mi/fi}. The reason why these forms are included --- albeit
as research exceptions --- is that their orthography also changed over the
thirteenth century\footnote{See \ref{sec:petrification}.}. Including them allows
us to gauge when this happened relatively to morphophonemic lenition.

It is not always clear what types of lenition applied in different
stages of \gls{mw}. Particularly some types of free
lenition\todo{define glossary term}, such as object lenition and NP
lenition tend to be used only irregularly, or only in later texts~\autocite{van_sluis_development_2014}, but
with no distinction for consonant type. Evidence for the existence of
lenition differs between texts here, precluding a catch-all policy for
the inclusion of such instances. I generally look for multiple positive
instances attesting to the lenition of objects and nominal predicates.
Consequently, I do not consider non-lenition of \mw[dead]{marỽ} in
Example~\ref{ex:bvmarw} to be relevant for the orthography of lenition,
because NP lenition is only found once elsewhere in this corpus in the
same sentence. This implies that lenition was grammatically optional here.
\mwcc[ex:bvmarw]{\gls{ll1} 38.9--10}
{en e tryded wlwydyn gwedy henny e bỽ ỽarỽ llyr.\ ac e \al{bỽ} marỽ aganyppỽs brenyn ffreync}{In the third year after this Ll\^yr died, and Aganippus, king of France, died.}


\section{Versions used}
Here, I discuss the manuscript witnesses of \mw{Brut y Brenhinedd}
chosen for analysis. \Gls{ll1} and \gls{p44} are some of the earliest
manuscripts, and they constitute independent translations from the
Latin:
\tqt{ Dingestow, Peniarth 44, and Llanstephan 1, all belong to the
  thirteenth century and all are close renderings of the Latin.  They
  represent three independent translations, although [\dots] the
  Dingestow and Llanstephan versions have influenced each other.
}{roberts_brut_1971}{xxix}
This makes these two manuscripts good independent snapshots of
thirteenth-century orthography.

\subsection{Llanstephan 1}
The earliest manuscript containing the \mw{Brut} is \gls{ll1}, which
is dated in the second quarter of the thirteenth century.  A part of
the manuscript contains another version, namely a fragment
of \gls{p44}. The scribe was not a translator, as may be judged from
some mistakes and omissions~\autocite[xxxvii]{roberts_brut_1971}.

This manuscript forms the basis of an edition
by \textcite{roberts_brut_1971}.  He notes the following on the
orthography of stops and on lenition, respectively:
\tqt{ Initially [b, d, g] are always deboted by \textit{b, d, g} ;
  medially they are usually represented by \textit{b, d, g}, with some
  examples of \textit{-p-, -t-, -k-}. Finally, [d] is always
  represented by \textit{t}, [g] by \textit{c} with the exception
  of \textit{og}, but final [b] is represented sometimes
  by \textit{p}, and sometimes by \textit{b, pob, pab, escyb}.  The
  unvoiced stops [p, t, k] occur initially and are written \textit{p,
    t, c/k}. \textit{c-} does not occur often and the scribe prefers
  to use \textit{k-}.  The convention of using \textit{k}
  with \textit{y, i,} or \textit{e} and \textit{c} with other vowels
  and consonants [\dots] does not seem to have been followed and the
  same word may appear with initial \textit{c-} or \textit{k-}.
  As \textit{b, d, g} medially denote [b, d, g] the corresponding
  unvoiced stops can be written \textit{p, t, tt, k, kc}
}{roberts_brut_1971}{xli}
\tqt{ Lenition of initial consonants is not always shown in the text
  but the scribe almost invariably denotes the spirant and nasal
  mutations }{roberts_brut_1971}{xlii}
It is obvious from these quotes that Roberts fails to problematise the
erratic representation of lenition.  Not only does he fail to account
for the distribution of represented and unrepresented lenitions, but
he also fails to see that there is a topic to be studied at all.

The \todo{Some notes on the geography of its composition, if
  possible.}

Lenition on pages 33 to 38 was analysed.

\subsection{Peniarth 44}
The \gls{p44} translation of the \mw{Brut} is roughly as old as the \gls{ll1} one.
\Textcite{lewis_brut_1942} states that \gls{p44} and \gls{ll1} are based on a common original translation,
but according to \textcite[xliii--xliv]{roberts_astudiaeth_1969}  they are based on separate translations.

Lenition was analysed for pages 23 to 28.
\subsection{Brut Dingestow}
\label{sec:dingestow}
\Gls{bd} is found in the following manuscript: \gls{nlw} MS 5266.  It
is dated to the end of the thirteenth century~\autocite[xliii]{roberts_astudiaeth_1969}.
An edition of this text is prepared by \textcite{lewis_brut_1942}.  His
introduction does not mention the orthography of lenition, however.

The translation found in \gls{bd} is more free than the \gls{ll1} and \gls{p44} ones.

\tqt{I ailadrodd, bu o leiaf dri pherson rywdro'n cyfieithu Historia
  Sieffre i'r Gymraeg, bob un yn ei ddull ei hun. Cynrychiolir gwaith
  un ohonynt yn y testun hwn, gwaith un arall yn Havod 2, a gwaith y
  trydydd yn Peniarth 44-Llanstephan 1}{lewis_brut_1942}{xxiv}

Lenition was analysed for pages 26 to 36.
\subsection{Cotton Cleopatra B V i}
\label{sec:cotton-cleopatra-b}
\Gls{bcc} is the latest manuscript under consideration.
It dates from the mid-fourteenth century~\autocite[xlv]{roberts_astudiaeth_1969}.
The scribe of this \mw{Brut y Brenhinedd} translation is identified as scribe X89\autocite{_tei_???}. 

Lenition on folios 16r to 19v were taken for analysis.

\section{Comparison of the versions}
\label{sec:comparison-versions}
The following sentence found in several manuscripts demonstrates how the orthography of lenition developed around the end of the thirteenth century.
Roughly speaking, Examples~\ref{ll1llyr} and \ref{p44llyr} from the mid-thirteenth century show that lenited voiceless stops were not written.
Example \ref{bdllyr} from the late thirteenth century shows lenition of \mw{g}, but not of \mw{t}, and thus forms an intermediate stage before voiceless stops were all written lenited.
The final stage of full lenition is found in Example~\ref{bccllyr}.

\begin{mwl}
  \mwc[ll1llyr]{\acrshort{ll1} 34.16--18}{Ac ena hep ỽn gohyr
    o \al{k}yghor y \al{w}yrda ef ar rodes e dwy \al{ỽ}erchet hynaf
    ydaỽ yr deỽ \al{t}ewyssaỽc. nyt amgen tewyssaỽc kernyw ac ỽn
    gogled}{And then without any delay of counsel of his noblemen, he
    gave his two eldest daughters to the two princes no other than the
    prince of Cornwall and one of [the] North.}
  \mwc[p44llyr]{\acrshort{p44} 24.21--23}{Ac ena ny \al{b}ỽ ỽn gohyr
    o \al{k}yghor y \al{w}yrda er rodes ef e dwy \al{ỽ}erchet hynaf y
    deỽ \al{t}ewyssaỽc nyt amgen a thewyssaỽc kernyw. a thewyssaỽc e
    gogled.}{And then there was no delay of counsel of his noblemen
    that he gave his two eldest daughters to two princes no other than
    the prince of Cornwall and the prince of the North.}
  \mwc[bdllyr]{\acrshort{bd} 31.6--7}{A hep \al{o}hir
    o \al{g}yt\al{g}yghor y \al{w}yrda y rodes ef y dỽy \al{u}erched
    hynhaf ydaỽ y \al{t}ywyssogyon yr alban. a chernyỽ. }{And without
    delay of counsel of his noblemen, he gave his two eldest daughters
    to the princes of Scotland and Cornwall.}
  \mwc[bccllyr]{\acrshort{bcc} 17r.21--24}{Ac yna yn diohir y rodes ef
    y dwy \al{v}erchet hynaf y deu \al{d}ywyssawc nyt amgen tywyssawc
    kernyw ar hwnn yr alban}{And then without delay he gave his two
    eldest daughters to two princes no other than the prince of
    Cornwall and the one of Scotland.}
\end{mwl}

Table~\ref{tab:perlenbrut} confirms these impressions: only a minority of lenited \mw{p, t} is written until \gls{bcc}, while a minority of lenited \mw{c} is only written until \gls{bd}.
This result shows that the behaviour of Scribe X86 is not unique to this scribe\todo{Refer to chapter}. 

Comparison between Table~\ref{tab:perlenbrut} and Table~\ref{tab:perlenbrutex} shows that most early instances of lenited \mw{c} written with \mw{g} are in fact instances of \gls{petr}. 

The only other number in Table~\ref{tab:perlenbrut} dipping below the fifty per cent mark is that of the representation of lenited \mw{g} in \gls{p44}.
\begin{table}[h]
  \centering
  \begin{tabular}{lddddddd}
    \toprule
    \tch{Manuscript} & \tch{\mw{b}} & \tch{\mw{g}} & \tch{\mw{ll}} & \tch{\mw{m}} & \tch{\mw{p}} & \tch{\mw{t}} & \tch{\mw{c}} \\
    \midrule
    \acrshort{ll1} & 82.1 & 72.1 & 83.3 & 98.2 & 0.0 & 9.1 & 35.8 \\
    \acrshort{p44} & 82.5 & 36.1 & 78.6 & 91.7 & 8.3 & 13.3 & 29.1 \\
    \acrshort{bd} & 89.7 & 98.4 & 87.5 & 100.0 & 22.2 & 33.3 & 92.2 \\
    \acrshort{bcc} & 67.9 & 98.5 & 85.7 & 100.0 & 95.8 & 83.3 & 86.2 \\
    \bottomrule
  \end{tabular}%
  \caption{Percentage of lenition represented in the different
    versions of the \mw{Brut}, divided by consonant, including research exceptions.}
  \label{tab:perlenbrut}
\end{table}

\begin{table}[h]
  \centering
  \begin{tabular}{lddddddd}
    \toprule
    \tch{Manuscript} & \tch{\mw{b}} & \tch{\mw{g}} & \tch{\mw{ll}} & \tch{\mw{m}} & \tch{\mw{p}} & \tch{\mw{t}} & \tch{\mw{c}} \\
    \midrule
\acrshort{ll1} & 82.1 & 72.1 & 83.3 & 98.1 & 0.0 & 0.0 & 8.7 \\
\acrshort{p44} & 82.5 & 36.1 & 78.6 & 91.4 & 8.3 & 0.0 & 7.3 \\
\acrshort{bd} & 89.7 & 98.4 & 87.5 & 100.0 & 22.2 & 0.0 & 88.0 \\
\acrshort{bcc} & 67.9 & 98.5 & 85.7 & 100.0 & 95.8 & 80.0 & 78.6 \\
    \bottomrule
  \end{tabular}%
  \caption{Percentage of lenition represented in the
    different versions of the \mw{Brut}, divided by consonant,
    excluding research exceptions}
  \label{tab:perlenbrutex}
\end{table}

\section{Lenited \mw{g}}
\label{sec:lenited-mwg}
The question rises why \mw{g} in \gls{p44} and to a lesser
extent \gls{ll1} is lenited so infrequently.  In \gls{ll1}
lenited \mw{g} is written in 49 out of 68 instances; in \gls{p44} it
is written in 26 out of 72 instances.  The element causing lenition
does not seem to be relevant, \eg verbal particle \mw{a} regularly
causes lenition to \mw[did]{oruc}, but not to \mw[did]{gwnaeth}.

The phonology of the lenited word does seem to play a role: when
initial \mw{g} is not followed by \mw{w}, it disappears in 48 out of
52 of such instances. The remaining four comprise the following
instances: \mw[they rested]{gorffowyssassant} (\gls{ll1} 38.25),
\mw[rested]{gor/ffowyssỽs} (\gls{p44} 23.9--10), \mw[gain]{gorescyn}
(\gls{p44} 27.15), and \mw[glorious]{gogonedỽs} (\gls{p44}
28.23). These words all used to start with \mw{g\cw}, but the \mw{\cw}
has disappeared by the end of the \gls{ow} period. However, such words
starting with \mw{go} do sometimes write lenition, such as in the
plentiful instances of \mw[did]{oruc}.

The remaining 88 instances are all words starting with \mw{(g)w}. 27 of these show lenition.
Table~\ref{tab:gwphon} shows that the phonological structure of these words plays an important role in dictating whether lenition is written.
If the \mw{w} following lenited \mw{g} is a vowel, lenition is usually shown.
An example of such a word showing lenition is \mw[husband]{wr} (\gls{ll1} 34.13).
Similarly, lenition is usually shown if the quality of the following vowel is \mw{\cw}, but  this semivowel is in turn followed by a vowel, \eg \mw[wear]{wyscaỽ} (\gls{p44} 27.7).
However, if the \mw{g} is followed by semivowel \mw{\cw}, and then followed by another consonant, lenition is usually not written, \eg \mw[make]{gwneỽthỽr} (\gls{p44} 27.28).
This means that not representing \mw{g} served a purpose: it served to distinguish consonantal \mw{w} from its syllabic counterpart if it followed \mw{g} and preceded a consonant.


\begin{table}[h]
  \centering
  \begin{tabular}{ldd}
    \toprule
    & /g\cw{}C/ & /g\cw{}V, gu/ \\
    \midrule
    {Lenition shown} & 3 & 24 \\
    {Lenition not shown} & 48 & 13 \\
    \bottomrule
  \end{tabular}%
  \caption{Lenition of \mw{gw} divided by phonological structure of the word.}
  \label{tab:gwphon}
\end{table}

\section{Lenition of voiceless stops}
\label{sec:lenit-voic-stops}

Table~\ref{tab:perlenbrutex} shows how rarely morphophonemic lenition of \mw{p, t, c} is represented in the earlier translations of the \mw{Brut}. The orthographical development of these consonants will be discussed in this section.

\subsection{Lenition in \acrshort{ll1} and \acrshort{p44}}
\label{sec:lenit-acrsh-acrsh}


The pattern is clear in \gls{ll1} and \gls{p44}: voiceless stops are not written lenited as a rule.
Given this rule, exceptions need to be accounted for.
These exceptions are found in Table~\ref{tab:ltrepll1p44}.
As can be seen in this table, both the form and meaning of the words themselves and their reason for being lenited differs, so accounting for them must be done on an ad-hoc basis, and a satisfying account may not always be found.

\begin{table}[h]
  \centering
  \begin{tabular}{lddwql}
    \toprule
    \tch{Source} & \tch{Page} & \tch{Line} & \tch{Word} & \tch{Translation} & \tch{Reason lenition} \\
    \midrule
    \acrshort{ll1} & 34 & 1 & gellweyr & jest & \mw{trwy} \\
    \acrshort{ll1} & 35 & 8 & gytdỽundep & agreement & \mw{o} \\
    \acrshort{ll1} & 36 & 4 & gof & memory & \mw{ar} \\
    \acrshort{ll1} & 37 & 17 & glaf & sick & \mw{en} \\
    \acrshort{p44} & 25 & 10 & gewylyd & shame & \mw{en} \\
    \acrshort{p44} & 26 & 2 & gynt & before & adv phrase \\
    \acrshort{p44} & 26 & 9 & bryt & moment & \mw{pa} \\
    \acrshort{p44} & 27 & 6 & glaf & sick & \mw{en} \\
    \bottomrule
  \end{tabular}%
  \caption{Instances of \lT\ represented in \acrshort{ll1} and \acrshort{p44}.}
  \label{tab:ltrepll1p44}
\end{table}

One word that may be accounted for, however, is \mw[before]{gynt} (\gls{p44} 26.2).
It stands out in that it is the only lenited adverbial phrase in this source.
Moreover, lenition of adverbial phrases of time is frequently petrified, \eg \gmow[yesterday]{ddoe}.
The writing of \mw{gynt} with \mw{g} may similarly point to \gls{petr}, rather than \gls{morphophonlen}.
If so, this word constitutes a research exception.

The case of \mw[what moment?]{pa bryt} may be explained in a similar manner.
Although lenition following \mw{pa} is morphophonemic and still grammatical, it occurs frequently in combination with \mw{bryt}, and orthographical lenition may thus be reminiscent of the \gls{petr} we find in research exception \mw[together]{y gyt}.
Moreover, the semantics of the phrase obviously have a temporal dimension similar to \mw{gynt}.
The phrase \mw{pa bryt} is also found with exceptional lenition in \gls{bd}, as can be seen in Table~\ref{tab:replenpbd}.



\subsection{Lenition in \acrshort{bd} }
\label{sec:lenition-acrshortbd-}
In \gls{bd}, lenition is generally not represented for \mw{p, t}, and it is represented for \mw{c}.
This pattern is exceptionless for \mw{t}, but both \mw{p} and \mw{c} have some exceptions to this rule.
For \mw{p}, these exceptions constitute instances where lenition is represented, and for \mw{c} these exceptions constitute instances where lenition is not represented.
Table~\ref{tab:replenpbd} shows the two instances of orthographically represented lenited \mw{p}, and Table~\ref{tab:nonlencbd} shows every instance where lenited \mw{c} is not orthographically represented.

\begin{table}[h]
  \centering
  \begin{tabular}{addwql}
    \toprule
    \tch{Source} & \tch{Page} & \tch{Line} & \tch{Word} & \tch{Translation} & \tch{Reason lenition} \\
    \midrule
    bd & 34 & 4 & bryt & moment & \mw{pa} \\
    bd & 36 & 2 & baraỽt & ready & \mw{yn} \\
    \bottomrule
  \end{tabular}
  \caption{Representation of lenition of \mw{p} in \acrshort{bd}}
  \label{tab:replenpbd}
\end{table}

\begin{table}[h]
  \centering
  \begin{tabular}{addwql}
    \toprule
    \tch{Source} & \tch{Page} & \tch{Line} & \tch{Word} & \tch{Translation} & \tch{Reason lenition} \\
    \midrule
    bd & 29 & 10 & caerussalem & (place name) & fem noun \\
    bd & 30 & 14 & kyuoeth & kingdom & \mw{y} ‘his' \\
    bd & 31 & 8 & kyuoeth & kingdom & \mw{y} ‘his' \\
    bd & 31 & 9 & kyuoeth & kingdom & \mw{y} ‘his' \\
    bd & 33 & 2 & caffei & received & \mw{na} \\
    bd & 35 & 1 & keissyaỽ & seek & \mw{y} ‘to' \\
    \bottomrule
  \end{tabular}%
  \caption{Non-lenition of \mw{c} in \acrshort{bd}}
  \label{tab:nonlencbd}
\end{table}

\subsection{Lenition in \acrshort{bcc}}
\label{sec:lenition-acrshortbcc}


Lenition of voiceless stops is usually represented in \gls{bcc}, so instances where lenition is not shown are the ones that need to be accounted for.
Table~\ref{tab:ltnotrepbcc} shows these instances of non-represented lenition.
Various reasons for lenition appear multiple times in this table.



\begin{table}[h]
  \centering
  \begin{tabular}{lddwql}
    \toprule
    \tch{Source} & \tch{Page} & \tch{Line} & \tch{Word} & \tch{Translation} & \tch{Reason lenition} \\
    \midrule
    \gls{bcc} & 16r & 29 & prosessio & procession & \mw{y} ‘to' \\
    \gls{bcc} & 16v & 6 & keluydodeu & arts & prep adj \\
    \gls{bcc} & 17r & 14 & kereis & I loved & \mw{th} \\
    \gls{bcc} & 17r & 15 & caraf & I love & \mw{th} \\
    \gls{bcc} & 17r & 16 & kerir & is loved & \mw{th} \\
    \gls{bcc} & 17v & 18 & tywyssauc & prince & apposition \\
    \gls{bcc} & 17v & 29 & tywyssawc & prince & apposition \\
    \gls{bcc} & 18r & 12 & trugarhae & mercy & \mw{y} ‘his' \\
    \gls{bcc} & 18r & 25 & Cordeilla & (personal name) & \mw{-ei} \\
    \gls{bcc} & 19r & 10 & Cordeilla & (personal name) & \mw{-ei} \\
    \gls{bcc} & 19v & 4 & tywyssawc & prince & apposition \\
    \gls{bcc} & 19v & 4 & tywyssawc & prince & apposition \\
    \gls{bcc} & 19v & 5 & calet & hard & prep adj \\
    \gls{bcc} & 19v & 13 & cordeilla & (personal name) & \mw{y} ‘to' \\
    \gls{bcc} & 19v & 25 & creftwyr & craftsmen & prep adj \\
    \bottomrule
  \end{tabular}%
  \caption{Instances of \lT\ not represented in \acrshort{bcc}.}
  \label{tab:ltnotrepbcc}
\end{table}

Within \gls{bcc}, \mw[prince]{tywyssawc} is found four times in apposition to a personal name.
A lenitable noun in apposition to a personal name is found eight times within this source, and is lenited only once in \mw[prophet]{broffwid} (\gls{bcc} 16r.26).
Lenition of nouns in apposition is not shown for other types of consonants either, as the word \mw[king]{brenhin} is found in unlenited form in this position also.

Personal name \mw{Cordeilla} is found unlenited where lenition would be expected three times. The fact that we are dealing with a personal name here may be of influence, as \mw{llyr} is found unlenited following \mw[fort]{caer} twice. Furthermore, personal names are not as a rule lenited in \gls{mow}. These instances of \mw{llyr} and \mw{Cordeilla} may be early examples of this \gls{mow} rule. 

Infixed object pronoun \mw{'th} should cause lenition \todo{or should it not?}. It does not do so here.

Several words following a preposed adjective fail to show lenition, although these instances are outnumbered by preposed adjectives shown to cause lenition.
Lenition following preposed adjectives is a type of free lenition, because any adjective principally causes lenition when it is used as a preposed adjective.
These same adjectives do not principally cause lenition when they are in their usual postnominal position.
This means that the correct application of lenition hinges on the morphosyntactic relationship between two elements in a clause rather than by simply checking whether the immediately preceding morpheme causes lenition.
This morphosyntactic relationship may not always be clear, either to us or the scribes and translators of the \gls{mw} text.
An example where I found it difficult to establish the exact relationship involving a supposedly preposed adjective is Example~\ref{ex:wychyrcalet}:
\mwcc[ex:wychyrcalet]{\gls{bcc} 19v.3--6}{ac yn ev herbyn wynt y doeth Maglawn tywyssawc yr alban. a henwyn tywyssawc kernyw ac ev holl allu. ac ymlad yn \al{wychyr calet} ac wynt.}{And against them came Maglawn, prince of Scotland, and Henwyn, prince of Cornwall, and their whole capability, and they fought them \al{violently hard}.}
What exactly is the grammatical relationship between \mw[violent]{wychyr} and \mw[hard]{caled}?
Is \mw{wychyr} a preposed adjective, translating to the translation given, or does \mw{caled} modify \mw{wychyr}?
In the latter case, `toughly violent' would be a more suitable translation. 
Judgments like these may not be made consistently, either by me when deciding where lenition was not written, or by the translators and copyists who produced this work.

%%% Local Variables:
%%% coding: utf-8
%%% mode: latex
%%% TeX-master: "../main"
%%% End:
