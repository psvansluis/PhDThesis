\section{Aneirin}\label{sectionaneirin}
The Book of Aneirin is a particularly interesting showcase because the conclusions one may make on the basis of its lenition patterns are counterintuitive: failure to write lenition is comparatively rare in this manuscript, but the patterns in it are nevertheless evidence of an \gls{ow} source. This may be concluded on the basis of the specific pattern of lenition found, combined with some other text-critical observations.

The \mw{Gododdin} as found in the Book of Aneirin provides a case study on how the orthography of lenition may pinpoint the date a text was first composed. The \mw{Gododdin} is found in two different hands, called Hand A and B. Their work may be considered two different recensions of one original composition, as their contents overlap quite a bit. These two parts of the \mw{Gododdin} are found in three different locations within the manuscript. I will follow Ifor Williams (IW) in calling these versions A, B, and C~\autocite*{williams_canu_1938}, although it should be remembered that both version B and C are written by hand B. All these alphabetic abbreviations for texts, classifications, and hands easily lead to confusion, so Table~\ref{contentsllaneirin} gives a handy overview of the contents of the Book of Aneirin.



Isaac has already done some textual criticism on a portion which is found in all three versions~\autocite{isaac_canu_1993}, namely in \mw{awdl} LI. This \mw{awdl} is found separately on pages 13, 23, and 34 of the Book of Aneirin. It stands to reason that text found in the immediate vicinity of these three \mw{awdlau}  was handed down in a similar way to \mw{Awdl} LI. This assumption will prove fruitful in understanding some patterns of lenition. The triple existence of this \mw{awdl} in a slightly different form allows for a stemma reconstruction, which is given in Figure~\ref{stemmagododdin}.

\begin{figure}[h]
    \centering
\begin{tikzpicture}
\node{X} 
    child [level distance=20mm]{node {Y}
    child [level distance=20mm]{node {B}}
        child [level distance=10mm]{node {D}
            child {node {A}}
            child {node {C}}
            }
    }
    ;
\end{tikzpicture}
    \caption{A stemma of \mw{Awdl} LI of the Gododdin~\autocite[77]{isaac_canu_1993}.}
    \label{stemmagododdin}
\end{figure}

In addition to this stemma, \textcite[87]{isaac_canu_1993} shows that Gododdin B (IW C and B) may also be subdivided into two parts in another way. One part barely shows any \gls{ow} features, and  is found on pages 23--24, 30--33, and line 34.8 of the Book of Aneirin. The rest of Gododdin B, i.e.\ pages 34--38, shows \gls{ow} features abundantly. Isaac gives the following comment on this difference: 
\tqt{I cannot see that we could possibly suppose the difference to be due to the scribe starting off modernizing his text thoroughly and then gradually slipping. There is nothing gradual about the contrast between these two sections: at p.~34 the orthography of the text suddenly and consistently takes on a far more archaic appearance. Conclusion: the B text consists of material from at least two sources, one of which was considerably more archaic than the other.}{isaac_canu_1993}{87} 
This means that part of Gododdin B was transferred reasonably directly from an \gls{ow} or Northern British exemplar, while the rest of the \mw{Gododdin} went through an intermediate stage where the text was modernised. Figure~\ref{stemmagododdin} confirms this, as B in this figure corresponds to an \mw{Awdl} as found on page 34 in \gls{ow} orthography, while C in this figure is found in the \gls{mw} orthography of page 24 of the Book of Aneirin (and indeed in Table~\ref{gododdinc}).

\Textcite{koch_gododdin_1997} follows Isaac in grouping the first half of IW B with IW C. He calls C along with the first half of B `B\textsuperscript{1}', and the rest of B `B\textsuperscript{2}', as shown in Table~\ref{contentsllaneirin}. Koch expands on Isaac in proposing dates for the common exemplar of A and B\textsuperscript{1}. This exemplar, D (or in Koch's notation: AB\textsuperscript{1}), must have been written in the period between 642 and 655, based on the existence of the \mw{Srath Caruin}-awdl, the reciter's prologue, and the collected elegies in both A and B\textsuperscript{1}, but not in B\textsuperscript{2}\todo{Absence of evidence in B1=evidence of absence?}~\autocite[lxv]{koch_gododdin_1997}. The reasoning behind this date is that the \mw{Srath Caruin}-awdl must have served Eidin-Votadinian interests\todo{expand}.

Moreover, Koch argues on linguistic grounds that c.~750 must serve as a rough \emph{terminus ante quem} for the writing of the AB\textsuperscript{1} exemplar. 

However, AB\textsuperscript{1} also show signs of a post-Old North transmission in \mw{Gwynedd}, and written exemplar AB\textsuperscript{1} may have arrived in Wales as early as 655~\autocite[xcii]{koch_gododdin_1997}. \Textcite[lxxii]{koch_gododdin_1997} notes that the first awdl of Text A must be post c.~900, because \mw{vrein} `crows' and \mw{Ywein} (personal name) rhyme with \mw{argyurein} `laying out of a corpse'. The final vowel of the first two words are formed by final \mw{i}-affection, while the final vowel of \mw{argyurein} descends from a vocalised \mw{-g-}. In earlier times, these environments would produce /ej/ and /e/, respectively. Koch additionally states that the dating of this poem cannot be \emph{too} late, as  scribe A of our text must have copied from a source which he understood imperfectly: \mw{oed}  `age' is spelled with a final \mw{d}, even though the scribe's practice was to use \mw{t} for final /d/. Moreover, the word \mw{ethy} (\gls{mow} \mw{eddi}) `threadwork' is found. An \gls{ow} exemplar would have written \mw{d} for /ð/ here. In the ninth and early tenth century, /θ/ would also commonly be represented by \mw{d}. It seems that scribe A has wrongly modernised a text in a way that only makes sense if this text contains \mw{d} for /θ/, and is therefore at least as recent as the ninth century~\autocite[lxxii--lxxiii]{koch_gododdin_1997}.


\begin{table}[h]
    \centering
        \begin{tabular}{@{}lllll@{}}
        \toprule
        \textbf{Text} & \textbf{Pages} & \textbf{IW} & \textbf{Koch} & \textbf{Hand} \\ \midrule
        \mw{Gododdin A} & 1.1--23.5 & A & A & A \\
        \mw{Gododdin B} & 23.6--24.21 & C & B\textsuperscript{1} & B \\
        \mw{Gwarchan Tudfwlch} & 25.1--26.7 & & & A \\
        \mw{Gwarchan Adebon} & 26.8--16 & & & A \\
        \mw{Gwarchan Cynfelyn} & 26.18--28.6 & & & A \\
        \mw{Gwarchan Maeldderw} & 28.18--30.11 & & & A \\
        \mw{Gododdin B} & 30.12--34.6 & B & B\textsuperscript{1} & B \\
        \mw{Gododdin B} & 34.6--38.22 & B & B\textsuperscript{2} & B \\ \bottomrule
        \end{tabular}
    \caption{The contents of the Book of Aneirin}
    \label{contentsllaneirin}
\end{table}

\todo[inline]{This discussion may need to be expanded with some more secondary literature on authenticity and transmission of the \textit{Gododdin}. HOW DO WE SQUARE B1'S SUPPOSED CONSERVATISM WITH HIGH RATE OF REPRESENTED LENITION}

\subsection{Hypothesis and method}
I propose that the orthography of lenition may aid in reconstructing and dating the transmission of the \mw{Gododdin}. Here, I do not presume to staunchly confirm or deny its sixth-century roots, but I do expect to shed light upon its transmission in intermediate stages. Irrespectively of the original date the \mw{Gododdin} was composed or written down, it is obvious that the current texts we have of it are modernised to varying degrees. Because of this, what can be treated as different `manuscript witnesses' of this story may therefore be compared and contrasted in how they represent lenition.

I have gathered roughly 100 data points each from IW A, C, and B in order to make this comparison. For each word that I consider should be lenited by the time of the manuscript, I counted why it should be lenited, and whether lenition is represented orthographically in the manuscript. 

In all three versions some instances of lenition are found where I see no grammatical reason for lenition to apply. These instances are marked with `??' in the tables. 

In the IW B\todo{just b1 or b2 too?} version of \mw{Gododdin} B (Table~\ref{gododdinb}), /ɬ/ and /l/ are not differentiated. Instead, they are both represented with \graph{l}, as they would be in \gls{ow}. Consequently, it is impossible to tell whether a word starting with \mw{l} in this section should be considered lenited or not. For this reason, Table~\ref{gododdinb} does not contain any words whose radical starts with /ɬ/. Nevertheless, this fact is interesting in itself. One might argue that a word starting with \mw{l} in this section does not have its lenition represented when we would expect it, and that it should therefore be included in the table. However, a modernising scribe would not change the \mw{l} in such a case. Alternatively, one could mark down /ɬ/'s represented with \mw{l} as hypercorrect lenition, but it would be thoroughly wrong to classify an archaism as a hypercorrect modernisation, so I consider it wiser to simply leave them out of the equation\footnote{In this manner, they are treated similarly to lenition of /d/ and /\rh/, which I do not treat anywhere in this thesis unless explicitly mentioned.}. 


\subsection{Results}
Tables~\ref{gododdina}, \ref{gododdinc}, and \ref{gododdinb} (found from page~\pageref{tablesgododdin} onwards) are summarised in Table~\ref{sumgododdin}. This table shows that no version of the \mw{Gododdin} shows non-lenition of voiceless stops in particular. Instead, we find that a  81.0\% of voiceless stops have their lenition represented, while this is the case in only 70.7\% of other cases. If anything, we see the reverse.

What we do see, however, is a large contrast between IW's versions A and C on one hand, and B on the other. Lenition in B is represented at a much lower rate: 82.6\% and 80.6\% against 61.4\%, respectively. This contrast is all the more jarring when considering that /ɬ/ and /l/ were not differentiated in the latter version. This shows that IW B is considerably more archaic-looking than A and C. Obviously, however, this does not mean that B has an older original composition, because they are all recensions of the \mw{Gododdin}. Similarly, it also does not mean that we are dealing with an earlier `manuscript witness' here, for the similarly obvious reason that A, C, and B are all found in the same manuscript.

\begin{table}[h]
\centering
\begin{tabular}{@{}lllllllll@{}}
\toprule
\multirow{2}{*}{\textbf{Text (IW)}} & \multirow{2}{*}{\textbf{Table}} & \multicolumn{3}{l}{\textbf{Voiceless stops}} & \multicolumn{3}{l}{\textbf{Not voiceless stops}} & \multirow{2}{*}{\textbf{Total rep.\ (\%)}}\\
  &  & \textbf{rep.} & \textbf{not rep.} & \textbf{rep.~(\%)} & \textbf{rep.} & \textbf{not rep.} & \textbf{rep.~(\%).} \\ \midrule
\mw{Gododdin} A (A) & \ref{gododdina} & 43 & 5 & 89.6 & 66 & 18 & 78.6 & 82.6 \\
\mw{Gododdin} B (C) & \ref{gododdinc} & 29 & 4 & 87.9 & 25 & 9 & 73.5 & 80.6\\
\mw{Gododdin} B (B) & \ref{gododdinb} & 30 & 15 & 66.7 & 32 & 24 & 57.1 &61.4\\
\textbf{Total}  & \textbf{} & \textbf{102} & \textbf{24} & \textbf{81.0} & \textbf{123} & \textbf{51} & \textbf{70.7} & \textbf{75.0} \\ \bottomrule
\end{tabular}
\caption{Representation of lenition in the \mw{Gododdin}}
\label{sumgododdin}
\end{table}

These numbers do tell us about the textual history in the intermediate centuries between when the \mw{Gododdin} was first written (possibly as early as the sixth century) and when the manuscript before us was written (the thirteenth century). For one thing, the numbers for IW A and C are very close, even though one was penned  by scribe A, while the other was penned  by scribe B. By contrast, scribe B also penned  IW B, but the numbers are very different.

As shown by Isaac and Koch\todo{It might be more efficient to just do away with the ACB numbering and just go with Koch's A, B1, B2 numbering.}, IW B may be sub-divided into two halves, with the second half starting at line 6 of page 34. In the first half, lenition is represented with greater frequency than in the second half. 30 out of 42 (71.4\%) instances of lenition are represented  in the first half, while only 32 out of 59 (54.2\%) instances of lenition are represented in the second half\todo{These data are not statistically significant, meaning I may want to collect more Data from IW B}. This confirms Isaac's impression that IW B up to page 34 may be considered along with IW C rather than the rest of IW B, and that IW B from page 34 onwards must be considered separately~\autocite[87]{isaac_canu_1993}.

It is striking that scribe B wrote both IW C and the second part of IW B. The large difference in the extent to which lenition is represented demonstrates that scribe B must have had a readily-lenited exemplar to transcribe. The only alternative explanation would be that scribe B somehow `unlearned' how to modernise orthography of page 34. Additionally, the percentage to which lenition is represented in IW C roughly agrees with A. For \mw{Awdl} LI, Isaac assumes that they had a common exemplar not shared with IW B (this is manuscript D in Figure~\ref{stemmagododdin}). The appearance of lenition confirms this. Moreover, it implies that lenition of all consonants was added in manuscript D. This in turn means that D is a relatively young manuscript, only written after orthographic lenition of voiceless stops was invented. The date of this innovation may be considered a \emph{terminus post quem} for manuscript D.

The dearth of lenition in IW B after page 34 shows that it must have been copied from an exemplar which most likely had no lenition. This exemplar must be manuscript Y in Isaac's stemma (or perhaps an intermediate copy, but this chapter gives no evidence for this). This text does not show a particular tendency to lenite consonants other than voiceless stops. This means that manuscript Y was written at a time before any lenition was represented at all. The innovation of representing lenition for other consonants than voiceless stops may thus be considered a \emph{terminus ante quem} for the composition of manuscript Y.

\subsection{Comments on individual cases}


Tables~\ref{gododdina}, \ref{gododdinc}, and \ref{gododdinb} show that it is worthwhile to list both instances of non-lenition and lenition. Many instances of lenition are straightforward: \gls{mw} \mw{o} `from' causes lenition consistently\footnote{The only example is found in p.~37, l.~1, before	\mw{tan} `under', and is tellingly found in the second part of version B, the most \gls{ow}-looking part of the \mw{Gododdin}.}, while most instances of \gls{mw} \mw{y} `his, her, the, to' cause lenition. In these cases therefore, lenition may be expected to be presented consistently, especially where the meaning is unclear. 

In other cases, this is not the case. \mw{grud} `cheek' found in p.\ 24, l.\ 8 should be lenited, given how it follows \mw{ar} `on, against, in front of'. However, \mw{ar} could also stand for what would be spelled in \gls{mow} as \mw{a'r} `and the'. In such a case, lenition would not be found. A scribe encountering a word such as \mw{ar} is therefore much more likely to fail to modernize lenition following than he would be following \mw{o}. This demonstrates how lenition was at some point inserted by the scribe modernising an exemplar which did not have lenition written down. 

A closer look at individual cases of lenition shows that even where the text looks reasonably \gls{mw} in its orthography, the instances in which lenition fail to be represented  betray an \gls{ow} source. In some cases, lenition is not found where the word preceding the word to be lenited is corrupted or misinterpreted. The effect of such a mistake may be compounded by failure to represent the lenition following. This is found in p.\ 23, l.\ 6 in \mw{aberthach coel kerth}, which is a corruption of \mw{aberth am coel kerth} `a sacrifice for a bonfire'. \gls{mw} \mw{am} `about' causes lenition, but where \mw{am} is misread, lenition cannot be supplied by a scribe. This demonstrates that unlenited \mw{coel kerth} was the original. A similar corruption is found in the following example~(tr.\ \textcite[30]{jarman_y_1988}):
\mwcc[tynyho]{Book of Aneirin, p.~24, l.~19}%
{a chẏnẏho mwng bleid heb pre\={n} enẏ lav}%
{he who holds a wolf's mane without spear in his hand}
Here, two mistakes were made. The first is that initial \graph{t} was mistaken for a \graph{c} at some point\footnote{This mistake is not unprecedented and easily imaginable when considering the shape of the insular \emph{t}: \graph{ꞇ}. This mistake implies that an exemplar of the \mw{Gododdin} was written in insular minuscule, which had fallen out of favour by the \gls{mw} period.}. Reversing this mistake gives a subjunctive form of the verb \mw{tynnu}, which works well in this context. Also, verbal particle \mw{a} was misinterpreted as conjunction \mw{a} `and', and therefore causes spirantisation rather than lenition. This instance therefore also demonstrates that spirantisation was supplied at a later date in at least one instance. Yet another way in which copying errors may betray original non-lenition is found in the following line~(tr.\ \textcite[4]{jarman_y_1988}):
\mwcc[nytechei]{Book of Aneirin, p.~2, l.~13}%
{mal brwẏn gomẏnei gwẏr nẏt echei.}%
{Like rushes he cut down men who did not flee. }
The final verb here should be \mw{nẏ techei} `did not flee', rather than what we see above. The erroneous word division prevented lenition from being supplied at a later date here\footnote{The verb \mw{techei} is found in a relative position here. Originally, verbal particle \mw{ny} would cause lenition in this position, and cause spirantisation in a main clause.}.

Another point is object lenition. \Textcite{van_sluis_development_2014} considers object lenition a development which arose in the \gls{mw} period. The presence or absence of object lenition and its distribution across consonants may tell us more about when lenition was added relatively to the linguistic invention of object lenition. Table~\ref{objlengododdin} shows that object lenition is indeed represented, but not very regularly. Their appearance implies that lenition was added at a fairly late date, but their scarcity implies that object lenition may not have developed yet when lenition was added in the exemplar.

\begin{table}[h]
\centering
\begin{tabular}{@{}llllll@{}}
\toprule
\textbf{Folio} & \textbf{Line} & \textbf{Word} & \textbf{Translation} & \textbf{Represented} & \textbf{IW} \\ \midrule
20 & 4 & \mw{gwẏr} & `men' & no & A \\
20 & 10 & \mw{vudic} & `victor' & yes & A \\
20 & 13 & \mw{gwaetlin} & `bloodshed' & no & A \\
23 & 8 & \mw{wẏr} & `men' & yes & C \\
24 & 9 & \mw{uot} & `that … is' & yes & C \\
24 & 19 & \mw{mwng} & `mane' & no & C \\
31 & 4 & \mw{gwin} & `wine' & no & B \\
35 & 12 & \mw{molut} & `praise' & no & B \\ \bottomrule
\end{tabular}
\caption{Representation of object lenition in the \mw{Gododdin}.}
\label{objlengododdin}
\end{table}


% \subsection{Unsorted notes}
% The place name \mw{Gatraeth} refers to Catterick in Northumbria, and is held to be lenited in these texts, while the unlenited form would be \mw{Catraeth}. I rather think an early Welsh speaker would have understood \mw{Gatraeth} to be the unlenited form for several reasons. One is that an unlenited /g/ would alliterate with unlenited /g/ in \mw{gwyr} in any line starting with \mw{Gwyr a aeth Gatraeth}. Another is the fact that I have yet to see an example beside \mw{Gatraeth} showing lenition of the object of destination following a personal verbal ending, while at the same time regular object lenition is a comparatively late development~\autocite{van_sluis_development_2014}. Thirdly, \mw{Gatraeth} is in fact found in a different text where it is hard to imagine a grammatical reason for lenition:
% \mwcc[eilywedgatraeth]{Moliant Cadwallon, l.\ 31}{eilywed Gattraeth fawr vygedawc:}{the loss of the great nobles of Gatraeth:}
% Example \ref{eilywedgatraeth} shows \mw{Gattraeth} in a genitival relationship with preceding masculine noun \mw{eilywed} `loss', which does not cause lenition. On a related note, it takes some trouble explaining why Latin \mw{Cataracta} should yield medial \mw{t} in \mw{Gatraeth}, rather than \mw{d}, so confusion on the quality of consonants is not new in this word\footnote{See \Textcite[409--10]{jackson_language_1953} for this matter.}.


\subsection{The tables}\label{tablesgododdin}
\begin{mylongtable}{@{}llllll@{}}
\toprule
\textbf{Folio} & \textbf{Line} & \textbf{Word} & \textbf{Translation} & \textbf{Cause of lenition} & \textbf{Represented} \\ \midrule\endhead
1 & 3 & \mw{dan} & `under' & \mw{o} & yes \\
1 & 3 & \mw{vordw\.{y}t} & `thigh' & \mw{dan} & yes \\
1 & 4 & \mw{bedrein} & `horse's buttocks' & \mw{ar} & yes \\
1 & 4 & \mw{vuan} & `quick' & ??compound & yes \\
1 & 6 & \mw{wawt} & `mockery, poetry' & \mw{ar} & yes \\
1 & 6 & \mw{uoli} & `praise' & \mw{dẏ} & yes \\
1 & 6 & \mw{waet} & `blood' & \mw{ẏ} `to' & yes \\
1 & 6 & \mw{elawr} & `stretcher?' & ?? & yes \\
1 & 7 & \mw{vwẏt} & `food' & \mw{ẏ} `to' & yes \\
1 & 7 & \mw{vrein} & `ravens' & \mw{ẏ} `to' & yes \\
1 & 7 & \mw{kẏueillt} & `foster brother' & preposed adjective & no \\
1 & 8 & \mw{uot} & `that … is' & \mw{ẏ} `his' & yes \\
1 & 8 & \mw{dan} & `under' & \mw{a} `from' & yes \\
1 & 8 & \mw{vrein} & `ravens' & \mw{dan} & yes \\
1 & 8 & \mw{vro} & `land' & \mw{pa} & yes \\
1 & 11 & \mw{dalhei} & `would pay' & \mw{a} & yes \\
1 & 11 & \mw{awr} & `shout' & \mw{-ei} & yes \\
1 & 12 & \mw{gamhawn} & `battle' & \mw{o} & yes \\
1 & 13 & \mw{verei} & `dripped' & \mw{oni} & yes \\
1 & 13 & \mw{waet} & `blood' & \mw{-ei} & yes \\
1 & 13 & \mw{gwẏr} & `men' & \mw{-ei} & no \\
1 & 13 & \mw{techei} & `would retreat' & \mw{nẏ} & no \\
1 & 14 & \mw{llawr} & `floor' & \mw{ar} & no \\
1 & 15 & \mw{gant} & `hundred' & \mw{o} & yes \\
1 & 17 & \mw{vu} & `was' & \mw{a} & yes \\
1 & 17 & \mw{gatwẏt} & `was kept' & \mw{a} & yes \\
1 & 18 & \mw{wnaeth} & `did' & \mw{a} & yes \\
1 & 18 & \mw{gilẏwẏt} & `was fled' & \mw{nẏ} & yes \\
1 & 18 & \mw{ododin} & (place name) & feminine noun & yes \\
1 & 18--19 & \mw{ode/chwẏt} & `was retreated' & [\mw{a}] & yes \\
1 & 19 & \mw{gẏmhell} & `pressure' & preposed adjective & yes \\
1 & 19 & \mw{vreithel} & `land' & \mw{ar} & yes \\
1 & 19 & \mw{vanawẏt} & (personal name) & feminine noun & yes \\
1 & 20 & \mw{ellir} & `was able' & \mw{nẏ} & yes \\
1 & 20 & \mw{vaethpwẏt} & `was nurtured' & \mw{rẏ} & yes \\
2 & 2 & \mw{vann} & `cup' & [\mw{o}] & yes \\
2 & 3 & \mw{wẏned} & (place name) & \mw{-ei} & yes \\
2 & 3 & \mw{guſſẏl} & `planning' & \mw{o} & yes \\
2 & 6 & \mw{gwr} & `man' & \mw{e} `to' & no \\
2 & 7 & \mw{pẏm} & `five' & \mw{-ei} & no \\
2 & 7 & \mw{lafnawr} & `blades' & \mw{ẏ} `his' & yes \\
2 & 7 & \mw{wẏr} & `men' & \mw{o} & yes \\
2 & 9 & \mw{gic} & `meat' & \mw{ẏ} `to' & yes \\
2 & 9 & \mw{vleid} & `wolf' & \mw{e} `to' & yes \\
2 & 9 & \mw{vud} & `booty' & \mw{e} `to' & yes \\
2 & 9 & \mw{vran} & `raven' & \mw{e} `to' & yes \\
2 & 10 & \mw{waet} & `blood' & \mw{e} `to' & yes \\
2 & 10 & \mw{lawr} & `down' & \mw{e} `to' & yes \\
2 & 11 & \mw{gan} & `with' & petrified lenition & yes \\
2 & 11 & \mw{lliwedawr} & `hosts' & \mw{gan} & no \\
2 & 12 & \mw{vo} & `is' & \mw{tra} & yes \\
2 & 13 & \mw{Ododin} & (place name) & object of destination & yes \\
2 & 13 & \mw{ognaw} & `provoking' & ??compound & yes \\
2 & 14 & \mw{vlẏned} & `years' & preposed adjective & yes \\
2 & 15 & \mw{daw} & `quiet' & \mw{ẏn} & yes \\
2 & 16 & \mw{law} & `hand' & \mw{e} `his' & yes \\
2 & 16 & \mw{lanneu} & `parishes' & \mw{e} `to' & yes \\
2 & 16 & \mw{benẏdẏaw} & `do penance' & \mw{e} `to' & yes \\
2 & 18 & \mw{Ododin} & (place name) & object of destination & yes \\
2 & 18 & \mw{wanar} & `eager' & ??compound & yes \\
2 & 19--20 & \mw{va-wr} & `much' & \mw{heb} & yes \\
2 & 20 & \mw{drẏdar} & `noise' & preposed adjective & yes \\
2 & 21 & \mw{gatraeth} & (place name) & object of destination & yes \\
2 & 22 & \mw{vu} & `was' & [\mw{a}] & yes \\
2 & 22 & \mw{beirẏant} & `command' & \mw{trwẏ} & yes \\
3 & 1 & \mw{vu} & `was' & [\mw{a}] & yes \\
3 & 1 & \mw{lanneu} & `parishes' & \mw{e} `to' & yes \\
3 & 2 & \mw{benẏdu} & `do penance' & \mw{e} `to' & yes \\
3 & 3 & \mw{gatraeth} & (place name) & object of destination & yes \\
3 & 3 & \mw{veduaeth} & `mead-nourished' & ?? & yes \\
3 & 3 & \mw{uedwn} & `hosts' & preposed adjective & yes \\
3 & 4 & \mw{lavnawr} & `blades' & \mw{am} & yes \\
3 & 6 & \mw{deulu} & `retinue' & \mw{ar} & yes \\
3 & 6 & \mw{vẏw} & `alive' & \mw{yn} & yes \\
3 & 7 & \mw{golleiſ} & `I lost' & \mw{a} & yes \\
3 & 8 & \mw{adwn} & `I left' & \mw{ri} & yes \\
3 & 8 & \mw{mennwſ} & `wished' & \mw{nẏ} & no \\
3 & 8 & \mw{gwadawl} & `endowment' & preposed adjective & no \\
3 & 9 & \mw{gian} & (personal name) & \mw{ẏ} `to' & yes \\
3 & 9 & \mw{vaen} & (place name) & \mw{o} & yes \\
3 & 10 & \mw{gatraeth} & (place name) & object of destination & yes \\
3 & 10 & \mw{gan} & `with' & petrified lenition & yes \\
3 & 10 & \mw{wawr} & `dawn' & \mw{gan} & yes \\
3 & 12 & \mw{waewawr} & `spears' & plural verbal ending & yes \\
3 & 14 & \mw{gatraeth} & (place name) & object of destination & yes \\
3 & 14 & \mw{gan} & `with' & petrified lenition & yes \\
3 & 17 & \mw{bennawr} & `headstall' & preposed adjective & yes \\
3 & 19 & \mw{gatraeth} & (place name) & object of destination & yes \\
3 & 19 & \mw{gan} & `with' & petrified lenition & yes \\
3 & 19 & \mw{gadeu} & `battles' & \mw{o} & yes \\
3 & 19--20 & \mw{ge-wilid} & `shame' & parenthesis & yes \\
3 & 20 & \mw{geugant} & `certain' & \mw{en} & yes \\
4 & 1 & \mw{ododin} & (place name) & feminine noun & yes \\
4 & 1 & \mw{vudẏd} & `strikes' & \mw{pan} & yes \\
20 & 2 & \mw{benn tir} & (place name) & \mw{o} & yes \\
20 & 2 & \mw{goelkerth} & `conflagration' & \mw{am} & yes \\
20 & 3 & \mw{dref} & `town' & \mw{ar} & yes \\
20 & 3 & \mw{degein} & `descend' & \mw{re} & yes \\
20 & 4 & \mw{golleſſẏn} & `lost' & \mw{rẏ} & yes \\
20 & 4 & \mw{gwẏr} & `men' & object lenition & no \\
20 & 4 & \mw{gan} & `with' & petrified lenition & yes \\
20 & 4 & \mw{awr} & `dawn' & \mw{gan} & yes \\
20 & 6 & \mw{vudic} & `victorious' & preposed adjective & yes \\
20 & 6 & \mw{laſſawc} & `blue sword' & \mw{e} `his' & yes \\
20 & 6--7 & \mw{tebe-dawc} & `repelling' & feminine noun & no \\
20 & 7 & \mw{alon} & `enemies' & preposed adjective & yes \\
20 & 7 & \mw{lu} & `host' & \mw{ẏ} `his' & yes \\
20 & 8 & \mw{vronn} & `breast' & preposed adjective & yes \\
20 & 8 & \mw{gẏnnedẏf} & `custom' & \mw{ẏ} `his' & yes \\
20 & 10 & \mw{vudic} & `victor' & object lenition & yes \\
20 & 10 & \mw{vu} & `was' & \mw{a} & yes \\
20 & 10 & \mw{lew} & `valiant' & ??dvandva compound & yes \\
20 & 11 & \mw{geſſevin} & `foremost' & adv.\ clause & yes \\
20 & 12 & \mw{win} & `wine' & feminine noun & yes \\
20 & 13 & \mw{leaſ} & `green' & \mw{e} `his' & yes \\
20 & 13 & \mw{laſ} & `blue sword' & \mw{e} `his' & yes \\
20 & 13 & \mw{gwaetlin} & `bloodshed' & object lenition & no \\
20 & 14 & \mw{gledẏf} & `sword' & \mw{e} `his' & yes \\
20 & 16 & \mw{gennẏf} & `with me' & petrified lenition & yes \\
20 & 17 & \mw{gennẏf} & `with me' & petrified lenition & yes \\
20 & 18 & \mw{draet} & `feet' & \mw{o} & yes \\
20 & 21 & \mw{wedẏ} & `after' & petrified lenition & yes \\
20 & 21 & \mw{kẏnnwẏſ} & `welcome' & parenthesis & no \\ \bottomrule
\caption{Representation of lenition in \mw{Gododdin} A (IW A) pp.\ 1--3, 20} \label{gododdina}
\end{mylongtable}


\begin{mylongtable}{@{}llllll@{}}
\toprule
\textbf{Folio} & \textbf{Line} & \textbf{Word} & \textbf{Translation} & \textbf{Cause of lenition} & \textbf{Represented} \\ \midrule\endhead
23 & 6 & \mw{bentir} & (place name) & \mw{o} & yes \\
23 & 6 & \mw{coel kerth} & `conflagration' & \mw{am} & no \\
23 & 7 & \mw{gwẏdẏn} & `they descended' & \mw{rẏ} & yes \\
23 & 7 & \mw{eir} & `word' & \mw{o} & yes \\
23 & 8 & \mw{godeſſẏn} & `they rose' & \mw{rẏ} & yes \\
23 & 8 & \mw{wẏr} & `men' & object lenition & yes \\
23 & 8 & \mw{gan} & `by' & petrified lenition & yes \\
23 & 9 & \mw{wavr} & `dawn' & \mw{gan} & yes \\
23 & 9 & \mw{vrẏch} & (personal name) & epithet & yes \\
23 & 10 & \mw{blegẏt} & `behalf' & \mw{th} & yes \\
23 & 11 & \mw{wrhyt} & `courage' & preposed adjective & yes \\
23 & 12 & \mw{want} & `was slain' & \mw{pan} & yes \\
23 & 15 & \mw{ware} & `play' & preposed adjective & yes \\
23 & 16 & \mw{win} & `wine' & \mw{o} & yes \\
23 & 16 & \mw{bebẏll} & `tent' & ??preposed adjective & yes \\
23 & 18 & \mw{gadꝛawt} & `war band' & preposed adjective & yes \\
23 & 20 & \mw{vu} & `was' & [\mw{a}] & yes \\
23 & 21 & \mw{bop} & `every' & \mw{ẏ} `to' & yes \\
23 & 22 & \mw{ohir} & `delay' & preposed adjective & yes \\
24 & 1 & \mw{w[a]ith} & `time' & \mw{eil} & yes \\
24 & 1 & \mw{gat veirch} & `warhorses' & \mw{ẏ} `his' & yes \\
24 & 2 & \mw{greulet} & `bloody' & feminine noun & yes \\
24 & 2 & \mw{get uoꝛon} & `powerful battlers' & ?? & yes \\
24 & 3 & \mw{godet} & `was angered' & \mw{rẏ} & yes \\
24 & 3 & \mw{ladei} & `slaid' & \mw{ẏt} & yes \\
24 & 4 & \mw{gat} & `battle' & \mw{o} & yes \\
24 & 4 & \mw{gant} & `hundred' & \mw{-ei} & yes \\
24 & 5 & \mw{dan} & `under' & petrified lenition & yes \\
24 & 5 & \mw{vab} & `son' & \mw{dan} & yes \\
24 & 5 & \mw{dan} & `under' & petrified lenition & yes \\
24 & 5 & \mw{dwrch} & `boar' & \mw{dan} & yes \\
24 & 8 & \mw{grud} & `cheek' & \mw{ar} & no \\
24 & 9 & \mw{get} & `generosity' & \mw{ẏ} `his' & yes \\
24 & 9 & \mw{glot} & `fame' & \mw{e} `his' & yes \\
24 & 9 & \mw{uot} & `that … is' & object lenition & yes \\
24 & 10 & \mw{oꝛthir} & (personal name) & \mw{o} & yes \\
24 & 11 & \mw{dꝛẏnni} & `battle' & \mw{am} & yes \\
24 & 11 & \mw{drẏlav} & `?earth' & ?? & yes \\
24 & 11 & \mw{dꝛẏlen} & `?battle' & ??preposed adjective & yes \\
24 & 11 & \mw{lwẏs} & `fair' & \mw{am} & yes \\
24 & 11--12 & \mw{dẏ-warchen} & `turf' & preposed adjective & yes \\
24 & 12 & \mw{dreiſ} & `?oppressor' & preposed adjective & yes \\
24 & 12 & \mw{lec hen} & `slate' & ?? & yes \\
24 & 13 & \mw{guia[len]} & `offshoot' & \mw{e} `his' & no \\
24 & 13 & \mw{ureuer} & `loud' & ?? & yes \\
24 & 13 & \mw{urag denn} & `army-member' & preposed adjective & yes \\
24 & 14 & \mw{gu} & `dear' & \mw{at} & yes \\
24 & 14 & \mw{kelein} & `corpse' & preposed adjective & no \\
24 & 15 & \mw{guen} & `white' & feminine noun & no \\
24 & 15 & \mw{mab} & `son' & epithet & no \\
24 & 16 & \mw{ginẏav} & `dinner' & \mw{am} & yes \\
24 & 16 & \mw{drẏlav} & `?earth' & feminine noun & yes \\
24 & 16 & \mw{dꝛẏlen} & `?battle' & feminine noun & yes \\
24 & 16--17 & \mw{dẏ-warchen} & `turf' & preposed adjective & yes \\
24 & 27 & \mw{lumen} & `banner' & \mw{e} `his' & yes \\
24 & 17 & \mw{vreith} & `speckled' & feminine noun & yes \\
24 & 18 & \mw{goꝛuchẏd} & `uplifter' & preposed genitive & no \\
24 & 18 & \mw{lav} & `hand' & \mw{ẏ} `his' & yes \\
24 & 18 & \mw{loflen} & `grasp' & preposed genitive & yes \\
24 & 18 & \mw{gynt} & (personal name) & \mw{ar} & no \\
24 & 19 & \mw{[t]ẏnẏho} & `pulls' & \mw{a} & no \\
24 & 19 & \mw{mwng} & `mane' & object lenition & no \\
24 & 20 & \mw{pren} & `wood' & \mw{heb} & no \\
24 & 20 & \mw{lav} & `hand' & \mw{ẏ} `his' & yes \\
24 & 20 & \mw{lenn} & `covering' & \mw{e} `his' & yes \\
24 & 21 & \mw{marw} & `dead' & \mw{-ei} & no \\
24 & 21 & \mw{mab} & `son' & epithet & no \\ \bottomrule
\caption{Representation of lenition in \mw{Gododdin} B (IW C) pp.\ 23--24}
\label{gododdinc}
\end{mylongtable}

\begin{mylongtable}{@{}llllll@{}}
\toprule
\textbf{Folio} & \textbf{Line} & \textbf{Word} & \textbf{Translation} & \textbf{Cause of lenition} & \textbf{Represented} \\ \midrule\endhead
30 & 12 & \mw{wnelei} & `did' & \mw{a} & yes \\
30 & 13 & \mw{vratwen} & (personal name) & \mw{-ei} & yes \\
30 & 14 & \mw{weleiſ} & `I saw' & \mw{nẏ} & yes \\
30 & 15 & \mw{bwẏ} & `who' & parenthesis & yes \\
30 & 15 & \mw{vei} & `was' & \mw{a} & yes \\
30 & 15 & \mw{waeth} & `worse' & \mw{-ei} & yes \\
30 & 16 & \mw{grẏſſẏaſſant} & `attacked' & \mw{a} & yes \\
30 & 18 & \mw{oꝛfen} & `end' & \mw{hẏt} & yes \\
30 & 18 & \mw{vẏdant} & `will be' & [\mw{a}] & yes \\
30 & 19 & \mw{gẏt garant} & `co-love' & \mw{o?} & yes \\
30 & 22 & \mw{grẏſſẏwſ} & `attacked' & \mw{a} & yes \\
30 & 22 & \mw{ganthud} & `with them' & petrified lenition & yes \\
31 & 3 & \mw{dang} & `truce' & ?? & yes \\
31 & 3 & \mw{weith} & `struggle' & parenthesis & yes \\
31 & 4 & \mw{gocheli} & `shunned' & [\mw{a}] & no \\
31 & 4 & \mw{mab} & `son' & epithet & no \\
31 & 4 & \mw{gwin} & `wine' & object lenition & no \\
31 & 5 & \mw{wneei} & `did' & \mw{a} & yes \\
31 & 6 & \mw{dreghi} & `death' & \mw{e} `his' & yes \\
31 & 7 & \mw{gadeu} & `battles' & ẏ `to' & yes \\
31 & 8 & \mw{crẏſgwẏdẏat} & `attack' & preposed adjective & no \\
31 & 8 & \mw{brẏt} & `time' & ?? & yes \\
31 & 8 & \mw{goꝛlew} & `??' & \mw{am} & no \\
31 & 9 & \mw{oꝛwẏlam} & `happy fate' & \mw{ẏ} `his' & yes \\
31 & 9 & \mw{gigleu} & `heard' & \mw{rẏ} & yes \\
31 & 9 & \mw{gwneei} & `made' & [\mw{a}] & no \\
31 & 10 & \mw{gwẏr} & `men' & \mw{-ei} & no \\
31 & 11 & \mw{mab} & `son' & epithet & no \\
31 & 11 & \mw{beri} & `cause' & \mw{ẏ} `to' & yes \\
31 & 12 & \mw{guodeo} & `conduct' & preposed adjective & no \\
31 & 12 & \mw{celẏo} & `??wild beast' & \mw{e} `to' & no \\
31 & 12 & \mw{vẏhẏr} & `spears' & preposed adjective & yes \\
31 & 13 & \mw{baub} & `everyone' & \mw{-ei} & yes \\
31 & 14 & \mw{wanut} & `you wounded' & \mw{a} & yes \\
31 & 14 & \mw{teir} & `three' & adv.\ clause & no \\
31 & 15 & \mw{vavr} & `large' & \mw{en} & yes \\
31 & 15 & \mw{uaer} & `officer' & \mw{t} `your' & yes \\
31 & 16 & \mw{uo} & `is' & \mw{th} & yes \\
31 & 16 & \mw{di} & `you' & petrified lenition & yes \\
31 & 16 & \mw{gwaſ} & `servant' & parenthesis & yes \\
31 & 17 & \mw{kẏuadraud} & `famed' & feminine noun & no \\
31 & 17 & \mw{glut} & `baggage' & preposed genitive & yes \\
34 & 4 & \mw{cẏuarchant} & `meeting' & \mw{ẏ} `to' & no \\
34 & 5 & \mw{ceinẏo} & `adorned' & compound & no \\
34 & 5 & \mw{urẏthon} & `Britons' & ?? & yes \\
34 & 10 & \mw{briv} & `wound' & compound & no \\
34 & 10 & \mw{bu} & `was' & [\mw{a}] & no \\
34 & 11 & \mw{mur} & `wall' & \mw{ar} & no \\
34 & 13 & \mw{mirein} & `fair' & \mw{o(id)} & no \\
34 & 13 & \mw{uo} & `is' & \mw{a} & yes \\
34 & 13 & \mw{bẏv} & `alive' & NP lenition & no \\
34 & 14 & \mw{gueinieit} & `' & \mw{odam} & no \\
34 & 15 & \mw{uo} & `is' & \mw{a} & yes \\
34 & 15 & \mw{biu} & `alive' & NP lenition & no \\
34 & 19 & \mw{deẏrnet} & `princes' & \mw{o} & yes \\
34 & 19--20 & \mw{grimbu-iller} & `is praised' & \mw{pan} & yes \\
34 & 20 & \mw{wẏr} & `men' & \mw{can} & yes \\
34 & 20 & \mw{gatraeth} & `(place name)' & \mw{a} `from' & yes \\
35 & 1 & \mw{geuin} & `back' & \mw{ar} & yes \\
35 & 1 & \mw{gauall} & `horse' & \mw{e} `his' & yes \\
35 & 1 & \mw{wisguis} & `wore' & \mw{ny} & yes \\
35 & 2 & \mw{guaiu} & `spear' & epithet & no \\
35 & 2 & \mw{gledẏf} & `sword' & \mw{e} `his' & yes \\
35 & 2 & \mw{gẏllell} & `knife' & \mw{e} `his' & yes \\
35 & 3 & \mw{ab} & `son' & epithet & yes \\
35 & 3 & \mw{uei} & `would be' & \mw{a} & yes \\
35 & 3 & \mw{well} & `better' & \mw{-ei} & yes \\
35 & 5 & \mw{guoꝛeu} & `acted' & \mw{a} & no \\
35 & 9 & \mw{guledic} & `prince' & \mw{-ei} & no \\
35 & 9 & \mw{tat} & `father' & \mw{i} `his' & no \\
35 & 12 & \mw{ueiri} & `stewards' & subject of plural verb & yes \\
35 & 12 & \mw{molut} & `clamour' & object lenition & no \\
35 & 13 & \mw{oꝛthoꝛet} & `outbreak' & feminine noun & yes \\
35 & 13 & \mw{drui} & `through' & petrified lenition & yes \\
35 & 13 & \mw{cinneuet} & `kindling' & \mw{drui} & no \\
35 & 14 & \mw{duhet} & `covering' & preposed adjective & yes \\
35 & 16 & \mw{ciuriuet} & `was reckoned' & \mw{a} & no \\
35 & 18 & \mw{ben} & `end' & \mw{hẏt} & yes \\
35 & 22 & \mw{celeo} & `??' & \mw{e} `to' & no \\
36 & 2 & \mw{uoleit} & `praised' & epithet & yes \\
36 & 2 & \mw{map} & `son' & epithet & no \\
36 & 5 & \mw{goll} & `loss' & \mw{o} & yes \\
36 & 7 & \mw{cr[ẏ]ſ} & `??' & \mw{pan} & no \\
36 & 8 & \mw{douiſ} & `lead' & \mw{o} & yes \\
36 & 8--9 & \mw{ce/ſeuin} & `first' & adverbial clause & no \\
36 & 9 & \mw{guanauc} & `desirous' & \mw{moꝛ} & no \\
36 & 10 & \mw{win} & `wine' & \mw{neu} & yes \\
36 & 11 & \mw{wanei} & `stabbed' & \mw{ẏt} & yes \\
36 & 14 & \mw{garat} & `??' & ?? & yes \\
36 & 17 & \mw{galet} & `hard' & ?? & yes \\
36 & 18 & \mw{vre} & `hill' & parenthesis & yes \\
36 & 18 & \mw{gaffei} & `took' & \mw{a} & yes \\
36 & 21 & \mw{drei} & `broken' & ?? & yes \\
36 & 21 & \mw{griniec} & `withered' & \mw{pan} & no \\
36 & 22 & \mw{guanei} & `stabbed' & \mw{ri} & no \\
36 & 22 & \mw{guanet} & `stabbed' & \mw{ri} & no \\
36 & 23 & \mw{cimluin} & `gift' & \mw{i} `his' & no \\
37 & 1 & \mw{olo} & `covering' & \mw{i} `his' & yes \\
37 & 1 & \mw{tan} & `under' & \mw{a} `from' & no \\
37 & 1 & \mw{titguet} & `soil' & \mw{tan} & no \\
37 & 2 & \mw{ued iuet} & `mead-drinking' & \mw{i} `his' & yes \\ \bottomrule
\caption{Representation of lenition in \mw{Gododdin} B (IW B) pp.\ 30--31, 34--37}
\label{gododdinb}
\end{mylongtable}
%%% Local Variables:
%%% mode: latex
%%% TeX-master: "../main"
%%% End:
