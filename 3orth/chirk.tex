 \chapter{Evidence from the Black Book of Chirk}\label{chirk}
At the start of the \gls{mw} period, orthographical representation of lenition arose. If, at the same time, the distinction between \lT\ and \xD\ was maintained into the \gls{mw} period, then this must have caused trouble for scribes trying to represent the existence of three different stop series: \xT, \lT, and \xD. The Latin alphabet does not provide a third set of stops, so they necessarily merged orthographically with their unlenited counterparts\footnote{Similar trouble existed for fricatives, cf.\ \textcite[28]{russell_rowynniauc_2003}, who states that `[i]t has long been observed that, because of the rise of fricatives and spirants within the history of British, early Welsh was seriously understocked in signs to represent the full consonantal inventory.' The significance of the rise of these fricatives and spirants is that the limited Latin set of stops had to be used not only to denote the presumably larger Welsh stop inventory, but also had to provide graphemes for sounds such as \lT. This competition further narrowed chances of the opposition between lenited voiceless and unlenited voiced stops to be represented in writing.}. 

As a result of the new-found habit to distinguish lenition combined with the inability of the Latin alphabet to represent such a wide stop inventory, Early \gls{mw} orthography represented lenition only for other consonants than voiceless stops\footnote{Although lenition was not written for \mw{d} and \mw{rh} either, but orthographical representation for those consonants only became standard by the end of the \gls{mw} period.}. This chapter treats the extent to which orthographical lenition is found, but not for voiceless stops. This limited orthographical system for lenition is indeed found in the \gls{bbch}. I will illustrate the pattern of lenition found in this manuscript with the following example: 

\mwcc[peniarth29]{\acrshort{bbch},~p.~4,~ll.~9--10}{sef eu aylodeu e brenyn. y ueybyon ay neyeynt ay keuenderu.}{These are the members of the king: his sons and his nephews and his cousins.}

Example \ref{peniarth29} shows lenition after \mw{y} `his' in \mw{ueybyon} `sons', but not in \mw{keuenderu} `cousins', even though \mw{y} must also be translated as `his' before this word. Obviously, then, non-lenition of voiceless stops only after \ei\ and \oes\ is found here, but this fact is unremarkable given the wholesale lack of representation of lenition of voiceless stops in this manuscript. 

 The early three-way distinction between stops as proposed by Koch explains why lenition of voiceless stop <k> is not represented orthographically in Example \ref{peniarth29}, while <m> is. However, there are a handful isolated instances where lenited voiceless stops are written with <b, d, g>. The exact extent to which lenition was represented orthographically for different consonants deserves investigation. The results of this investigation may provide insight into whether non-lenition of voiceless stops after verbs was a matter of orthography, or really represented a system of limited postverbal lenition in Early \gls{mw}.

The orthography of the Black Book of Chirk suggests orthography may after all play a role in explaining why voiceless stops are sometimes not written lenited after postverbal lenition. It is possible that lack of writing of lenition of voiceless stops after verbs was maintained when copying from an older source, while the writing of lenition was innovated in (most) other contexts. This pattern would be explainable: lenition has no function in distinguishing meaning after \ei\ and \oes, but it does have such a function after \mw{y} `his', as seen in Example \ref{peniarth29}. For this reason, the orthography of lenition stands a larger chance of being modernised in the latter case than in the former case. If this is true, limited writing of postverbal lenition may not represent any stage of \gls{mw}, but may instead give clues as to the age of a particular text before being written down in its final form, and may give insight into the process of scribal modernisation.

 If Peniarth 29 demonstrates that non-lenition of voiceless stops after verbs is indeed a matter of orthography, then the unique survival of this pattern after verbs only in later \gls{mw} implies that this is the result of imperfect modernisation of an earlier source when the scribe copied a text. This necessarily implies that there was a scribe using a textual source from before orthographical lenition of voiceless stops at some point in the transmission of the text. In this case, non-lenition of voiceless stops after verbs may provide a \textit{terminus ante quem} for its original composition, if we can date the start of the period when lenited voiceless stops were represented with <b, d, g> rather than their voiceless counterparts. In this case, the differing treatment of voiceless stops after verbs in \mw{Pwyll Pendeuic Dyuet} and \mw{Culhwch ac Olwen} compared to the other three branches of the Mabinogi would imply that the former two would be copied from a text written before the orthographical merger of lenited voiceless stops and unlenited voiced stops, while the other three branches would be copied from a text written after this merger. 
 
 The reason why the orthography of lenition would not be modernised in this context needs to be investigated, but may be related to the lack of grammatical saliency of postverbal lenition: in Early \gls{mw}, postverbal lenition did not serve to disambiguate meaning (as opposed to words like \mw{y}, which may either mean `the', `his', and `her', among others, and meaning may be disambiguated on the basis of the mutation it causes)\todo{Paragraph needs to be moved to chapter about postverbal lenition}. 
 
 Another question is from what point onwards the phonemic merger of lenited voiceless stops and unlenited voiced stops took place, and whether it coincided with the orthographical merger. Clues may be taken from alliteration patterns in Early \gls{mw} poetry. In the the earliest \gls{mw} poetry, lenited consonants could alliterate with their unlenited counterparts. It stands to reason that this possibility to alliterate lenited and unlenited consonants lasted up until roughly the end of the period when lenited voiceless stops and unlenited voiced stops merged phonologically. If this assumption is true, then changes in rules governing alliteration may pinpoint changes in the phonology of lenition in \gls{mw}. This assumption may be wrong, though: aspirated and nasalised consonants also alliterated with their radical counterparts, and nasalised consonants could also alliterate with their respective radical counterparts. This suggests that the Early \gls{mw} listener to poetry would judge correct alliteration more on the radical lexemes of the alliterating words than on the actual sounds uttered~\parencite[339]{rowland_early_1990}. This would be a learned feature not influenced by the exact phonological or phonetic reality of Welsh.
 
% If, however, non-lenition of voiceless stops after verbs was a feature of postverbal lenition, then a voiceless stop following a verbal ending must have stayed unlenited phonetically, i.e.\ their value was [p\textsuperscript{h}-, k\textsuperscript{h}-, t\textsuperscript{h}-], not [\bd, \gd, \dd]. Otherwise, non-lenition of voiceless stops would not have stood out after verbs in texts such as the White Book recension of \mw{Pwyll Pendeuic Dyuet}.
 
 \section{Potentially new research questions}
 \begin{itemize}
 \item How is lenition represented in the orthography of Peniarth 29?
 \begin{itemize}
 \item To what extent exactly is lenition written for voiceless stops, and how does this compare to other consonants?
 \item Does lenition after \ei\ and \oes\ show the same distribution as elsewhere?
 \end{itemize}
 \item If non-lenition of voiceless stops after verbs is purely a matter of archaic orthography, then what was the (both relative and absolute) chronology of orthographical lenition?
 \begin{itemize}
 \item When did lenited voiceless stops start to be written with <b, d, g>?
 \item How may this be used to elucidate upon the history of the Four Branches of the Mabinogi? \item Does this mean \mw{Pwyll Pendeuic Dyuet} was copied from a different (earlier) source than the other tales? 
 \item Does the orthographical merger coincide with the phonological merger of lenited voiceless stops and unlenited voiced stops?
 \end{itemize}
 \end{itemize}
 
 \section{Lenition in the Black Book of Chirk}
 
 In this section, I will analyze how lenition is represented orthographically in the Black Book of Chirk. I have taken the first page as a sample. For this end, I will use the transcription prepared by Isaac et al. \parencite*{isaac_rhyddiaith_2013}.
 
 The orthography of the Black Book of Chirk has already been studied by Russell~\parencite*{russell_scribal_1995}. In this article, he notes the following on voiceless stops: \tqt{Latin \textit{t} and \textit{c} had been used in British
before lenition, and after lenition they were used to represent internal and
final /d/ and /ɡ/, later to be replaced by \textit{d} and \textit{g}. In BBCh, \textit{t} and \textit{c} are
still being used for the voiced stops; it is, therefore, worth looking at how
secondary /t/ and /k/ were represented in the absence of \textit{t} and\textit{ c}.}{russell_scribal_1995}{135} In his article, Russell does not make explicit what he means by internal and final /d/ and /ɡ/, i.e.\ whether he considers lenited voiceless stops belonging to this set, or also unlenited voiced stops (e.g. historical geminates in medial or final position).

He also made the following comment: \tqt{The voiced fricative /ð/ arose through the lenition of /d/, and
\gls{ow} orthography did not reflect the change in pronunciation; thus,
before lenition \textit{d} in internal position represented /d/ and \textit{t} represented /t/, but after lenition \textit{d} stood for /ð/ and \textit{t} for /d/. In initial position, however, \textit{t} and \textit{d} still represented /t/ and /d/ respectively. But within \gls{mw} there was a gradual shift to a system where \textit{t} represented /t/ and \textit{d} /d/ in all positions. The pressure would have increased with the rise of secondary \textit{t} from /tt/, etc. in internal and final position, since this secondary \textit{t} could not be spelt \textit{t} if represented /d/. Hence the various attempts to represent /t/ by \textit{tt} and \textit{th}, etc.\ in early \gls{mw}. [\dots]

Initial mutations were vital to preserve the marking of grammatical categories, and so it is reasonable to
suppose that the pressure from the initial spellings of for /t/ and \textit{d} for
/d/ led to a gradual regularization in their favour, i.e.\ \textit{t} for /t/ and \textit{d} for
/d/ in all positions, and from early \gls{mw} onwards the rise of a new
spelling for /ð/, namely \textit{dd}. In \gls{ow}, and apparently also in BBCh, there
is some indication of variation but the major shifts have not yet taken place.}{russell_scribal_1995}{142} I understand this statement to be applicable mainly to the later and more standardized \gls{mw} orthography, as it is not applicable to the manuscript's first page, which shows that exactly this development did not yet take place word-initially. As a result, marking of grammatical categories was not preserved. His statement does have value in a more general sense: initial mutations were vital to mark grammatical categories, and marking of grammatical categories is vital for understanding the contents of a text. Because of this, instances of failure to mark lenition must be regarded as an archaism, not only because we know that this was the \gls{ow} system, but also because this is the \textit{lectio difficilior}.

An interesting point made by Russell is that the rise of medial and final /t/ in words such as \mw{eto} may have had decisive influence in spelling medial/final /d/ with \mw{d} rather than OW \mw{t}.
 
 In the analysis, I list all words which should be lenited, why, and whether lenition is represented orthographically. This list does not include words starting with \mw{d, r}, since lenition of these consonants was only written by the end of the \gls{mw} period. The list also does not consider the orthography intra-word lenition, although a similar pattern seems to be visible there. 

\begin{mwl}\item\onesp{\label{textpeniarthpone}\mw{%
1. Heuel da uab kadell teuyhauc ke\{m\}ry\newline
2. oll a | uelles e | kemry en kam!arueru\newline
3. or kefreythyeu ac a | deuenus atau\newline
4. uy. guyr o | pop kemud en | y tehuyo/\newline
5. kaet e | pduuar en lleycyon ar deu\newline
6. e\{n\} | scolecyon sef achaus e | ue\{n\}nuyt er escleyc/\newline
7. yon rac gossod or lleycyn dym a | uey en er/\newline
8. byn er escrftur lan sef amser e | doythant en/\newline
9. o e | garauuys sef amser achaus e | doyant e | ga/\newline
10. rauuys eno urth delehu o | paup bod en | yaun en\newline
11. er amser glan hunnu. Ac na guenelhey ka\{m\}\newline
12. en amser gleyndyt Ac o | kyd kaghor a kyd\newline
13. synedycaeth. e | doytho\{n\} a | doytant eno er hen\newline
14. kefreythyeu a | esteryasant a | rey onadunt\newline
15. a | adassant y | redec a | rey a | emendassant ac\newline
16. ereyll en kubyl a dyleassant ac ereyll o | ne/\newline
17. uuyt a | hosodassant. a | guedy ho\{n\}ny onadunt\newline
18. e | kefreythyeu a | uarnassant eu cadu. heuel\newline
19. a r<o>des y | audurdaut uthu\{n\}t ac a orckemenus\newline
20. en kadarn eu kadu en craf. a | heuel ar doy/\newline
21. thyon a | uuant ykyd ac ef. a | ossodassant\newline
22. eu hemendyth ar hon kamry holl ar e\newline
23. nep eg | kemry a | lecrey heb eu kadu e | ke/\newline
24. freythyeu. Ac a | dodassant eu heme\{n\}dyt ar er\newline
25. egnat a | kamero dyofryt braut ac\newline
26. ar er argluyt ay rodhey ydau ar ny hu/\newline
27. ypey teyr kolhouen kefreyth a | guerth //\newline%
}(Peniarth 29, p. 1)}\end{mwl}
\begin{table}[h]
\centering
\begin{tabular}{@{}lllll@{}}
\toprule
\textbf{\textbf{Line}} & \textbf{\textbf{Word}} & \textbf{\textbf{Translation}} & \textbf{\textbf{Reason for lenition}} & \textbf{Represented} \\ \midrule
1 & \mw{uab} & `son' & apposition & yes \\
1 & \mw{teuyhauc} & `prince' & apposition & no \\
2 & \mw{welles} & `saw' & follows VP \mw{a} & yes \\
2 & \mw{kemry} & `Welshmen' & follows \mw{e} `his' & no \\
4 & \mw{pop} & `every' & follows \mw{o} `from' & no \\
4--5 & \mw{tehuyo/kaet} & principality & follows \mw{y} `his' & no \\
6 & \mw{ue\{n\}nuyt} & `were procured' & follows \mw{e} ( = \mw{y(r), (r)y}?) & yes \\
7 & \mw{uey} & `would be' & follows VP \mw{a} & yes \\
8 & \mw{lan} & `holy' & follows femine \mw{escrftur} & yes \\
9 & \mw{garauuys} & `Lent' & petrified lenition & yes \\
9--10 & \mw{ga/rauuys} & `Lent' & petrified lenition & yes \\
10 & \mw{paup} & `everyone' & follows \mw{o} `from' & no \\
11 & \mw{ka\{m\}} & `wrong' & follows \ei. & no \\
12 & \mw{kyd} & `union' & follows \mw{o} `from' & no \\
14 & \mw{kefreythyeu} & `laws' & follows \mw{hen} `old' & no \\
16 & \mw{kubyl} & `whole' & follows predicate marker \mw{en} & no \\
18 & \mw{uarnassant} & `judged' & follows VP \mw{a} & yes \\
19 & \mw{orckemenus} & `commanded' & follows VP \mw{a} & yes \\
20 & \mw{kadarn} & `strong' & follows predicate marker \mw{en} & no \\
20 & \mw{craf} & `firm' & follows predicate marker \mw{en} & no \\
21 & \mw{uant} & `were' & follows VP \mw{a} & yes \\
21 & \mw{ossodassant} & `put' & follows VP \mw{a} & yes \\
22 & \mw{kamry} & `Welshmen' & follows \textit{sangiad} & no \\
23 & \mw{lecrey} & `spoiled' & follows VP \mw{a} & yes \\
25 & \mw{kamero} & `may seize' & follows VP \mw{a} & no \\
26--27 & \mw{hu/ypey} & `would know' & follows VP \mw{ny} & yes \\
27 & \mw{teyr} & `three' & follows \ei & no \\ \bottomrule
\end{tabular}
\caption{Representation of lenition found in Example \ref{textpeniarthpone} (Peniarth 29, p. 1).}
\label{lenitionpeniarthpone}
\end{table}
Generally speaking, the first page of Peniarth 29 shows that its orthography does not represent lenition of voiceless stops word-initially. Table \ref{lenitionpeniarthpone} on page \pageref{lenitionpeniarthpone} nevertheless yields some complications worth commenting.

Firstly, \mw{pop} `every' found in l.~4 of Example \ref{textpeniarthpone} is not lenited. Earlier, I considered lenition of \mw{pop} a research exception after verbs due to its erratic behaviour~\parencite[24]{van_sluis_development_2014}, but this erratic behaviour might be caused by precisely the same factor that sometimes gives postverbal lenition limited representation in manuscripts, i.e.\ orthography such as that found in the Black Book of Chirk.

Secondly, \mw{garauuys} and \mw{ga/rauuys} `Lent' in ll.~9, 9--10, resp.\ of Example~\ref{textpeniarthpone} are lenited, but not as a result of either contact lenition or syntactically motivated lenition. Rather, we are dealing with \gls{petr} here.
Lenition of this word apparently occurred at least as early as the thirteenth century, as there are no instances of unpetrified *\mw{C(a)rawys} attested, so \gls{petr} may even date centuries before the thirteenth century~\parencite[Grawys, Garawys]{bevan_geiriadur_2014}.
\Gls{petr} is explainable here, because this word comes from feminine \glat{quadragēsima}, so the feminine article could cause lenition which was then petrified. The choice to write <g> here instead of <c> for hypothesized [\gd] demonstrates that this word was no longer lenited on a synchronic level. 

Thirdly, the relative consistency in failing to represent lenition of voiceless stops word-initially is contrasted by the inconsistency by which this happens word-medially and word-finally. Compare, for example l.~1 \mw{uab} `son' and l.~4 \mw{pop} `every'. Their respective final consonants have the same phonetic and phonological value, i.e.\ [\bd], and both go back to a Proto-Brittonic intervocalic */p/. If lenited /p/ and unlenited /b/ had not yet merged by the time this manuscript was written. Similarly, lenited voiceless stops in word-medial position are not represented consistently. For example, <k> in ll.~4–5 \mw{tehuyo/kaet} goes back to Proto-Brittonic */k/, while <d> in l.~20 \mw{kadarn} goes back to Proto-Brittonic */t/. This implies that lenited voiceless stops and unlenited voiced stops had merged by the time of the Black Book of Chirk, at least in non-word-initial position\footnote{The pattern found in the Black Book of Chirk corroborates Schrijver's position on lenited voiceless stops that they were kept separate from unlenited voiced stops, but only word-initially~\autocite[31]{schrijver_old_2011}}.

The first page of Peniarth 29 leads me to conclude that, word-medially and word-finally, lenited voiceless stops and unlenited voiced stops had already merged by the time \mw{Llyfr Iorwerth} was written. Word-initially, grammatically non-petrified lenition is not represented for voiceless stops, implying that this same merger did not yet take place in word-initial position. Lenition of one word-initial voiceless stop is found in \mw{garauuys} `Lent', but lenition is a petrified property of this word, meaning it did not stand in opposition to unlenited *\mw{carauuys} at a synchronic level\todo{redundant!}.

\subsection{Position of the Peniarth 29 text within the Venedotian Code}

Although the orthography in the Black Book of Chirk certainly looks archaic, it is most likely not the oldest manuscript, and other recensions do not seem to be derived from it. According to Wiliam, the common ancestor of all extant manuscripts is lost: \tqt{It should perhaps be stated at this point that no MS. of those now extant can be regarded as the original Book of Iorwerth, for none of them is such that the others may reasonably be held to be derived from it. Nevertheless, it is probable that all of the Venedotian MSS. were copied directly or indirectly from a common archetype of the class, for there is too much similarity between them to allow a belief in their independent origin.}{wiliam_llyfr_1960}{xxi}
There are several more manuscripts that Wiliam considers of roughly equal age to the Black Book of Chirk\footnote{i.e.\ around the thirteenth century}, and that Wiliam considers descendants from the same archetype~\parencite[xxix]{wiliam_llyfr_1960}. These are: BL Cotton Titus D. ii, NLW Peniarth 35, and BL Cotton Caligula A.

\section{BL Cotton Titus D. ii}
Here, I discuss the same piece of text as it is found in a roughly contemporary manuscript.
\begin{mwl}\item\onesp{\label{textcottontitus}
\mw{%
1.	Hewel uab kadell tywyssaỽc kemry oll a elwys ataỽ \newline
2.	chwegỽyr o pob cantref yg kemry hyt y ty gỽyn ar \newline
3.	taf. a henny or gỽyr doethaf yn | y kyuoeth petwar \newline
4.	onadunt yn lleygyon. ar deu yn esgolheygyon. Sef \newline
5.	achaỽs y ducpỽyt yr esgolheygyon rac dody or lleygyon peth/\newline
6.	eu a uey yn erbyn yr yscrythur glan. a | henny o achaỽs gỽelet \newline
7.	y kemry yn camarueru or kyureythyeu. Sef amser y doe/\newline
8.	thant yno y garawys. a sef achaỽs y doethant y garawys \newline
9.	ỽrth delyu o paỽb bot yn glan yn yr amser gleyndyt hỽn/\newline
10.	nỽ. ac na wnelhey gam. ac o gyt!kyghor a kytsynnyediga/\newline
11.	eth y doythyon a doethant yno yr hen kyureythyeu a edrecha/\newline
12.	sant a rey onadunt a adassant y redec. ac ereyll a emendas/\newline
13.	sant. ac ereyll yn kỽbyl a dyleassant. ac ereyll o newyd a osso/\newline
14.	dassant. a guedy honny onadunt y kyureythyeu a uarnas/\newline
15.	sant hewel a rodes y audurdaỽt udunt ac a orkymynnỽs eu \newline
16.	kadv yn graf ac yn gadarn. a hewel ar doethyon a wuant y \newline
17.	gyt ac ef a ossodassant eu hemendyth ar hon kymry oll ar y nep \newline
18.	yg kymry a lycrey y kyureythyeu hep eu cadỽ ac a ossodassant \newline
19.	eu hemendyt ar yr egnat a kymerey dyouryt braỽt ac ar yr \newline
20.	arglỽyd ay rodhey. yny ỽypey teyr koloỽyn kyureyth a guer/\newline
21.	th\newline}
\newline
(BL Cotton Titus D. ii, f. 1r.)
}\end{mwl}

\begin{table}[h]
\centering
\begin{tabular}{@{}lllll@{}}
\toprule
\textbf{\textbf{Line}} & \textbf{\textbf{Word}} & \textbf{\textbf{Translation}} & \textbf{\textbf{Reason for lenition}} & \textbf{Represented} \\ \midrule
1 & \mw{uab} & `son' & apposition & yes \\
1 & \mw{tywyssaỽc} & `prince' & apposition & no \\
3 & \mw{taf} & (place name) & follows \mw{ar} `on' & no \\
3 & \mw{kyuoeth} & `territory' & follows \mw{y} `his' & no \\
6 & \mw{uey} & `would be' & follows VP \mw{a} & yes \\
6 & \mw{glan} & `holy' & follows femine \mw{yscrythur} & no \\
8 & \mw{garawys} & `Lent' & petrified lenition & yes \\
8 & \mw{garawys} & `Lent' & petrified lenition & yes \\
9 & \mw{paỽb} & `everyone' & follows \mw{o} `from' & no \\
9 & \mw{glan} & `clean' & follows predicate marker \mw{yn} & no \\
10 & \mw{gam} & `wrong' & follows \ei & yes \\
10 & \mw{gyt} & `union' & follows \mw{o} `from' & yes \\
11 & \mw{kyureythyeu} & `laws' & follows \mw{hen} `old' & no \\
12 & \mw{adassant} & `left' & follows VP \mw{a} & yes \\
13 & \mw{kỽbyl} & `whole' & follows adverb marker \mw{yn} & no \\
13--14 & \mw{osso/dassant} & `put' & follows VP \mw{a} & yes \\
14--15 & \mw{uarnas/sant} & `judged' & follows VP \mw{a} & yes \\
15 & \mw{orkymynnỽs} & `commanded' & follows VP \mw{a} & yes \\
16 & \mw{graf} & `firm' & follows predicate marker \mw{yn} & yes \\
16 & \mw{gadarn} & `strong' & follows predicate marker \mw{yn} & yes \\
16 & \mw{wuant} & `were' & follows VP \mw{a} & yes \\
17 & \mw{gyt} & `(to)gether' & follows \mw{i} `to' & yes \\
17 & \mw{ossodassant} & `put' & follows VP \mw{a} & yes \\
18 & \mw{lycrey} & `spoiled' & follows VP \mw{a} & yes \\
18 & \mw{ossodassant} & `put' & follows VP \mw{a} & yes \\
19 & \mw{kymerey} & `would seize' & follows VP \mw{a} & no \\
20 & \mw{ỽypey} & `would know' & follows \mw{yny} & yes \\
20 & \mw{teyr} & `three' & follows \ei & no \\ \bottomrule
\end{tabular}
\caption{Representation of lenition found in Example \ref{textcottontitus} (BL Cotton Titus D.\ ii, f.\ 1r).}
\label{lenitioncottontitus}
\end{table}

Table \ref{lenitioncottontitus} on page \pageref{lenitioncottontitus} shows that, in contrast to Peniarth 29, Cotton Titus D. ii, represents or fails to represent lenition inconsistently. Not only do we see lenited voiceless stops being written with <b, d, g>, but we also see lenition of other consonant types not represented orthographically. Examples of the former type include l.~10 \mw{gam} `wrong' and l.~16 \mw{gadarn} `strong'. The latter type is found in l.~9 \mw{glan} `clean'.

Nevertheless, lenition of voiceless stops is still partially not represented in this text: the radical consonant is written for lenition word-initially in six cases, while the consonant's voiced counterpart is also written six times to denote lenition. 

It is obvious, then, that the Black Book of Chirk is orthographically speaking more conservative than Cotton Titus D.\ ii. However, the latter manuscript has proven that even when an effort is made to modernise, archaic features are not completely removed from a text. Apparently, failing to write lenition did not get in the way of understanding the manuscript's contents too much. 

\section{Orthography in the Black Book of Chirk: conservative or eccentric?}

A fundamental assumption in the above conclusion is that the Black Book of Chirk is so different orthographically because it is conservative, rather than being just plain strange. Morgan Watkin considers its orthography a result of Norman French influence\parencite*{watkin_black_1966}. Russell considers this unlikely on the basis of considerations unrelated to the representation of voiceless stops\parencite*[143--144]{russell_scribal_1995}. However, the problems the strange orthography tries to solve are the same as those which mark the difference between \gls{ow} and \gls{mw}. Welsh has sounds that the Latin alphabet does not have when it is used to represent Latin. Moreover, French orthography never had the need to represent a three-way stop distinction as it did in this stage of Welsh. It seems this same set is the set that is strange in the Black Book of Chirk, and is similar to \gls{ow} orthography in that it does not follow the \gls{mw} conventions of writing non-Latin sounds. It is possible that a scribe was unfamiliar with them, but that they existed. However, it is more likely that the archetypical \mw{Llyfr Iorwerth} text similarly dates from before these conventions spread. In other words, it is simpler to assume that the archetypical version of \mw{Llyfr Iorwerth} predated the later \gls{mw} conventions on spelling non-Latin sounds than assuming that it was somehow ignorant of them.

On the other hand, it is quite possible that the scribe of the Black Book of Chirk himself had imperfect knowledge of Welsh, or of Welsh orthography, seeing as how orthographical lenition was so consistently not modernized, even though other texts from the same period did modernize the orthography of lenition.

\section{The Black Book of Carmarthen}
The Black Book of Carmarthen shows a pattern of lenition that is broadly similar to the Black Book of Chirk. The manuscript is thought to have been compiled around 1250~\autocite[xxiv]{jones_rhagymadrodd_1982}. This assertion will be demonstrated on the basis of lenition seen in several poems found in this book.
\subsection{\mw{Ymddiddan Myrddin a Thaliesin}}
The first poem in this book, \mw{Ymddiddan Myrddin a Thaliesin}~(YMATh), illustrates this statement. Table \ref{lenitionymath} shows every word in YMATh which either shows lenition, or would be expected to show lenition in \gls{mw}. Line numbers refer to the line numbers as found in \textcite{jarman_llyfr_1982}. The table excludes words whose radical starts with \mw{r} or \mw{d}, because no orthographical means to represent their lenition was found for these sounds, but their orthography is a separate matter. 

Lenition of other consonants than \mw{p, t, c} is generally written, and only one exception is found, which is \mw{brivher} `is broken' found in line 27, although lenition is held to be optional following \mw{pan}~\autocite[380]{morgan_y_1952}. By contrast, lenition of voiceless stops in this text is generally not written, except for \gls{petr} in \mw{gan} `with, by' and \mw{ban} `when'. There is one exception to this rule, which is found in line 29: \mw{brouher} `is proved'. A potential explanation for this is that \mw{brouher} `is proved' echoes \mw{brivher} `is broken' two lines earlier, cf.\ the following four lines:

\mwcc[ymath2730]{YMATh ll.\ 27--30, emphasis added}{Llyaus ban \emph{brivher}, llyaus ban foher,\\
Llyaus ev hymchuel in eu hymvan.\\
Seith meib eliffer, seith guir ban \emph{brouher},\\
Seith guaew ny ochel in eu seithran.}{%
Hosts when they are broken, hosts when they are made to scatter,\\
Hosts returning, in their combat.\\
Seven sons of Eliffer, seven men when it is proved,\\
Seven spears they do not avoid in seven companies.}
As can be seen in Example \ref{ymath2730}, both words appear in a similar syntactic context, but also in a very similar semantic context. It is after all the proof (\ie test) that breaks the very same men. Nevertheless, the question rises how it was possible that a lenited voiceless stop could be made to alliterate with an unlenited voiced stop, while at the same time, they were consistently kept apart in the orthography. The answer to this conundrum may lie in the \mw{r} following both words: it is seen in Chapter \ref{oldwelsh} that it is exactly resonants next to stops that create confusion as to which series a stop belongs to. Furthermore, the exceptions \mw{brivher} and \mw{brouher} also show the same behaviour as \mw{grawys} discussed earlier in this chapter.
\begin{table}[h]
\centering
\begin{tabular}{@{}lllll@{}}
\toprule
\textbf{Line} & \textbf{Word}  & \textbf{Translation} & \textbf{Reason for lenition}  & \textbf{Represented} \\ \midrule
1   & \textit{truan} & `sad'   & follows \mw{mor}    & no    \\
1   & \textit{genhẏf} & `with me'  & petrified lenition    & yes   \\
1   & \textit{truan} & `sad'   & follows \mw{mor}    & no    \\
2   & \textit{keduyv} & (personal name) & follows \mw{am}     & no    \\
4   & \textit{tryuruyd} & `bloodstained'  & follows \mw{o}    & no    \\
4   & \textit{tryuan} & `shattered'  & follows \mw{o}    & no    \\
5   & \textit{uelun} & `I saw'   & follows VP \mw{a}    & yes   \\
6   & \textit{teulu} & `retinue'  & follows \mw{y} `his'   & no    \\
8   & \textit{welugan} & `white horse'  & follows \mw{ar}     & yes   \\
10  & \textit{gan} & `by'    & petrified lenition    & yes   \\
11  & \textit{leith} & `death'   & follows \mw{o'e} `of his'   & yes   \\
11  & \textit{teith} & `journey'  & follows \mw{a} `from'    & no    \\
12  & \textit{tarian} & `shield' & follows \mw{y} `his'   & no    \\
15  & \textit{uuan}  & `soon'   & follows \mw{tra} `very'   & yes  \\
16  & \textit{gan} & `by'  & petrified lenition    & yes   \\
19  & \textit{kyulauan} & `battle'   & follows preposed adjective & no    \\
20  & \textit{uerin} & `folk'   & follows \mw{a'e} `and his'  & yes   \\
20  & \textit{wnaethan} & `they did'   & follows VP \mw{a}    & yes   \\
23  & \textit{vit} & `will be'  & follows \mw{pan}    & yes   \\
24  & \textit{wuchit} & `life'   & follows \mw{y} `his'   & yes   \\
26  & \textit{vidan} & `they will be'  & follows non-written VP \mw{a} & yes   \\
27  & \textit{ban} & `when'   & petrified lenition    & yes   \\
27  & \textit{brivher} & `is broken'  & follows \mw{ban}    & no    \\
27  & \textit{ban} & `when'   & petrified lenition    & yes   \\
29  & \textit{ban} & `when'   & petrified lenition    & yes   \\
29  & \textit{brouher} & `is proved'  & follows \mw{pan}    & yes   \\
30  & \textit{ochel} & `avoids'   & follows VP \mw{ni}    & yes   \\
31  & \textit{kyuerbin} & `opposing'   & follows feminine \mw{kad}   & no    \\
31  & \textit{pop} & `every'   & follows \mw{y} `to'    & no    \\
33  & \textit{loneid} & `fullness'   & follows \mw{seith}    & yes   \\
34  & \textit{guaed} & `blood'   & follows \mw{o}    & no    \\ \bottomrule
\end{tabular}
\caption{Representation of lenition in \mw{Ymddiddan Myrddin a Thaliesin} as found in the Black Book of Carmarthen}
\label{lenitionymath}
\end{table}

\subsection{\mw{Marunad Madauc Fil' Maredut}}
Additionally, the poem \mw{Marunad Madauc Fil' Maredut}~(MMFM) has been chosen to demonstrate the same principle of non-representation of lenition of voiceless stops. This poem has been chosen because it is an elegy mourning a known person: Madog ap Maredudd died in 1160~\autocite[82]{jones_gwaith_1991}. Additionally, we know this poem was written by Cynddelw Brydydd Mawr, who was active in the late twelfth century~\autocite[xxx]{jones_gwaith_1991}.


\begin{mwl}
 \item%
  \begin{minipage}{0.45\textwidth}
  \textbf{Black Book of Carmarthen}\\
  \mw{Kywarchaw im ri.\ rad wobeith\\
  Kywarchaw kywercheis e \al{c}anweith.\\
  Y \al{p}rowi prydv.\ o \al{p}riwieth.\\
  Eurgert. ym argluit \al{k}edymteith.\\
  Y \al{c}vinav madauc.\ metweith y alar\\}
  % \end{block}
  \end{minipage}
  \begin{minipage}{0.45\textwidth}
  \textbf{Hendregadredd manuscript}\\
  \mw{Kyuarchaf y'm Ri rad wobeith,\\
  Kyuarchaf, kyuercheis \al{g}anweith,\\
  Y \al{b}roui prydu o'm prifyeith --- eurgert\\
  Y'm arglwyt \al{g}edymdeith,\\
  Y \al{g}wynaỽ Madaỽc metueith --- y alar, \\}
  % \end{block}
  \end{minipage}
 % \caption{The first five lines of \mw{Marwnad Madog ap Maredudd}, with lenited voiceless stops marked.}
 \label{marwnadcomparison}
\end{mwl}


Analysis of the poem is helped by the existence of the very same poem in a similarly old manuscript, but which shows lenition of voiceless stops for the most part, \ie \gls{h}. This manuscript dates from about 1300~\autocite{huws_llawysgrif_1981}. By way of illustration, Figure~\ref{marwnadcomparison} shows first five lines of this poem as found in both manuscripts, and easily demonstrates the difference in orthography. Since the Hendregadredd version does typically write lenition of voiceless stops, and forms the basis of the analysis by \textcite[82--91]{jones_gwaith_1991}, it is easy to make sense of all the grammatical preconditions for lenition as found in Table \ref{lenitionmmfm}. Line numbers refer to \textcite[78-79]{jarman_llyfr_1982}, but also generally agree with \textcite[82--91]{jones_gwaith_1991} for the Hendregadredd version.

Table \ref{lenitionmmfm} shows that voiceless stops are not lenited as a rule, while other consonants are typically lenited. Furthermore, there are cases where it is not clear why lenition was written. In the table, these instances are marked with `?', optionally followed by a suggestion why it may be lenited. In all of these cases, lenition is confirmed by the \gls{h} recension. Lenition is confirmed by \gls{h} in these cases, so I will not delve into the why of these lenitions further. However, the case of \mw{kedymteith} `companion' in line 4 may prove interesting. It is written \mw{gedymdeith} in \gls{h}, but it is not clear what grammatical reason there is for lenition. Lenition in \mw{gedymdeith} may therefore be regarded as a hypercorrect insertion of orthographical representation for lenition of voiceless stops. This in turn suggests that orthographical representation of lenition of voiceless stops was indeed an innovation which occurred after the death of Madog ap Maredudd. 

Despite the general pattern of non-lenition of voiceless stops, and lenition of other consonants, there are several exceptions to this rule.
These exceptions show lenition of voiceless stops in \mw{gar}~(l.~22) and \mw{dan}~(l.~23), as well as non-lenition of other consonants in \mw{lledieith}~(l.~17) and \mw{madauc}~(l.~28).
Line 22, in which \mw{gar} `loves' is found, is written in the margin of the page on which it is found.
This might point to a later insertion, but the insertion is written by the same hand.
This leaves this matter unsolved. The other lenited voiceless stop, in \mw{dan} `under', falls under \gls{petr}, which is already seen to be represented in this chapter.
The words \mw{lledieith} `foreign' and \mw{madauc} `Madog' are both used as epithets, which are known to lenite, but this is not done consistently\todo{I should refer to my results from CO here, which may or may not end up in my final thesis.}.

\begin{table}[h]
\centering
\begin{tabular}{@{}lllll@{}}
\toprule
\textbf{Line} & \textbf{Word} & \textbf{Translation} & \textbf{Reason for lenition} & \textbf{Represented} \\ \midrule
1 & \textit{wobeith} & `hope' & follows \mw{o} found in \gls{h} & yes \\
2 & \textit{canweith} & `hundred times' & adverbial clause & no \\
3 & \textit{prowi} & `prove' & follows \mw{y} `to' & no \\
3 & \textit{priwieith} & `best language' & follows \mw{o} & no \\
4 & \textit{kedymteith} & `companion' & ?dvandva compound & no \\
5 & \textit{cvinav} & `mourn' & follows \mw{y} `to' & no \\
5 & \textit{alar} & `lament' & follows \mw{y} `his' & yes \\
6 & \textit{alon} & `enemies' & follows \mw{y} `his' & yes \\
7 & \textit{canhimteith} & `escort' & follows feminine \mw{yscvid} & no \\
11 & \textit{wobeith} & `hope' & ? & yes \\
12 & \textit{kedimteith} & `companion' & follows preposed adjective & no \\
13 & \textit{leith} & `death' & follows \mw{no'e} `than his' & yes \\
16 & \textit{wisscoet} & `vestments' & follows preposed adjective & yes \\
16 & \textit{wessgvin} & `Gascon horse' & ? & yes \\
16 & \textit{canhimteith} & `escort' & follows preposed genitive & no \\
17 & \textit{vab} & `son' & epithet following PN & yes \\
17 & \textit{lledieith} & `foreign' & epithet following PN & no \\
18 & \textit{clod} & `fame' & follows \mw{y} `his' & no \\
19 & \textit{vaon} & `subjects' & follows preposed adjective & yes \\
19 & \textit{oleith} & `retreats' & follows VP \mw{ni} & yes \\
20 & \textit{wastad} & `constant' & follows \mw{rhad} & yes \\
20 & \textit{canhimteith} & `escort' & follows preposed genitive & no \\
22 & \textit{gar} & `loves' & follows VP \mw{a} & yes \\
22 & \textit{kidweith} & `joint work' & follows \mw{o} & no \\
23 & \textit{dan} & `under' & petrified lenition & yes \\
23 & \textit{calchwreith} & `vari-coloured' & follows feminine \mw{yscvd} & no \\
25 & \textit{owin} & `wish' & follows preposed genitive & yes \\
26 & \textit{pedeirieith} & `four languages' & follows feminine \mw{yscvid} & no \\
27 & \textit{teirn} & `king' & follows feminine \mw{hil} & no \\
28 & \textit{madauc} & (personal name) & epithet following \mw{hael} & no \\
28 & \textit{veuder} & `dread' & ? & yes \\
29 & \textit{leith} & `death' & follows \mw{o'e} `of his' & yes \\
30 & \textit{kedymteith} & `companion' & follows \mw{darw} `ended' & no \\
33 & \textit{truited} & `welcome' & follows \mw{y} `his' & no \\
34 & \textit{kywarweith} & `fight' & follows \mw{o'e} `of his' & no \\
36 & \textit{kadieith} & (personal name) & follows feminine \mw{aerllin} & no \\
37 & \textit{kadarn} & `strong' & follows preposed adjective & no \\
37 & \textit{kedymdemteith} & `companion' & follows preposed adjective & no \\
38 & \textit{talheith} & `crown' & follows \mw{y} `his' & no \\
39 & \textit{leith} & `death' & follows \mw{y} `his' & yes \\ \bottomrule
\end{tabular}
\caption{Representation of lenition in \mw{Marunad Madauc Fil' Maredut} as found in the Black Book of Carmarthen}
\label{lenitionmmfm}
\end{table}

\section{Final remarks}
If we accept the following premises:
\begin{enumerate}
\item \gls{ow} did not represent lenition orthographically;
\item Early \gls{mw} did not represent lenition of voiceless stops orthographically;
\item Later \gls{mw} did represent lenition orthographically;
\item When a text with older orthography is copied, lenition is typically only modernised where it is vital to represent or disambiguate grammatical categories;
\item Lenition following verbs in Early \gls{mw} did not represent or disambiguate verbal categories;
\end{enumerate}

Then we must conclude the following:
\begin{enumerate}
\item In a text where postverbal lenition is represented orthographically, but not for voiceless stops, we may postulate a \textit{terminus ante quem} as well as a \textit{terminus post quem} for its original composition: it must originally have been composed before lenition of voiceless stops was represented orthographically, and after lenition started being represented at all.
\end{enumerate}

This conclusion necessarily follows from the premises above: if lenition following verbs was not modernised orthographically, then this must have been the case equally when copying from an \gls{ow} text and when copying from an Early \gls{mw} text. If, hypothetically speaking, an \gls{ow} text were copied into Later \gls{mw}, then we should not expect non-lenition of voiceless stops only. Rather, we should expect non-lenition of any consonant following verbs, because premise (4) above is as much applicable to premise (1) as it is to premise (2).


\subsection{The suddenness of changing orthographies}
The innovation of writing out lenition of voiceless stops around 1250 did not come alone. At around the same time, we see the spread of <y> to relieve <i> and <e> of multiple phonological loads, <w> stopped being used to denote /v/. In a similar vein, we see old irregular verbal inflections such as \mw[said]{amkawd}, \mw[did]{goruc} and \mw[heard]{kigleu} falling out of use\todo{Check dates of attestations}.

All these changes seemed to happen around a time of great societal unrest. Between 1277 and 1283, the Edwardian conquest of Wales took place. This period marked the defeat and annexation of the last native Welsh principalities and to their replacement with Marcher lords\todo{expand}.

Explaining linguistic innovation by means of societal developments 

It is not unheard of for a period of great social upheaval to also show a great linguistic upheaval. A well-known example is the rapid development of \gls{oir} from \gls{pi}. When comparing ogham inscriptions from the fourth century with sixth-century manuscript texts, one finds that the former type of texts looks more or less the same as how we reconstruct Proto-Celtic from a millennium earlier, while the Old Irish forms found two centuries later are still recognizable nowadays --- even more than a millenium later. The key lies in understanding that these two centuries saw an uprooting not in language, but in how language was represented in orthography:
\tqt{A colloquial language only becomes vulnerable to replacement if an entire society is uprooted or withers away, but an educated standard can be drastically altered or replaced through the far less disruptive replacement of an educated élite or even an élite's evolutionary reformation of itself (Cf.~Greene 1971). So, for example, a change of religion (or even the reformation of a continuing religion) can easily have the effect of removing an old sacred language from the curriculum, whether or not the pristhood is replaced, purged, or henceforth recruited from a different class.}{koch_conversion_1995}{47}

It is not difficult to assume a similar, although much less drastic upheaval of learned culture in late-thirteenth-century Wales. Before the Edwardian conquest, the Poets of the Princes had a fixed spot in Welsh courtly life, but after the uprooting of these structures by the conquest, the poets' positions were under threat. This may have provided an opportune moment to start representing linguistic innovations that were heretofore unrepresented in the conservative written standard\todo{Line of reasoning needs quite a bit of fine tuning. More info on the \mw{beirdd}, role of clergy, law in the period. Also: orthography was not wholly uprooted unlike Irish.}. 

Some quotations from Davies

\tqt{\begin{welsh}Ar lawer golwg, ni fu'r Goncwes yn andwyol i'r Eglwys ac ystyrir yr hanner canrif ar \^ol 1282 yn rhyw fath o oes aur yn ei hanes. Cafodd esgobion a oedd yn w\^yr galluog ac ymroddgar, y rhan fwyaf ohonynt yn Gymry neu'n ddynion a chysylltiadau Cymreig ganddynt. Cynhyrchwyd corff o lenyddiaeth grefyddol yn y Gymraeg, parhad o weithgarwch a oedd eisoes ar droed cyn y Goncwest. Tystia'r rhyddiaith a'r cerddi crefyddol yn \textit{Llyfr Gwyn Rhydderch} ac yn \textit{Llyfr yr Ancr} fod y Gymraeg, megis eraill o brif ieithoedd llenyddol Ewrop, yn datblygu'n `un o dafodieithoedd datguddiad Duw'. 
\end{welsh}
}{davies_hanes_1990}{167}

\tqt{The Cistercians also contributed handsomely to the literary culture of Wales. There has, it is true, been a tendency to ascribe all Welsh manuscripts indiscriminately to Cistercian \textit{scriptoria} and thereby to overlook the contribution of lay scribes. Thus it has recently been shown that the largest and single most valuable volume of Welsh medieval literature, \textit{The Red Book of Hergest}, which has been variously ascribed to the \textit{scriptoria} of Strata Florida and Neath abbeys, was in fact mainly written by a lay scribe from Builth for a lay patron. Yet this is not to gainsay the contribution of the Cistercians in literary matters. Much of their work was doubtless devotional, such as the concordance of St Bernard's Song of Songs compiled by an abbot of Margam, or the translation of the Athanasian creed into Welsh probably undertaken by a monk of Strata Florida. But the abbeys also became important centres for the conservation and transmission of secular learning. The library at Margam included copies of the works of Geoffrey of Monmouth and William of Malmesbury, while the survival of a single early Welsh verse or \textit{englyn} on the flap of a charter there possibly indicates an interest in vernacular literature also. Several Welsh literary manuscripts were probably copied at Cistercian abbeys: the include, for example, a Welsh version of the Charlemagne legend probably compiled at Strata florida and it was there also, around 1300, that the earliest most systematic and most comprehensive collection of medieval Welsh court poetry, now known as \textit{Llawysgrif Hendregadredd}, was assembled. Some of the survaving versions of Welsh law may also have been written and for monastic libraries. The Cistercian houses, from an early date, took upon themselves the role of compiling annals of events in Wales, thereby supplementing and extending the earlier annalistic tradition at St Davids and possibly at Llanbadarn. It was at Strata Florida that the lost Latin chronicle which forms the basis of the Chronicle of the Princes was compiled, in the late thirteenth century, and it was there also that at least one translation into Welsh, \textit{Brut y Tywysogyon}, was effected. But Strata Florida was not alone. Annals and chronicles were also composed at other Cistercian monasteries in Wales and some of them were circulated from one house to another. [\dots]

[D]uring the later thirteenth century, it had increasingly to share its supremacy with the mendicant friars. In many ways Wales was and remained unpromising ground for the friars; it had neither the large urban centres nor the university schools which were their natural habitat. It is not surprising, therefore, that only three Franciscan houses [\dots] were founded in Wales. Yet their impact was considerable. Already by the end of the thirteenth century they were outstripping the Cistercians in recruitment: there were twenty-three friars in the Dominican house at Rhuddlan. Thirty at Cardiff. [\dots] By teaching and, above all, by preaching, the friars introduced new standards into the life of parish clergy and laity alike; they undertook the translation of manuals of instruction for use in parochial work; they used the vernacular to communicate the intensity of their religious sentiments and vision [\dots]. }{davies_history_1987}{200--202}

% The Welsh law texts may help us in dating the transition between Early \gls{mw} showing no lenition of voiceless stops and later \gls{mw} showing lenition. If we accept Wiliam's~\parencite*[xx]{wiliam_llyfr_1960} date of the thirteenth century for BL Cotton Titus D. ii, then we must conclude that lenition of voiceless stops was represented by the thirteenth century.

% The orthographically more conservative first page of the Black Book of Chirk, however, shows that the archetype of these two manuscripts did not represent lenition of voiceless stops in its orthography. We must conclude from this that lenition of voiceless stops was not yet represented by the time it was written down. Perhaps the date of composition of the original Book of Iorwerth may serve as a \textit{terminus ante quem} for orthographical representation of lenition of voiceless stops.
% \newpage

