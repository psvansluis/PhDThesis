 \chapter{Evidence from the Black Book of Chirk}\label{chirk}
\todo[inline]{This paragraph makes for a good introduction to the part on orthography}

 
 \section{Lenition in the Black Book of Chirk}

 
 In this section, I will analyze how lenition is represented orthographically in the Black Book of Chirk. I have taken the first page as a sample. For this end, I will use the transcription prepared by Isaac et al. \parencite*{isaac_rhyddiaith_2013}.

\todo[inline]{Some insights may be salvaged and added to the laws chapter}
 


\todo[inline]{The data below are redundant with the laws chapter}

 In the analysis, I list all words which should be lenited, why, and whether lenition is represented orthographically. This list does not include words starting with \mw{d, r}, since lenition of these consonants was only written by the end of the \gls{mw} period. The list also does not consider the orthography intra-word lenition, although a similar pattern seems to be visible there. 

\begin{mwl}\item\onesp{\label{textpeniarthpone}\mw{%
1. Heuel da uab kadell teuyhauc ke\{m\}ry\newline
2. oll a | uelles e | kemry en kam!arueru\newline
3. or kefreythyeu ac a | deuenus atau\newline
4. uy. guyr o | pop kemud en | y tehuyo/\newline
5. kaet e | pduuar en lleycyon ar deu\newline
6. e\{n\} | scolecyon sef achaus e | ue\{n\}nuyt er escleyc/\newline
7. yon rac gossod or lleycyn dym a | uey en er/\newline
8. byn er escrftur lan sef amser e | doythant en/\newline
9. o e | garauuys sef amser achaus e | doyant e | ga/\newline
10. rauuys eno urth delehu o | paup bod en | yaun en\newline
11. er amser glan hunnu. Ac na guenelhey ka\{m\}\newline
12. en amser gleyndyt Ac o | kyd kaghor a kyd\newline
13. synedycaeth. e | doytho\{n\} a | doytant eno er hen\newline
14. kefreythyeu a | esteryasant a | rey onadunt\newline
15. a | adassant y | redec a | rey a | emendassant ac\newline
16. ereyll en kubyl a dyleassant ac ereyll o | ne/\newline
17. uuyt a | hosodassant. a | guedy ho\{n\}ny onadunt\newline
18. e | kefreythyeu a | uarnassant eu cadu. heuel\newline
19. a r<o>des y | audurdaut uthu\{n\}t ac a orckemenus\newline
20. en kadarn eu kadu en craf. a | heuel ar doy/\newline
21. thyon a | uuant ykyd ac ef. a | ossodassant\newline
22. eu hemendyth ar hon kamry holl ar e\newline
23. nep eg | kemry a | lecrey heb eu kadu e | ke/\newline
24. freythyeu. Ac a | dodassant eu heme\{n\}dyt ar er\newline
25. egnat a | kamero dyofryt braut ac\newline
26. ar er argluyt ay rodhey ydau ar ny hu/\newline
27. ypey teyr kolhouen kefreyth a | guerth //\newline%
}(Peniarth 29, p. 1)}\end{mwl}
\begin{table}[h]
\centering
\begin{tabular}{@{}lllll@{}}
\toprule
\textbf{\textbf{Line}} & \textbf{\textbf{Word}} & \textbf{\textbf{Translation}} & \textbf{\textbf{Reason for lenition}} & \textbf{Represented} \\ \midrule
1 & \mw{uab} & `son' & apposition & yes \\
1 & \mw{teuyhauc} & `prince' & apposition & no \\
2 & \mw{welles} & `saw' & follows VP \mw{a} & yes \\
2 & \mw{kemry} & `Welshmen' & follows \mw{e} `his' & no \\
4 & \mw{pop} & `every' & follows \mw{o} `from' & no \\
4--5 & \mw{tehuyo/kaet} & principality & follows \mw{y} `his' & no \\
6 & \mw{ue\{n\}nuyt} & `were procured' & follows \mw{e} ( = \mw{y(r), (r)y}?) & yes \\
7 & \mw{uey} & `would be' & follows VP \mw{a} & yes \\
8 & \mw{lan} & `holy' & follows femine \mw{escrftur} & yes \\
9 & \mw{garauuys} & `Lent' & petrified lenition & yes \\
9--10 & \mw{ga/rauuys} & `Lent' & petrified lenition & yes \\
10 & \mw{paup} & `everyone' & follows \mw{o} `from' & no \\
11 & \mw{ka\{m\}} & `wrong' & follows \ei. & no \\
12 & \mw{kyd} & `union' & follows \mw{o} `from' & no \\
14 & \mw{kefreythyeu} & `laws' & follows \mw{hen} `old' & no \\
16 & \mw{kubyl} & `whole' & follows predicate marker \mw{en} & no \\
18 & \mw{uarnassant} & `judged' & follows VP \mw{a} & yes \\
19 & \mw{orckemenus} & `commanded' & follows VP \mw{a} & yes \\
20 & \mw{kadarn} & `strong' & follows predicate marker \mw{en} & no \\
20 & \mw{craf} & `firm' & follows predicate marker \mw{en} & no \\
21 & \mw{uant} & `were' & follows VP \mw{a} & yes \\
21 & \mw{ossodassant} & `put' & follows VP \mw{a} & yes \\
22 & \mw{kamry} & `Welshmen' & follows \textit{sangiad} & no \\
23 & \mw{lecrey} & `spoiled' & follows VP \mw{a} & yes \\
25 & \mw{kamero} & `may seize' & follows VP \mw{a} & no \\
26--27 & \mw{hu/ypey} & `would know' & follows VP \mw{ny} & yes \\
27 & \mw{teyr} & `three' & follows \ei & no \\ \bottomrule
\end{tabular}
\caption{Representation of lenition found in Example \ref{textpeniarthpone} (Peniarth 29, p. 1).}
\label{lenitionpeniarthpone}
\end{table}
Generally speaking, the first page of Peniarth 29 shows that its orthography does not represent lenition of voiceless stops word-initially. Table \ref{lenitionpeniarthpone} on page \pageref{lenitionpeniarthpone} nevertheless yields some complications worth commenting.

\todo[inline]{The rest of this subsection, especially \mw[]{garawys}, may have some use when discussing research exceptions}

Firstly, \mw{pop} `every' found in l.~4 of Example \ref{textpeniarthpone} is not lenited. Earlier, I considered lenition of \mw{pop} a research exception after verbs due to its erratic behaviour~\parencite[24]{van_development14}, but this erratic behaviour might be caused by precisely the same factor that sometimes gives postverbal lenition limited representation in manuscripts, i.e.\ orthography such as that found in the Black Book of Chirk.

Secondly, \mw{garauuys} and \mw{ga/rauuys} `Lent' in ll.~9, 9--10, resp.\ of Example~\ref{textpeniarthpone} are lenited, but not as a result of either contact lenition or syntactically motivated lenition. Rather, we are dealing with \gls{petr} here.
Lenition of this word apparently occurred at least as early as the thirteenth century, as there are no instances of unpetrified *\mw{C(a)rawys} attested, so \gls{petr} may even date centuries before the thirteenth century~\parencite[Grawys, Garawys]{bevan_geiriadur_2014}.
\Gls{petr} is explainable here, because this word comes from feminine \glat{quadragēsima}, so the feminine article could cause lenition which was then petrified. The choice to write \graph{g} here instead of \graph{c} for hypothesized [\gd] demonstrates that this word was no longer lenited on a synchronic level. 

Thirdly, the relative consistency in failing to represent lenition of voiceless stops word-initially is contrasted by the inconsistency by which this happens word-medially and word-finally. Compare, for example l.~1 \mw{uab} `son' and l.~4 \mw{pop} `every'. Their respective final consonants have the same phonetic and phonological value, i.e.\ [\bd], and both go back to a Proto-Brittonic intervocalic */p/. If lenited /p/ and unlenited /b/ had not yet merged by the time this manuscript was written. Similarly, lenited voiceless stops in word-medial position are not represented consistently. For example, \graph{k} in ll.~4–5 \mw{tehuyo/kaet} goes back to Proto-Brittonic */k/, while \graph{d} in l.~20 \mw{kadarn} goes back to Proto-Brittonic */t/. This implies that lenited voiceless stops and unlenited voiced stops had merged by the time of the Black Book of Chirk, at least in non-word-initial position\footnote{The pattern found in the Black Book of Chirk corroborates Schrijver's position on lenited voiceless stops that they were kept separate from unlenited voiced stops, but only word-initially~\autocite[31]{schrijver_old_2011}}.

The first page of Peniarth 29 leads me to conclude that, word-medially and word-finally, lenited voiceless stops and unlenited voiced stops had already merged by the time \mw{Llyfr Iorwerth} was written. Word-initially, grammatically non-petrified lenition is not represented for voiceless stops, implying that this same merger did not yet take place in word-initial position. Lenition of one word-initial voiceless stop is found in \mw{garauuys} `Lent', but lenition is a petrified property of this word, meaning it did not stand in opposition to unlenited *\mw{carauuys} at a synchronic level\todo{redundant!}.

\subsection{Position of the Peniarth 29 text within the Venedotian Code}

\todo[inline]{delete whole subsection}
Although the orthography in the Black Book of Chirk certainly looks archaic, it is most likely not the oldest manuscript, and other recensions do not seem to be derived from it. According to Wiliam, the common ancestor of all extant manuscripts is lost: \tqt{It should perhaps be stated at this point that no MS. of those now extant can be regarded as the original Book of Iorwerth, for none of them is such that the others may reasonably be held to be derived from it. Nevertheless, it is probable that all of the Venedotian MSS. were copied directly or indirectly from a common archetype of the class, for there is too much similarity between them to allow a belief in their independent origin.}{wiliam_llyfr_1960}{xxi}
There are several more manuscripts that Wiliam considers of roughly equal age to the Black Book of Chirk\footnote{i.e.\ around the thirteenth century}, and that Wiliam considers descendants from the same archetype~\parencite[xxix]{wiliam_llyfr_1960}. These are: BL Cotton Titus D. ii, NLW Peniarth 35, and BL Cotton Caligula A.

\section{BL Cotton Titus D. ii}
\todo[inline]{Whole section can go}
Here, I discuss the same piece of text as it is found in a roughly contemporary manuscript.
\begin{mwl}\item\onesp{\label{textcottontitus}
\mw{%
1.	Hewel uab kadell tywyssaỽc kemry oll a elwys ataỽ \newline
2.	chwegỽyr o pob cantref yg kemry hyt y ty gỽyn ar \newline
3.	taf. a henny or gỽyr doethaf yn | y kyuoeth petwar \newline
4.	onadunt yn lleygyon. ar deu yn esgolheygyon. Sef \newline
5.	achaỽs y ducpỽyt yr esgolheygyon rac dody or lleygyon peth/\newline
6.	eu a uey yn erbyn yr yscrythur glan. a | henny o achaỽs gỽelet \newline
7.	y kemry yn camarueru or kyureythyeu. Sef amser y doe/\newline
8.	thant yno y garawys. a sef achaỽs y doethant y garawys \newline
9.	ỽrth delyu o paỽb bot yn glan yn yr amser gleyndyt hỽn/\newline
10.	nỽ. ac na wnelhey gam. ac o gyt!kyghor a kytsynnyediga/\newline
11.	eth y doythyon a doethant yno yr hen kyureythyeu a edrecha/\newline
12.	sant a rey onadunt a adassant y redec. ac ereyll a emendas/\newline
13.	sant. ac ereyll yn kỽbyl a dyleassant. ac ereyll o newyd a osso/\newline
14.	dassant. a guedy honny onadunt y kyureythyeu a uarnas/\newline
15.	sant hewel a rodes y audurdaỽt udunt ac a orkymynnỽs eu \newline
16.	kadv yn graf ac yn gadarn. a hewel ar doethyon a wuant y \newline
17.	gyt ac ef a ossodassant eu hemendyth ar hon kymry oll ar y nep \newline
18.	yg kymry a lycrey y kyureythyeu hep eu cadỽ ac a ossodassant \newline
19.	eu hemendyt ar yr egnat a kymerey dyouryt braỽt ac ar yr \newline
20.	arglỽyd ay rodhey. yny ỽypey teyr koloỽyn kyureyth a guer/\newline
21.	th\newline}
\newline
(BL Cotton Titus D. ii, f. 1r.)
}\end{mwl}

\begin{table}[h]
\centering
\begin{tabular}{@{}lllll@{}}
\toprule
\textbf{\textbf{Line}} & \textbf{\textbf{Word}} & \textbf{\textbf{Translation}} & \textbf{\textbf{Reason for lenition}} & \textbf{Represented} \\ \midrule
1 & \mw{uab} & `son' & apposition & yes \\
1 & \mw{tywyssaỽc} & `prince' & apposition & no \\
3 & \mw{taf} & (place name) & follows \mw{ar} `on' & no \\
3 & \mw{kyuoeth} & `territory' & follows \mw{y} `his' & no \\
6 & \mw{uey} & `would be' & follows VP \mw{a} & yes \\
6 & \mw{glan} & `holy' & follows femine \mw{yscrythur} & no \\
8 & \mw{garawys} & `Lent' & petrified lenition & yes \\
8 & \mw{garawys} & `Lent' & petrified lenition & yes \\
9 & \mw{paỽb} & `everyone' & follows \mw{o} `from' & no \\
9 & \mw{glan} & `clean' & follows predicate marker \mw{yn} & no \\
10 & \mw{gam} & `wrong' & follows \ei & yes \\
10 & \mw{gyt} & `union' & follows \mw{o} `from' & yes \\
11 & \mw{kyureythyeu} & `laws' & follows \mw{hen} `old' & no \\
12 & \mw{adassant} & `left' & follows VP \mw{a} & yes \\
13 & \mw{kỽbyl} & `whole' & follows adverb marker \mw{yn} & no \\
13--14 & \mw{osso/dassant} & `put' & follows VP \mw{a} & yes \\
14--15 & \mw{uarnas/sant} & `judged' & follows VP \mw{a} & yes \\
15 & \mw{orkymynnỽs} & `commanded' & follows VP \mw{a} & yes \\
16 & \mw{graf} & `firm' & follows predicate marker \mw{yn} & yes \\
16 & \mw{gadarn} & `strong' & follows predicate marker \mw{yn} & yes \\
16 & \mw{wuant} & `were' & follows VP \mw{a} & yes \\
17 & \mw{gyt} & `(to)gether' & follows \mw{i} `to' & yes \\
17 & \mw{ossodassant} & `put' & follows VP \mw{a} & yes \\
18 & \mw{lycrey} & `spoiled' & follows VP \mw{a} & yes \\
18 & \mw{ossodassant} & `put' & follows VP \mw{a} & yes \\
19 & \mw{kymerey} & `would seize' & follows VP \mw{a} & no \\
20 & \mw{ỽypey} & `would know' & follows \mw{yny} & yes \\
20 & \mw{teyr} & `three' & follows \ei & no \\ \bottomrule
\end{tabular}
\caption{Representation of lenition found in Example \ref{textcottontitus} (BL Cotton Titus D.\ ii, f.\ 1r).}
\label{lenitioncottontitus}
\end{table}

Table \ref{lenitioncottontitus} on page \pageref{lenitioncottontitus} shows that, in contrast to Peniarth 29, Cotton Titus D. ii, represents or fails to represent lenition inconsistently. Not only do we see lenited voiceless stops being written with \graph{b, d, g}, but we also see lenition of other consonant types not represented orthographically. Examples of the former type include l.~10 \mw{gam} `wrong' and l.~16 \mw{gadarn} `strong'. The latter type is found in l.~9 \mw{glan} `clean'.

Nevertheless, lenition of voiceless stops is still partially not represented in this text: the radical consonant is written for lenition word-initially in six cases, while the consonant's voiced counterpart is also written six times to denote lenition. 

It is obvious, then, that the Black Book of Chirk is orthographically speaking more conservative than Cotton Titus D.\ ii. However, the latter manuscript has proven that even when an effort is made to modernise, archaic features are not completely removed from a text. Apparently, failing to write lenition did not get in the way of understanding the manuscript's contents too much. 

\section{Orthography in the Black Book of Chirk: conservative or eccentric?}

\todo[inline]{This can probably be deleted because BBCh is not in fact eccentric for its time in terms of lenition of ptc}

A fundamental assumption in the above conclusion is that the Black Book of Chirk is so different orthographically because it is conservative, rather than being just plain strange. Morgan Watkin considers its orthography a result of Norman French influence\parencite*{watkin_black_1966}. Russell considers this unlikely on the basis of considerations unrelated to the representation of voiceless stops\parencite*[143--144]{Rus_Scribal95}. However, the problems the strange orthography tries to solve are the same as those which mark the difference between \gls{ow} and \gls{mw}. Welsh has sounds that the Latin alphabet does not have when it is used to represent Latin. Moreover, French orthography never had the need to represent a three-way stop distinction as it did in this stage of Welsh. It seems this same set is the set that is strange in the Black Book of Chirk, and is similar to \gls{ow} orthography in that it does not follow the \gls{mw} conventions of writing non-Latin sounds. It is possible that a scribe was unfamiliar with them, but that they existed. However, it is more likely that the archetypical \mw{Llyfr Iorwerth} text similarly dates from before these conventions spread. In other words, it is simpler to assume that the archetypical version of \mw{Llyfr Iorwerth} predated the later \gls{mw} conventions on spelling non-Latin sounds than assuming that it was somehow ignorant of them.

On the other hand, it is quite possible that the scribe of the Black Book of Chirk himself had imperfect knowledge of Welsh, or of Welsh orthography, seeing as how orthographical lenition was so consistently not modernized, even though other texts from the same period did modernize the orthography of lenition.



\subsection{The suddenness of changing orthographies}

\todo[inline]{This piece is little more than wild speculation, but may find some spot in the conclusion.}
The innovation of writing out lenition of voiceless stops around 1250 did not come alone. At around the same time, we see the spread of \graph{y} to relieve \graph{i} and \graph{e} of multiple phonological loads, \graph{w} stopped being used to denote /v/. In a similar vein, we see old irregular verbal inflections such as \mw[said]{amkawd}, \mw[did]{goruc} and \mw[heard]{kigleu} falling out of use\todo{Check dates of attestations}.

All these changes seemed to happen around a time of great societal unrest. Between 1277 and 1283, the Edwardian conquest of Wales took place. This period marked the defeat and annexation of the last native Welsh principalities and to their replacement with Marcher lords\todo{expand}.

Explaining linguistic innovation by means of societal developments 

It is not unheard of for a period of great social upheaval to also show a great linguistic upheaval. A well-known example is the rapid development of \gls{oir} from \gls{pi}. When comparing ogham inscriptions from the fourth century with sixth-century manuscript texts, one finds that the former type of texts looks more or less the same as how we reconstruct Proto-Celtic from a millennium earlier, while the Old Irish forms found two centuries later are still recognizable nowadays --- even more than a millenium later. The key lies in understanding that these two centuries saw an uprooting not in language, but in how language was represented in orthography:
\tqt{A colloquial language only becomes vulnerable to replacement if an entire society is uprooted or withers away, but an educated standard can be drastically altered or replaced through the far less disruptive replacement of an educated élite or even an élite's evolutionary reformation of itself (Cf.~Greene 1971). So, for example, a change of religion (or even the reformation of a continuing religion) can easily have the effect of removing an old sacred language from the curriculum, whether or not the pristhood is replaced, purged, or henceforth recruited from a different class.}{koch_conversion_1995}{47}

It is not difficult to assume a similar, although much less drastic upheaval of learned culture in late-thirteenth-century Wales. Before the Edwardian conquest, the Poets of the Princes had a fixed spot in Welsh courtly life, but after the uprooting of these structures by the conquest, the poets' positions were under threat. This may have provided an opportune moment to start representing linguistic innovations that were heretofore unrepresented in the conservative written standard\todo{Line of reasoning needs quite a bit of fine tuning. More info on the \mw{beirdd}, role of clergy, law in the period. Also: orthography was not wholly uprooted unlike Irish.}. 

Some quotations from Davies

\tqt{\begin{welsh}Ar lawer golwg, ni fu'r Goncwes yn andwyol i'r Eglwys ac ystyrir yr hanner canrif ar \^ol 1282 yn rhyw fath o oes aur yn ei hanes. Cafodd esgobion a oedd yn w\^yr galluog ac ymroddgar, y rhan fwyaf ohonynt yn Gymry neu'n ddynion a chysylltiadau Cymreig ganddynt. Cynhyrchwyd corff o lenyddiaeth grefyddol yn y Gymraeg, parhad o weithgarwch a oedd eisoes ar droed cyn y Goncwest. Tystia'r rhyddiaith a'r cerddi crefyddol yn \textit{Llyfr Gwyn Rhydderch} ac yn \textit{Llyfr yr Ancr} fod y Gymraeg, megis eraill o brif ieithoedd llenyddol Ewrop, yn datblygu'n `un o dafodieithoedd datguddiad Duw'. 
\end{welsh}
}{davies_hanes_1990}{167}

\tqt{The Cistercians also contributed handsomely to the literary culture of Wales. There has, it is true, been a tendency to ascribe all Welsh manuscripts indiscriminately to Cistercian \textit{scriptoria} and thereby to overlook the contribution of lay scribes. Thus it has recently been shown that the largest and single most valuable volume of Welsh medieval literature, \textit{The Red Book of Hergest}, which has been variously ascribed to the \textit{scriptoria} of Strata Florida and Neath abbeys, was in fact mainly written by a lay scribe from Builth for a lay patron. Yet this is not to gainsay the contribution of the Cistercians in literary matters. Much of their work was doubtless devotional, such as the concordance of St Bernard's Song of Songs compiled by an abbot of Margam, or the translation of the Athanasian creed into Welsh probably undertaken by a monk of Strata Florida. But the abbeys also became important centres for the conservation and transmission of secular learning. The library at Margam included copies of the works of Geoffrey of Monmouth and William of Malmesbury, while the survival of a single early Welsh verse or \textit{englyn} on the flap of a charter there possibly indicates an interest in vernacular literature also. Several Welsh literary manuscripts were probably copied at Cistercian abbeys: the include, for example, a Welsh version of the Charlemagne legend probably compiled at Strata florida and it was there also, around 1300, that the earliest most systematic and most comprehensive collection of medieval Welsh court poetry, now known as \textit{Llawysgrif Hendregadredd}, was assembled. Some of the survaving versions of Welsh law may also have been written and for monastic libraries. The Cistercian houses, from an early date, took upon themselves the role of compiling annals of events in Wales, thereby supplementing and extending the earlier annalistic tradition at St Davids and possibly at Llanbadarn. It was at Strata Florida that the lost Latin chronicle which forms the basis of the Chronicle of the Princes was compiled, in the late thirteenth century, and it was there also that at least one translation into Welsh, \textit{Brut y Tywysogyon}, was effected. But Strata Florida was not alone. Annals and chronicles were also composed at other Cistercian monasteries in Wales and some of them were circulated from one house to another. […]

[D]uring the later thirteenth century, it had increasingly to share its supremacy with the mendicant friars. In many ways Wales was and remained unpromising ground for the friars; it had neither the large urban centres nor the university schools which were their natural habitat. It is not surprising, therefore, that only three Franciscan houses […] were founded in Wales. Yet their impact was considerable. Already by the end of the thirteenth century they were outstripping the Cistercians in recruitment: there were twenty-three friars in the Dominican house at Rhuddlan. Thirty at Cardiff. […] By teaching and, above all, by preaching, the friars introduced new standards into the life of parish clergy and laity alike; they undertook the translation of manuals of instruction for use in parochial work; they used the vernacular to communicate the intensity of their religious sentiments and vision […]. }{davies_history_1987}{200--202}

% The Welsh law texts may help us in dating the transition between Early \gls{mw} showing no lenition of voiceless stops and later \gls{mw} showing lenition. If we accept Wiliam's~\parencite*[xx]{wiliam_llyfr_1960} date of the thirteenth century for BL Cotton Titus D. ii, then we must conclude that lenition of voiceless stops was represented by the thirteenth century.

% The orthographically more conservative first page of the Black Book of Chirk, however, shows that the archetype of these two manuscripts did not represent lenition of voiceless stops in its orthography. We must conclude from this that lenition of voiceless stops was not yet represented by the time it was written down. Perhaps the date of composition of the original Book of Iorwerth may serve as a \textit{terminus ante quem} for orthographical representation of lenition of voiceless stops.
% \newpage



%%% Local Variables:
%%% mode: latex
%%% TeX-master: "../main"
%%% End:
