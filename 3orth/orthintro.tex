\chapter{Introduction --- orthography}
\label{cha:intr-orth}
These are the goals of this chapter: 
\begin{itemize}
\item Argue that the \gls{mw} orthography of lenition is not internally consistent, just like how the \gls{ow} and \gls{mw} orthography are not consistent with each other.
  \begin{itemize}
  \item Exemplify this point by comparing BBC and H.
  \end{itemize}
\item Review some earlier literature in light of this new view on \gls{mw} lenition.
  \begin{itemize}
  \item Haycock on BT.
  \item Van Sluis on \ei, \oes.
  \item Evans' Grammar of Middle Welsh, who states that lenition is represented inconsistently in \gls{mw}, but without problematising it, or trying to find consistency in the inconsistency.
  \end{itemize}
\item Argue that orthographical and linguistic clues may be used to date texts, and present earlier literature that did just that.
  \begin{itemize}
  \item Willis
  \item PSW
  \item Nurmio
  \item Rodway
  \end{itemize}
\item Give an overview of the upcoming chapters, and note how they relate to one another.
\item Give some caveats on the methodology: counting instances where lenition is not written implies exhaustive knowledge of rules governing \gls{mw} lenition.
  \begin{itemize}
  \item In fact, some types of lenition are in flux. Such as postverbal lenition (MA thesis), and other types of free lenition (Schrijver).
  \end{itemize}
\end{itemize}
In Part~\todo{1}, I argued that \lT\ and \xD\ were phonologically distinct at least until and including when the Poets of the Princes were active. In this Part, I will argue that this phonological distinction had its impact in the orthography of lenition well into the Middle Welsh period. More specifically, I will demonstrate that lenition of \mw[]{p, t, c} was not written until about 1300. Thus, there existed a time from the Old Welsh period up until this point when lenition of most consonants other than voiceless stops wa written, but not of these voiceless stops.

Orthographical representation of lenition arose at the end of the \gls{ow} period. At this time time, the distinction between \lT\ and \xD\ was still maintained. As a result, scribes trying to represent the existence of three different stop series (\xT, \lT, and \xD) were in trouble, because the Latin alphabet  provides two, not three sets of stop consonants. They were thus unable to keep apart in writing all three stop series that Welsh had at the time%
\footnote{Similar trouble existed for fricatives, cf.\ \textcite[28]{russell_rowynniauc_2003}, who states that `[i]t has long been observed that, because of the rise of fricatives and spirants within the history of British, early Welsh was seriously understocked in signs to represent the full consonantal inventory.'}.

Word-initially, there was no natural association between \lT\ and \xD\ until the point when they merged phonologically, so the natural result of this phonology and the limitations of the Latin alphabet was to write lenited voiceless stops with \mw[]{p, t, c}. Thus, Early \gls{mw} orthography represented lenition only for other consonants than voiceless stops\footnote{Of course, lenition was not written for \mw{d} and \mw{rh} either, but orthographical representation for those consonants only became standard by the end of the \gls{mw} period.}.


%%SECTION?
One manuscript illustrating this limited orthographical system of lenition is found in MS \gls{sA}, the Black Book of Chirk, for which Example~\ref{ex:aylodeubrenyna} may serve as an example: 

\mwcc[ex:aylodeubrenyna]{\gls{sA}~4.9--10}{sef eu aylodeu e brenyn. \al{y u}eybyon ay neyeynt a\al{y k}euenderu.}{These are the members of the king: his sons and his nephews and his cousins.}

Here, lenition is represented following  \mw[his]{y} in \mw[sons]{ueybyon}, but not in \mw[cousin]{keuenderu}, even though \mw{y} must obviously be translated as `his' before both words. In this example, the parallelism makes it obvious that \mw[]{keuenderu} should be lenited even though there is no orthographical lenition to back this up.

The fact that lenited \mw[]{p, t, c} are written exactly the same as their radical counterparts makes it difficult to identify the exact geographical and chronological extent to which the non-merger of \xD\ and \lT\ existed. Usually, the only way to identify this non-merger is by seeing where in a \gls{mw} text we would expect lenition, but is not written. Naturally, these identifications require thorough knowledge on lenition in  the grammar of early \gls{mw}\footnote{I discuss my policy on where I may confidently expect lenition in Chapter~\todo[inline]{ref to chapter on lenition}.}. Fortunately, the early habit of not writing lenition of voiceless stops can be discerned even in the absence of such thorough knowledge in a few cases.

% \section{Two compositions of \mow{Marwnad Madog ap Maredudd}}
% \label{sec:two-exampl-mowm}
The basic fact that there was an early \gls{mw} period with orthographic lenition, but not of \lT, may be established even without knowledge of when exactly to expect lenition. This may be done on the basis of two manuscript copies of the elegy of Madog ap Maredudd. The opening lines of this poem as they are found in \gls{bbc}  and \gls{h}, respectively, are given in Example~\ref{ex:marwnadcomparison}.
\begin{mwl}
\item%
  \begin{minipage}{0.45\textwidth}
    \mw{%
      Kẏwarchaw im ri.\ rad wobeith.\\
      Kẏwarchaw kẏwercheiſ e \al{c}anweith.\\
      Ẏ \al{p}rowi prẏdv.\ o\abbr{m} priwieth eurgert.\\
      ẏm argluit \al{k}edẏmteith.\\
      Ẏ \al{c}vinav madauc.\ metweith ẏ alar\\
      ae alon ẏm pop ieith.\\
      Doꝛ yſgoꝛ ẏſcvid \al{c}anhimteith.}\\
    (\acrshort{bbc}~52v.3--7)
  \end{minipage}~
  \begin{minipage}{0.45\textwidth}
    \mw{%
      Kẏuarchaf ẏm ri rad o obeith.\\
      kẏuarchaf, kẏuercheis \al{g}anweith.\\
      ẏ \al{b}ꝛoui pꝛẏdu om pꝛifẏeith eurgert.\\
      ẏm arglwẏt \al{g}edymdeith.\\
      ẏ \al{G}wẏnaỽ madaỽc metueith.\ ẏ alar\\
      ae alon ẏm pob ẏeith\\
      Doꝛ ẏſgoꝛ ẏſgwẏd \al{g}anhẏmdeith.}\\
    (\acrshort{h}~47v.8--13)
  \end{minipage}
  \label{ex:marwnadcomparison}
\end{mwl}
Here, we see that \gls{bbc} does write lenition to some extent, \eg in \mw[ mourning him]{ẏ alar}, but wherever \gls{h} writes lenition of \mw[]{p, t, c}, \gls{bbc} preserves the radical. I have marked these instances.

Incidentally, 



from about 1250~\autocite[xxiv]{jones_rhagymadrodd_1982}
 from about 1300~\autocite{huws_llawysgrif_1981}

The manuscript is thought to have been compiled around 1250\todo{ref to Huws' repertory}. This assertion will be demonstrated on the basis of lenition seen in several poems found in this book.

Additionally, the poem \mw{Marunad Madauc Fil' Maredut}~(MMFM) has been chosen to demonstrate the same principle of non-representation of lenition of voiceless stops. This poem has been chosen because it is an elegy mourning a known person: Madog ap Maredudd died in 1160~\autocite[82]{jones_gwaith_1991}. Additionally, we know this poem was written by Cynddelw Brydydd Mawr, who was active in the late twelfth century~\autocite[xxx]{jones_gwaith_1991}.


Analysis of the poem is helped by the existence of the very same poem in a similarly old manuscript, but which shows lenition of voiceless stops for the most part, \ie \gls{h}. This manuscript dates from about 1300. By way of illustration, Example~\ref{ex:marwnadcomparison} shows first five lines of this poem as found in both manuscripts, and easily demonstrates the difference in orthography. Since the Hendregadredd version does typically write lenition of voiceless stops, and forms the basis of the analysis by \textcite[82--91]{jones_gwaith_1991}, it is easy to make sense of all the grammatical preconditions for lenition as found in Table \ref{lenitionmmfm}. Line numbers refer to \textcite[78-79]{jarman_llyfr_1982}, but also generally agree with \textcite[82--91]{jones_gwaith_1991} for the Hendregadredd version.

Table \ref{lenitionmmfm} shows that voiceless stops are not lenited as a rule, while other consonants are typically lenited. Furthermore, there are cases where it is not clear why lenition was written. In the table, these instances are marked with `?', optionally followed by a suggestion why it may be lenited. In all of these cases, lenition is confirmed by the \gls{h} recension. Lenition is confirmed by \gls{h} in these cases, so I will not delve into the why of these lenitions further. However, the case of \mw{kedymteith} `companion' in line 4 may prove interesting. It is written \mw{gedymdeith} in \gls{h}, but it is not clear what grammatical reason there is for lenition. Lenition in \mw{gedymdeith} may therefore be regarded as a hypercorrect insertion of orthographical representation for lenition of voiceless stops. This in turn suggests that orthographical representation of lenition of voiceless stops was indeed an innovation which occurred after the death of Madog ap Maredudd. 

Despite the general pattern of non-lenition of voiceless stops, and lenition of other consonants, there are several exceptions to this rule.
These exceptions show lenition of voiceless stops in \mw{gar}~(l.~22) and \mw{dan}~(l.~23), as well as non-lenition of other consonants in \mw{lledieith}~(l.~17) and \mw{madauc}~(l.~28).
Line 22, in which \mw{gar} `loves' is found, is written in the margin of the page on which it is found.
This might point to a later insertion, but the insertion is written by the same hand.
This leaves this matter unsolved. The other lenited voiceless stop, in \mw{dan} `under', falls under \gls{petr}, which is already seen to be represented in this chapter.
The words \mw{lledieith} `foreign' and \mw{madauc} `Madog' are both used as epithets, which are known to lenite, but this is not done consistently\todo{I should refer to my results from CO here, which may or may not end up in my final thesis.}.

\begin{table}[h]
\centering
\begin{tabular}{@{}lllll@{}}
\toprule
\textbf{Line} & \textbf{Word} & \textbf{Translation} & \textbf{Reason for lenition} & \textbf{Represented} \\ \midrule
1 & \textit{wobeith} & `hope' & follows \mw{o} found in \gls{h} & yes \\
2 & \textit{canweith} & `hundred times' & adverbial clause & no \\
3 & \textit{prowi} & `prove' & follows \mw{y} `to' & no \\
3 & \textit{priwieith} & `best language' & follows \mw{o} & no \\
4 & \textit{kedymteith} & `companion' & ?dvandva compound & no \\
5 & \textit{cvinav} & `mourn' & follows \mw{y} `to' & no \\
5 & \textit{alar} & `lament' & follows \mw{y} `his' & yes \\
6 & \textit{alon} & `enemies' & follows \mw{y} `his' & yes \\
7 & \textit{canhimteith} & `escort' & follows feminine \mw{yscvid} & no \\
11 & \textit{wobeith} & `hope' & ? & yes \\
12 & \textit{kedimteith} & `companion' & follows preposed adjective & no \\
13 & \textit{leith} & `death' & follows \mw{no'e} `than his' & yes \\
16 & \textit{wisscoet} & `vestments' & follows preposed adjective & yes \\
16 & \textit{wessgvin} & `Gascon horse' & ? & yes \\
16 & \textit{canhimteith} & `escort' & follows preposed genitive & no \\
17 & \textit{vab} & `son' & epithet following PN & yes \\
17 & \textit{lledieith} & `foreign' & epithet following PN & no \\
18 & \textit{clod} & `fame' & follows \mw{y} `his' & no \\
19 & \textit{vaon} & `subjects' & follows preposed adjective & yes \\
19 & \textit{oleith} & `retreats' & follows VP \mw{ni} & yes \\
20 & \textit{wastad} & `constant' & follows \mw{rhad} & yes \\
20 & \textit{canhimteith} & `escort' & follows preposed genitive & no \\
22 & \textit{gar} & `loves' & follows VP \mw{a} & yes \\
22 & \textit{kidweith} & `joint work' & follows \mw{o} & no \\
23 & \textit{dan} & `under' & petrified lenition & yes \\
23 & \textit{calchwreith} & `vari-coloured' & follows feminine \mw{yscvd} & no \\
25 & \textit{owin} & `wish' & follows preposed genitive & yes \\
26 & \textit{pedeirieith} & `four languages' & follows feminine \mw{yscvid} & no \\
27 & \textit{teirn} & `king' & follows feminine \mw{hil} & no \\
28 & \textit{madauc} & (personal name) & epithet following \mw{hael} & no \\
28 & \textit{veuder} & `dread' & ? & yes \\
29 & \textit{leith} & `death' & follows \mw{o'e} `of his' & yes \\
30 & \textit{kedymteith} & `companion' & follows \mw{darw} `ended' & no \\
33 & \textit{truited} & `welcome' & follows \mw{y} `his' & no \\
34 & \textit{kywarweith} & `fight' & follows \mw{o'e} `of his' & no \\
36 & \textit{kadieith} & (personal name) & follows feminine \mw{aerllin} & no \\
37 & \textit{kadarn} & `strong' & follows preposed adjective & no \\
37 & \textit{kedymdemteith} & `companion' & follows preposed adjective & no \\
38 & \textit{talheith} & `crown' & follows \mw{y} `his' & no \\
39 & \textit{leith} & `death' & follows \mw{y} `his' & yes \\ \bottomrule
\end{tabular}
\caption{Representation of lenition in \mw{Marunad Madauc Fil' Maredut} as found in the Black Book of Carmarthen}
\label{lenitionmmfm}
\end{table}

\section{Final remarks}
\todo[inline]{These remarks should come at the end of the introduction to the orthography. These points roughly correspond to chapters which can be referred to.}
If we accept the following premises:
\begin{enumerate}
\item \gls{ow} did not represent lenition orthographically;
\item Early \gls{mw} did not represent lenition of voiceless stops orthographically;
\item Later \gls{mw} did represent lenition orthographically;
\item When a text with older orthography is copied, lenition is typically only modernised where it is vital to represent or disambiguate grammatical categories;
\item Lenition following verbs in Early \gls{mw} did not represent or disambiguate verbal categories;
\end{enumerate}

Then we must conclude the following:
\begin{enumerate}
\item In a text where postverbal lenition is represented orthographically, but not for voiceless stops, we may postulate a \textit{terminus ante quem} as well as a \textit{terminus post quem} for its original composition: it must originally have been composed before lenition of voiceless stops was represented orthographically, and after lenition started being represented at all.
\end{enumerate}

This conclusion necessarily follows from the premises above: if lenition following verbs was not modernised orthographically, then this must have been the case equally when copying from an \gls{ow} text and when copying from an Early \gls{mw} text. If, hypothetically speaking, an \gls{ow} text were copied into Later \gls{mw}, then we should not expect non-lenition of voiceless stops only. Rather, we should expect non-lenition of any consonant following verbs, because premise (4) above is as much applicable to premise (1) as it is to premise (2).


%%% Local Variables:
%%% mode: latex
%%% TeX-master: "../main"
%%% coding: utf-8
%%% End:
