\chapter{Introduction --- orthography}
\label{cha:intr-orth}
These are the goals of this chapter: 
\begin{itemize}
\item Argue that the \gls{mw} orthography of lenition is not internally consistent, just like how the \gls{ow} and \gls{mw} orthography are not consistent with each other.
  \begin{itemize}
  \item Exemplify this point by comparing BBC and H.
  \end{itemize}
\item Review some earlier literature in light of this new view on \gls{mw} lenition.
  \begin{itemize}
  \item Haycock on BT.
  \item Van Sluis on \ei, \oes.
  \item Evans' Grammar of Middle Welsh, who states that lenition is represented inconsistently in \gls{mw}, but without problematising it, or trying to find consistency in the inconsistency.
  \end{itemize}
\item Argue that orthographical and linguistic clues may be used to date texts, and present earlier literature that did just that.
  \begin{itemize}
  \item Willis
  \item PSW
  \item Nurmio
  \item Rodway
  \end{itemize}
\item Give an overview of the upcoming chapters, and note how they relate to one another.
\item Give some caveats on the methodology: counting instances where lenition is not written implies exhaustive knowledge of rules governing \gls{mw} lenition.
  \begin{itemize}
  \item In fact, some types of lenition are in flux. Such as postverbal lenition (MA thesis), and other types of free lenition (Schrijver).
  \end{itemize}
\end{itemize}
In Part~\todo{1}, I argued that \lT\ and \xD\ were phonologically distinct at least until and including when the Poets of the Princes were active. In this Part, I will argue that this phonological distinction had its impact in the orthography of lenition well into the Middle Welsh period. More specifically, I will demonstrate that lenition of \mw[]{p, t, c} was not written until about 1300. Thus, there existed a time from the Old Welsh period up until this point when lenition of most consonants other than voiceless stops wa written, but not of these voiceless stops.

Orthographical representation of lenition arose at the end of the \gls{ow} period. At this time time, the distinction between \lT\ and \xD\ was still maintained. As a result, scribes trying to represent the existence of three different stop series (\xT, \lT, and \xD) were in trouble, because the Latin alphabet  provides two, not three sets of stop consonants. They were thus unable to keep apart in writing all three stop series that Welsh had at the time%
\footnote{Similar trouble existed for fricatives, cf.\ \textcite[28]{russell_rowynniauc_2003}, who states that `[i]t has long been observed that, because of the rise of fricatives and spirants within the history of British, early Welsh was seriously understocked in signs to represent the full consonantal inventory.'}.

Word-initially, there was no natural association between \lT\ and \xD\ until the point when they merged phonologically, so the natural result of this phonology and the limitations of the Latin alphabet was to write lenited voiceless stops with \mw[]{p, t, c}. Thus, Early \gls{mw} orthography represented lenition only for other consonants than voiceless stops\footnote{Of course, lenition was not written for \mw{d} and \mw{rh} either, but orthographical representation for those consonants only became standard by the end of the \gls{mw} period.}.


\section{Two compositions of \mow{Marwnad Madog ap Maredudd}}
\label{sec:two-exampl-mowm}
One manuscript illustrating this limited orthographical system of lenition is found in MS \gls{sA}, the Black Book of Chirk, for which Example~\ref{ex:aylodeubrenyna} may serve as an example: 

\mwcc[ex:aylodeubrenyna]{\gls{sA}~4.9--10}{sef eu aylodeu e brenyn. \al{y u}eybyon ay neyeynt a\al{y k}euenderu.}{These are the members of the king: his sons and his nephews and his cousins.}

Here, lenition is represented following  \mw[his]{y} in \mw[sons]{ueybyon}, but not in \mw[cousin]{keuenderu}, even though \mw{y} must obviously be translated as `his' before both words. In this example, the parallelism makes it obvious that \mw[]{keuenderu} should be lenited even though there is no orthographical lenition to back this up.

The fact that lenited \mw[]{p, t, c} are written exactly the same as their radical counterparts makes it difficult to identify the exact geographical and chronological extent to which the non-merger of \xD\ and \lT\ existed. Usually, the only way to identify this non-merger is by seeing where in a \gls{mw} text we would expect lenition, but is not written. Naturally, these identifications require thorough knowledge on lenition in  the grammar of early \gls{mw}\footnote{I discuss my policy on where I may confidently expect lenition in Chapter~\todo[inline]{ref to chapter on lenition}.}. Fortunately, the early habit of not writing lenition of voiceless stops can be discerned even in the absence of such thorough knowledge in a few cases.


The basic fact that there was an early \gls{mw} period with orthographic lenition, but not of \lT, may be established even without knowledge of when exactly to expect lenition. This may be done on the basis of two manuscript copies of the elegy of Madog ap Maredudd. The opening lines of this poem as they are found in \gls{bbc}  and \gls{h}, respectively, are given in Example~\ref{ex:marwnadcomparison}.
\begin{mwl}
\item%
  \begin{minipage}{0.45\textwidth}
    \mw{%
      Kẏwarchaw im ri.\ rad wobeith.\\
      Kẏwarchaw kẏwercheiſ e \al{c}anweith.\\
      Ẏ \al{p}rowi prẏdv.\ o\abbr{m} priwieth eurgert.\\
      ẏm argluit \al{k}edẏmteith.\\
      Ẏ \al{c}vinav madauc.\ metweith ẏ alar\\
      ae alon ẏm pop ieith.\\
      Doꝛ yſgoꝛ ẏſcvid \al{c}anhimteith.}\\
    (\acrshort{bbc}~52v.3--7)
  \end{minipage}~
  \begin{minipage}{0.45\textwidth}
    \mw{%
      Kẏuarchaf ẏm ri rad o obeith.\\
      kẏuarchaf, kẏuercheis \al{g}anweith.\\
      ẏ \al{b}ꝛoui pꝛẏdu om pꝛifẏeith eurgert.\\
      ẏm arglwẏt \al{g}edymdeith.\\
      ẏ \al{G}wẏnaỽ madaỽc metueith.\ ẏ alar\\
      ae alon ẏm pob ẏeith\\
      Doꝛ ẏſgoꝛ ẏſgwẏd \al{g}anhẏmdeith.}\\
    (\acrshort{h}~47v.8--13)
  \end{minipage}
  \label{ex:marwnadcomparison}
\end{mwl}
Here, we see that \gls{bbc} does write lenition to some extent, \eg in \mw[ mourning him]{ẏ alar}, but wherever \gls{h} writes lenition of \mw[]{p, t, c}, \gls{bbc} preserves the radical. I have marked these instances.

Both manuscripts are datable. \Gls{bbc} dates from about 1250~\autocite[xxiv]{jones_rhagymadrodd_1982}, while \gls{h} dates from about 1300~\autocite{huws_llawysgrif_1981}. Additionally, the original composition of the poem is datable, because it mourns the death of a known person. It must thus have been written shortly after Madog ap Maredudd's death in 1160~\autocite[82]{jones_gwaith_1991}. Additionally, we know this poem was written by Cynddelw Brydydd Mawr, who was active in the late twelfth century~\autocite[xxx]{jones_gwaith_1991}. From these dates we may conclude that lenition of \mw[]{p, t, c} was  not written in 1160, and this pattern did not need to be updated by 1250. By 1300, lenition of voiceless stops could be written for all consonants.

lenition is written in \gls{bbc} in a handful of instances:
\begin{mwl}
  \mwc[ex:bbcdan]{\gls{bbc}~53r.4--5}{Llav eſcud. \al{dan} iſcud calchwreith.}{}
  \mwc[ex:bbcagar]{\gls{bbc}~53r.4 (margin)}{llawin gviar \al{a gar}.\ o kidweith.}{}
\end{mwl}
Here, Example~\ref{ex:bbcdan} contains an instance of \mw[under]{dan}, which is lenited because it is a reduced clitic. In Chapter~\ref{cha:some-phon-issu} I argue that such clitic reduction is not in fact lenition, and Chapter~\ref{cha:indep-comp-mwbr} demonstrates that these reduced clitics as well as instances of petrified lenition are represented with \mw[]{b, d, g} from an earlier date onwards than morphophonemically lenited voiceless stops. It is therefore no surprise that \mw[under]{dan} is written the way it is here. Example~\ref{ex:bbcagar} has \mw[loves]{a gar}, and is a genuine instance of morphophonemic lenition. It is written in the margin, which may point to a later insertion, but the insertion is written by the same hand, so it cannot have been much later. This one exception shows that scribes were aware of the possibility to represent lenited voiceless stops with \mw[]{b, d, g}, but that they chose not to do so\footnote{Section~\ref{sec:lenited-mwg} gives an insight into why this might have been the case.}.

\todo[inline]{Some remarks that the earliest orthographical lenition of voiceless stops is found in braint teilo, albeit irregularly.}

\section{Earlier literature}
\label{sec:earlier-literature}
Earlier scholars mentioned that lenition is represented inconsistently, and some even noted that words beginning in specific consonants are often not lenited. However, none of these have problematised this issue.

\Textcite[\S 18]{evans_grammar_1964} notes that `[i]n MW orthography lenition is not regularly indicated.' He also notes the same for nasalisation, but not for spirantisation~\autocite[\S\S 24--25]{evans_grammar_1964}. This is unexpected: given how lenition is  more common and serves to disambiguate more meanings than either nasalisation or spirantisation, we would expect lenition to be represented the most frequently of the three. Some examples of lenition given by Evans have counterexamples starting in voiceless stops. For example, \mw[Pwyll Prince of Dyfed]{Pwyll \al{P}endeuic Dyuet} is given as a counterexample to the rule that a noun in apposition to a personal name is lenited~\autocite[\S 19]{evans_grammar_1964}. However, this example may also precede orthographic lenition of voiceless stops\todo{early text in late MS}. Evans does attempt to find some amount of consistency in the inconsistency with which lenition is represented:
\tqt{A tenuis often remains unchanged after \mw[]{'th}: \mw[and thy sword]{a'th \al{c}ledeu} [\dots], \mw[to thy castle]{y'th \al{c}astell}. [\dots] Unvoicing of a media after a vocieless spirant is quite common in MW; note the following examples: \mw[thy body]{dy gorff \al{t}i} [\dots], \mw[all the chains]{yr holl \al{k}adwyneu}.}{evans_grammar_1964}{\S 20N}
Here, Evans may be quite right that \mw[]{'th} and the other voiceless spirants may undo lenition of voiceless stops, but they may also simply be traces of an earlier, more general situation where lenition of voiceless stop was not written\footnote{Chapter~\ref{cha:welsh-laws} and Chapter~\ref{cha:stemm-mwbuch-dewi} give a more detailed discussion of how the orthography of an early text may leave traces in later manuscripts.}.

The prefaces of textual editions typically give an overview of the orthography used in these texts. The \gls{ll1} manuscript, containing \mw[]{Brut y Brenhinedd}, forms the basis of an edition
by \textcite{roberts_brut_1971}, which I will take as an example. The orthography of lenition in this manuscript is analysed in Chapter~\ref{cha:indep-comp-mwbr}, and it is just like the Black Book of Carmarthen in that lenition of \mw[]{p, t, c} is not represented. Roberts notes the following on the
orthography of stops and on lenition, respectively:
\tqt{Initially [b, d, g] are always denoted by \textit{b, d, g} ;
  medially they are usually represented by \textit{b, d, g}, with some
  examples of \textit{-p-, -t-, -k-}. Finally, [d] is always
  represented by \textit{t}, [g] by \textit{c} with the exception
  of \textit{og}, but final [b] is represented sometimes
  by \textit{p}, and sometimes by \textit{b, pob, pab, escyb}.  The
  unvoiced stops [p, t, k] occur initially and are written \textit{p,
    t, c/k}. \textit{c-} does not occur often and the scribe prefers
  to use \textit{k-}.  The convention of using \textit{k}
  with \textit{y, i,} or \textit{e} and \textit{c} with other vowels
  and consonants [\dots] does not seem to have been followed and the
  same word may appear with initial \textit{c-} or \textit{k-}.
  As \textit{b, d, g} medially denote [b, d, g] the corresponding
  unvoiced stops can be written \textit{p, t, tt, k, kc}.
}{roberts_brut_1971}{xli}
There is no mention of how lenition is written under the discussion of the orthography of stops. 
\tqt{Lenition of initial consonants is not always shown in the text
  but the scribe almost invariably denotes the spirant and nasal
  mutations }{roberts_brut_1971}{xlii}
Roberts, like Evans, is able to tell that lenition is represented irregularly, and that this is unlike the spirant and nasal mutations. He similarly does not mention how problematic this difference is.

Not only did this generation of scholars fail to find a pattern in when lenition was and was not represented, but they also failed to see the need to find such a pattern at all, \ie they not only failed to find the solution to the topic of this thesis, but they even failed to find the topic at all.

Haycock, in her edition of the legendary poems in the Book of Taliesin goes a bit further and notices the following: `Lenition of initial p, t (and d) are not generally realized'~\autocite[p.~7, n.~18]{haycock_legendary_2015}. This is to my knowledge the first instance where a previous author saw that representation of lenition depended on whether the initial consonant is a voiceless stop\footnote{Here, lenited \mw[]{c} was already written, but not \mw[]{p, t}. This pattern is found more generally in the late thirteenth century.}. She does not discuss this pattern any further. \todo{If I do not include Scribe X86 as a separate chapter, I should expand on how this scribe has this orthography all around, and how he is fairly late with this orthography.}

The verbal endings \ei\ and \oes\ are noticed by~\textcite{van_development14} to cause lenition to immediately subjects and objects alike, but not to \mw[]{p, t, c}. This behaviour contrasts with other types of postverbal lenition, such as object lenition and \gls{np} lenition. This pattern is observed in some of the earliest Mabinogion texts, \ie \mow{Culhwch ac Olwen} and \mow{Pwyll Pendefig Dyfed}. These texts are found in the \gls{wbr}, which dates from the early fourteenth century~\todo{huws}, but are thought to have been composed a few centuries earlier~\todo{rodway}. Because non-lenition of \mw{p, t, c} is found following postverbal contact lenition, but not following object lenition or \gls{np} lenition, \textcite{van_development14} held this non-lenition to be a specific feature of \ei\ and \oes\ only. In light of the difference in orthography between \gls{bbc} and \gls{h}, however, it may make more sense to think of non-lenition of \mw[]{p, t, c} following \ei\ and \oes\ as an older orthographical stratum, and to think of object and \gls{np} lenition as later innovations to the text\footnote{In fact, \textcite{van_development14} describes how object lenition and \gls{np} lenition gained ground in the \gls{mw} period. Chapter~\ref{cha:welsh-laws} explores how older orthographical strata may be transmitted in later \gls{mw}.}.

\section{Lenited voiceless stops as a dating criterion}
\label{sec:lenit-voic-stops-1}
Any student of Welsh quickly learns to recognise the difference between \gls{ow}, \gls{mw} and \gls{mow}. \Gls{ow} has no orthographic lenition even though we know it existed in the spoken language, while \gls{mw} does have orthographic lenition, except for \mw{d, rh}. Thus, a student will easily be able to identify just on the basis of lenition whether any text before him dates from the \gls{ow} period (9th--11th centuries), the \gls{mw} period (12th--14th centuries), or the \gls{mow} period (15th century onwards)\todo{think through these dates for periods some more}.


\section{Final remarks}
\todo[inline]{These remarks should come at the end of the introduction to the orthography. These points roughly correspond to chapters which can be referred to.}
If we accept the following premises:
\begin{enumerate}
\item \gls{ow} did not represent lenition orthographically;
\item Early \gls{mw} did not represent lenition of voiceless stops orthographically;
\item Later \gls{mw} did represent lenition orthographically;
\item When a text with older orthography is copied, lenition is typically only modernised where it is vital to represent or disambiguate grammatical categories;
\item Lenition following verbs in Early \gls{mw} did not represent or disambiguate verbal categories;
\end{enumerate}

Then we must conclude the following:
\begin{enumerate}
\item In a text where postverbal lenition is represented orthographically, but not for voiceless stops, we may postulate a \textit{terminus ante quem} as well as a \textit{terminus post quem} for its original composition: it must originally have been composed before lenition of voiceless stops was represented orthographically, and after lenition started being represented at all.
\end{enumerate}

This conclusion necessarily follows from the premises above: if lenition following verbs was not modernised orthographically, then this must have been the case equally when copying from an \gls{ow} text and when copying from an Early \gls{mw} text. If, hypothetically speaking, an \gls{ow} text were copied into Later \gls{mw}, then we should not expect non-lenition of voiceless stops only. Rather, we should expect non-lenition of any consonant following verbs, because premise (4) above is as much applicable to premise (1) as it is to premise (2).


%%% Local Variables:
%%% mode: latex
%%% TeX-master: "../main"
%%% coding: utf-8
%%% End:
