\chapter{Introduction --- orthography}
\label{cha:intr-orth}

 At  the start of the \gls{mw} period, orthographical representation of lenition arose. If, at the same time, the distinction between \lT\ and \xD\ was maintained into the \gls{mw} period, then this must have caused trouble for scribes trying to represent the existence of three different stop series: \xT, \lT, and \xD. The Latin alphabet does not provide a third set of stops, so they necessarily merged orthographically with their unlenited counterparts\footnote{Similar trouble existed for fricatives, cf.\ \textcite[28]{russell_rowynniauc_2003}, who states that `[i]t has long been observed that, because of the rise of fricatives and spirants within the history of British, early Welsh was seriously understocked in signs to represent the full consonantal inventory.' The significance of the rise of these fricatives and spirants is that the limited Latin set of stops had to be used not only to denote the presumably larger Welsh stop inventory, but also had to provide graphemes for sounds such as \lT. This competition further narrowed chances of the opposition between lenited voiceless and unlenited voiced stops to be represented in writing.}. 

As a result of the new-found habit to distinguish lenition combined with the inability of the Latin alphabet to represent such a wide stop inventory, Early \gls{mw} orthography represented lenition only for other consonants than voiceless stops\footnote{Although lenition was not written for \mw{d} and \mw{rh} either, but orthographical representation for those consonants only became standard by the end of the \gls{mw} period.}. This chapter treats the extent to which orthographical lenition is found, but not for voiceless stops. This limited orthographical system for lenition is indeed found in the \gls{bbch}. I will illustrate the pattern of lenition found in this manuscript with the following example: 

\mwcc[peniarth29]{\acrshort{bbch},~p.~4,~ll.~9--10}{sef eu aylodeu e brenyn. y ueybyon ay neyeynt ay keuenderu.}{These are the members of the king: his sons and his nephews and his cousins.}

Example \ref{peniarth29} shows lenition after \mw{y} `his' in \mw{ueybyon} `sons', but not in \mw{keuenderu} `cousins', even though \mw{y} must also be translated as `his' before this word. Obviously, then, non-lenition of voiceless stops only after \ei\ and \oes\ is found here, but this fact is unremarkable given the wholesale lack of representation of lenition of voiceless stops in this manuscript. 

 The early three-way distinction between stops as proposed by Koch explains why lenition of voiceless stop \graph{k} is not represented orthographically in Example \ref{peniarth29}, while \graph{m} is. However, there are a handful isolated instances where lenited voiceless stops are written with \graph{b, d, g}. The exact extent to which lenition was represented orthographically for different consonants deserves investigation. The results of this investigation may provide insight into whether non-lenition of voiceless stops after verbs was a matter of orthography, or really represented a system of limited postverbal lenition in Early \gls{mw}.

The orthography of the Black Book of Chirk suggests orthography may after all play a role in explaining why voiceless stops are sometimes not written lenited after postverbal lenition. It is possible that lack of writing of lenition of voiceless stops after verbs was maintained when copying from an older source, while the writing of lenition was innovated in (most) other contexts. This pattern would be explainable: lenition has no function in distinguishing meaning after \ei\ and \oes, but it does have such a function after \mw{y} `his', as seen in Example \ref{peniarth29}. For this reason, the orthography of lenition stands a larger chance of being modernised in the latter case than in the former case. If this is true, limited writing of postverbal lenition may not represent any stage of \gls{mw}, but may instead give clues as to the age of a particular text before being written down in its final form, and may give insight into the process of scribal modernisation.

 If Peniarth 29 demonstrates that non-lenition of voiceless stops after verbs is indeed a matter of orthography, then the unique survival of this pattern after verbs only in later \gls{mw} implies that this is the result of imperfect modernisation of an earlier source when the scribe copied a text. This necessarily implies that there was a scribe using a textual source from before orthographical lenition of voiceless stops at some point in the transmission of the text. In this case, non-lenition of voiceless stops after verbs may provide a \textit{terminus ante quem} for its original composition, if we can date the start of the period when lenited voiceless stops were represented with \graph{b, d, g} rather than their voiceless counterparts. In this case, the differing treatment of voiceless stops after verbs in \mw{Pwyll Pendeuic Dyuet} and \mw{Culhwch ac Olwen} compared to the other three branches of the Mabinogi would imply that the former two would be copied from a text written before the orthographical merger of lenited voiceless stops and unlenited voiced stops, while the other three branches would be copied from a text written after this merger. 
 
 The reason why the orthography of lenition would not be modernised in this context needs to be investigated, but may be related to the lack of grammatical saliency of postverbal lenition: in Early \gls{mw}, postverbal lenition did not serve to disambiguate meaning (as opposed to words like \mw{y}, which may either mean `the', `his', and `her', among others, and meaning may be disambiguated on the basis of the mutation it causes)\todo{Paragraph needs to be moved to chapter about postverbal lenition}. 
 
 Another question is from what point onwards the phonemic merger of lenited voiceless stops and unlenited voiced stops took place, and whether it coincided with the orthographical merger. Clues may be taken from alliteration patterns in Early \gls{mw} poetry. In the the earliest \gls{mw} poetry, lenited consonants could alliterate with their unlenited counterparts. It stands to reason that this possibility to alliterate lenited and unlenited consonants lasted up until roughly the end of the period when lenited voiceless stops and unlenited voiced stops merged phonologically. If this assumption is true, then changes in rules governing alliteration may pinpoint changes in the phonology of lenition in \gls{mw}. This assumption may be wrong, though: aspirated and nasalised consonants also alliterated with their radical counterparts, and nasalised consonants could also alliterate with their respective radical counterparts. This suggests that the Early \gls{mw} listener to poetry would judge correct alliteration more on the radical lexemes of the alliterating words than on the actual sounds uttered~\parencite[339]{rowland_early_1990}. This would be a learned feature not influenced by the exact phonological or phonetic reality of Welsh.
 
% If, however, non-lenition of voiceless stops after verbs was a feature of postverbal lenition, then a voiceless stop following a verbal ending must have stayed unlenited phonetically, i.e.\ their value was [p\textsuperscript{h}-, k\textsuperscript{h}-, t\textsuperscript{h}-], not [\bd, \gd, \dd]. Otherwise, non-lenition of voiceless stops would not have stood out after verbs in texts such as the White Book recension of \mw{Pwyll Pendeuic Dyuet}.
 
 \section{Potentially new research questions}
 \begin{itemize}
 \item How is lenition represented in the orthography of Peniarth 29?
 \begin{itemize}
 \item To what extent exactly is lenition written for voiceless stops, and how does this compare to other consonants?
 \item Does lenition after \ei\ and \oes\ show the same distribution as elsewhere?
 \end{itemize}
 \item If non-lenition of voiceless stops after verbs is purely a matter of archaic orthography, then what was the (both relative and absolute) chronology of orthographical lenition?
 \begin{itemize}
 \item When did lenited voiceless stops start to be written with \graph{b, d, g}?
 \item How may this be used to elucidate upon the history of the Four Branches of the Mabinogi? \item Does this mean \mw{Pwyll Pendeuic Dyuet} was copied from a different (earlier) source than the other tales? 
 \item Does the orthographical merger coincide with the phonological merger of lenited voiceless stops and unlenited voiced stops?
 \end{itemize}
\end{itemize}

\todo[inline]{All of this above can serve as a skeleton for a new introduction on orthgraphy}


 The orthography of the Black Book of Chirk has already been studied by Russell~\parencite*{Rus_Scribal95}. In this article, he notes the following on voiceless stops: \tqt{Latin \textit{t} and \textit{c} had been used in British
before lenition, and after lenition they were used to represent internal and
final /d/ and /ɡ/, later to be replaced by \textit{d} and \textit{g}. In BBCh, \textit{t} and \textit{c} are
still being used for the voiced stops; it is, therefore, worth looking at how
secondary /t/ and /k/ were represented in the absence of \textit{t} and\textit{ c}.}{Rus_Scribal95}{135} In his article, Russell does not make explicit what he means by internal and final /d/ and /ɡ/, i.e.\ whether he considers lenited voiceless stops belonging to this set, or also unlenited voiced stops (e.g. historical geminates in medial or final position).

He also made the following comment: \tqt{The voiced fricative /ð/ arose through the lenition of /d/, and
\gls{ow} orthography did not reflect the change in pronunciation; thus,
before lenition \textit{d} in internal position represented /d/ and \textit{t} represented /t/, but after lenition \textit{d} stood for /ð/ and \textit{t} for /d/. In initial position, however, \textit{t} and \textit{d} still represented /t/ and /d/ respectively. But within \gls{mw} there was a gradual shift to a system where \textit{t} represented /t/ and \textit{d} /d/ in all positions. The pressure would have increased with the rise of secondary \textit{t} from /tt/, etc. in internal and final position, since this secondary \textit{t} could not be spelt \textit{t} if represented /d/. Hence the various attempts to represent /t/ by \textit{tt} and \textit{th}, etc.\ in early \gls{mw}. [\dots]

Initial mutations were vital to preserve the marking of grammatical categories, and so it is reasonable to
suppose that the pressure from the initial spellings of for /t/ and \textit{d} for
/d/ led to a gradual regularization in their favour, i.e.\ \textit{t} for /t/ and \textit{d} for
/d/ in all positions, and from early \gls{mw} onwards the rise of a new
spelling for /ð/, namely \textit{dd}. In \gls{ow}, and apparently also in BBCh, there
is some indication of variation but the major shifts have not yet taken place.}{Rus_Scribal95}{142} I understand this statement to be applicable mainly to the later and more standardized \gls{mw} orthography, as it is not applicable to the manuscript's first page, which shows that exactly this development did not yet take place word-initially. As a result, marking of grammatical categories was not preserved. His statement does have value in a more general sense: initial mutations were vital to mark grammatical categories, and marking of grammatical categories is vital for understanding the contents of a text. Because of this, instances of failure to mark lenition must be regarded as an archaism, not only because we know that this was the \gls{ow} system, but also because this is the \textit{lectio difficilior}.

An interesting point made by Russell is that the rise of medial and final /t/ in words such as \mw{eto} may have had decisive influence in spelling medial/final /d/ with \mw{d} rather than OW \mw{t}.

\section{The Black Book of Carmarthen}
\todo[inline]{BBC will serve to give a foretaste of what the Part is about.}
The Black Book of Carmarthen shows a pattern of lenition that is broadly similar to the Black Book of Chirk. The manuscript is thought to have been compiled around 1250~\autocite[xxiv]{jones_rhagymadrodd_1982}. This assertion will be demonstrated on the basis of lenition seen in several poems found in this book.
\subsection{\mw{Ymddiddan Myrddin a Thaliesin}}
The first poem in this book, \mw{Ymddiddan Myrddin a Thaliesin}~(YMATh), illustrates this statement. Table \ref{lenitionymath} shows every word in YMATh which either shows lenition, or would be expected to show lenition in \gls{mw}. Line numbers refer to the line numbers as found in \textcite{jarman_llyfr_1982}. The table excludes words whose radical starts with \mw{r} or \mw{d}, because no orthographical means to represent their lenition was found for these sounds, but their orthography is a separate matter. 

Lenition of other consonants than \mw{p, t, c} is generally written, and only one exception is found, which is \mw{brivher} `is broken' found in line 27, although lenition is held to be optional following \mw{pan}~\autocite[380]{morgan_y_1952}. By contrast, lenition of voiceless stops in this text is generally not written, except for \gls{petr} in \mw{gan} `with, by' and \mw{ban} `when'. There is one exception to this rule, which is found in line 29: \mw{brouher} `is proved'. A potential explanation for this is that \mw{brouher} `is proved' echoes \mw{brivher} `is broken' two lines earlier, cf.\ the following four lines:

\mwcc[ymath2730]{YMATh ll.\ 27--30, emphasis added}{Llyaus ban \emph{brivher}, llyaus ban foher,\\
Llyaus ev hymchuel in eu hymvan.\\
Seith meib eliffer, seith guir ban \emph{brouher},\\
Seith guaew ny ochel in eu seithran.}{%
Hosts when they are broken, hosts when they are made to scatter,\\
Hosts returning, in their combat.\\
Seven sons of Eliffer, seven men when it is proved,\\
Seven spears they do not avoid in seven companies.}
As can be seen in Example \ref{ymath2730}, both words appear in a similar syntactic context, but also in a very similar semantic context. It is after all the proof (\ie test) that breaks the very same men. Nevertheless, the question rises how it was possible that a lenited voiceless stop could be made to alliterate with an unlenited voiced stop, while at the same time, they were consistently kept apart in the orthography. The answer to this conundrum may lie in the \mw{r} following both words: it is seen in Chapter \ref{oldwelsh} that it is exactly resonants next to stops that create confusion as to which series a stop belongs to. Furthermore, the exceptions \mw{brivher} and \mw{brouher} also show the same behaviour as \mw{grawys} discussed earlier in this chapter.
\begin{table}[h]
\centering
\begin{tabular}{@{}lllll@{}}
\toprule
\textbf{Line} & \textbf{Word}  & \textbf{Translation} & \textbf{Reason for lenition}  & \textbf{Represented} \\ \midrule
1   & \textit{truan} & `sad'   & follows \mw{mor}    & no    \\
1   & \textit{genhẏf} & `with me'  & petrified lenition    & yes   \\
1   & \textit{truan} & `sad'   & follows \mw{mor}    & no    \\
2   & \textit{keduyv} & (personal name) & follows \mw{am}     & no    \\
4   & \textit{tryuruyd} & `bloodstained'  & follows \mw{o}    & no    \\
4   & \textit{tryuan} & `shattered'  & follows \mw{o}    & no    \\
5   & \textit{uelun} & `I saw'   & follows VP \mw{a}    & yes   \\
6   & \textit{teulu} & `retinue'  & follows \mw{y} `his'   & no    \\
8   & \textit{welugan} & `white horse'  & follows \mw{ar}     & yes   \\
10  & \textit{gan} & `by'    & petrified lenition    & yes   \\
11  & \textit{leith} & `death'   & follows \mw{o'e} `of his'   & yes   \\
11  & \textit{teith} & `journey'  & follows \mw{a} `from'    & no    \\
12  & \textit{tarian} & `shield' & follows \mw{y} `his'   & no    \\
15  & \textit{uuan}  & `soon'   & follows \mw{tra} `very'   & yes  \\
16  & \textit{gan} & `by'  & petrified lenition    & yes   \\
19  & \textit{kyulauan} & `battle'   & follows preposed adjective & no    \\
20  & \textit{uerin} & `folk'   & follows \mw{a'e} `and his'  & yes   \\
20  & \textit{wnaethan} & `they did'   & follows VP \mw{a}    & yes   \\
23  & \textit{vit} & `will be'  & follows \mw{pan}    & yes   \\
24  & \textit{wuchit} & `life'   & follows \mw{y} `his'   & yes   \\
26  & \textit{vidan} & `they will be'  & follows non-written VP \mw{a} & yes   \\
27  & \textit{ban} & `when'   & petrified lenition    & yes   \\
27  & \textit{brivher} & `is broken'  & follows \mw{ban}    & no    \\
27  & \textit{ban} & `when'   & petrified lenition    & yes   \\
29  & \textit{ban} & `when'   & petrified lenition    & yes   \\
29  & \textit{brouher} & `is proved'  & follows \mw{pan}    & yes   \\
30  & \textit{ochel} & `avoids'   & follows VP \mw{ni}    & yes   \\
31  & \textit{kyuerbin} & `opposing'   & follows feminine \mw{kad}   & no    \\
31  & \textit{pop} & `every'   & follows \mw{y} `to'    & no    \\
33  & \textit{loneid} & `fullness'   & follows \mw{seith}    & yes   \\
34  & \textit{guaed} & `blood'   & follows \mw{o}    & no    \\ \bottomrule
\end{tabular}
\caption{Representation of lenition in \mw{Ymddiddan Myrddin a Thaliesin} as found in the Black Book of Carmarthen}
\label{lenitionymath}
\end{table}

\subsection{\mw{Marunad Madauc Fil' Maredut}}
Additionally, the poem \mw{Marunad Madauc Fil' Maredut}~(MMFM) has been chosen to demonstrate the same principle of non-representation of lenition of voiceless stops. This poem has been chosen because it is an elegy mourning a known person: Madog ap Maredudd died in 1160~\autocite[82]{jones_gwaith_1991}. Additionally, we know this poem was written by Cynddelw Brydydd Mawr, who was active in the late twelfth century~\autocite[xxx]{jones_gwaith_1991}.


\begin{mwl}
 \item%
  \begin{minipage}{0.45\textwidth}
  \textbf{Black Book of Carmarthen}\\
  \mw{Kywarchaw im ri.\ rad wobeith\\
  Kywarchaw kywercheis e \al{c}anweith.\\
  Y \al{p}rowi prydv.\ o \al{p}riwieth.\\
  Eurgert. ym argluit \al{k}edymteith.\\
  Y \al{c}vinav madauc.\ metweith y alar\\}
  % \end{block}
  \end{minipage}
  \begin{minipage}{0.45\textwidth}
  \textbf{Hendregadredd manuscript}\\
  \mw{Kyuarchaf y'm Ri rad wobeith,\\
  Kyuarchaf, kyuercheis \al{g}anweith,\\
  Y \al{b}roui prydu o'm prifyeith --- eurgert\\
  Y'm arglwyt \al{g}edymdeith,\\
  Y \al{g}wynaỽ Madaỽc metueith --- y alar, \\}
  % \end{block}
  \end{minipage}
 % \caption{The first five lines of \mw{Marwnad Madog ap Maredudd}, with lenited voiceless stops marked.}
 \label{marwnadcomparison}
\end{mwl}


Analysis of the poem is helped by the existence of the very same poem in a similarly old manuscript, but which shows lenition of voiceless stops for the most part, \ie \gls{h}. This manuscript dates from about 1300~\autocite{huws_llawysgrif_1981}. By way of illustration, Figure~\ref{marwnadcomparison} shows first five lines of this poem as found in both manuscripts, and easily demonstrates the difference in orthography. Since the Hendregadredd version does typically write lenition of voiceless stops, and forms the basis of the analysis by \textcite[82--91]{jones_gwaith_1991}, it is easy to make sense of all the grammatical preconditions for lenition as found in Table \ref{lenitionmmfm}. Line numbers refer to \textcite[78-79]{jarman_llyfr_1982}, but also generally agree with \textcite[82--91]{jones_gwaith_1991} for the Hendregadredd version.

Table \ref{lenitionmmfm} shows that voiceless stops are not lenited as a rule, while other consonants are typically lenited. Furthermore, there are cases where it is not clear why lenition was written. In the table, these instances are marked with `?', optionally followed by a suggestion why it may be lenited. In all of these cases, lenition is confirmed by the \gls{h} recension. Lenition is confirmed by \gls{h} in these cases, so I will not delve into the why of these lenitions further. However, the case of \mw{kedymteith} `companion' in line 4 may prove interesting. It is written \mw{gedymdeith} in \gls{h}, but it is not clear what grammatical reason there is for lenition. Lenition in \mw{gedymdeith} may therefore be regarded as a hypercorrect insertion of orthographical representation for lenition of voiceless stops. This in turn suggests that orthographical representation of lenition of voiceless stops was indeed an innovation which occurred after the death of Madog ap Maredudd. 

Despite the general pattern of non-lenition of voiceless stops, and lenition of other consonants, there are several exceptions to this rule.
These exceptions show lenition of voiceless stops in \mw{gar}~(l.~22) and \mw{dan}~(l.~23), as well as non-lenition of other consonants in \mw{lledieith}~(l.~17) and \mw{madauc}~(l.~28).
Line 22, in which \mw{gar} `loves' is found, is written in the margin of the page on which it is found.
This might point to a later insertion, but the insertion is written by the same hand.
This leaves this matter unsolved. The other lenited voiceless stop, in \mw{dan} `under', falls under \gls{petr}, which is already seen to be represented in this chapter.
The words \mw{lledieith} `foreign' and \mw{madauc} `Madog' are both used as epithets, which are known to lenite, but this is not done consistently\todo{I should refer to my results from CO here, which may or may not end up in my final thesis.}.

\begin{table}[h]
\centering
\begin{tabular}{@{}lllll@{}}
\toprule
\textbf{Line} & \textbf{Word} & \textbf{Translation} & \textbf{Reason for lenition} & \textbf{Represented} \\ \midrule
1 & \textit{wobeith} & `hope' & follows \mw{o} found in \gls{h} & yes \\
2 & \textit{canweith} & `hundred times' & adverbial clause & no \\
3 & \textit{prowi} & `prove' & follows \mw{y} `to' & no \\
3 & \textit{priwieith} & `best language' & follows \mw{o} & no \\
4 & \textit{kedymteith} & `companion' & ?dvandva compound & no \\
5 & \textit{cvinav} & `mourn' & follows \mw{y} `to' & no \\
5 & \textit{alar} & `lament' & follows \mw{y} `his' & yes \\
6 & \textit{alon} & `enemies' & follows \mw{y} `his' & yes \\
7 & \textit{canhimteith} & `escort' & follows feminine \mw{yscvid} & no \\
11 & \textit{wobeith} & `hope' & ? & yes \\
12 & \textit{kedimteith} & `companion' & follows preposed adjective & no \\
13 & \textit{leith} & `death' & follows \mw{no'e} `than his' & yes \\
16 & \textit{wisscoet} & `vestments' & follows preposed adjective & yes \\
16 & \textit{wessgvin} & `Gascon horse' & ? & yes \\
16 & \textit{canhimteith} & `escort' & follows preposed genitive & no \\
17 & \textit{vab} & `son' & epithet following PN & yes \\
17 & \textit{lledieith} & `foreign' & epithet following PN & no \\
18 & \textit{clod} & `fame' & follows \mw{y} `his' & no \\
19 & \textit{vaon} & `subjects' & follows preposed adjective & yes \\
19 & \textit{oleith} & `retreats' & follows VP \mw{ni} & yes \\
20 & \textit{wastad} & `constant' & follows \mw{rhad} & yes \\
20 & \textit{canhimteith} & `escort' & follows preposed genitive & no \\
22 & \textit{gar} & `loves' & follows VP \mw{a} & yes \\
22 & \textit{kidweith} & `joint work' & follows \mw{o} & no \\
23 & \textit{dan} & `under' & petrified lenition & yes \\
23 & \textit{calchwreith} & `vari-coloured' & follows feminine \mw{yscvd} & no \\
25 & \textit{owin} & `wish' & follows preposed genitive & yes \\
26 & \textit{pedeirieith} & `four languages' & follows feminine \mw{yscvid} & no \\
27 & \textit{teirn} & `king' & follows feminine \mw{hil} & no \\
28 & \textit{madauc} & (personal name) & epithet following \mw{hael} & no \\
28 & \textit{veuder} & `dread' & ? & yes \\
29 & \textit{leith} & `death' & follows \mw{o'e} `of his' & yes \\
30 & \textit{kedymteith} & `companion' & follows \mw{darw} `ended' & no \\
33 & \textit{truited} & `welcome' & follows \mw{y} `his' & no \\
34 & \textit{kywarweith} & `fight' & follows \mw{o'e} `of his' & no \\
36 & \textit{kadieith} & (personal name) & follows feminine \mw{aerllin} & no \\
37 & \textit{kadarn} & `strong' & follows preposed adjective & no \\
37 & \textit{kedymdemteith} & `companion' & follows preposed adjective & no \\
38 & \textit{talheith} & `crown' & follows \mw{y} `his' & no \\
39 & \textit{leith} & `death' & follows \mw{y} `his' & yes \\ \bottomrule
\end{tabular}
\caption{Representation of lenition in \mw{Marunad Madauc Fil' Maredut} as found in the Black Book of Carmarthen}
\label{lenitionmmfm}
\end{table}

\section{Final remarks}
\todo[inline]{These remarks should come at the end of the introduction to the orthography. These points roughly correspond to chapters which can be referred to.}
If we accept the following premises:
\begin{enumerate}
\item \gls{ow} did not represent lenition orthographically;
\item Early \gls{mw} did not represent lenition of voiceless stops orthographically;
\item Later \gls{mw} did represent lenition orthographically;
\item When a text with older orthography is copied, lenition is typically only modernised where it is vital to represent or disambiguate grammatical categories;
\item Lenition following verbs in Early \gls{mw} did not represent or disambiguate verbal categories;
\end{enumerate}

Then we must conclude the following:
\begin{enumerate}
\item In a text where postverbal lenition is represented orthographically, but not for voiceless stops, we may postulate a \textit{terminus ante quem} as well as a \textit{terminus post quem} for its original composition: it must originally have been composed before lenition of voiceless stops was represented orthographically, and after lenition started being represented at all.
\end{enumerate}

This conclusion necessarily follows from the premises above: if lenition following verbs was not modernised orthographically, then this must have been the case equally when copying from an \gls{ow} text and when copying from an Early \gls{mw} text. If, hypothetically speaking, an \gls{ow} text were copied into Later \gls{mw}, then we should not expect non-lenition of voiceless stops only. Rather, we should expect non-lenition of any consonant following verbs, because premise (4) above is as much applicable to premise (1) as it is to premise (2).


%%% Local Variables:
%%% mode: latex
%%% TeX-master: "../main"
%%% coding: utf-8
%%% End:
