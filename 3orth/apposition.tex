\subsection{\textit{Englynion y Clywaid}}
One text I will analyze is \mw{Englynion y Clywaid} (abbreviated as \mw{EngCl}) as found in its edition by Haycock \autocite*[313--337]{haycock_blodeugerdd_1994}, based on the text as found in NLW MS.\ Llanstephan 27, also known as the Red Book of Talgarth. The manuscript was produced in the late fourteenth or early fifteenth century~\autocite[60]{huws_medieval_2000}. The text is also found in Oxford Jesus College MS.\ 3, which may be dated to the beginning of the fourteenth century~\autocite[297]{haycock_blodeugerdd_1994}.

The \mw{Englynion y Clywaid} are a series of three-line englynion that are very similar to each other in structure. The first line asks `Did you hear what [person] said?', followed by a description of said person in the second line and a quote of this character in the third line. Naturally, these poems contain many epithets which may or may not be lenited. The first two lines of relevant examples are given below. 

The examples below show lenition. Note that Examples \ref{engcl30} and \ref{engcl35} contain two instances of lenition.
\begin{mwl}
\mwc[engcl30]{\mw{EngCl} st.\	30}{A glyweist-di a gant Haernwed~| Vradawc, vilwr t\"eyrned?}{Did you hear what Haernwedd said,~| treacherous, soldier of princes?}
% \mwc{\mw{EngCl} st.\	30}{A glyweist-di a gant Haernwed~| Vradawc, vilwr t\"eyrned?}{Did you hear what Heaernwedd said,~| treacherous, soldier of princes?}
\mwc[engcl35]{\mw{EngCl} st.\	35}{A glyweist-di a gant Auaon~| Uab Talyessin, gerd gofyon?}{Did you hear what Afaon said,~| son of Taliesin, thoughts of song?}
\mwc{\mw{EngCl} st.\	60}{A gly<w>eist-di a gant Credeilat~| Uerch Lud, riein wastat?}{Did you hear what Creiddylad said~| daughter of Lludd, elegant woman?}
\mwc{\mw{EngCl} st.\	72}{A glyweisti a gant Kad<r>eith~| Uab Porthawr, milwr areith?}{Did you hear what Cadriaith said~| son of Porthor, with the speech of a soldier?}
\end{mwl}

The examples below show non-lenition. Examples \ref{engcl21},
\ref{engcl32},
\ref{engcl36},
\ref{engcl39},
\ref{engcl58},
\ref{engcl61}, and
\ref{engcl66} each contain two examples of non-lenition:

\begin{mwl}
\mwc{\mw{EngCl} st.\ 2}{A glyweist-di a gant Kynrein,~| Penn kyssul Ynys Brydein?}{Did you hear what Cynrain said,~| the head of the council of Britain?}
\mwc{\mw{EngCl} st.\ 3}{A glyweist-di a gant Itloes,~| Gwr gwar, hygar y einyoes?}{Did you hear what Idloes said,~|  warm man, whose life was lovely?}
\mwc{\mw{EngCl} st.\ 4}{A glyweist-di a gant Kynnlwc,~| Gwr llwyt, llydan y olwc?}{Did you hear what Kynllwc said,~|  holy man, whose view was broad?}
\mwc{\mw{EngCl} st.\ 13}{A glyweist-di a gant Dewi,~| Gwr llwyt, llydan y deithi?}{Did you hear what  Dewi said,~|  holy man, whose qualities were ample?}
\mwc{\mw{EngCl} st.\ 14}{A glyweist-di a gant Teilaw,~| Gwr a vu yn penytyaw?}{Did you hear from Teilo,~|  man who was doing penance?}
\mwc{\mw{EngCl} st.\ 15}{A glyweist-di a gant <Padarn>,~| Pregethwr kywir kadarn?}{Did you hear what Padarn said,~|  true and strong preacher}
\mwc{\mw{EngCl} st.\ 17}{A glyweist-di a gant Marthin,~| Brenhinawl, sant y gyfrin?}{Did you hear what Martin said,~| the kingly, whose secret was holy}
\mwc{\mw{EngCl} st.\ 18}{A glyweist-di a gant Gwynlliw,~| Tat Katwc, kywir ymliw?}{Did you hear what Gwynllyw said,~| father of Cadwg, whose reproach was true?}
\mwc{\mw{EngCl} st.\ 19}{A glyweist-di a gant Anarawt,~| Milwr donyawc, did<l>awt?}{Did you hear what Anarawd said,~| gifted soldier, not poor?}
\mwc[engcl21]{\mw{EngCl} st.\ 20}{A glyweist-di a gant Gwrhir,~| Gwalstawt pob ieithydd}{Did you hear what Gwrhir said,~| interpreter of all people skilled in languages?}
\mwc{\mw{EngCl} st.\ 21}{A glyweist-di a gant Gereint~| Mab Erbin, kywir kywir kywreint?}{Did you hear what Geraint said,~| Son of Erbin, faithful clever man?}
% \mwc{\mw{EngCl} st.\ 21}{A glyweist-di a gant Gereint~| Mab Erbin, kywir kywir kywreint?}{Did you hear what Geraint said,~| Son of Erbin, faithful clever man?}
\mwc{\mw{EngCl} st.\ 22}{A glyweist-di a gant Ryderch,~| Trydyd hael, serchawc serch?}{Did you hear what Rhydderch said,~| third lord, affectionate love?}
\mwc{\mw{EngCl} st.\ 25}{A glyweist-di <a gant>  Cormoc,~| Brenhinawl gyfreith annoc?}{Did you hear what Cormac said,~| instigator of royal law?}
\mwc{\mw{EngCl} st.\ 28}{A glyweist-di a gant Hyled~| Merch Kyndrwyn, mawr y ryfed?}{Did you hear what Heledd said,~| daughter of Cyndrwyn, great her riches?}
\mwc[engcl32]{\mw{EngCl} st.\ 32}{A glyweist-di a gant Hueil~| Mab Kaw, kymwyll areil?}{Did you hear what Hueil said,~| son of Caw, who guarded his praise?}
% \mwc{\mw{EngCl} st.\ 32}{A glyweist-di a gant Hueil~| Mab Kaw, kymwyll areil?}{Did you hear what Hueil said,~| son of Caw, who guarded his praise?}
\mwc[engcl36]{\mw{EngCl} st.\ 36}{A glyweist-di a gant Yscafnell~| Mab Dysgyfdawt, kat gymmell?}{Did you hear what Ysgafnell said,~| son of Dysgyfdawd, who instigated battle?}
% \mwc{\mw{EngCl} st.\ 36}{A glyweist-di a gant Yscafnell~| Mab Dysgyfdawt, kat gymmell?}{Did you hear what Ysgafnell said,~| son of Dysgyfdawd, who instigated battle?}
\mwc[engcl39]{\mw{EngCl} st.\ 39}{A glyweist-di a gant Bangar~| Mab Kaw, milwr clotgar?}{Did you hear what Bangar said~| son of Caw, famous soldier?}
% \mwc{\mw{EngCl} st.\ 39}{A glyweist-di a gant Bangar~| Mab Kaw, milwr clotgar?}{Did you hear what Bangar said~| son of Caw, famous soldier?}
\mwc{\mw{EngCl} st.\ 41}{A glyweist-di a gant Goliffer~| Gosgorduawr, gwymp y niuer?}{Did you hear what Goliffer said~| having a large retinue, his host was splendid?}
\mwc{\mw{EngCl} st.\ 44}{A glyweisti a gant Dirmic,~| Milwr doeth detholedic}{Did you hear what Dirmyg said,~| wise and chosen soldier?}
\mwc{\mw{EngCl} st.\ 46}{A glyweisti a gant Gwiawn,~| Dremynwr golwc unyawn?}{Did you hear what Gwion said,~| observant watcher?}
\mwc{\mw{EngCl} st.\ 47}{A glyweisti a gant Llenllyawc~| Gwydel, urdawl eurdorchawc?}{Did you hear what Llenlleog said~| Irishman, honoured golden-torque-wearer?}
\mwc{\mw{EngCl} st.\ 55}{A glyweisti a gant Kynan, ~| Gwledic sant y anyan?}{Did you hear what Cynan said,~| prince whose nature was saintly?}
\mwc{\mw{EngCl} st.\ 56}{A glyweisti a gant Kunllaw,~| Gwr a vu yn llauuryaw?}{Did you hear what Cynllo said,~| man who was toiling?}
\mwc[engcl58]{\mw{EngCl} st.\ 58}{A glyweisti a gant Gildas~| Mab y Gaw, milwr atkas?}{Did you hear what Gildas said~| son to Caw, hateful soldier?}
% \mwc{\mw{EngCl} st.\ 58}{A glyweisti a gant Gildas~| Mab y Gaw, milwr atkas?}{Did you hear what Gildas said~| son to Caw, hateful soldier?}
\mwc{\mw{EngCl} st.\ 59}{A glyweisti a gant Garselit~| Gwydel, diogel ymlit?}{Did you hear what Garselit said?~| Irishman, certain to be persecuted?}
\mwc[engcl61]{\mw{EngCl} st.\ 61}{A glyweist-di a gant Llyaws~| Mab Nwyfre, milwr hynaws?}{Did you hear what Lliaws said~| son of Nwyfre, kind soldier?}
% \mwc{\mw{EngCl} st.\ 61}{A glyweist-di a gant Llyaws~| Mab Nwyfre, milwr hynaws?}{Did you hear what Lliaws said~| son of Nwyfre, kind soldier?}
\mwc{\mw{EngCl} st.\ 62}{A glyweist-di a gant Kyndrwyn,~| Tangnefedwr a mat mwyn?}{Did you hear what Cyndrwyn said,~| peacemaker and good gentle man?}
\mwc{\mw{EngCl} st.\ 65}{A glyweist-di a gant Europa~| Mab Custeon, cas westua?}{Did you hear what Europa said,~| son of Custeon, his accommodation was hateful?}
\mwc[engcl66]{\mw{EngCl} st.\ 66}{A glyweist-di a gant Eheubryt~| Mab Kyfwlch, kyfyawn yspryt?}{Did you hear what Eheubryd said,~| son of Cyfwlch, righteous spirit?}
% \mwc{\mw{EngCl} st.\ 66}{A glyweist-di a gant Eheubryt~| Mab Kyfwlch, kyfyawn yspryt?}{Did you hear what Eheubryd said,~| son of Cyfwlch, righteous spirit?}
\mwc{\mw{EngCl} st.\ 67}{A glyweist-di a gant Drystan,~| Gobeithwr prud y anyan?}{Did you hear what Drystan said,~| hoper whose nature was wise?}
\end{mwl}
\subsection{\mw{Culhwch ac Olwen}}
One text which contains plenty of names and is also considered to be very old is \gls{co}. However, lenition of names in apposition does not appear strongly correlated to consonant type. Voiceless stops in apposition to a personal name are lenited 11 out of 63 times, while other consonants are lenited 45 of of 246 times. There are several patterns to be discerned nevertheless.

Among all the epithets starting with a voiceless stop, only velars are written lenited, although they still constitute a minority: 11 out of 30 epithets starting with /k/ are lenited. 

Among epithets starting with another consonant, two patterns may be discerned. One is that /g\cw/ is lenited frequently when this phoneme goes back to pre-apocope \pc{*\cw} or sometimes \pc{*g\cw}: 28 out of 45 of these consonants are lenited. This fact is significant, because initial /g\cw/ is in fact an innovation to distinguish the radical from lenited /\cw/, while the latter represents the initial pre-apocope form. Epithets starting with /g/ not followed by /\cw/ are never lenited. 

Of all the remaining epithets, 17 out of 200 are lenited. These are all lenited /b/ and /m/, but these two consonants constitute the vast majority of remaining initial consonants for epithets anyway.
\subsection{The four branches of the Mabinogi}
As it has already been established that voiceless stops are not lenited following \ei\ and \oes\ in \mw{Pwyll Pendeuic Dyuet}, it follows that epithets to nouns may behave similarly in terms of lenition. This pattern is already found in how we name each of the four branches: \mw{Pwyll Pendeuic Dyuet} has an unlenited epithet, while \mw{Branwen Uerch Lyr, Manawydan Uab Llyr} and \mw{Math Uab Mathonwy} each show lenition. 

\begin{table}[h]
  \centering
    \begin{tabular}{rrrrr}
    \toprule
    \tch{} & \tch{\acrshort{ppd}} & \tch{\acrshort{bul}} & \tch{\acrshort{mul}} & \tch{\acrshort{mum}} \\
    \midrule
    \textbf{T} & 0/15 & 0/3 & 0/1 & 4/7 \\
    \textbf{¬T} & 15/15 & 22/30 & 5/10 & 27/35 \\
    \bottomrule
    \end{tabular}%
    \caption{Lenition of epithets after personal names in the four branches of the Mabinogi, separated between epithets starting with a voiceless stop and other consonants.}
  \label{lenitionepithetspkm}%
\end{table}%
 Table \ref{lenitionepithetspkm} shows that lenition of voiceless stops is only found in \mw{Math Uab Mathonwy}. In this text, all epithets starting with voiceless stops relate to the character Gronw Pebyr, as shown in Table \ref{gronwpebyr}.
 \begin{table}[h]
\centering
\begin{tabular}{@{}llll@{}}
\toprule
\tch{Column} & \tch{Line} & \tch{Name}   & \tch{Apposition} \\ \midrule
101             & 30            & \textit{Gronỽ}  & \textit{pebẏr}      \\
104             & 32            & \textit{gronỽ}  & \textit{pebẏr}      \\
109             & 32            & \textit{gronỽẏ} & \textit{pebẏr}      \\
110             & 10            & \textit{gronỽ}  & \textit{bebẏr}      \\
110             & 23            & \textit{gronỽẏ} & \textit{bebẏr}      \\
110             & 27            & \textit{gronỽẏ} & \textit{bebẏr}      \\
111             & 5             & \textit{gronỽẏ} & \textit{bebẏr}      \\ \bottomrule
\end{tabular}
\caption{Epithets starting with a voiceless stop in \mw{Math Uab Mathonwy}.}
\label{gronwpebyr}
\end{table}

In addition, there is an irregularity in the second to fourth branch in that not all other consonants than voiceless stops are lenited. A list of unlenited epithets starting with one of these consonants is given in Table \ref{unlennontepithetspkm}.

% Please add the following required packages to your document preamble:
% \usepackage{booktabs}
\begin{table}[]
\centering
\begin{tabular}{@{}lllll@{}}
\toprule
\tch{Branch} & \tch{Column} & \tch{Line} & \tch{Name}                   & \tch{Apposition}         \\ \midrule
\acrshort{bul}             & 39              & 32            & \textit{matholỽch}              & \textit{brenhin iỽerdon}    \\
\acrshort{bul}             & 41              & 18            & \textit{matholỽch}              & \textit{brenhin iỽerdon}    \\
\acrshort{bul}             & 43              & 21            & \textit{unic}                   & \textit{gleỽ ẏscỽẏd}        \\
\acrshort{bul}             & 45              & 19            & \textit{llassar}                & \textit{llaes gẏfneỽit}     \\
\acrshort{bul}             & 50              & 5             & \textit{unic}                   & \textit{gleỽẏscỽẏd}         \\
\acrshort{bul}             & 50              & 7             & \textit{ỽlch}                   & \textit{minasgỽrn}          \\
\acrshort{bul}             & 50              & 8             & \textit{llaẏssar}               & \textit{llaesgẏgỽẏt}        \\
\acrshort{bul}             & 57              & 1             & \textit{ẏnaỽc}                  & \textit{grudẏeu uab murẏel} \\
\acrshort{mul}             & 62              & 17            & \textit{ỽẏn}                    & \textit{gloẏỽ}              \\
\acrshort{mul}             & 65              & 31            & \textit{lassar}                 & \textit{llaes gẏgnỽẏt}      \\
\acrshort{mul}             & 65              & 35            & \textit{lassar}                 & \textit{llaes gẏgnỽẏt}      \\
\acrshort{mul}             & 71              & 16            & \textit{gỽẏn}                   & \textit{gloeỽ}              \\
\acrshort{mul}             & 71              & 16            & \textit{kicua uerch gỽẏn gloeỽ} & \textit{gỽreic prẏderi}     \\
\acrshort{mum}             & 88              & 5             & \textit{gỽrgi}                  & \textit{guastra}            \\
\acrshort{mum}             & 97              & 17            & \textit{lleỽ}                   & \textit{llaỽgẏffes}         \\
\acrshort{mum}             & 97              & 35            & \textit{lleỽ}                   & \textit{llaỽgẏffes}         \\
\acrshort{mum}             & 108             & 10            & \textit{lleỽ}                   & \textit{llaỽ gẏffes}        \\
\acrshort{mum}             & 109             & 19            & \textit{leỽ}                    & \textit{llaỽ gẏffes}        \\
\acrshort{mum}             & 109             & 36            & \textit{leỽ}                    & \textit{llaỽ gẏffes}        \\
\acrshort{mum}             & 110             & 24            & \textit{lleỽ}                   & \textit{llaỽ gyffes}        \\
\acrshort{mum}             & 111             & 10            & \textit{lleỽ}                   & \textit{llaỽ gẏffes}        \\ \bottomrule
\end{tabular}
\caption{Non-lenited epithets in the four branches of the Mabinogi starting with a consonant other than a voiceless stop.}
\label{unlennontepithetspkm}
\end{table}