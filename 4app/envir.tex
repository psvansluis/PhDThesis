\chapter{Environments that cause lenition}
\label{cha:envir-that-cause}

Appendix~\ref{cha:database-lenition} and Appendix~\ref{cha:datab-lenit-mwbuch} contain all instances of lenition in the texts discussed in Part~\ref{part:orthography}. These databases contain instances of orthographical lenition, and instances where  lenition is expected, but not found. Here, all instances where lenition is expected but not found, and where lenition is found are discussed.

\section{Contact lenition}
\label{sec:contact-lenition-1}
Here, I list all types of contact lenition encountered. Here, I emend, \ie I argue lenition was in fact spoken even where it is not represented orthographically. Exceptions include unproductive categories such as lenition of nouns following a plural article \eg \mow{y bobl}, and cases where spirantisation may be expected instead of lenition, \eg verbs starting with \mow{p, t, c} following \mow{ni, na} or a compound thereof in a relative clause. The list is given in \gls{mow} orthography. 

\begin{itemize}
\item\mow{a} (interrogative particle): \textcite[\S\S 23, 196]{evans_grammar_1964}.
\item\mow{a} (verbal particle): \textcite[\S\S 23, 6, 192]{evans_grammar_1964}.
\item\mow{a} (vocative particle): \textcite[\S 19]{evans_grammar_1964}.
\item\mow[to]{add}: \textcite[\S 20]{evans_grammar_1964}.
\item\mow{-ai} (imperfect ending): \textcite[\S 21]{evans_grammar_1964}; \textcite[42--45]{van_development14}.
\item\mow[second]{ail}: \textcite[\S 249]{evans_grammar_1964}.
\item\mow[about]{am}: \textcite[\S 20]{evans_grammar_1964}.
\item\mow[on]{ar}: \textcite[\S 20]{evans_grammar_1964}.
\item\mow[to]{at}: \textcite[\S 20]{evans_grammar_1964}.
\item\mow[with, by]{can}: \textcite[\S 20]{evans_grammar_1964}.
\item\mow[although]{cani}: See \mow{ni}.
\item Equative \mow[as]{cyn}: \textcite[\S\S 22, 43]{evans_grammar_1964}.
\item\mow[around]{dam}: See \mow{am}.
\item\mow[two]{dau}: \textcite[\S 20]{evans_grammar_1964}.
\item\mow[two]{dwy}: \textcite[\S 20]{evans_grammar_1964}.
\item\mow[your]{dy}: \textcite[\S 20]{evans_grammar_1964}.
\item\mow{-ed} (imperative ending): \textcite[\S 21]{evans_grammar_1964}.
\item\mow[his]{ei}: \textcite[\S 20]{evans_grammar_1964}.
\item feminine article \textcite[\S 19]{evans_grammar_1964}.
\item feminine noun: \textcite[\S\S 19, 22]{evans_grammar_1964}.
\item\mow{-fu} (preterite ending): \textcite[\S 21]{evans_grammar_1964}; \textcite[50--51]{van_development14}.
\item\mow[without]{heb}: \textcite[\S 20]{evans_grammar_1964}.
\item\mow[until]{hyd}: \textcite[\S 20]{evans_grammar_1964}.
\item\mow[his]{'i}: \textcite[\S 20]{evans_grammar_1964}.
\item\mow[to]{i}: \textcite[\S 20]{evans_grammar_1964}.
\item\mow[under]{is}: \textcite[\S 20]{evans_grammar_1964}.
\item\mow[behold]{llyma}: \textcite[\S 280]{evans_grammar_1964}; actually parenthesis i.e.\ free lenition.
\item\mow[so, as]{mor}: \textcite[\S\S 22, 41]{evans_grammar_1964}.
\item\mow{na} (negative particle): See \mow{ni}.
\item\mow[or]{neu}: \textcite[\S 20]{evans_grammar_1964}.
\item\mow{ni} (negative particle): Before \mow{p, t, c} not included, except where represented; \textcite[\S\S 23, 24]{evans_grammar_1964}.
\item\mow{nir} (negative perfect particle): cf. \mow{rhy}.
\item\mow[from]{o}: \textcite[\S 20]{evans_grammar_1964}.
\item\mow[there is]{oes}: \textcite[\S 21]{evans_grammar_1964}; \textcite[39]{van_development14}.
\item\mow[until]{oni}: See \mow{ni}.
\item\mow[what]{pa}: \textcite[\S 20]{evans_grammar_1964}.
\item\mow[when]{pan}: \textcite[\S 23]{evans_grammar_1964}.
\item plural article: \textcite[10--11]{morgan_y_1952}.
\item\mow[perfect particle]{rhy}: \textcite[\S 23]{evans_grammar_1964}.
\item\mow[seven]{saith}: \textcite[\S 20]{evans_grammar_1964}.
\item\mow[until, under]{tan}: \textcite[\S 20]{evans_grammar_1964}.
\item\mow[your]{'th}: \textcite[\S 20]{evans_grammar_1964}.
\item\mow[while]{tra}: \textcite[\S 23]{evans_grammar_1964}.
\item\mow[over]{tros}: \textcite[\S 20]{evans_grammar_1964}.
\item\mow[through]{trwy}: \textcite[\S 20]{evans_grammar_1964}.
\item\mow[third]{trydedd}: \textcite[\S 52]{evans_grammar_1964}.
\item Feminine \mow[one]{un}: \textcite[\S 20]{evans_grammar_1964}.
\item\mow[above]{uwch}: \textcite[\S 20]{evans_grammar_1964}.
\item\mow[against, by]{wrth}: \textcite[\S 20]{evans_grammar_1964}.
\item\mow[eight]{wyth}: \textcite[\S 20]{evans_grammar_1964}.
\item\mow[verbal particle]{yd}: \textcite[\S 23]{evans_grammar_1964}.
\item\mow{yn} (adverbial particle): \textcite[\S 22]{evans_grammar_1964}.
\end{itemize}

\section{Free lenition}
\label{sec:free-lenition-1}
As many types of free lenition are in flux through the \gls{mw} period, I do not emend instances of free lenition, except where noted. That is, I do not take most instances where we could expect free lenition,  but where it is not written, to be an instance of unwritten object lenition. This is because object lenition itself was not written consistently even where object lenition is known to have been written occasionally and where representation of lenition does not depend on consonant type.

Emended instances of free lenition are: apposition and preposed adjectives. \gls{np} lenition is sometimes included where the text as a whole is shown to have it, as is adverb lenition.
\begin{itemize}
\item adverbial phrase \textcite[\S 19, 22]{evans_grammar_1964}.
\item apposition: \textcite[\S 19]{evans_grammar_1964}; \textcite[116--9, 122--3]{morgan_y_1952}; but especially \textcite{schrijver_free_2010}. Examples include \mw[Macsen the lord]{Macsen \al{W}ledic}, \mw[whatever]{beth \al{b}ynnac}, and vocative expressions without the vocative particle, \eg \mw[What is this, Margaret?]{Pa beth yw hynn, \al{V}argret?}.
\item compound: \textcite[\S 20, 22]{evans_grammar_1964}. 
\item existential verb:  \textcite[29--30]{van_development14}. Any inflected form of \mow[to be]{bod} may cause lenition to its subject when it has the semantics of an existential verb, analogous to lenition after \mow[there is]{oes}.
\item \gls{np} lenition: \textcite[\S 21]{evans_grammar_1964}.
\item object of destination: \textcite[227]{morgan_y_1952}; \textcite[\S 21]{evans_grammar_1964}. 
\item object lenition: \textcite[\S 21]{evans_grammar_1964}.
\item parenthesis: \textcite[429]{morgan_y_1952}, \textcite[\S 21]{evans_grammar_1964}, \textcite[5--6]{schrijver_free_2010}.
\item preposed adjective: \textcite[35]{morgan_y_1952}. Often difficult to distinguish from preposed nouns, because nouns may be used as adjectives and vice versa. Mostly found in poetry, except in the case of adjectives such as \mw[old]{hen} which always precede the noun  they specify.
\item preposed noun: a noun that stands in a genitival relationship with the noun following usually causes lenition, \eg \mow[leaves of Autumn]{Hydref ddail}~\autocite{daniel_cyfuniadau_2003}.
\item subject to plural verb: \textcite[\S 21]{evans_grammar_1964}; \textcite[2]{schrijver_free_2010}, although \textcite{van_development14} fails to find lenition in this environment.
\end{itemize}


%%% Local Variables:
%%% coding: utf-8
%%% mode: latex
%%% TeX-master: "../main"
%%% End:
