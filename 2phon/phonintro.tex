\chapter{Introduction — phonology}
\label{cha:introduction-phonology}

All present-day Celtic languages share grammatical lenition, whereby the first consonant of a word may change into a more sonorous one as a result of its morphosyntactic environment. The way in which a consonant is made into a more sonorous one differs between the Goidelic and the Brittonic branches of Celtic. In Goidelic, all stop consonants turn into their fricative counterparts, while present-day Brittonic voiceless stops turn into their voiced counterparts, and voiced stops turn into their respective fricatives. Table~\ref{tab:lenitionwelshirish} shows the differing treatments with  Welsh and  Irish as examples.

\begin{table}[h]
  \centering
  \begin{tabular}{lllll}
    \toprule
    & \multicolumn{2}{c}{Modern Welsh} & \multicolumn{2}{c}{Old Irish} \\
    & Radical & Lenited & Radical & Lenited \\\midrule
    \gls{T} & /p t k/ & \tikz[remember picture,anchor=base,baseline=(current bounding box.base)]{\node[draw,rounded corners,fill=black,fill opacity=0.1, text opacity=1](lt){/b d ɡ/};}   & /p t k/ & /f θ x/  \\
    \gls{D} & \tikz[remember picture,anchor=base,baseline=(current bounding box.base)]{\node[draw, rounded corners,fill=black,fill opacity=0.1, text opacity=1](xd){/b d ɡ/};} &  /v ð \zero/  & /b d ɡ/ & /β ð ɣ/ \\
    \bottomrule
  \end{tabular}%
  \caption[Radical and lenited stop consonants in Welsh and Irish.]{Radical and lenited stop consonants in Welsh and Irish. The chart is simplified for Irish in that it does not include both broad and slender phonemes. }
  \label{tab:lenitionwelshirish}%
\end{table}%

The shaded areas in Table~\ref{tab:lenitionwelshirish} highlight a merger in Welsh. Judging from their Irish counterparts alone we can see that lenited voiceless stops (\lT) and radical voiced stops (\xD) must at some point have been separate.  I will argue that the merger between \lT\ and \xD\ occurred in the historical period, and that the phonetic variables keeping these phonemes apart are recoverable. Part~\ref{part:phonology-phonetics} of this thesis treats which phonological variables kept these sounds apart in Welsh, and Part~\ref{part:orthography} treats the date up until when they were separated. This chapter will introduce the history of the idea that \lT\ did not equal \xD\ word-initially.

\section{A two-stage development of lenition}
\label{sec:two-stage-devel}
A typical historical grammar describing lenition is \textcite{Mor_Welsh13}, who describes it as follows:
\tqt{Brit.\ and Lat. \textit{p, t, k, b, d, g, m} between vowels became \textit{b, d, g, f, δ, ᵹ, f} respectively in W.}{Mor_Welsh13}{§~103}
This description summarises the totality of changes between \gls{pc} and \gls{mow} correctly, but does not break it up into intermediate stages. 
When he posits that present-day Brittonic languages /p t k/ lenite to /b d ɡ/, he says two things: the first is that lenition applied to voiceless stops at some point in the history of Brittonic; the second is that the phonemes to which these voiceless stops lenited merged with the pre-existing phonemes /b d ɡ/. These two developments are often taken together.  Not breaking these two developments up creates a few problems.

One obvious problem is Irish. If the two Brittonic developments occurred at the same time, then lenition in Goidelic and Brittonic cannot be described as a single development, because Goidelic voiceless stops lenite into /f θ x/, and it is hard to derive the Goidelic  voiceless fricatives from the Brittonic voiced stops, or vice versa. Another problem is Brittonic-internal. If voiceless stops after lenition came to equal unlenited voiced stops immediately, then would these not have to be lenited into lenited voiced stops in turn? 

This issue was recognised  by \textcite{loth_les_1892}, who proposed that voiceless stops underwent lenition at a later date than other consonants:
\tqt{\textfrench{Les explosives sonores \textit{b, d, g}, entre deux voyelles ont dû \textit{commencer} leur mouvement vers les spirantes correspondantes, avant que les explosives sourdes, \textit{p, t, c} ne fussent devenues \textit{b, d, g}; autrement, celles-ci auraient eu le même sort qu'elles. Si le latin \textit{opera} avait donné \textit{obera} au moment où \textit{labore} était encore \textit{labure}, le \textit{b} d'\textit{ober} eût été traité comme celui de \textit{labur}; c'est-à-dire fût devenu \textit{v}; on aurait aujourd'hui \textit{over, lavur} et non \textit{ober, lavur}. Des trois explosives sonores, \textit{g} paraît la première être devenue spirante.}\footnote{The voiced stops \textit{b, d, g} between two vowels should have \emph{started} their movement towards their corresponding spirants before the voiceless stops became \textit{b, d, g}; otherwise, these ones would have had the same fate as those. If Latin \textit{opera} had given \textit{obera} at the moment when \textit{labore} was still \textit{labure}, the \textit{b} of \textit{ober} would have been treated as the one of \textit{labur}; \ie it would have become \textit{v}; today we would have \textit{over, lavur} and not \textit{ober, lavur}. Of the three voiced stops, \textit{g} seems to be the first to have become a spirant.}}{loth_les_1892}{87}
Loth's explanation explains why we do not have anything like /p/ > /b/ > /v/, but it has its drawbacks. Lenition may be described as a single process whereby intervocalic consonants become more sonorous, so it would be uneconomical to suggest two separate events of lenition. Incidentally, his account solves the issue of Goidelic lenition:  lenited voiceless stops to voiceless spirants could be just one more independent development of lenition. Yet by now we have to assume three independent events to account for all the lenition in Goidelic and Brittonic.

Alternatively, Loth's explanation could be criticised for not considering chain shifts: it would be economical to assume that intervocalic /p/ and /b/ would shift at the same time in a push chain or a drag chain. Accounting for this possibility, Loth only demonstrates  that lenition of voiceless stops did not  occur \emph{before} other types of lenition, but not necessarily that it must have occurred \emph{afterwards}. Considering Brittonic lenition a chain shift could work for Brittonic itself, but it would force one to consider Goidelic lenition a wholly separate affair.

\Textcite[162]{Foer_Flussname41} and \Textcite{sims-williams_dating_1990} also argue for two separate stages of lenition, with /p t k/ undergoing lenition at a later date than other consonants.
% \tqt{I shall argue that `lenition' — the conventional name for the spirantization of [b d g m] > [β δ γ μ] and voicing of [p t k] > [b d g] — occurred in two distinct stages: (1) spirantization, then (2) voicing […]. This will seem heretical to Brittonicists, who instinctively regard [t] > [d], etc., and [d] > [δ], etc., as a single phenomenon because they constitute the morphophonemic alternation of `lenition' or `soft mutation'.}{sims-williams_dating_1990}{221}
They both apply Loth's line of reasoning on how /p t k/ cannot have preceded other consonants because that would have led to these consonants being lenited twice over~\autocite[\eg][232]{sims-williams_dating_1990}. This shows how  lenition of voiceless stops is often silently equated with the merger of lenited voiceless stops with radical voiced stops. Förster's argument is rejected by \textcite[§~131]{jackson_language_1953}, who argues that both series could have shifted in a chain shift. Sims-Williams' argument also rests on early loans from Brittonic into Goidelic, but this analysis is rejected by \textcite{isaac_chronology_2004}, because Sims-Williams conflates the roles  phonetics and phonology play in the process of loaning.

To solve the problem of Goidelic and Brittonic lenition, \textcite{Ped_Aspirationen97} put forward complicated, all-inclusive theory attempting to relate the Irish fricatives to the Brittonic voiced stops, and even brought Brittonic voiceless fricatives descending from geminates into the mix. His theory — especially the Brittonic part — was described as `obscure' and rejected in a review by~\autocite{Str_Erschienene99}. Later, \textcite[§§~149,~303]{Ped_Vergleichende09} proposed the following three-way opposition for the shared ancestor of Goidelic and Brittonic: \xT\ was voiceless aspirated, \ie [pʰ tʰ kʰ]; \lT\ was voiceless, but unaspirated, \ie [p t k]; and \xD\ was voiced, \ie [b d ɡ]. Later again, \textcite{LP_Concise37} are silent about it altogether~\autocite[§~131]{jackson_language_1953}.

Even if Pedersen's theory as a whole never found traction, he was not the only scholar to propose this three-way distinction:
\tqt{In Britannic single stops underwent a change of character after vowels. Probably in all dialects the voiceless stops […] first became unaspirated lenes, which were then voiced […] at an early period in some dialects. The old voiced stops […], on the other hand, became spirants […]. In Irish, on the other hand, single \textit{c} and \textit{t} after vowels in native words turn into the spirants \textit{ch} and \textit{th} […], which in certain circumstances become voiced \textit{ɣ} and \textit{δ}.}{Thu_grammar46}{§~915}
Thurneysen proposes this system in order to understand how the phonology of Brittonic and British Latin influenced the stop orthography of Old Irish, but his proposal also presents us with a viable common ancestor of the Goidelic and Brittonic lenited voiceless stops. At any rate, he expects a only a short period in which \lT\ and \xD\ were pronounced differently, and there is no evidence whether he conceives of them as separate phonemes, or as pre-apocope allophones.

\section{Lenition and gemination}
\label{sec:martinet}

\Textcite{martinet_celtic_1952} describes how lenition arose in Insular Celtic and Western Romance, and connects it to the feature of gemination, \ie the existence of long consonants such as \lat{-cc-} in \glat[cheek]{bucca}. He describes lenition as the appearance of a set of allophones, whereby every consonants may be articulated in two different ways: as a weaker articulation in intervocalic contexts, and as its original articulation in non-leniting contexts. Crucially, this development does not entail the creation of new phonemes, so no new distinctions may be expressed as a result~\autocite[192]{martinet_celtic_1952}.

It is only following syncope and apocope — the loss of some internal and all final syllables — that we may speak of lenited phonemes rather than allophones, or lenition as a morphophonemic aspect of the Celtic languages. The term `lenition' has been used to refer to either process, or both, yet `the use of the same word to designate two synchronically quite different phenomena is apt to create confusion'~\autocite[193–194]{martinet_celtic_1952}.

The phonetic stage of lenition obviously occurred before its phonemicisation, but Martinet is unsure how much earlier it occurred. It could either be considered an early, \gls{pc}, development, or a later Pan-Celtic areal feature developing due to parallel development, a common substrate or spread from one dialect to another. He leans towards a Pan-Celtic scenario, due to the problem of different treatments of \lT\ in Goidelic and Brittonic, where intervocalic \pc{-t-} ultimately yields [θ] in Goidelic, but [d] in Brittonic. And it is difficult to infer [d] from [θ], or vice versa. Yet he then notes:
\tqt{But this of course is not decisive: intervocalic \textit{t} may have been weakened in Proto-Celtic, let us say, to a voiceless media (a lenis stop) from which both [θ] and [d] developed at a later date\footnote{I am unsure what Martinet means by ‘voiceless media (a lenis stop)’. The term ‘media’ is a dated synonym for a voiced stop consonant, yet the term is preceded by ‘voiceless’, its antonym.}.}{martinet_celtic_1952}{195}
Martinet is the first to posit a short lenited allophone of \lT\ which differed from \xD\, which equalled neither its Goidelic reflexes /f θ x/ nor its ultimate Brittonic reflex /b d g/. There is no evidence that Martinet believed these shared Goidelic-Brittonic lenis stops ever survived to be a phoneme in post-apocope Celtic.

He is able to posit a \lT\ allophone differing in length because he connects lenition with gemination. Martinet argues that it was under the pressure of these geminated consonants  that the articulation of singletons was relaxed. Subsequently, unweakened single consonants merged with geminates, and not with their weakened counterparts~\autocite[212]{martinet_celtic_1952}. This reidentification gives insight into how radical and lenited consonants were different. Geminates were pronounced long, so the radical consonants which came to be identified with geminates must have been long also, compared to their lenited counterparts. Thus, the distinction between radical and lenited must originally have been one of length.

One way in which the reidentification of geminates with unlenited consonants played out was the merger of \lT\  with \xD, which I argue did not occur word-initially, but did occur elsewhere. Martinet gives a convincing account of how and why this merger occurred in Romance languages, and notes that his arguments are equally applicable to Brittonic. The explanation for this early merger lies in the rarity of voiced geminates in Brittonic as well as in Greek, Germanic, and Latin: 
\tqt{Geminated voiced stops [in Brittonic] probably existed, but hardly at other points than at morphemic junctures. The situation must have been very much like the one which prevailed in Latin, where the gemination of surds was common both at the juncture of morphemes (as in \textit{at-tingo, ap-pello}) and elsewhere (as in \textit{bucca, puppa, mitto}), whereas geminated voiced stops were most exceptional except at morpheme juncture (as in \textit{ag-ger, ab-brevio, red-do}). Even here there was a tendency to eliminate them as soon as the feeling for composition became blurred; cf.\ \textit{credō} as opposed to Skt.\ \textit{\c{c}rad-dadhāmi}, and later \textit{reddō > rendo} with dissimilation, probably suggested by \textit{pre(he)ndō}; cf.\ also Ital.\ \textit{argine} `dam levee' from \textit{agger} (once \textit{arger}). A very similar situation must have prevailed in classical Greek, where \textit{-ππ-, -ττ-, -κκ-, -πφ-, -τθ-, -κχ-} are frequent (both as the reflexes of normal sound shifts and in hypocoristic formation as a result of some expressive process) but where \textit{-ββ-, -δδ-, -γγ-} are so exceptional nothing prevented the use of \textit{-γγ-} for [ŋg]. A tendency to unvoice geminates must have existed in the older stages of Germanic, at least in cases of expressive gemination, as is shown for instance by the geminated surd of OE \textit{liccian}, OHG \textit{lecchōn}, as opposed to Goth.\ \textit{bilaigon} (with regular \textit{-g-} from *\textit{-\^gh-}; cf.\ Gk.\ \textit{λείχω}, OIr.\ \textit{ligim}, etc.).
}{martinet_celtic_1952}{198}
In short: voiced geminates in other positions than word-initial were so rare that the functional value of a three-way stop distinction would have been minimal, so they merged with lenited voiceless stops. Hence, a word like \gpc{*ad-beros} > \pc{*abberos} > \gmw[estuary]{aber} was spelled as \mw{aper} in \gls{ow}, because the word would have merged with a hypothetical \pc{**aperos} by this stage, and because \gls{ow} would orthographically represent any sound existing in a lenited form using its radical counterpart\footnote{This point is confirmed in Section~\ref{bdgwithptc}}.

The consequence of this development is that \lT\ and \xD\ were only distinguished in those environments where they both appeared with a somewhat comparable frequency: word-initially\footnote{There is no order-of-magnitude difference in the amount of words starting with \eg \mow[]{t-} and \mow[]{d-} in a typical Welsh text.}. Martinet does not make this point himself, but it follows logically from his point\footnote{He implies the opposite, \ie that word-initial \xD\ and \lT\ also merged, when he describes the evolution of Brittonic stops, he simply notes that `\textit{-p-, -t-, -k-} were voiced to \textit{-b-, -d-, -g-}' without mentioning the position of these consonants within a word~\autocite[198]{martinet_celtic_1952}.}.
% He does note, however, that `the phonemic stability of word initials was restored by the analogical extension of one and the same phoneme to all syntactic situations'~\autocite[212]{martinet_celtic_1952} in Western Romance. Thus he implies % what exactly?
Proposing that a merger only occurred non-word-initially requires a definition of `word', which is a non-trivial matter\footnote{This point is discussed in Chapter~\ref{cha:some-phon-issu}.}. 

% \todo[inline]{If I had infinite space, I should also discuss gemination and spirantisation from the perspective of Greene, Jackson, Schrijver, Isaac, Sims-Williams and Thomas. I do not however, so I need to decide how to concisely summarise this discussion, if I mention it at all. There is no point in dropping their names without engaging with their argument in any form.}

\section{Experimental evidence from Breton}
\label{sec:falchun}
\Textcite{falchun_systeme_1951} was first author to propose that word-initial \lT\ and \xD\ were indeed separate phonemes at some point in Brittonic. This insight came from experimental evidence. He noted that word-initial lenited voiceless stops and unlenited voiced stops are kept separate by length in his own Breton dialect of Le Bourg Blanc.
\tqt{\begin{french}
    Elles [les occlusives sonores initiales] sont toujours fortes, à moins qu'elles no proviennent de la mutation de \textit{p, t, k}. Pour plus de clarté, nous les transcrirons par \textit{bb, dd, gg}, et réserverons les signes \textit{b, d, g} pour les occlusives intervocaliques, ou initiales résultant de la mutation \textit{p, t, k}.
  \end{french}
  \footnote{They [the initial voiced stops] are always strong, at least when they do not come from the mutation of \mob{p, t, k}. For extra clarity, we transcribe them with \mob{bb. dd. gg}, and reserve the signs \mob{b, d, g} for the intervocalic stops, or intial stops resulting from the mutation of \mob{p, t, k}.}}{falchun_systeme_1951}{63}
Using Falc'hun's notation, Breton has \mob{bb, dd, gg} for word-initial \xD, while word-medial and word-final \gls{D}, as well as word-initial \lT, are represented by \mob{b, d, g}. The effect of having different phonemes initial positions is that some minimal pairs may be discerned. He gives the following examples:
\begin{mwl}
  \langc[]{\cite[64]{falchun_systeme_1951}}{\mob{tôrrèd éó é gār; tôrrèd éó é ggār}\footnote{Note that this example is not technically a minimal pair, because the /r/ is long in \mob[car]{karr}, but short in \mob[leg]{gar}. However, this difference may be neutralised word-finally~\autocite[34]{carlyle_syllabic_1988}.}}{his car is broken; her leg is broken}
  \langc[]{\cite[64]{falchun_systeme_1951}}{\mob{trṓèd éó e dūr; trṓèd éó e ddūr}}{his tower is tilted; he has turned into water}
  \langc[]{\cite[64]{falchun_systeme_1951}}{\mob{kwēźed éó é bāz; kwēźed éó é bbāz}}{his cough has dropped; her stick has dropped}
\end{mwl}
% \tqt{\begin{french}
%   Prenons les deux suivantes, qui ont été expérimentées, avec bien d'autres du même genre, sur des auditoires bretonnants du Léon, du Tréguier ou de la Cornouaille:

%   1. \textit{(tôrrèd éó é g\={\c{a}}r)}; 2. \textit{(tôrrèd éó é gg\={\c{a}}r)}. Les bretonnants traduisent sans hésitation: 1. «Sa charette à lui est cassée», \textit{torret eo e garr.} 2. «Sa jambe à elle est cassée», \textit{torret eo he gar.}
% \end{french}
% \footnote{Let us take the following two well-tested phrases, and there are plenty of the same type, to Breton-speaking audiences from Léon, Tréguier, or from Cornouaille:

% 1. \textit{(tôrrèd éó é g\={\c{a}}r)}; 2. \textit{(tôrrèd éó é gg\={\c{a}}r)}. Breton-speakers translate without hesitation: 1. ``his car is broken'', \textit{torret eo e garr.} 2. ``her leg is broken'', \textit{torret eo he gar.}  }}{falchun_systeme_1951}{64}
The distinct pronunciation of these consonants has some limitations. Following a consonant, \xD\ and \lT\ merge, and the differentiation is not always historically grounded:
\tqt{
  \begin{french}
    Toutefois, la distinction n'est possible qu'après voyelle. Le \textit{d} initial de \textit{an dud}, «les gens», de \textit{tud} ne diffère rien de celui de \textit{an dour}, «l'eau».  
    Et si une occlusive sonore provenant de la mutation de \textit{p, t, k}, se trouve à l'initiale absolue, elle se prononce forte, ainsi le \textit{b} de \textit{breman}, «à présent», qui provient du \textit{p} de \textit{pred}, «moment».
    Le \textit{b} sera doux dans \textit{abred, (abr\c{ē}d)}, «de bonne heure», ais fort dans \textit{ha breman (a bbr\`{\c{ē}}mã)}, «et à présent», parce que \textit{ha}, réduction de \textit{hag}, n'adoucit pas la consonne suivante […].
  \end{french}\footnote{At any rate, the distinction is only possible following a vowel. The initial \mob{d} of \mob[the people]{an dud}, from \mob{tud} does not differ at all from that of \mob[the water]{an dour}. And if a voiced stop originating from the mutation of \mob{p, t, k} is found at absolute initial position, it is pronounced strongly, thus the \mob{b} in \mob[at present]{breman}, which comes from \mob{p} of \mob[moment]{pred}. The \mob{b} is soft in \mob[in time]{abred, (abr\c{ē}d)}, but strong in \mob[and at present]{ha breman (a bbr\`{\c{ē}}mã)}, because \mob{ha}, reduction of \mob{hag}, does not lenite the following consonant.}}{falchun_systeme_1951}{64} 
Falc'hun's examples involving \mob[moment]{pred} and related words  are important in understanding how the Breton differentiation between \xD\ and \lT\ works on a synchronic level. Historically, \mob{bremañ} is the lenited form, being lenited because it forms an adverbial clause. At some point, lenition was petrified, meaning  \mob{b} was reanalysed as the radical form. The fact that \xD\ and \lT\ are differentiated elsewhere allows us to confirm that it was indeed reanalysed as such. This reanalysis results in initial \xD\ and not \lT, which shows us that \lT\ is only differentiated from \xD\ as long as the speaker lays a connection with \xT. In other words: a Breton speaker only uses \lT\ if his vocabulary contains the same word beginning with \xT. Apparently, there is no such a word as \mob{**premañ}, and the connection with \mob[moment]{pred} is lost. Hence, this example shows us that the difference between \lT\ and \xD\ is only found as a feature of \gls{morphophonlen}, and instances of \lT\ lenited as \gls{petr} are treated as \xD\footnote{ Section~\ref{sec:lenition} treats the differences between these types of lenition.}. The instance of \mob[in time]{abred} shows us that \lT\ and \xD\ are only differentiated word-initially, and reanalysis of word boundaries may cause the distinction between \lT\ and \xD\ to collapse\footnote{Section~\ref{sec:indet-word-separ} discusses what a word is.}.


% Falc'hun gives the following measurements on word-initial stop length:

% \tqt{\begin{french}
%   Des enregistrements ont permis de vérifier les durées des occlusives sonores \`a lintérieur de la phrase, mais au début du mot: 

%   \begin{tabular}{rr}
      %       \textit{bb} 8,6 centisecondes (13 ex.) & \textit{b} 6,8 centisecondes (10 ex.) \\
      %       \textit{dd} 9,5 centisecondes (12 ex.) & \textit{b[d]} 5,6 centisecondes (11 ex.) \\
      %       \textit{gg} 8,5 centisecondes (10 ex.) & \textit{g} 5,2 centisecondes (14 ex.) \\
      %     \end{tabular}%

      %       A l'intervocalique dans le mot, apr\`es voyelle accentuée, la durée moyenne des occlusives sourdes \textit{p, t, k} a été de 10,80; celle des sonores \textit{b, d, g}, de 5,64.
      %       \end{french}}{falchun_systeme_1951}{65}

      % %       Question: duration of what exactly, in terms of phonetics? Answer: most likely, the duration the airway remains closed before the release of the air signifying the stop.

      % %       Question: does this phenomenon of lengthening voiced stops occur purely as a measure to disambiguate length only where ambiguity might otherwise arise (most notably after \textit{e} `his' or `her', depending on the presence or absence of lenition), or is there a quantitative opposition between long and short voiced stops word-initially throughout Le Bourg Blanc Breton? The former situation would argue for Harvey's position that the opposition was analogically reintroduced, since using lenition exclusively to disambiguate echoes later grammatical innovations such as syntactic lenition more than it does inherited patterns. Falc'hun  implies this position in his quote two paragraphs below.

      %       \tqt{\begin{french}Les occlusives sonores fortes ont tendance à s'assourdir: on entend fréquemment\textit{ (va t\={\c{u}}é) va Doue} «mon Dieu!» pour \textit{(va dd\={\c{u}}é)}. Même quand elles sont entièrement sonores, leur explosion peut être suivie d'un souffle sourd, qui dure jusqu'à 5 centisecondes (cf. infra p. 159). Leur hauteur explosive est toujours plus grande que celle de \textit{b, d, g}. \end{french}}{falchun_systeme_1951}{65}

      %       The above quote seems to imply a relationship between fortition and distinguishing unlenited voiced stops and lenited voiceless stops.
According to Falc'hun, the ability to distinguish lenited voiceless stops from unlenited voiced stops is dependent on understanding lenition as a system, which gives a synchronic explanation for why the distinction \lT\ and \xD\ only occurs word-initially:
\tqt{\begin{french}
    Cette opposition entre deux séries d'occlusives sonores, l'une forte et l'autre douce, ne joue dans la langue qu'un rôle négligeable. Son existence fait cependant mieux comprendre la logique du mécanisme des mutations tel qu'il sera décrit plus loin. 
    % Sans elle, dans un système consonantique où toute consonne est forte ou douce, on ne saurait où classer \textit{b, d, g,} qui sont des douces, puisque provenant de l'adoucissement de \textit{p, t, k,} et qui seraient en même temps des fortes, puisque s'adoucissant elles-mêmes en \textit{v, z, h}.
  \end{french}\footnote{
    This opposition between two series of voiced stops, one strong and the other weak, only plays a negligible role in the language. Its existence nevertheless allows for a better understanding of the logic of the mechanism of the mutations as it will be described further on.}}{falchun_systeme_1951}{65}
He then notes that radical \mob{n-, l-, r-} are fortis consonants, and they are identical to medial long \mob{nn, ll, rr} following the accent, while lenited \mob{n-, l-, r-} are identical to short \mob{n, l, r} following the accent. These long and short sonorants are similarly phonemically distinct, so they form minimal pairs:
\begin{mwl}
  \langc[]{\cite[66]{falchun_systeme_1951}}{\mob{ãnn īni nnĕ̀ta; ãnn īni nĕ̀ta}}{the cleanest (male) one; the cleanest (female) one}
\end{mwl}
Falc'hun thus produces evidence for a Brittonic dialect maintaining the distinction between word-initial \lT\ and \xD. However, he shows that this distintion is firmly embedded into how lenition operates as a morphophonemic process synchronically. He also shows that this distinction of length between radical and lenited is not restricted to the stop system, because \mob{n, l, r} also employ length to distinguish between radical and lenited.

%%% end of falc'hun%%% SUBSEQUENT LITERATURE ON FALC'HUN SHOULD ONLY INCLUDE WHAT IS RELEVANT TO BRETON, BUT NOT TO BRITTONIC AS A WHOLE. THUS: JACKSON AND PROBABLY HARVEY SHOULD NOT BE DISCUSSED HERE, BUT CARLYLE, KENNARD, AND IOSAD DEFINITELY SHOULD.

%%% CARLYLE
A further experimental study on Léonais Breton comes from~\textcite[27--28]{carlyle_syllabic_1988}, whose data on the dialect of Lanhouarneau confirm the existence of a length contrast in both resonants and stop consonants. She also confirms that it is phonetically best described as a difference in duration. This contrast is similarly employed to distinguish between \lT\ and \xD.

\Textcite{carlyle_syllabic_1988} offers a series of assumptions and derivation rules by which this three-way distinction may be derived. She argues that word-medial fortis obstruents, \ie \gls{T}, are long and lenis obstruents, \ie \gls{D}, are short; the distinction in voice is secondarily supplied by a redundancy rule~\autocite[46]{carlyle_syllabic_1988}. Lenition is simply the process of degemination, \ie shortening. She argues that the same underlying structure is found word-initially: \gls{D} is secondarily voiced because it is short, while \gls{T}, which is long, is not. In word-initial position, moreover, there is a rule according to which obstruents lengthen. This word-initial gemination is applied in absolute initial position, but also following an element which does not cause lenition. It is not applied following lenition, however. These rules together allow for a phonological distinction of \lT\ and \xD, as is shown in Table~\ref{tab:carlylederiv}.
\begin{table}[h]
  \centering
  \begin{tabular}{lllll}
    \toprule
    & `stick'      & `his stick'  & `cough'      & `his cough' \\
    \midrule
    Underlying form & \mob{pas}  & \mob{e\gls{l} pas} & \mob{pːas} & \mob{e\gls{l} pːas} \\
    Lenition & \mob{pas}  & \mob{e\gls{l} fas} & \mob{pːas} & \mob{e\gls{l} pas} \\
    Voicing & \mob{bas}  & \mob{e\gls{l} vas} & \mob{pːas} & \mob{e\gls{l} bas} \\
    Word-initial gemination & \mob{bːas} & \mob{e\gls{l} vas} & \mob{pːas} & \mob{e\gls{l} bas} \\
    \bottomrule
  \end{tabular}%
  \caption{Overview of phonological rules causing distinct \lT\ and \xD\ according to \textcite{carlyle_syllabic_1988}.}
  \label{tab:carlylederiv}
\end{table}

The result of the application of these synchronic rules is that word-medial \gls{D} is identified with word-initial \lT, and word-initial \xD\ is a separate phoneme. This is expected under Martinet's account, who argued that word-medial \xD\ must have merged with \lT\ early on.

It should be noted that both Falc'hun's findings and Carlyle's model constitute only a synchronic description of their respective Breton dialects. They do not describe to what extent this three-way stop distinction was applicable to earlier stages of Brittonic. The observations given above are also far from universal even within present-day Breton, as Jackson found:
\tqt{Since initial consonants are always short (lenis) there is no initial distinction of fortis-lenis, and therefore no system such as that described by Falc'hu for Le Bourg Blanc. He gives \textit{he bbaz} <<her stick>> and \textit{he ggar} <<her leg>>, from \textit{baz} and \textit{gar}, versus \textit{e baz} <<his cough>> and \textit{e garr} <<his cart>>, from \textit{paz} and \textit{karr} […]. % Similarly, \textit{an hini nneta, an hini llousa}, and \textit{an hini rruz} when masculine, but \textit{an hini neta, an hini lousa,} and \textit{an hini ruz} when feminine, and compare \textit{he lleur} <<her threshing-floor>> versus \textit{e leur} <<his threshing-floor>>, etc. […]
  At Plougrescant, \textit{he baz} <<her stick>> and \textit{e baz} <<his cough>> are both [\textit{i \textsuperscript{l}ba̤·s}]; […]
  % and there is no difference between \textit{an hini nevez, an hini lousañ,} and \textit{an hini ruz} whether masculine or feminine. [̣\dots]
  In Falc'hun's system the distinction must be phonemic, since it carries with it a distinction in meaning which the speakers recognise, but at Plougrescant there is certainly no difference and therefore no phonemic difference.}{Jac_Phonology61}{332}



\section{From Breton to Brittonic}
\label{sec:jackson}
Falc'hun's measurements of length distinctions between radical and lenited consonants quickly found their way into historical linguistics. \Textcite[§~132]{jackson_language_1953} proposes a Common Celtic language with comparatively long consonants in absolute initial position, internal geminates, and in certain consonant clusters. Consonants initially after proclitics ending in vowels, or internally between vowels or in combination with some resonants were comparatively short sounds. He describes the next step as follows:
\tqt{What seems to have happened is that at a certain stage yet to be determined the Common Celtic short consonants, being mostly invervocal, underwent a loosening or weakening of articulation which resulted in the voiceless stops \textit{p , t, c} become voiced to \textit{b, d, g}; the voiced stops \textit{b, d, g} becoming the spirants \textit{ƀ, đ, ʒ} […] The long consonants, however, whether intervocal or in absolute initial, were energetic enough to resist this loosening and remained unaffected at first; though later and as a quite separate evolution \textit{-pp-, -tt-}, and \textit{-cc-} became \textit{f, th, ch}, and \textit{-bb-, -dd- -gg-} were simplified. The half-long consonants in initial position have lasted to the present day in Breton, being now fully long, but in Welsh they were subsequently shortened. So, for example, […] Brit.~*\textit{adbero-} > *\textit{abbero-} gave Welsh \& Breton \textit{aber}, ``river-mouth'', in which the geminate resisted lenition, but was later shortened and so fell together with \textit{b} the lenition of \textit{p}.
}{jackson_language_1953}{§~132}
From this we may gather that Jackson had roughly the following relative chronology of lenition for stop consonants:
\begin{enumerate}
\item Shortening  of intervocalic non-geminate consonants;
\item Voicing of shortened \pbr{p t k} to \pbr{b d g}; spirantisation of \pbr{b d g} to \pbr{β ð ɣ};
\item Spirantisation of \pbr{pp tt kk} to \pbr{f th ch}; simplification, \ie shortening of \pbr{bb dd gg};
\end{enumerate}
These steps preceded apocope according to~\textcite[§~133]{jackson_language_1953}.
% Although he is equipped with Falc'hun's experimental results, Jackson fails to see here what Martinet sees: the connection between the first and the last step, \ie between lenition and gemination.
Taken at face value, Jackson would have intervocalic non-geminate consonants shorten, but the shortening of intervocalic geminates would have to wait until after the Goidelic-Brittonic split so that it can be considered in one go with spirantisation. The result of these developments was that Jackson envisages a three-way length allophonic distinction of voiceless stops before apocope and spirantisation kicked in. For example, phoneme /t/ would have three realisations: [t] in intervocalic position, [T] in absolute initial position, and [TT] for intervocalic geminates and following resonants\footnote{This full array of symbols is used in \textcite{Jac_Gemination60} and \textcite{Jac_Historical67}.}. The first of these would be lenited later and the third spirantised. Jackson needs such a three-way opposition in order to account for the difference between initial stops remaining stops and geminates becoming spirants. 

\Textcite{greene_gemination_1956}, following \textcite{martinet_celtic_1952}, notes how Jackson fails to see that geminates can only be long in opposition to the shortness of non-geminates. It is therefore nonsense to speak of the simplification of voiced geminates centuries after non-geminates were shortened intervocalically. \Textcite{Gre_Spirant66} notes that it is impossible to envisage a threefold quantitative distinction, and Jackson's [TT] and [T] must be considered a single length grade. The later reflexes of Jackson's [TT, T] as voiceless spirants and voiceless stops, respectively, was because it was normalised as morphophonemic mutation in only a limited range of cases.

\Textcite{jackson_language_1953}'s \emph{Language and History in Early Britain} primarily focuses on dating sound changes. Jackson also attempted to do this with lenition, which makes it important to know what exactly he understood to be lenition:
\tqt{What is meant, then, by the date of lenition is the time when e.g.~\textit{t} became a full \textit{d} in British and a full \textit{th} in Irish, and these were felt as phonemes distinct from non-lenited \textit{tt, t(t)-}.
}{jackson_language_1953}{§~132}
So what he seemingly tries to date (in one go) is the merger of \lT\ and \xD\ when he talks about \pbr[]{t} becoming full \pbr[]{d}, but also when they were distinct phonemes. I take the second criterion to mean the moment when lenited and radical allophones were sufficiently different that they could be reinterpreted as separate phonemes, rather than them actually being separate post-apocope phonemes. This is because when he sets out the evidence for his date, the first thing he notes is:
\tqt{The chief certain fact about the \textit{relative} dating of lenition in Brittonic is that it must be older than the loss of final syllables, since otherwise consonants which came to stand at the ends of words by that loss would no longer be intervocal.}{jackson_language_1953}{§~133}
Apparently, the only way Jackson is able to conceive \lT\ being sufficiently different from \xT\ is by merging with \xD. Yet there is no reason why this should be the case, and Falc'hun has shown how three different series of consonants may stand in a three-way distinction. One way to make sense of Jackson's choice to define lenition as the merger of \lT\ and \xD\  is to look at his evidence for his date of lenition. His evidence comprises loans from Brittonic and British Latin into Goidelic, inscriptions, personal names and place-names~\autocite[§§~133–134]{jackson_language_1953}. None of these pieces of evidence may shed light on what exactly the morphophonemic process of lenition did to initial consonants at any given moment. The only thing that can be learned from this evidence is at what point \lT\ and \xD\ merged word-medially and word-finally. \Textcite[§~142]{jackson_language_1953} concludes with `the Late British lenition of \textit{p, t, c, b, d, g, m} to \textit{b, d, g, ƀ, đ, ʒ, μ} took place in the second half of the fifth century'\todo{pa mor dda ydy dyddiad Jackson yn cadw lan heddiw?}. Given the nature of his evidence, I take his date to refer only to the non-initial merger of \lT\ with \xD\footnote{The date may also refer to when the remaining lenited consonants became spirants, but this matter is outside the scope of this thesis.}.

For the word-initial merger of \lT\ with \xD, \textcite{jackson_language_1953} does not give evidence one way or another. In the absence of evidence that word-initial \lT\ was distinct from \xD, one could well analogously assert that the word-initial merger occurred at the same as elsewhere. Yet \textcite{martinet_celtic_1952} demonstrated how non-lenited voiced stops in other positions than word-initial were exceedingly rare, while they were reasonably common word-initially. At the same time, \textcite{falchun_systeme_1951} gave insight into how a word-initial contrast between \xD\ and \lT\ might work synchronically. 

\todo[inline]{mae'r geiriau isod yn rhy llym, ond dyma beth dw i eisiau dweud}
\Textcite{greene_gemination_1956} is the first to notice how Jackson's theory is untenable because it fails to follow Martinet in equating gemination with non-lenition. Greene argues convincingly that, in \gls{oir}, there was no length distinction between any of the stops. This is to be expected, because all lenited stops in Goidelic became fricatives, so the cumulative result of (1) the realignment of single/geminate to lenited/radical as described by Martinet; and (2) the Goidelic realisation of lenited stops as fricatives was that formerly long stops were  realised as stops in \gls{oir} and formerly short stops were realised as fricatives. This means that the formerly long stops had no other series of stops to which it could stand in opposition, so it is nonsense to speak of these stops as either long or short.

For Brittonic, the second step described above never occurred, but the first certainly did. Greene criticises Jackson for considering the spirantisation of geminate voiceless stops and the simplification of geminate voiced stops a single development, even though they are quite different in realisation. Greene then proposes that Falc'hun's description of Breton may be applied to \gls{pbr}:
\tqt{{[}T]he Brythonic languages possessed after the lenition two sets of mediae but only one set of tenues. One set of mediae was strong, \textit{B, D, G}, representing original initial and geminated sounds; the other weak, \textit{b, d, g}, representing the lenited forms of \textit{p, t, k}. This system still survives in Breton and has been described by Falc'hun.}{greene_gemination_1956}{289}

Greene was quite right in his criticism, and his proposal was simple enough. However, 

\subsection{Harvey}\todo{I should move extensive discussion of Harvey elsewhere, and merely introduce his position in the introduction}
\label{harveylenition}
Harvey argues that \xD\ and \lT\ had already merged in early Brittonic. Features of Le Bourg Blanc Breton may be explained by analogy.
                                                                                                                                                                                                                                                                                                                                                                                                                                                                                                                                                                                                                                                                                                                                                                                                                                                                                                                                             %                                                                                                                                                                                                                                                                                                                                                                                                                                                                                                                                                                                                                                                                                                                                                                                                                                                                                                                                              \tqt{It is no longer possible, from synchronic considerations, to tell whether a given internal intervocalic voiced stop (\textit{g, b, d}) is the product of the lenition of an unvoiced simplex (/k, p, t/) or of the non-lenition of a voiced genimate (/gg, bb, dd/); hence the disagreement about whether Welsh, Cornish, and Breton \textit{ober} `work' derives from Latin \textit{opera} or from a native /obber-/ < /od+ber-/.}{harvey_aspects_1984}{96}
                                                                                                                                                                                                                                                                                                                                                                                                                                                                                                                                                                                                                                                                                                                                                                                                                                                                                                                                             %                                                                                                                                                                                                                                                                                                                                                                                                                                                                                                                                                                                                                                                                                                                                                                                                                                                                                                                                              What Harvey states here is correct for Modern Welsh and Middle Welsh: the \textit{b} in e.g.\ \mw{ebol}, which goes back to lenited \textit{p},  and \mw{aber}, which goes back to geminate \textit{bb}, had already merger in both the phonology and the orthography of \gls{ow}~\autocite[33]{schrijver_old_2011}. 


\tqt{This treatment [i.e.\ apocope] also destroys the distinction between the relevant elements of examples [/V+\textit{d}\"{u}ːd/] and [/V+\textit{d}uβr/], since if gemination has indeed disappeared, the lenition of an unvoiced stop gives the same phoneme as the non-lenition of the corresponding voiced one and both are now in identical environments in initial sandhi. But once again, we find that this prediction is generally borne out in practice: thus in Welsh `his stags' (\textit{ceirw}, plural of \textit{carw}) and `her waves' (\textit{geirw}) are indistinguishably \textit{ei geirw}. The only counter-evidence comes from the Breton of le Bourg Blanc, where an opposition based on gemination exists in this position (for example, the \textit{g} in \textit{torret eo he gar} `her leg is broken' is pronounced much longer than that in \textit{ torret eo e garr} (<\textit{karr}) `his cart is broken'). But this phenomenon is not attested in any other Modern Celtic dialect, and even in this one it is not found in other environments such as internal intervocalic, the relevant stops […] having merged there as elsewhere. It thus seems likely that the alternation is not original but is the result of analogical extension of the mutation system
  \begin{center}
    \begin{tabular}{l@{~=~}l}
      \textsc{radical}  & [+gemination]\\
      \textsc{lenition} & [-gemination]
    \end{tabular}
  \end{center}
  to the stops from the resonants \textit{l(l), n(n), and r(r)}, which, as we have seen, are its proper domain because in their case the character of the opposition was not altered at the time of lenition.}{harvey_aspects_1984}{96--97}
In short: Harvey believes that lenited voiceless stops and unlenited voiced stops merged in the immediate post-acope period. He provides three arguments why the dissimilarity in Breton \lT\ and \xD\ are not inherited. The first is that it is only attested in Le Bourg Blanc Breton, and in none of the other modern Celtic languages. The second is that it is only found word-initially, but not word-medially. The third argument is that the resonants in Le Bourg Blanc Breton provide a solid analogical base. 

The only Brittonic evidence of a non-merger is in the Breton dialect of le Bourg Blanc, which uses length to distinguish these consonants. The fact that no present-day Celtic languages except for some Breton dialects maintain a distinction between lenited voiceless stops and unlenited voiced stops does not mean that there was none. Absence of evidence is not evidence of absence. Moreover, \textcite{koch_*cothairche_1990} and this thesis  provide several strands of evidence showing that \lT\ and \xD\ were in fact kept separate in Welsh, even if not up until the present day.

The point that the distinction between \lT\ and \xD\ was only maintained word-initially does not by itself mean the system arose as a result of analogy. Pre-apocope lenition operated without respect to word boundaries pre-apocope, and the maintenance of this rule is still why Celtic languages have lenition. However, nasalisation and spirantisation do show difference in application word-internally and across word boundaries at times. Compare, for example, the development of the consonant cluster *\mw{ntr} in Welsh \mw[nail]{cethr} < \glat{centrum}, where this cluster becomes /θr/ as a result of spirantisation, and \mow{fy nhref} < \gpbr[my town]{*men trebā}, where /-n tr-/ becomes /-n̥r/ after nasalisation\footnote{For more information on the details of this process, see \textcite{schrijver_spirantization_1999,isaac_chronology_2004}.}. Additionally, the very fact that this three-way stop distinction is only found word-initially is shared between Le Bourg Blanc Breton and the earliest Middle Welsh. It therefore suggests that the phonology of the distinction between lenited voiceless and unlenited voiced stops developed separately word-initially and word-medially, and that they may have merged word-medially as early as the common Brittonic period.


The argument that resonants provided a solid analogical base in using length to distinguish fortis from lenis is true. However, a three-way stop distinction is cumbersome. This is hardly the kind of system one should expect from analogy, which is a force that simplifies things rather than complicating them. This same cumbersomeness is not found in these resonants themselves, which typically give phonemic contrast based on length, but do not also have a voiced-voiceless distinction to complicate the system into a three-way contrast. More generally, the precise fact that a three-way stop distinction is so counter-intuitive as a result of analogy means that it is simpler to assume that this cumbersome system arose as a result of a linguistic process which does not necessarily simplify matters: phonological change. Therefore, it is simpler to regard the distinction between \lT\ and \xD\ as a direct result of phonemicisation of post-apocope consonants, which was subsequently eroded in most of the Brittonic dialects.  

Harvey bases his argument that analogy drove the distinction between \lT\ and \xD\ on the sonorants \mw{r, l, }and \mw{n}. These consonants have two phonemes each: a long fortis, and a short lenis. Naturally then, it is length that distinguishes fortis from lenis here. This identification of length as a marker for fortis quality was then expanded to voiced stops. This matter is of relevance to Welsh, precisely because Welsh does not use length to distinguish fortis and lenis consonants for \mw{r} and \mw{l}. Rather, voice is the common factor in distinguishing \graph{rh} and \graph{r}, and \graph{ll} and \graph{l}, respectively, while \graph{n} does not distinguish between fortis and lenis word-initially\footnote{\gls{mw} does distinguish /nː/, /rː/, and /lː/ from their short counterparts. This distinction is lost before the end of the Middle Welsh period, see also: \textcite[127--128]{schumacher_mittel-_2011}. The relevance of this phonemic opposition for the opposition between lenited voiceless stops and unlenited voiced stops is doubtful, however, because sonorants only use length to distinguish fortis and lenis in non-initial position, whereas it is precisely in word-initial position that lenited voiceless stops and unlenited voiced stops are kept separate in the Early \gls{mw} orthography and in Le Bourg Blanc Breton.}. Resonants  \graph{ll} /ɬ/ and \graph{l} /l/ also differ in manner of articulation: the former sound is a lateral fricative while the latter is a lateral approximant. 

For Middle Welsh, \textcite*{koch_*cothairche_1990} believes that the primary phonetic property distinguishing lenited voiceless stops and unlenited voiced stops was voice: lenited voiceless stops were voiceless (but unaspirated) and unlenited voiced stops were voiced. Early Middle Welsh and Le Bourg Blanc Breton apparently both distinguish \lT\ and \xD, but realize this distinguishing in a radically different way phonetically. For both Welsh and Breton, this distinction runs analogous to the distinction between fortis and lenis sonorants. Welsh sonorants have voiceless and voiced variants, and by the same mechanism are lenited voiceless stops and unlenited voiced stops kept separate. Breton sonorants have long and short variants, and lenited voiceless stops and unlenited voiced stops are similarly distinguished by length. If the Breton opposition between stops came into being as a result of analogy with sonorants, then the same would sensibly the case for Welsh. For reasons of economy, however, it is more sensible to regard the Welsh and Breton systems as having a common origin. The break-up of Brittonic into Welsh and Breton constitutes continuity in the phonemic oppositions here, and only phonetic innovation in one of the languages needs to be hypothesised\footnote{Loss of aspiration yielding length distinction in Breton is easily justifiable, given the absence of aspiration in the whole of continental western Europe and particularly France.}. 

\subsection{Koch}
\label{kochvoiceless}
\Textcite{koch_*cothairche_1990} gives several strands of evidence supporting the non-merger of word-initial lenited voiceless stops and radical voiced stops in Welsh. According to Koch, lenited voiceless stops did not merge with unlenited voiced stops until some time into the historical period, and were likely separate consonants up until part of the Early \gls{mw} period. The evidence for this is that these consonants did not merge up until after the split between Western Brittonic and Southwestern Brittonic, since some Breton dialects still distinguish between them: 
\tqt{As Falc'hun has shown, these series never completely converged in Breton, at least not in some dialects, where the historical voiced fortes still contrast with the historical voiceless lenes in minimal pairs by a clinically measurable difference in duration: e.g. [\c{e} g:a:r] `her leg' vs. [\c{e} ga:r] `his car'.}{koch_*cothairche_1990}{\S26} 

Because this merger had not yet taken place, it would not be sensible to write lenited voiceless stops as voiced stops. This in part explains why \gls{ow} orthography does not represent lenition. Koch notes that this non-representation of lenition of voiceless stops is completely consistent in \gls{ow} and \gls{ob}, although many errors would be expected if lenited voiceless stops and unlenited voiced stops had already merged by this point. Moreover, Koch states, the cynghanedd demonstrates that these sets only fell together by the end of the fourteenth century~\autocite[\S26]{koch_*cothairche_1990}. In terms of exact phonetic reality, Koch proposes the following three-way distinction in the immediate post-acope period: \tqt{lenited [\bd, \gd, \dd] was distinct from radical [b-, g-, d-] by being voiceless and from radical [pʰ-, kʰ-, tʰ-] by being unaspirated.}{koch_*cothairche_1990}{\S30} This account is phonetically different from the one proposed by \textcite{falchun_systeme_1951} and \textcite{jackson_language_1953} in that Koch proposes voice as the variable distinguishing lenited voiceless stops from radical voiced stops, whereas the earlier authors propose length. 

In addition to the above, Koch states:
\tqt{it is not common in the languages of the world for the duration of consonant segments in absolute phrase-initial position to be phonologically significant. A Common Neo-Brittonic phonemic opposition of \textit{b-, g, d-} = /\textit{b:-, g:-, d:-}/ vs \textit{-p-, -c-, -t-} = /\textit{-b-, -g-, -d-}/ is accordingly unlikely, though a phonetic concomitant of duration was probably present. Anglo-Saxon borrowings of the \textit{Cerdic} type suggest that voicing in internal position was as old as the Migration Period, but it is likely that Brittonic speakers were still keeping the series distinct on the basis of \textbf{degree of voicing}.}{koch_*cothairche_1990}{\S29} 

I have three comments about this\todo{These comments should be part of a chapter discussing the phonetics of my thesis}. The first is that absolute initial positions do not exist for lenited consonants during the Common Neo-Brittonic period: lenition exists as a feature of the preceding word (and syntactically conditioned types of lenition are later developments), so a lenited voiceless stop always has a preceding phonological context of some sort in this stage.  We are therefore left with two possible scenario's: either the merger of \xD\ and \lT\ was a post-apocope development, or the merger of \xD\ and \lT\ was followed by a sound law disambiguating these consonants in word-initial contexts. The former scenario is postulated by Koch (and generally accepted), while the latter scenario is proposed by \textcite{harvey_aspects_1984}, but, crucially, both scenario's are implied to be post-apocope innovations.

The second is that what we call voiced stops in Modern Welsh are not in fact defined by voicedness, and are typically unvoiced. Their quality is determined by non-aspiration, which is exactly the phonetic value Koch had in mind for lenited voiceless stops. Accepting Koch's proposal concerning the phonetic difference between \lT\ and \xD\ implies that present-day word-initial \xD\ is phonetically more similar to \lT\ than to \xD. In other words, the phonological merger of \lT\ and \xD\ was the result of the phonetic shift in \xD\ towards \lT.

Thirdly, a quantitative distinction is exactly what we see in Modern Breton dialects. If we see a pattern of quantitative distinction in those Modern Brittonic dialects that distinguish lenited voiceless stops and radical voiced stops, then surely it is not such a stretch to assume the same for Common Neo-Brittonic. However, the Breton phonetic oppositions may have emerged under Latin influence~\autocite[31]{schrijver_old_2011}.


\section{Voicing versus aspiration}
\label{sec:voic-vers-aspir}

%%% On voicing and aspiration in Breton.
Many Breton dialects have voicing rather than aspiration to distinguish voiced from voiceless~\autocite[221]{Ios_Representation13}. \Textcite[114]{Bot_Etude82} finds only voicing contrast in Argol Breton. \Textcite[177--178]{Hum_Phonologie95a} explicitly denies that aspiration may be present in Bothoa.

\Textcite[9]{carlyle_syllabic_1988} finds that the degree of voicing may vary from speaker to speaker.

\Textcite[157--167]{falchun_systeme_1951} finds at least some aspiration in  Léon Breton. \Textcite{Ter_Grammaire70} finds traces of it in Vannetais.


\section{Phonetics and phonology of lenition in present-day Welsh} 
Unlike Breton, present-day Welsh does not preserve a three-way stop distinction either word-initially or elsewhere. Nevertheless, the way in which the contrast between voiced and voiceless stops is realised phonetically differs between dialects. In some Southeastern dialects of Welsh, the consonants themselves have phonetically merged, but they are kept separate by the length of neighboring vowels: e.g.\ \mw{ebol} /'eːpol/ `foal' v.s. \mw{capel} /'kapel/ `chapel'~\autocite[85]{awbery_phonotactic_1984}. One may wonder whether preceding vowel length similarly played a role in disambiguating the three Early Middle Welsh stop series as it does in disambiguating the two stop series remaining in Modern SE Welsh, and perhaps the modified survival of such a pattern in southeastern Welsh implies that this system survived for longer in these dialects. However, the idea that preceding vowel length may have served to disambiguate lenited voiceless stops from unlenited voiced stops has one drawback: the distinction between three different stop series was only maintained word-initially. The vowel lengthening would therefore have occurred in the word preceding the word to be lenited. The phonetic upshot of this would be (to use Koch's examples~\autocite*[§~26]{koch_*cothairche_1990}): [\c{e}ː g(ː)ar] `her leg' vs. [\c{e} gar] `his car'. This supposition is problematic, because only stressed syllables distinguish between short and long vowels, and the posessive pronouns in the above examples are both unstressed. 

\subsection{Schrijver}
\label{sec:schrijver}

The following Old British stop system has been reconstructed so far: 
\begin{table}[h]
  \centering
  \begin{tabular}{lll}
    \begin{tabular}[c]{@{}l@{}}{[}-voice{]}\\ {[}long{]}\\ {[}asp{]}\end{tabular} & \begin{tabular}[c]{@{}l@{}}{[}+voice{]}\\ {[}long{]}\end{tabular} & {[}short{]} \\\hline
    p{[}ʰː{]} & bː & b \\
    t{[}ʰː{]} & dː & d \\
    k{[}ʰː{]} & gː & g
  \end{tabular}
  \caption{The Old British stop system between the 7th and 11th centuries according to \textcite[33]{schrijver_old_2011}.}
  \label{oldbritishconsonantsystem}
\end{table}



\section{Voiceless stops after verbs}

                     %                      This work deals with the system of postverbal lenition in Early Middle Welsh poetry. In particular, it deals with the type of lenition following verbs ending in \mw{–ei} and \mw{oes}\footnote{It should be noted that there is no known phonetic or phonological distinction between voiceless stops following \mw{s} and when their voiced counterparts follow \mw{s} in Welsh. Erratic behaviour of lenition following \oes\ is therefore problematic: if a scribe could not hear the distinction between lenited and unlenited consonants in this position, lack of lenition in writing is expected, but not exclusively following \oes. As such, the opposition in lenition between examples \ref{storch} and \ref{sbont} may be merely orthographical in nature.}. 
\Textcite{van_development14} uncovers a relevant pattern after two Middle Welsh verbal endings: third person singular \ei\ and existential verb \oes\ are followed by lenition, but words starting with a voiceless stop are not typically lenited. The following examples serve to illustrate the system:
\begin{mwl}
  \mwc{\acrshort{wbr}~2.2-4}%
  {ac ual ẏ llathrei ỽynnet ẏ cỽn ẏ llathrei cochet ẏ clusteu}%
  {And as white as the dogs shone, so red shone their ears.}%
  \mwc{\acrshort{wbr}~481.22}%
  {canẏt oes lestẏr ẏn ẏ bẏt a dalhẏo ẏ llẏn cadarn hỽnnỽ namẏn hi.}%
  {Since there is no vessel in the world that may keep the strong drink except for this one.}%
  \mwc[storch]{\acrshort{wbr}~483.23}%
  {Nẏt oes torch ẏn ẏ bẏt a dalhẏo ẏ gẏnllẏuan namẏn torch canastẏr kanllaỽ}%
  {There is no collar in the world that may hold the leash except for the collar of Canastr Canllaw.}%
\end{mwl}
So far, this pattern has been found in the White Book recensions of \mw{Culhwch ac Olwen} and \mw{Pwyll Pendeuic Dyuet}. 
What makes these two verbal endings special is that they are the only ones causing lenition as a contact lenition. Contact lenition may as a rule be considered an inherited grammatical feature that may be traced back to pre-apocope Brittonic, when lenition was not yet phonemicized. All other types of lenition, by contrast, occur depending on the grammatical relationship of the verb to the following word and may not be traced back to when lenition was allophonic. The latter type of lenition is typically found as lenition of the object or nominal predicate, and is not limited to consonants other than voiceless stops~\autocite[70]{van_development14}.

This system of limited lenition following \ei\ and \oes\ had already broken down by the time the White Book of Rhydderch itself was written. In the White Book recension of \mw{Branwen Uerch Lyr}, lenition following these verbs is haphazard, and no regularity may be discerned in it~\autocite[42]{van_development14}. Furthermore, the scribe departs from this system on a handful of occasions, which suggests that this feature was not a part of the scribe's writing, but merely copied this feature from an earlier text. These exceptions imply that this system where voiceless stops are not lenited disappeared at some point in the Middle Welsh period. Therefore, it must have disappeared at some point in time between the date of composition of these tales and when the White Book of Rhydderch was composed. The date of composition of these tales is still debated, e.g.\ by \textcite*{rodway_date_2005}, but the date the manuscript itself was written may be dated fairly safely to the middle of the fourteenth century~\autocite[228]{huws_medieval_2000}. 

The fact that postverbal contact lenition was not written with \graph{b, d, g} in \mw{Culhwch ac Olwen} and \mw{Pwyll Pendeuic Dyuet} in these fossilized cases implies that these letters could not be used to represent lenited voiceless stops, which are written with \graph{b, d, g} already. This, in turn, implies that lenited voiceless stops had not yet merged with unlenited voiced stops when these stories were written down. I will discuss the context and the implications of this issue in further detail in Chapter \ref{postverballenition}. 
                     %                      A description of the lenition found after these verbs is given by \textcite{morgan_y_1952}. He analyses these verbal endings as causing lenition unconditionally, irrespectively of consonant type or grammatical function of the following word. Exceptions to this rule are explained as the result of imperfect representation of lenition in the orthography. 
                     %                      \Textcite{van_development14}, however, finds that lenition following these verbs is not wholly unconditioned in the earliest Middle Welsh prose. Rather, these verbs only cause lenition to consonants other than voiceless stops. 

\section{Related issues}
\todo{To be expanded, obviously}
\subsection{The related issues of gemination and spirantization}
\cite{jackson_language_1953,martinet_celtic_1952,schrijver_spirantization_1999,isaac_old-_2004}

Greene on gemination, with comments in square brackets: 
\tqt{One of the difficulties about the spirant mutation is that the forms of it are not those normally given by \textit{-sk-, -sp-, -st-} in word-interior~; a second is that it affects only the tenues [=voiceless stops]. Jackson deals with the first point that a final lost Σ (from -s) caused a gemination, and that the resulting \textit{pp, tt, kk} later gaven \textit{f, th, ch,} a change well attested from word-interior. But the lack of a similar development for the mediae [=voiced stops] is left unexplained, except for the suggestion that ``all such geminate groups tended to become simiplified quite early in Pr.WCB, and in initial position no doubt much earlier than elsewhere''. Furthermore, the spirant mutation must be connected, not only with the reduction of internal \textit{-pp-, -tt-, -kk-} to \textit{f, th, ch}, but also with the change of ungeminated \textit{p, t, k} to the same sounds after \textit{r, l}. There is no doubt that these changes which took place after the lenition and the loss of final syllables ; Jackson places them in the ``mid or later sixth century'' in his chronology and gives convincing reasons.

  I think the origin of these changes may be found in the fact that the Brythonic languages possessed after lenition two sets of mediae but only one set of tenues. One set of mediae was strong \textit{B, D, G,} representing original initial and geminated sounds ; the other weak, \textit{b, d, g,} representing the lenited forms of \textit{p, t, k}. This system still survives in Breton and has been described by Falc'hun. On the other hand, the tenues had only \textit{P, T, K,} representing original initial and geminated sounds and single unlenited \textit{p, t, k} after \textit{l} and \textit{r} ; in this system there could be no opposition of \textit{P} and \textit{p} and the oriiginal strong sound was weakened to \textit{p}. I take it that consonants in sandhi with final \textit{-s, -t, -k} in Welsh, as well as final \textit{-n} in Breton, were preserved strong, in all cases. After the loss of final syllables the tenues were weakened for lack of opposition and were then further weakened to \textit{f, th, ch} in all leniting positions—a weakening of the same type as the earlier lenition of the mediae. The same weakening occurred after particles which now ended in a vowel (final \textit{-s, -t, -k} and, in Breton, \textit{-n}, having dropped before consonants)—e.g.\ \textit{y} ``her'', \textit{tra, a}, and so on. […] I take the development of this mutation as more ore less contemporary with the rise of provection. It was the latter process which provided a new series of tenues (\textit{p} from \textit{b + b, b + h}, etc.) in inlaut ; these are still geminated after the accent in \gls{mow}, and it is not impossible that it was their appearance which hastened the weakening of the old tenues—at least it is certain that the two series are never confused.

  This Brythonic evidence justifies the statement that, once lenition had become phonemic in insular Celtic, the opposition \textit{geminated : single} was replaced by the opposition \textit{unlenited : lenited} and gemination ceased to have any phonemic function [except that the phonemic function of gemination was that it caused the morphophonemic rules governing spirantization, which may hardly be called `ceasing of phonemic function' PS].
}{greene_gemination_1956}{288--289}

%%% Local Variables:
%%% mode: latex
%%% TeX-master: "../main"
%%% End:
