\chapter{Introduction — phonology}
\label{cha:introduction-phonology}

All present-day Celtic languages share grammatical lenition, whereby the first consonant of a word may change into a more sonorous one as a result of its morphosyntactic environment. The way in which a consonant is made into a more sonorous one differs between the Goidelic and the Brittonic branches of Celtic. In Goidelic, all stop consonants turn into their fricative counterparts, while present-day Brittonic voiceless stops turn into their voiced counterparts, and voiced stops turn into their respective fricatives. Table~\ref{tab:lenitionwelshirish} shows the differing treatments with  Welsh and  Irish as examples.

\begin{table}[h]
  \centering
  \begin{tabular}{lllll}
    \toprule
    & \multicolumn{2}{c}{Modern Welsh} & \multicolumn{2}{c}{Old Irish} \\
    & Radical & Lenited & Radical & Lenited \\\midrule
    \gls{T} & /p t k/ & \tikz[remember picture,anchor=base,baseline=(current bounding box.base)]{\node[draw,rounded corners,fill=black,fill opacity=0.1, text opacity=1](lt){/b d ɡ/};}   & /p t k/ & /f θ x/  \\
    \gls{D} & \tikz[remember picture,anchor=base,baseline=(current bounding box.base)]{\node[draw, rounded corners,fill=black,fill opacity=0.1, text opacity=1](xd){/b d ɡ/};} &  /v ð \zero/  & /b d ɡ/ & /β ð ɣ/ \\
    \bottomrule
  \end{tabular}%
  \caption[Radical and lenited stop consonants in Welsh and Irish.]{Radical and lenited stop consonants in Welsh and Irish. The chart is simplified for Irish in that it does not include both broad and slender phonemes. }
  \label{tab:lenitionwelshirish}%
\end{table}%

The shaded areas in Table~\ref{tab:lenitionwelshirish} highlight a merger in Welsh. Judging from their Irish counterparts alone we can see that lenited voiceless stops (\lT) and radical voiced stops (\xD) must at some point have been separate.  I argue that \lT\ and \xD\ were still separate in Welsh up until halfway the Middle Ages, and that the phonetic variables keeping these phonemes apart are recoverable. Part~\ref{part:phonology-phonetics} of this thesis treats which phonological variables kept these sounds apart in Welsh, and Part~\ref{part:orthography} treats the date when \lT\ and \xD\ separated. This chapter introduces the history of the idea that \lT\ did not equal \xD\ word-initially.

\section{A two-stage development of lenition}
\label{sec:two-stage-devel}
A typical historical grammar describing lenition is \textcite{Mor_Welsh13}, who describes it as follows:
\tqt{Brit.\ and Lat. \textit{p, t, k, b, d, g, m} between vowels became \textit{b, d, g, f, δ, ᵹ, f} respectively in W.}{Mor_Welsh13}{§~103}
This description summarises the totality of changes between \gls{pc} and \gls{mow} correctly, but does not break it up into intermediate stages. 
When he posits that present-day Brittonic languages /p t k/ lenite to /b d ɡ/, he says two things: the first is that lenition applied to voiceless stops at some point in the history of Brittonic; the second is that the phonemes to which these voiceless stops lenited merged with the pre-existing phonemes /b d ɡ/. These two developments are often taken together. They are taken together because that is the final result in present-day Brittonic, but not breaking these two developments up creates a few problems.

One obvious problem is Irish. If the two Brittonic developments occurred at the same time, then lenition in Goidelic and Brittonic cannot be described as a single development, because Goidelic voiceless stops lenite into /f θ x/, and it is hard to derive the Goidelic  voiceless fricatives from the Brittonic voiced stops, or vice versa. Another problem is Brittonic-internal. If voiceless stops after lenition came to equal unlenited voiced stops immediately, then would these not have to be lenited into lenited voiced stops in turn? 

This issue was recognised  by \textcite{loth_les_1892}, who proposed that voiceless stops underwent lenition at a later date than other consonants:
\tqt{\textfrench{Les explosives sonores \textit{b, d, g}, entre deux voyelles ont dû \textit{commencer} leur mouvement vers les spirantes correspondantes, avant que les explosives sourdes, \textit{p, t, c} ne fussent devenues \textit{b, d, g}; autrement, celles-ci auraient eu le même sort qu'elles. Si le latin \textit{opera} avait donné \textit{obera} au moment où \textit{labore} était encore \textit{labure}, le \textit{b} d'\textit{ober} eût été traité comme celui de \textit{labur}; c'est-à-dire fût devenu \textit{v}; on aurait aujourd'hui \textit{over, lavur} et non \textit{ober, lavur}. Des trois explosives sonores, \textit{g} paraît la première être devenue spirante.}\footnote{The voiced stops \textit{b, d, g} between two vowels should have \emph{started} their movement towards their corresponding spirants before the voiceless stops became \textit{b, d, g}; otherwise, these ones would have had the same fate as those. If Latin \textit{opera} had given \textit{obera} at the moment when \textit{labore} was still \textit{labure}, the \textit{b} of \textit{ober} would have been treated as the one of \textit{labur}; \ie it would have become \textit{v}; today we would have \textit{over, lavur} and not \textit{ober, lavur}. Of the three voiced stops, \textit{g} seems to be the first to have become a spirant.}}{loth_les_1892}{87}
Loth's explanation explains why we do not have anything like /p/ > /b/ > /v/, but it has its drawbacks. Lenition may be described as a single process whereby intervocalic consonants become more sonorous, so it would be uneconomical to suggest two separate events of lenition depending on the consonant. Incidentally, his account solves the issue of Goidelic lenition:  lenited voiceless stops to voiceless spirants could be just one more independent development of lenition. Yet by now we have to assume three independent events to account for all the lenition in Goidelic and Brittonic.

\Textcite[162]{Foer_Flussname41} and \Textcite{sims-williams_dating_1990} also argue for two separate stages of lenition, with /p t k/ undergoing lenition at a later date than other consonants.
% \tqt{I shall argue that `lenition' — the conventional name for the spirantization of [b d g m] > [β δ γ μ] and voicing of [p t k] > [b d g] — occurred in two distinct stages: (1) spirantization, then (2) voicing […]. This will seem heretical to Brittonicists, who instinctively regard [t] > [d], etc., and [d] > [δ], etc., as a single phenomenon because they constitute the morphophonemic alternation of `lenition' or `soft mutation'.}{sims-williams_dating_1990}{221}
They both apply Loth's line of reasoning on how /p t k/ cannot have preceded other consonants because that would have led to these consonants being lenited twice over~\autocite[\eg][232]{sims-williams_dating_1990}. This shows how  lenition of voiceless stops is often silently equated with the merger of lenited voiceless stops with radical voiced stops. Förster's argument is rejected by \textcite[§~131]{jackson_language_1953}, who argues that original \pbr{b, d, g} could well have shifted away from their original pronunciation as \pbr{p, t, k} was shifting towards \pbr{b, d, g}. \Textcite[5]{Tho_Brythonic90} argues that Jackson's scenario would have led to `a certain amount of inappropriate feeding and bleeding', but \textcite[243]{Rus_Introduction95} notes that such feeding and bleeding could not possibly have happened, because that would have led to lexical chaos. In my view, Jackson meant to describe a chain shift, which is a well-known cross-linguistic phenomenon that does not necessarily cause feeding and bleeding. In a chain-shift scenario, intervocalic /p/ and /b/ would shift at the same time in a push chain or a drag chain. If we allow for chain shifts, Loth and Förster only demonstrate that voicing of voiceless stops did not occur prior to the moment when lenited voiced stops became fricatives, but not necessarily that the fricativisation of voiced stops occurred afterwards.  Sims-Williams' argument also rests on early loans from Brittonic into Goidelic, but this analysis is rejected by \textcite{isaac_chronology_2004}, because Sims-Williams conflates the roles  phonetics and phonology play in the process of loaning.

Not all lenited consonants differ between Goidelic and Brittonic. Lenited voiced stops, for example, became voiced fricatives in both branches. It would be economical to posit a scenario where the common ancestor of these branches also had the lenited allophones of \gls{D} realised as fricatives. In such a scenario, the date at which Goidelic and Brittonic lenited allophones of \gls{T} were realised as voiceless fricatives and voiced stops, respectively, must be uncoupled from the date lenited allophones of \gls{D} became fricatives. However, in such a scenario, \gls{T} would still be expected to have both radical and lenited allophones.  After these considerations, a two-stage development of lenited consonants may be posited, but the change occurring in two stages is not the creation of radical and lenited allophones, or the phonemicisation thereof, but the allophonic realisation of these lenited consonants. The first stage would include the realisation of lenited allophones of \gls{D} as fricatives, and the second stage would comprise the different realisations of \lT\ in Brittonic and Goidelic. This raises the question what allophone could yield both Goidelic and Brittonic \lT.

\Textcite{Ped_Aspirationen97} put forward a complicated, all-inclusive theory attempting to relate the Irish fricatives to the Brittonic voiced stops, and even brought Brittonic spirantisation — the voiceless fricatives descending from geminates — into the mix, but the theory found few supporters\footnote{His theory — especially the Brittonic part — is described as `obscure' and rejected in a review by~\textcite{Str_Erschienene99}.}. A later evolution of his theory can be found in a short note in  \textcite[§§~149,~303]{Ped_Vergleichende09}. Here, he proposes the following three-way opposition for the shared ancestor of Goidelic and Brittonic: \xT\ was voiceless aspirated, \lT\ was voiceless, but unaspirated, and \xD\ was voiced\footnote{\Textcite{LP_Concise37} are silent about it altogether~\autocite[§~131]{jackson_language_1953}.}. Table~\ref{tab:pedersenstops} gives an overview of this system.

\begin{table}[h]
  \centering
  \begin{tabular}{lllllll}
    \toprule
    &/kʷ = p & k & t & b & d & ɡ/ \\\midrule
    Radical &[kʷʰ = pʰ& kʰ& tʰ& b & d & ɡ] \\ 
    Lenited &[kʷ = p & k & t & β & ð & ɣ]\\
    \bottomrule
  \end{tabular}
\caption{Radical and lenited allophones of \gls{pc} stops after phonetic lenition according to \textcite[§§~149,~303]{Ped_Vergleichende09}.}
\label{tab:pedersenstops}
\end{table}


Even though Pedersen's theory as a whole never found traction, he is not the only scholar to propose this three-way distinction:
\tqt{In Britannic single stops underwent a change of character after vowels. Probably in all dialects the voiceless stops […] first became unaspirated lenes, which were then voiced […] at an early period in some dialects. The old voiced stops […], on the other hand, became spirants […]. In Irish, on the other hand, single \textit{c} and \textit{t} after vowels in native words turn into the spirants \textit{ch} and \textit{th} […], which in certain circumstances become voiced \textit{ɣ} and \textit{δ}.}{Thu_grammar46}{§~915}
Thurneysen proposes this system in order to understand how the phonology of Brittonic and British Latin influenced the stop orthography of Old Irish, but his proposal also presents us with a viable common ancestor of the Goidelic and Brittonic lenited voiceless stops. At any rate, he expects only a short period during which \lT\ and \xD\ were pronounced differently, and there is no evidence that he conceives of them as separate phonemes rather than pre-apocope allophones.

\section{Lenition and gemination}
\label{sec:martinet}

\Textcite{martinet_celtic_1952} describes how lenition arose in Insular Celtic and Western Romance, and connects it to the feature of gemination. Gemination is the presence of long stop phonemes due to the existence of doubled consonants such as \lat{-cc-} in \glat[cheek]{bucca}. He describes lenition as the appearance of a set of allophones, whereby every consonant may be articulated in two different ways: as a weaker articulation in intervocalic contexts and as its original articulation in non-leniting contexts. Crucially, this development does not entail the creation of new phonemes, so no new distinctions may be expressed as a result~\autocite[192]{martinet_celtic_1952}.

It is only following syncope and apocope — the loss of some internal and all final syllables — that we may speak of lenited phonemes rather than allophones, or lenition as a morphophonemic aspect of the Celtic languages. The term `lenition' has been used to refer to either process, or both, yet `the use of the same word to designate two synchronically quite different phenomena is apt to create confusion'~\autocite[193–194]{martinet_celtic_1952}.

The phonetic stage of lenition obviously occurred before its phonemicisation, but Martinet is unsure how much earlier it occurred. Lenition may either be considered an early \acrlong{pc} development, or a later Pan-Celtic areal feature developing due to parallel development. Such a parallel development could be the result of a common substrate or spread from one dialect to another. He leans towards a Pan-Celtic scenario, due to the problem of different treatments of \lT\ in Goidelic and Brittonic, where intervocalic \pc{-t-} ultimately yields [θ] in Goidelic, but [d] in Brittonic. It is difficult to infer [d] from [θ] or vice versa. Yet he then notes:
\tqt{But this of course is not decisive: intervocalic \textit{t} may have been weakened in Proto-Celtic, let us say, to a voiceless media (a lenis stop) from which both [θ] and [d] developed at a later date\footnote{I am unsure what Martinet means by ‘voiceless media (a lenis stop)’. The term ‘media’ is a dated synonym for a voiced stop consonant, yet the term is preceded by ‘voiceless’, its antonym.}.}{martinet_celtic_1952}{195}
These words demonstrate that Martinet is the first to posit lenited allophones of \gls{T} differing in length from unlenited allophones of \gls{D}, and which equalled neither its Goidelic reflexes /f θ x/ nor its ultimate Brittonic reflexes /b d g/. There is no evidence that Martinet believed these shared Goidelic-Brittonic lenis stops ever survived to be a phoneme in post-apocope Celtic.

He is able to posit a \lT\ allophone differing in length because he connects lenition with gemination, which allowed for length distinction in the first place. Martinet argues that it was under the pressure of these geminated consonants  that the articulation of singletons was relaxed. Subsequently, unweakened single consonants merged with geminates, and not with their weakened counterparts~\autocite[212]{martinet_celtic_1952}. This reidentification gives insight into how radical and lenited consonants were different. Geminates were pronounced long, so the radical consonants which came to be identified with geminates must have been long also, compared to their lenited counterparts. Thus, the distinction between radical and lenited must originally have been one of length.


One way in which the reidentification of geminates with unlenited consonants played out was the merger of \lT\  with \xD. I argue that this merger occured much later word-initially than it did elsewhere. Martinet gives a convincing account of how and why this merger occurred in Romance languages, and notes that his arguments are equally applicable to Brittonic. The explanation for this early merger lies in the rarity of voiced geminates in Brittonic as well as in Greek, Germanic, and Latin: 
\tqt{Geminated voiced stops [in Brittonic] probably existed, but hardly at other points than at morphemic junctures. The situation must have been very much like the one which prevailed in Latin, where the gemination of surds was common both at the juncture of morphemes (as in \textit{at-tingo, ap-pello}) and elsewhere (as in \textit{bucca, puppa, mitto}), whereas geminated voiced stops were most exceptional except at morpheme juncture (as in \textit{ag-ger, ab-brevio, red-do}). Even here there was a tendency to eliminate them as soon as the feeling for composition became blurred; cf.\ \textit{credō} as opposed to Skt.\ \textit{\c{c}rad-dadhāmi}, and later \textit{reddō > rendo} with dissimilation, probably suggested by \textit{pre(he)ndō}; cf.\ also Ital.\ \textit{argine} `dam levee' from \textit{agger} (once \textit{arger}). A very similar situation must have prevailed in classical Greek, where \textit{-ππ-, -ττ-, -κκ-, -πφ-, -τθ-, -κχ-} are frequent (both as the reflexes of normal sound shifts and in hypocoristic formation as a result of some expressive process) but where \textit{-ββ-, -δδ-, -γγ-} are so exceptional nothing prevented the use of \textit{-γγ-} for [ŋg]. A tendency to unvoice geminates must have existed in the older stages of Germanic, at least in cases of expressive gemination, as is shown for instance by the geminated surd of OE \textit{liccian}, OHG \textit{lecchōn}, as opposed to Goth.\ \textit{bilaigon} (with regular \textit{-g-} from *\textit{-\^gh-}; cf.\ Gk.\ \textit{λείχω}, OIr.\ \textit{ligim}, etc.).
}{martinet_celtic_1952}{198}
In short: voiced geminates in other positions than word-initial were so rare that the functional value of a three-way stop distinction would have been minimal, so they merged with lenited voiceless stops. Hence, a word like \gpc{*ad-beros} > \pc{*abberos} > \gmw[estuary]{aber} was spelled as \mw{aper} in \gls{ow}, because the word would have merged with a hypothetical \gpc{**aperos} by this stage, and because \gls{ow} would orthographically represent any sound existing in a lenited form using its radical counterpart\footnote{This point is confirmed in Section~\ref{bdgwithptc}}. This merger may have been early according to Martinet, but it must have postdated the Brittonic-Goidelic split, because voiced geminates and lenited voiceless stops have distinct reflexes in Old Irish. Underlined phonemes in Table~\ref{tab:goidvoicedgems} show the differing reflexes of \lT\ and \xD\ non-word-initially in Irish, but not in Welsh.

\begin{table}[h]
  \centering
    \begin{tabular}{lllll}
    \toprule
      & \tchh{\lT} & \tchh{\xD} \\
      & \tchh{`brother'} & \tchh{`believes'} \\    \midrule
    Old Irish & \oi{bráthair} & /braː\al{θ}arʲ/  & \oi{creitid} & /kʲrʲe\al{dʲ}əðʲ/ \\
    Middle Welsh & \mw{brawd} & /brau\al{d}/  & \mw{cred} & /kre\al{d}/ \\
    \bottomrule
    \end{tabular}%
    \caption{Realisation of \lT\ and \xD\ word-medially in Irish and Welsh.}
    \label{tab:goidvoicedgems}%
\end{table}%

The consequence of this development is that \lT\ and \xD\ were only distinguished in those environments where they both appeared with a somewhat comparable frequency: word-initially. There is no order-of-magnitude difference in the amount of words starting with \eg radical \mow[]{t-} and \mow[]{d-} in a typical Welsh text. Martinet does not make this point himself, but it follows logically from his argument\footnote{He implies the opposite, \ie that word-initial \xD\ and \lT\ also merged, when he describes the evolution of Brittonic stops, he simply notes that `\textit{-p-, -t-, -k-} were voiced to \textit{-b-, -d-, -g-}' without mentioning the position of these consonants within a word~\autocite[198]{martinet_celtic_1952}.}.
% He does note, however, that `the phonemic stability of word initials was restored by the analogical extension of one and the same phoneme to all syntactic situations'~\autocite[212]{martinet_celtic_1952} in Western Romance. Thus he implies % what exactly?
Proposing that a merger only occurred non-word-initially requires a definition of `word', which is a non-trivial matter which is discussed in Chapter~\ref{cha:some-phon-issu}. 

% \todo[inline]{If I had infinite space, I should also discuss gemination and spirantisation from the perspective of Greene, Jackson, Schrijver, Isaac, Sims-Williams and Thomas. I do not however, so I need to decide how to concisely summarise this discussion, if I mention it at all. There is no point in dropping their names without engaging with their argument in any form.}

\section{Experimental evidence from Breton}
\label{sec:falchun}
\Textcite{falchun_systeme_1951} was first author to propose that word-initial \lT\ and \xD\ were indeed separate phonemes at some point in Brittonic. This insight came from experimental evidence. He noted that word-initial lenited voiceless stops and unlenited voiced stops are kept separate by length in his own Breton dialect of Le Bourg Blanc.
\tqt{\begin{french}
    Elles [les occlusives sonores initiales] sont toujours fortes, à moins qu'elles no proviennent de la mutation de \textit{p, t, k}. Pour plus de clarté, nous les transcrirons par \textit{bb, dd, gg}, et réserverons les signes \textit{b, d, g} pour les occlusives intervocaliques, ou initiales résultant de la mutation \textit{p, t, k}.
  \end{french}
  \footnote{They [the initial voiced stops] are always strong, at least when they do not come from the mutation of \mob{p, t, k}. For extra clarity, we transcribe them with \mob{bb. dd. gg}, and reserve the signs \mob{b, d, g} for the intervocalic stops, or intial stops resulting from the mutation of \mob{p, t, k}.}}{falchun_systeme_1951}{63}
Using Falc'hun's notation, Breton has \mob{bb, dd, gg} for word-initial \xD, while word-medial and word-final \gls{D}, as well as word-initial \lT, are represented by \mob{b, d, g}. The effect of having different phonemes initial positions is that some minimal pairs may be discerned. He gives the following examples:
\begin{mwl}
  \langc[]{\cite[64]{falchun_systeme_1951}}{\mob{tôrrèd éó é gār; tôrrèd éó é ggār}\footnote{Note that this example is not technically a minimal pair, because the /r/ is long in \mob[car]{karr}, but short in \mob[leg]{gar}. However, this difference may be neutralised word-finally~\autocite[34]{carlyle_syllabic_1988}.}}{his car is broken; her leg is broken}
  \langc[]{\cite[64]{falchun_systeme_1951}}{\mob{trṓèd éó e dūr; trṓèd éó e ddūr}}{his tower is tilted; he has turned into water}
  \langc[]{\cite[64]{falchun_systeme_1951}}{\mob{kwḗzed éó é bāz; kwḗzed éó é bbāz}}{his cough has dropped; her stick has dropped}
\end{mwl}
% \tqt{\begin{french}
%   Prenons les deux suivantes, qui ont été expérimentées, avec bien d'autres du même genre, sur des auditoires bretonnants du Léon, du Tréguier ou de la Cornouaille:

%   1. \textit{(tôrrèd éó é g\={\c{a}}r)}; 2. \textit{(tôrrèd éó é gg\={\c{a}}r)}. Les bretonnants traduisent sans hésitation: 1. «Sa charette à lui est cassée», \textit{torret eo e garr.} 2. «Sa jambe à elle est cassée», \textit{torret eo he gar.}
% \end{french}
% \footnote{Let us take the following two well-tested phrases, and there are plenty of the same type, to Breton-speaking audiences from Léon, Tréguier, or from Cornouaille:

% 1. \textit{(tôrrèd éó é g\={\c{a}}r)}; 2. \textit{(tôrrèd éó é gg\={\c{a}}r)}. Breton-speakers translate without hesitation: 1. ``his car is broken'', \textit{torret eo e garr.} 2. ``her leg is broken'', \textit{torret eo he gar.}  }}{falchun_systeme_1951}{64}
The distinct pronunciation of these consonants has some limitations. Following a consonant, \xD\ and \lT\ merge, and the differentiation is not always historically grounded:
\tqt{
  \begin{french}
    Toutefois, la distinction n'est possible qu'après voyelle. Le \textit{d} initial de \textit{an dud}, «les gens», de \textit{tud} ne diffère rien de celui de \textit{an dour}, «l'eau».  
    Et si une occlusive sonore provenant de la mutation de \textit{p, t, k}, se trouve à l'initiale absolue, elle se prononce forte, ainsi le \textit{b} de \textit{breman}, «à présent», qui provient du \textit{p} de \textit{pred}, «moment».
    Le \textit{b} sera doux dans \textit{abred, (abr\c{ē}d)}, «de bonne heure», ais fort dans \textit{ha breman (a bbr\`{\c{ē}}mã)}, «et à présent», parce que \textit{ha}, réduction de \textit{hag}, n'adoucit pas la consonne suivante […].
  \end{french}\footnote{At any rate, the distinction is only possible following a vowel. The initial \mob{d} of \mob[the people]{an dud}, from \mob{tud} does not differ at all from that of \mob[the water]{an dour}. And if a voiced stop originating from the mutation of \mob{p, t, k} is found at absolute initial position, it is pronounced strongly, thus the \mob{b} in \mob[at present]{breman}, which comes from \mob{p} of \mob[moment]{pred}. The \mob{b} is soft in \mob[in time]{abred, (abr\c{ē}d)}, but strong in \mob[and at present]{ha breman (a bbr\`{\c{ē}}mã)}, because \mob{ha}, reduction of \mob{hag}, does not lenite the following consonant.}}{falchun_systeme_1951}{64} 
Falc'hun's examples involving \mob[moment]{pred} and related words  are important in understanding how the Breton differentiation between \xD\ and \lT\ works on a synchronic level. Historically, \mob{bremañ} is the lenited form, being lenited because it forms an adverbial clause. At some point, lenition was petrified, meaning  \mob{b} was reanalysed as the radical form. The fact that \xD\ and \lT\ are differentiated elsewhere allows us to confirm that it was indeed reanalysed as such. This reanalysis results in initial \xD\ and not \lT, which shows us that \lT\ is only differentiated from \xD\ as long as the speaker lays a connection with \xT. In other words: a Breton speaker only uses \lT\ if his vocabulary contains the same word beginning with \xT. Apparently, there is no such a word as \mob{**premañ}, and the connection with \mob[moment]{pred} is lost. Hence, this example shows us that the difference between \lT\ and \xD\ is only found as a feature of \gls{morphophonlen}, and instances of \lT\ lenited as \gls{petr} are treated as \xD\footnote{ Section~\ref{sec:lenition} treats the differences between these types of lenition.}. The instance of \mob[in time]{abred} shows us that \lT\ and \xD\ are only differentiated word-initially, and reanalysis of word boundaries may cause the distinction between \lT\ and \xD\ to collapse\footnote{Section~\ref{sec:indet-word-separ} discusses what a word is.}.


% Falc'hun gives the following measurements on word-initial stop length:

% \tqt{\begin{french}
%   Des enregistrements ont permis de vérifier les durées des occlusives sonores \`a lintérieur de la phrase, mais au début du mot: 

%   \begin{tabular}{rr}
%     \textit{bb} 8,6 centisecondes (13 ex.) & \textit{b} 6,8 centisecondes (10 ex.) \\
%     \textit{dd} 9,5 centisecondes (12 ex.) & \textit{b[d]} 5,6 centisecondes (11 ex.) \\
%     \textit{gg} 8,5 centisecondes (10 ex.) & \textit{g} 5,2 centisecondes (14 ex.) \\
%   \end{tabular}%

%   A l'intervocalique dans le mot, apr\`es voyelle accentuée, la durée moyenne des occlusives sourdes \textit{p, t, k} a été de 10,80; celle des sonores \textit{b, d, g}, de 5,64.
% \end{french}}{falchun_systeme_1951}{65}

% % Question: duration of what exactly, in terms of phonetics? Answer: most likely, the duration the airway remains closed before the release of the air signifying the stop.

% % Question: does this phenomenon of lengthening voiced stops occur purely as a measure to disambiguate length only where ambiguity might otherwise arise (most notably after \textit{e} `his' or `her', depending on the presence or absence of lenition), or is there a quantitative opposition between long and short voiced stops word-initially throughout Le Bourg Blanc Breton? The former situation would argue for Harvey's position that the opposition was analogically reintroduced, since using lenition exclusively to disambiguate echoes later grammatical innovations such as syntactic lenition more than it does inherited patterns. Falc'hun  implies this position in his quote two paragraphs below.

      %       \tqt{\begin{french}Les occlusives sonores fortes ont tendance à s'assourdir: on entend fréquemment\textit{ (va t\={\c{u}}é) va Doue} «mon Dieu!» pour \textit{(va dd\={\c{u}}é)}. Même quand elles sont entièrement sonores, leur explosion peut être suivie d'un souffle sourd, qui dure jusqu'à 5 centisecondes (cf. infra p. 159). Leur hauteur explosive est toujours plus grande que celle de \textit{b, d, g}. \end{french}}{falchun_systeme_1951}{65}

      %       The above quote seems to imply a relationship between fortition and distinguishing unlenited voiced stops and lenited voiceless stops.
According to Falc'hun, the ability to distinguish lenited voiceless stops from unlenited voiced stops is dependent on understanding lenition as a system, which gives a synchronic explanation for why the distinction \lT\ and \xD\ only occurs word-initially:
\tqt{\begin{french}
    Cette opposition entre deux séries d'occlusives sonores, l'une forte et l'autre douce, ne joue dans la langue qu'un rôle négligeable. Son existence fait cependant mieux comprendre la logique du mécanisme des mutations tel qu'il sera décrit plus loin. 
    % Sans elle, dans un système consonantique où toute consonne est forte ou douce, on ne saurait où classer \textit{b, d, g,} qui sont des douces, puisque provenant de l'adoucissement de \textit{p, t, k,} et qui seraient en même temps des fortes, puisque s'adoucissant elles-mêmes en \textit{v, z, h}.
  \end{french}\footnote{
    This opposition between two series of voiced stops, one strong and the other weak, only plays a negligible role in the language. Its existence nevertheless allows for a better understanding of the logic of the mechanism of the mutations as it will be described further on.}}{falchun_systeme_1951}{65}
He then notes that radical \mob{n-, l-, r-} are fortis consonants, and they are identical to medial long \mob{nn, ll, rr} following the accent, while lenited \mob{n-, l-, r-} are identical to short \mob{n, l, r} following the accent. These long and short sonorants are similarly phonemically distinct, so they form minimal pairs:
\begin{mwl}
  \langc[]{\cite[66]{falchun_systeme_1951}}{\mob{ãnn īni nnĕ̀ta; ãnn īni nĕ̀ta}}{the cleanest (male) one; the cleanest (female) one}
\end{mwl}
Falc'hun thus produces evidence for a Brittonic dialect maintaining the distinction between word-initial \lT\ and \xD. However, he shows that this distintion is firmly embedded into how lenition operates as a morphophonemic process synchronically. He also shows that this distinction of length between radical and lenited is not restricted to the stop system, because \mob{n, l, r} also employ length to distinguish between radical and lenited.

%%% end of falc'hun%%% SUBSEQUENT LITERATURE ON FALC'HUN SHOULD ONLY INCLUDE WHAT IS RELEVANT TO BRETON, BUT NOT TO BRITTONIC AS A WHOLE. THUS: JACKSON AND PROBABLY HARVEY SHOULD NOT BE DISCUSSED HERE, BUT CARLYLE, KENNARD, AND IOSAD DEFINITELY SHOULD.

%%% CARLYLE
A further experimental study on Léonais Breton comes from~\textcite[27--28]{carlyle_syllabic_1988}, whose data on the dialect of Lanhouarneau confirm the existence of a length contrast in both resonants and stop consonants. She also confirms that it is phonetically best described as a difference in duration. This contrast is similarly employed to distinguish between \lT\ and \xD.

\Textcite{carlyle_syllabic_1988} offers a series of assumptions and derivation rules by which this three-way distinction may be derived. She argues that word-medial fortis stops, \ie \gls{T}, are long while lenis stops, \ie \gls{D}, are short; the distinction in voice is secondarily supplied by a redundancy rule~\autocite[46]{carlyle_syllabic_1988}. Lenition  of voiceless stops and of resonants constitutes degemination, \ie shortening; lenition of \gls{D} is fricativisation. She argues that the same underlying structure is found word-initially: \gls{D} is secondarily voiced because it is short, while \gls{T}, which is long, is not. In word-initial position, moreover, there is a rule according to which obstruents may be lengthened. This word-initial gemination is applied in absolute initial position, but also following an element which does not cause lenition. It is not applied following lenition, however. These rules together allow for a phonological distinction of \lT\ and \xD, as is shown in Table~\ref{tab:carlylederiv}.
\begin{table}[h]
  \centering
  \begin{tabular}{lllll}
    \toprule
    & \mob{baz}    & \mob{e vaz}  & \mob{paz}    & \mob{e baz} \\
    & `stick'      & `his stick'  & `cough'      & `his cough' \\
    \midrule
    Underlying form & \mob{pas}  & \mob{e\gls{l} pas} & \mob{pːas} & \mob{e\gls{l} pːas} \\
    Lenition & \mob{pas}  & \mob{e\gls{l} fas} & \mob{pːas} & \mob{e\gls{l} pas} \\
    Redundant voicing & \mob{bas}  & \mob{e\gls{l} vas} & \mob{pːas} & \mob{e\gls{l} bas} \\
    Word-initial gemination & \mob{bːas} & \mob{e\gls{l} vas} & \mob{pːas} & \mob{e\gls{l} bas} \\
    \bottomrule
  \end{tabular}%
  \caption{Overview of phonological rules causing distinct \lT\ and \xD\ according to \textcite{carlyle_syllabic_1988}.}
  \label{tab:carlylederiv}
\end{table}

The result of the application of these synchronic rules is that word-medial \gls{D} is identified with word-initial \lT, and word-initial \xD\ is a separate phoneme. This is expected under Martinet's account, who argued that word-medial \xD\ must have merged with \lT\ early on.

It should be noted that both Falc'hun's findings and Carlyle's model constitute only a synchronic description of their respective Breton dialects. They do not describe to what extent this three-way stop distinction was applicable to earlier stages of Brittonic. The observations given above are also far from universal even within present-day Breton, as Jackson observes for the dialect of Plougrescant:
\tqt{Since initial consonants are always short (lenis) there is no initial distinction of fortis-lenis, and therefore no system such as that described by Falc'hun for Le Bourg Blanc. He gives \textit{he bbaz} <<her stick>> and \textit{he ggar} <<her leg>>, from \textit{baz} and \textit{gar}, versus \textit{e baz} <<his cough>> and \textit{e garr} <<his cart>>, from \textit{paz} and \textit{karr} […]. % Similarly, \textit{an hini nneta, an hini llousa}, and \textit{an hini rruz} when masculine, but \textit{an hini neta, an hini lousa,} and \textit{an hini ruz} when feminine, and compare \textit{he lleur} <<her threshing-floor>> versus \textit{e leur} <<his threshing-floor>>, etc. […]
  At Plougrescant, \textit{he baz} <<her stick>> and \textit{e baz} <<his cough>> are both [\textit{i \textsuperscript{l}ba̤·s}]; […]
  % and there is no difference between \textit{an hini nevez, an hini lousañ,} and \textit{an hini ruz} whether masculine or feminine. [̣\dots]
  In Falc'hun's system the distinction must be phonemic, since it carries with it a distinction in meaning which the speakers recognise, but at Plougrescant there is certainly no difference and therefore no phonemic difference.}{Jac_Phonology61}{332}



\section{From Breton to Brittonic}
\label{sec:jackson}
Falc'hun's measurements of length distinctions between radical and lenited consonants quickly found their way into historical linguistics. \Textcite[§~132]{jackson_language_1953} proposes a Common Celtic language with comparatively long consonants in absolute initial position, internal geminates, and in certain consonant clusters. Consonants initially after proclitics ending in vowels, or internally between vowels or in combination with some resonants were comparatively short sounds. He describes the next step as follows:
\tqt{What seems to have happened is that at a certain stage yet to be determined the Common Celtic short consonants, being mostly invervocal, underwent a loosening or weakening of articulation which resulted in the voiceless stops \textit{p , t, c} become voiced to \textit{b, d, g}; the voiced stops \textit{b, d, g} becoming the spirants \textit{ƀ, đ, ʒ} […] The long consonants, however, whether intervocal or in absolute initial, were energetic enough to resist this loosening and remained unaffected at first; though later and as a quite separate evolution \textit{-pp-, -tt-}, and \textit{-cc-} became \textit{f, th, ch}, and \textit{-bb-, -dd- -gg-} were simplified. The half-long consonants in initial position have lasted to the present day in Breton, being now fully long, but in Welsh they were subsequently shortened. So, for example, […] Brit.~*\textit{adbero-} > *\textit{abbero-} gave Welsh \& Breton \textit{aber}, ``river-mouth'', in which the geminate resisted lenition, but was later shortened and so fell together with \textit{b} the lenition of \textit{p}.
}{jackson_language_1953}{§~132}
From this we may gather that Jackson had roughly the following relative chronology of lenition for stop consonants:
\begin{enumerate}
\item Shortening  of intervocalic non-geminate consonants;
\item Voicing of shortened \pbr{p t k} to \pbr{b d g}; spirantisation of \pbr{b d g} to \pbr{β ð ɣ};
\item Spirantisation of \pbr{pp tt kk} to \pbr{f th ch}; simplification, \ie shortening of \pbr{bb dd gg};
\end{enumerate}
These steps preceded apocope according to~\textcite[§~133]{jackson_language_1953}.
% Although he is equipped with Falc'hun's experimental results, Jackson fails to see here what Martinet sees: the connection between the first and the last step, \ie between lenition and gemination.
Taken at face value, Jackson would have intervocalic non-geminate consonants shorten, but the shortening of intervocalic geminates would have to wait until after the Goidelic-Brittonic split so that it can be considered simultaneously with spirantisation. The result of these developments was that Jackson envisages a three-way allophonic length distinction of voiceless stops before apocope and spirantisation kicked in. For example, phoneme /t/ would have three realisations: [t] in intervocalic position, [T] in absolute initial position, and [TT] for intervocalic geminates and following resonants\footnote{This full array of symbols is used in \textcite{Jac_Gemination60} and \textcite{Jac_Historical67}.}. The first of these would be lenited later and the third spirantised. Jackson needs such a three-way opposition in order to account for the difference between initial stops remaining stops and geminates becoming spirants. 


\Textcite{jackson_language_1953}'s \emph{Language and History in Early Britain} primarily focuses on dating sound changes. Jackson also attempted to do this with lenition, which makes it important to know what exactly he understood to be lenition:
\tqt{What is meant, then, by the date of lenition is the time when e.g.~\textit{t} became a full \textit{d} in British and a full \textit{th} in Irish, and these were felt as phonemes distinct from non-lenited \textit{tt, t(t)-}.
}{jackson_language_1953}{§~132}
What Jackson tries to date is the merger of \lT\ and \xD, because he talks about \pbr[]{t} becoming full \pbr[]{d}, but he also mentions the date when \lT\ and \xD\ were distinct phonemes. I take the second criterion to mean the moment when lenited and radical allophones were sufficiently different that they could be reinterpreted as separate phonemes, rather than them actually being separate phonemes. This is because when he sets out the evidence for his date, the first thing he notes is:
\tqt{The chief certain fact about the \textit{relative} dating of lenition in Brittonic is that it must be older than the loss of final syllables, since otherwise consonants which came to stand at the ends of words by that loss would no longer be intervocal.}{jackson_language_1953}{§~133}
So apocope is what made word-initial lenited consonants phonemic in the first place. Apparently, Jackson equates  \lT\ being sufficiently different from \xT\ with its merger with \xD. Yet there is no reason why this should be the case, and Falc'hun has shown how three different series of consonants may stand in a three-way distinction. One way to make sense of Jackson's choice to define lenition as the merger of \lT\ and \xD\  is to look at his evidence for his date of lenition. His evidence comprises loans from Brittonic and British Latin into Goidelic, inscriptions, personal names and place-names~\autocite[§§~133–134]{jackson_language_1953}. None of these pieces of evidence may shed light on what exactly the morphophonemic process of lenition did to initial consonants at any given moment. The only thing that can be learned from this evidence is at what point \lT\ and \xD\ merged word-medially and word-finally. \Textcite[§~142]{jackson_language_1953} concludes with `the Late British lenition of \textit{p, t, c, b, d, g, m} to \textit{b, d, g, ƀ, đ, ʒ, μ} took place in the second half of the fifth century'\todo{pa mor dda ydy dyddiad Jackson yn cadw lan heddiw?}. Given the nature of his evidence, I take his date to refer only to the non-initial merger of \lT\ with \xD\footnote{The date could also refer to when the remaining lenited consonants became spirants, but this is unlikely to be this late, as this development is shared with Goidelic.}. For the word-initial merger of \lT\ with \xD, \textcite{jackson_language_1953} does not give evidence one way or another. In the absence of evidence that word-initial \lT\ was distinct from \xD, one could well analogously assert that the word-initial merger occurred at the same as elsewhere. Yet \textcite{martinet_celtic_1952} demonstrated how non-lenited voiced stops in other positions than word-initial were exceedingly rare, while they were reasonably common word-initially. At the same time, \textcite{falchun_systeme_1951} gave insight into how a word-initial contrast between \xD\ and \lT\ might work synchronically. 


\Textcite{greene_gemination_1956}, following \textcite{martinet_celtic_1952}, notes how Jackson fails to see that geminates can only be long in opposition to the shortness of non-geminates. It is therefore nonsense to speak of the simplification of voiced geminates centuries after non-geminates were shortened intervocalically. \Textcite{Gre_Spirant66} notes that it is impossible to envisage a threefold quantitative distinction, and Jackson's [TT] and [T] must be considered a single length grade. The  reflexes of Jackson's [TT, T] as both voiceless spirants and voiceless stops, respectively, is because the sound law turning non-lenited voiceless stops into spirants  only applied to a subset of these stops, resulting in a situation where spirantisation is `in its origins, only a special case of non-lenition'~\autocite[119]{Gre_Spirant66}. Greene puts spirantisation following apocope, and the Brittonic phonemes descending from stops immediately after apocope are: short voiced stops representing \lT, long voiced stops representing \xD, and voiceless stops representing \xT. Thus in Greene's conception, \lT\ and \xD\ differed from one another in length, and \xT\ differed from the other stops in voice, but it was unmarked for length because short voiceless stops had become voiced stops earlier. Greene does not specify why these lenited voiceless stops became voiced earlier. He simply takes it for granted because this is what we see in present-day Brittonic dialects\footnote{These assumptions are not defended because the topic of disagreement between Jackson and Greene was not the historical phonology of lenition, but of spirantisation.}.


The debate between Jackson and Greene is reconsidered by \textcite{harvey_aspects_1984}. He joins Greene in rejecting Jackson's three-way length distinction and notes that it gives information about several periods at once rather than reflecting an opposition at any given time. Harvey then traces how consonants in different environments go through various successive stages, and which features were actually contrastive in these stages. He takes the dental phonemes as an example.

Harvey's first stage is what he calls `prelenition Celtic', by which he means the stage in which lenition was allophonic, the lenited allophones were shorter than their radical counterparts, and they could supply their Goidelic as well as their Brittonic descendants. So what Harvey means with `lenition' is the fricativisation of phonetically lenited consonants in Goidelic, and the voicing, shortening, and/or fricativisation of these consonants in Brittonic. He notes that in this stage double consonants only occur in strictly internal, intervocalic position. He then follows Martinet in positing that any consonant that was not in leniting position would merge with these double consonants phonetically. This stage is represented in Table~\ref{tab:harveypreap}.

\begin{table}[h]
  \centering
  \begin{tabular}{cccccccccccc}
    \toprule
     & \multicolumn{4}{c}{Initial sandhi} & \tchh{Abs.~initial} & \multicolumn{4}{c}{Internal intervocalic} & \tch{RT} \\
    \midrule
    / / & s\#t & o\#t & s\#d & o\#d & \#t & \#d & -t- & -d- & -tt- & -dd- & -Rt- \\
    {[ ]} & s\#tt & o\#t & s\#dd & o\#d & \#tt & \#dd & -t- & -d- & -tt- & -dd- & -Rtt- \\
      \bottomrule
    \end{tabular}%
    \caption{Pre-apocope Celtic phonemic and phonetic stop arrays according to \textcite[90, 93]{harvey_aspects_1984}.}
    \label{tab:harveypreap}%
\end{table}%
Harvey proposes the phonetic arrangement of Table~\ref{tab:harveypreap} because this is in exact accordance with the Goidelic reflexes. Harvey notes that intervocalic non-geminate stops became fricatives in Goidelic, giving the result of Table~\ref{tab:harveygoid}\footnote{Harvey silently assumes that fricativisation of lenited consonants preceded apocope, but this is not \lat{a priori} necessary.}. The remaining consonants after Goidelic lenited stops became fricatives were not distinguished in length, so an unlenited /t/ had the same length irrespectively of whether it was historically a geminate /-tt-/, absolute initial /\#t-/, or following a resonant /-Rt-/. 

\begin{table}[h]
  \centering
  \begin{tabular}{cccccccccccc}
    \toprule
     & \multicolumn{4}{c}{Initial sandhi} & \tchh{Abs.~initial} & \multicolumn{4}{c}{Internal intervocalic} & \tch{RT} \\
    \midrule
    {/ /} & s\#t & o\#θ & s\#d & o\#ð & \#t & \#d & -θ- & -ð- & -t- & -d- & -Rt- \\
      \bottomrule
    \end{tabular}%
    \caption{Goidelic array of \gls{pc} stops according to \textcite[91]{harvey_aspects_1984}.}
    \label{tab:harveygoid}%
\end{table}%

This confirms that phonetic lenition entailed the identification of intervocalic non-geminate stops as short. Geminate stops remained long and remaining consonants that were not specified for length came to be identified with the geminates. This distinction was originally one of length. This may be internally reconstructed because lenition entailed the identification of unlenited stops with long geminates, but Harvey also finds further evidence for this in Goidelic. Non-lenited stops are pronounced long in a number of present-day Goidelic dialects; this length is not phonemic synchronically, but may be diachronically connected with geminates. The opposition between lenited and unlenited resonants is largely one of length in multiple branches of Celtic. Also, Latin loanwords into Irish consistently show that the Classical Latin opposition geminate/single is adopted as unlenited/lenited.

For Brittonic, Harvey asserts that  lenition  occurred prior to apocope. With this he means that short stops in intervocalic position lost distinctive length, and former voiceless stops became voiced stops while former voiced stops became fricatives. As a result, geminate voiced stops in words such as \mw[estuary]{aber} fell together with lenited voiceless stops. This merger between word-medial \xD\ and \lT\ is indicated with arrows in Table~\ref{sfig:harveybrit1}. He argues that this merger is borne out by the later evidence:
\tqt{It is no longer possible, from synchronic considerations, to tell whether a given internal intervocalic voiced stop (\textit{g, b, d}) is the product of the lenition of an unvoiced simplex (/k, p, t/) or of the non-lenition of a voiced genimate (/gg, bb, dd/); hence the disagreement about whether Welsh, Cornish, and Breton \textit{ober} `work' derives from Latin \textit{opera} or from a native /obber-/ < /od+ber-/.}{harvey_aspects_1984}{96}
The later evidence indeed shows that word-medial \xD\ and \lT\ merged. What it does not show is when this merger occurred or indeed whether this merger occurred prior to apocope, as Harvey suggests. 

\begin{table}[h]
  \centering
  \subfloat[Following `lenition']{%
    \label{sfig:harveybrit1}
    \begin{tabular}{cccccccccccc}
      \toprule
      & \multicolumn{4}{c}{Initial sandhi} & \tchh{Abs.~initial} & \multicolumn{4}{c}{Internal intervocalic} & \tch{RT} \\
      \midrule
      {/ /} & s\#t & o\#d & s\#d & o\#ð & \#t & \#d & \tikz[remember picture,anchor=base,baseline=(current bounding box.base)]{\node(lt){-d-};} & -ð- & -t- & \tikz[remember picture,anchor=base,baseline=(current bounding box.base)]{\node(xd){-d-};} & -Rt- \\[.5cm]
      \bottomrule
    \end{tabular}%
    \tikz[overlay,remember picture]{
      \draw[<->,thick](lt) -- ++ (0,-.5cm) -| (xd);}
  }

  \subfloat[Following apocope]{%
    \label{sfig:harveybrit2}
    \begin{tabular}{cccccccccccc}
      \toprule
      & \multicolumn{4}{c}{Initial sandhi} & \tchh{Abs.~initial} & \multicolumn{4}{c}{Internal intervocalic} & \tch{RT} \\
      \midrule
      {/ /} & V\#t & \tikz[remember picture,anchor=base,baseline=(current bounding box.base)]{\node(lt){V\#d};} & \tikz[remember picture,anchor=base,baseline=(current bounding box.base)]{\node(xd){V\#d};} & V\#ð & \#t & \#d & -d- & -ð- & -t- & -d- & -Rt- \\[.5cm]
      \bottomrule
    \end{tabular}%
    \tikz[overlay,remember picture]{
      \draw[<->,thick](lt) -- ++ (0,-.5cm) -| (xd);}
  }
  \caption{Brittonic arrays of \gls{pc} stops following `lenition' and apocope respectively, according to \textcite[96]{harvey_aspects_1984}.}
  \label{tab:harveybrit}%
\end{table}%

It follows logically from Harvey's assertions that apocope would lead to the same merger between \lT\ and \xD\ word-initially. This merger is indicated with arrows in Table~\ref{sfig:harveybrit2}\footnote{Harvey then describes how a subset of the \pbr{t}'s in Table~\ref{sfig:harveybrit2} would  undergo spirantisation, but this step is irrelevant here.}. This is indeed what Harvey argues:
\tqt{This treatment also destroys the distinction between the relevant elements of examples [/V+\textit{d}\"{u}ːd/] and [/V+\textit{d}uβr/], since if gemination has indeed disappeared, the lenition of an unvoiced stop gives the same phoneme as the non-lenition of the corresponding voiced one and both are now in identical environments in initial sandhi. But once again, we find that this prediction is generally borne out in practice: thus in Welsh `his stags' (\textit{ceirw}, plural of \textit{carw}) and `her waves' (\textit{geirw}) are indistinguishably \textit{ei geirw}.}{harvey_aspects_1984}{96--97}

He argues that only Breton provides counter-evidence, but he provides three arguments why the dissimilarity in Breton \lT\ and \xD\ is not inherited. The first is that it is only attested in Le Bourg Blanc Breton\footnote{\Textcite{carlyle_syllabic_1988} finds it in Lanhouarneau also, but her thesis came out after \textcite{harvey_aspects_1984}.}, and in none of the other modern Celtic languages, but linguistic reconstruction does not work by tyranny of the majority and archaic features may well be preserved in only a minority of dialects. His second argument is that distinct \lT\ and \xD\ are only found word-initially, but not word-medially. Yet Martinet already identified word-medial voiced geminates as exceedingly rare compared to non-lenited voiced stops.

The third argument is that the resonants in Le Bourg Blanc Breton provide a solid analogical base. This argument holds more water. Harvey notes that the resonants \mob{n, l ,r} can be both long and short word-initially. Thus, length is the only feature keeping radical and lenited resonants apart word-initially, so length in stops may not be original but the outcome of the analogical extension of the system
\begin{center}
  \begin{tabular}{l@{~=~}l}
    \textsc{radical}  & [+gemination]\\
    \textsc{lenition} & [-gemination]
  \end{tabular}
\end{center}
from resonants to stops~\autocite{harvey_aspects_1984}. In \textcite{carlyle_syllabic_1988}'s terms, this analogical extension may be formulated as the spread of Word-initial gemination from resonants to stops (Table~\ref{tab:carlylederiv}). All the other rules formulated by Carlyle are in agreement with Harvey.

Harvey's proposed analogy easily explains the Breton evidence, but his account can easily be modified to not require this analogy in the first place. For Common Celtic pre-phonemic lenition, Harvey argues at length that the original distinction between radical and lenited stops was one of length. Then, when he discusses Brittonic, he does away with length distinctions for stops altogether and posits that length was wholly replaced by voice prior to apocope. Then, accounting for the Breton evidence, he posits that they were analogically reintroduced. Yet it would be simpler to conceive of a simplified chronology where the loss of length did not occur at any time between Proto-Brittonic and Le Bourg Blanc Breton in the first place. This simplification would be that the voicing of lenited voiceless stops does not necessarily equal the loss of distinctive length of stop consonants. Then later, this word-initial length distinction was lost at some point in Welsh and in the Breton dialects where a distinction between \lT\ and \xD\ was not maintained.

To conclude: the result of the arguments between Jackson, Greene and Harvey is that length may be identified as the original variable distinguishing pre-apocope radical and lenited stops. These distinctions could then phonemicise with the loss of final syllables. Greene demonstrated that length distinctions were \emph{only} used for distinguishing between fortis and lenis consonants. Equipped with this knowledge, it can be observed that length could be considered a feature marking lenition across \gls{archphon}s following apocope\footnote{\Gls{archphon}s are further discussed in Subsection~\ref{sec:from-allit-text}.}. Here, shortness marked lenited and longness marked radical phonemes.

Yet all three of them assume that this length distinction was replaced or at least supplemented by a voicing distinction early on in pre-apocope Brittonic. For stops, Harvey even posits a wholesale replacement of length with voicing for stops. In other words, they understand that the only use of length was to distinguish between what was to become radical and lenited consonants, but they cannot conceive that length could be the sufficient as a variable in distinguishing radical from lenited consonants stops in post-apocope Brittonic. Because of this, they all place the merger of \lT\ with \xD\ prior to apocope.

Yet there is evidence to the contrary. \Textcite{carlyle_syllabic_1988} finds that the voicing of \lT\ should be described as a derived feature from length; not vice versa. If we compare the three word-initial stop series of Le Bourg Blanc (and Lanhouarneau) Breton with Harvey's reconstruction of Common Celtic allophones, then it is found that two regular developments suffice in describing the differences. These rules are apocope, which phonemicises the Common Celtic length distinctions, and a redundant rule which voices short \ie lenited consonants is all that is needed. This rule must be placed following apocope. This model is illustrated in Figure~\ref{fig:pctolbbbstops}.

\begin{figure}[h]
  \centering
  \begin{tikzpicture}
    %%% NODES
    \node(topl){};
    \node[right= 1.5cm of topl](pcxt){\#\xT};
    \node[right= 1.5cm of pcxt](pclt){\#\lT};
    \node[right= 1.5cm of pclt](pcxd){\#\xD};
    \node[below= 1cm of topl,align=left](1l){%
      Phonetic\\lenition};
    \node(1xt) at (1l-|pcxt){{[}pː tː kː]};
    \node(1lt) at (1l-|pclt){{[}p t k]};
    \node(1xd) at (1l-|pcxd){{[bː dː ɡː]}};
    \node[below=1.5cm of 1l](2l){%
      Apocope};
    \node(2xt) at (2l-|pcxt){/pː tː kː/};
    \node(2lt) at (2l-|pclt){/p t k/};
    \node(2xd) at (2l-|pcxd){/bː dː ɡː/};
    \node[below=1.5cm of 2l,align=left](3l){%
      Redundant\\voicing\\of \lT};
    \node(3xt) at (3l-|pcxt){/pː tː kː/};
    \node(3lt) at (3l-|pclt){/b d g/};
    \node(3xd) at (3l-|pcxd){/bː dː ɡː/};
    %%% ARROWS
    \foreach \x in {xt,lt,xd}{%
      \draw[->] (1\x) -- (2\x);
      \draw[->] (2\x) -- (3\x);
    }
    %%% OPPOSITIONS
    \draw[<->,dashed, bend left = 50] (2xt) to node[midway,above]{±length} (2lt);
    \draw[<->,dashed, bend left = 50] (2lt) to node[midway,above]{±voice\vphantom{g}} (2xd);
    \draw[<->,dashed, bend left = 50] (3xt) to node[midway,above]{±voice\vphantom{g}} (3lt);
    \draw[<->,dashed, bend left = 50] (3lt) to node[midway,above]{±length} (3xd);
  \end{tikzpicture}
  \caption[From Proto-Celtic to Léon Breton word-initial stops.]{From Proto-Celtic to Léon Breton word-initial stops. Dashed lines indicate the phonological variable distinguishing pairs of stop series in a given period.}
  \label{fig:pctolbbbstops}
\end{figure}
The model given in Figure~\ref{fig:pctolbbbstops} may easily be adapted to \gls{mow} and the dialects of \gls{mob} that do not distinguish \lT\ from \xD. These dialects may be derived from this model by positing that distinctive length was lost following the voicing of \lT. Under this model, the voicing of \lT\ would occur after apocope in all Brittonic dialects.


\section{From Brittonic to Welsh}
\label{sec:from-brittonic-welsh}
The first scholar to critique the unspoken assumption that Brittonic lenition must entail the voicing of voiceless stops prior to apocope is \textcite{koch_*cothairche_1990}. Instead he posits an Old Celtic lenition where the initially allophonic opposition between radical and lenited was one of aspiration and of voice.  In his conception, \xT\ was voiceless aspirated, \lT\ was voiceless and unaspirated, and \xD\ was voiced. This system is identical to the proposals by Pedersen and Thurneysen shown in Table~\ref{tab:pedersenstops}\footnote{\Textcite{koch_*cothairche_1990} uses a slightly different notation for \lT: [b̥ d̥ ɡ̥]. In the International Phonetic Alphabet, a circle below is used to indicate devoicing, so these symbols may be considered equivalent to [p t k]. The circled forms may also be used to indicate half-voiced. \Textcite[§~29]{koch_*cothairche_1990}| leaves open such a possibility due to the conflicting evidence of loans into Old English.}.

According to \textcite{koch_*cothairche_1990} archaic Brittonic names and phrases borrowed into Irish such as \oi{Coirthech}, which was to survive into Welsh as \mow{Ceredig}, were introduced into Irish at a time when all \gls{pc} stops had fortis and lenis allophones. These lenis allophones must have had such a pronunciation that they could be interpreted as Irish lenited consonants, so that they could later become spirantised when lenited voiceless stops became voiceless spirants in Goidelic. This Goidelic spirantisation of lenited voiceless stops occurred in the mid to later fifth century. Thus Brittonic had a voiceless but lenis pronunciation of word-medial \lT\ at least up until the fifth century.

A number of Brittonic personal names and place names have been borrowed into Old English as well. Old English personal names such as \textit{Dinoot, Madoc, Cerdic} imply that final \lT\ was voiceless, but medial \lT\ was voiced; voiced word-initial \xD\ was correctly identified as voiced. Place names borrowed into Old English, however, show that \lT\ could also be interpreted as voiceless word-medially.

Koch thus regards the voicing of \lT\ to be a process which ocurred substantially later than apocope. Because it is apocope which causes the phonemicisation of lenition, it follows logically that \lT\ existed not only as allophones separate from \xD, but also as separate phonemes. He gives some clues as to how long this situation may have persisted. One clue is that the orthography of nin-century \gls{ow} and \gls{ob} consistently spell \lT\ with \ow{p, t, c}, which makes it unlikely that they had fallen together with initial \xD\footnote{In Chapter~\ref{oldwelsh}, I demonstrate that this is correct. Word-initial \lT\ is kept distinct from \xD, and the phoneme resulting from the non-word-initial merger of \lT\ and \xD\ was spelled in analogy with the word-initial phonemes.}. Koch argues that the later Welsh cynghanedd demonstrates that the latest date at which \lT\ may have merged with \xD\ must be the fourteenth century. In some Breton dialects, 

Koch also argues convincingly for the existence of aspiration as a variable in early Celtic. He notes that absolute initial voiceless stops are aspirated in all Celtic languages, and that this is old. This is suggestions by early confusions where a liquid robs the stop of its aspiration. This may lead to aspirated stops being reinterpreted as unaspirated. Examples of this sort are  \gpc[sword]{*kladjos} loaned as \glat{gladius}, the doublet \pc{Pretani} vs.~\pc{Bretanni}, and \lat{mons Graupius} for earlier \lat{Craupius}.

Koch gives a convincing evidence that absolute word-initial \xT\ was aspirated, and that non-word-initial \lT\ was not voiced, but did have a lenis quality to it somehow. This raises the question what this lenis quality was. Here, Koch asserts that lack of aspiration denoted the lenis quality of \lT. That is, \lT\ could be distinguished from \xT\ because it was unaspirated. He gives no positive arguments for this lack of aspiration, and argues against length instead.

Koch rejects Jackson's view that the original distinction between radical and lenited was one of length in early Brittonic. Instead in Koch's view, the \gls{mob} system where length serves to distinguish \lT\ from \xD\ is an innovation. This innovation is in turn explained by the fact that Breton has moved in the direction of voicing historically voiceless sounds, and no longer aspirating historically aspirated sounds. He also notes that distinctive stop length is typologically rare in absolute initial position. 

In Koch's model, early Brittonic \xT\ was voiceless aspirated, \lT\ plain voiceless, and \xD\ was voiced. In \gls{mob}, \xT\ is voiceless and long, \lT\ is voiced and short, and \xD\ is voiced and long. When Koch notes that Breton has moved in the direction of voicing, he means that aspirating consonants have become voiceless, and voiceless consonants have become voiced. Starting at Koch's reconstruction, Breton moving in the direction of voicing would yield voiceless \xT, voiced \lT, and voiced \xD\todo{I should refer to a chapter/section on voicing and aspiration here}. Thus we would expect a merger between \lT\ and \xD\ in Breton of all Brittonic dialects, yet it is only Breton which still distinguishes \lT/\xD. Koch argues that length would then be introduced to prevent this merger, but he gives no mechanism by which loss of aspiration in Breton would lead to the adoption of length. 

Even if Koch had given a convincing mechanism by which length could arise from loss of aspiration, then his argument would still hinge on the correctness of the assumption that aspiration is lost in Breton. This is not a clear-cut matter. Many Breton dialects use voicing rather than aspiration to distinguish voiced from voiceless~\autocite[221]{Ios_Representation13}. \Textcite[114]{Bot_Etude82} finds only voicing contrast in Argol Breton and \textcite[177--178]{Hum_Phonologie95a} explicitly denies that aspiration may be present in Bothoa.  \Textcite[9]{carlyle_syllabic_1988} finds that the degree of voicing may vary from speaker to speaker in Lanhouarneau. \Textcite[157--167]{falchun_systeme_1951} finds at least some aspiration in  Léon Breton. \Textcite{Ter_Grammaire70} finds traces of it in Vannetais. Here it may be observed that those dialects keeping \lT\ and \xD\ distinct have at least some trace of aspiration.

Moreover, the idea that length served to distinguish radical from lenited consonants is not based exclusively on Le Bourg Blanc Breton. It also follows from Martinet's arguments discussed in Section~\ref{sec:martinet}. So, in order to reject length as the variable distinguishing radical from lenited consonants, one would need to face these arguments as well. 

Koch's argument that absolute intial stop length is typologically rare does not hold either, because it loses sight of a key aspect of when length may serve to distinguish in Léon Breton. In Le Bourg Blanc, a phrase-initial constituent may never be lenited. That is, if \lT\ appears at the beginning of a phrase, it is reinterpreted as \xD. Falc'hun already noted that historical /\gls{l}p/ in \mob{bremañ} is in fact pronounced with initial /\gls{x}b/\footnote{See Seciton~\ref{sec:falchun}.}. Thus, distinctive stop length serves to distinguish \lT\ from \xD\ in word-initial position, but not in phrase-initial position. But even setting aside this consideration, the argument that we must avoid positing typologically rare configurations should apply only when it is done out of thin air, but in this case we do see distinctive length in an actual related dialect. 

The following points summarise our knowledge of post-apocope Welsh based on the contribution made by \textcite{koch_*cothairche_1990} and my discussion thereof:
\begin{itemize}
\item Absolute initial \gls{T} was strongly aspirated.
\item \lT\ was voiceless, but could be interpreted by Goidelic speakers as equivalent to Goidelic \lT.
\item Initial \lT\ was distinct from \xD\ in \gls{ow}, and merged before the fourteenth century.
\item Koch's rejection of phonemic length cannot be sustained.
\end{itemize}
This leaves two variables 


\todo[inline]{some more work}

\begin{table}[h]
  \centering
  \begin{tabular}{lll}
    \toprule
    \begin{tabular}[c]{@{}l@{}}{[}-voice{]}\\ {[}long{]}\\ {[}asp{]}\end{tabular} & \begin{tabular}[c]{@{}l@{}}{[}+voice{]}\\ {[}long{]}\end{tabular} & {[}short{]} \\\midrule
    p{[}ʰː{]} & bː & b \\
    t{[}ʰː{]} & dː & d \\
    k{[}ʰː{]} & ɡː & ɡ \\\bottomrule
  \end{tabular}
  \caption{The Old British stop system between the 7th and 11th centuries according to \textcite[33]{schrijver_old_2011}.}
  \label{oldbritishconsonantsystem}
\end{table}


\begin{itemize}
\item Finish with Koch:
  \begin{itemize}
  \item Account of the evidence Koch finds for \lT≠\xD\ in Welsh
  \item Account of why Koch argues for /tʰ, t, d/
  \end{itemize}
\item Note on Breton use of voice or aspiration to mark T and D
\item More general discussion of what the difference between a voicing language and an aspirating language are
\item Perhaps a section on phonological length in MW
  %%% A Section on phonological length in MW???
  % Harvey bases his argument that analogy drove the distinction between \lT\ and \xD\ on the sonorants \mw{r, l, }and \mw{n}. These consonants have two phonemes each: a long fortis, and a short lenis. Naturally then, it is length that distinguishes fortis from lenis here. This identification of length as a marker for fortis quality was then expanded to voiced stops. This matter is of relevance to Welsh, precisely because Welsh does not use length to distinguish fortis and lenis consonants for \mw{r} and \mw{l}. Rather, voice is the common factor in distinguishing \graph{rh} and \graph{r}, and \graph{ll} and \graph{l}, respectively, while \graph{n} does not distinguish between fortis and lenis word-initially\footnote{\gls{mw} does distinguish /nː/, /rː/, and /lː/ from their short counterparts. This distinction is lost before the end of the Middle Welsh period, see also: \textcite[127--128]{schumacher_mittel-_2011}. The relevance of this phonemic opposition for the opposition between lenited voiceless stops and unlenited voiced stops is doubtful, however, because sonorants only use length to distinguish fortis and lenis in non-initial position, whereas it is precisely in word-initial position that lenited voiceless stops and unlenited voiced stops are kept separate in the Early \gls{mw} orthography and in Le Bourg Blanc Breton.}. Resonants  \graph{ll} /ɬ/ and \graph{l} /l/ also differ in manner of articulation: the former sound is a lateral fricative while the latter is a lateral approximant. %% this point is true, but it does not fit here. it talks about the MW situation specifically, so it is more suitable for the conclusion

\item Conclusion of earlier literature with `state of the art' \ie overview of different viewpoints
\item Introduction to my own chapters attempting to uncover the phonological variables distinguishing \xT, \lT, \xD.
\end{itemize}


\section{Voicing versus aspiration}
\label{sec:voic-vers-aspir}

%%% Aspiration in Common Celtic
Baudiš: Grammar of Early Welsh p.~78
\Textcite[40, 43, 45]{LP_Concise37}

%%% On voicing and aspiration in Breton.


\section{Phonetics and phonology of lenition in present-day Welsh} 
Unlike Breton, present-day Welsh does not preserve a three-way stop distinction either word-initially or elsewhere. Nevertheless, the way in which the contrast between voiced and voiceless stops is realised phonetically differs between dialects. In some Southeastern dialects of Welsh, the consonants themselves have phonetically merged, but they are kept separate by the length of neighboring vowels: e.g.\ \mw{ebol} /'eːpol/ `foal' v.s. \mw{capel} /'kapel/ `chapel'~\autocite[85]{awbery_phonotactic_1984}. One may wonder whether preceding vowel length similarly played a role in disambiguating the three Early Middle Welsh stop series as it does in disambiguating the two stop series remaining in Modern SE Welsh, and perhaps the modified survival of such a pattern in southeastern Welsh implies that this system survived for longer in these dialects. However, the idea that preceding vowel length may have served to disambiguate lenited voiceless stops from unlenited voiced stops has one drawback: the distinction between three different stop series was only maintained word-initially. The vowel lengthening would therefore have occurred in the word preceding the word to be lenited. The phonetic upshot of this would be (to use Koch's examples~\autocite*[§~26]{koch_*cothairche_1990}): [\c{e}ː g(ː)ar] `her leg' vs. [\c{e} gar] `his car'. This supposition is problematic, because only stressed syllables distinguish between short and long vowels, and the posessive pronouns in the above examples are both unstressed. 


\subsection{The related issues of gemination and spirantization}
\cite{jackson_language_1953,martinet_celtic_1952,schrijver_spirantization_1999,isaac_old-_2004}
%%% Local Variables:
%%% mode: latex
%%% TeX-master: "../main"
%%% End:
