\chapter{Evidence from Old Welsh}
\label{oldwelsh}
This chapter discusses the orthography of stop consonants in \gls{ow}, and the implication of this orthography for its phonology and phonetics. \Gls{ow} does not typically write lenition word-initially, so lenited \mw{p, t, k} are represented by \graph{p, t, c}. Word-medially, lenition is not found, but note: 
\tqt{\begin{welsh}Ni ddynodir y treigladau dechreuol yn orgraff Hen Gymraeg fel arfer, fodd bynnag. Gw.\ e.e.\ \textit{ir bloidin} ‘y flwyddyn’ (Comp 20), enw benywaidd ar ôl y fannod heb ddangos y treiglad meddal. Fodd bynnag, ceir enghreifftiau prin lle y dangosir treigladau, e.e.\ \textit{hendat} ‘tad-cu, taid’ gl.\ \textit{auus} Ox 2, lle y dangosir y treiglad \textit{tat} > \textit{dat} ar ôl \textit{hen}.\end{welsh}\footnote{`The initial mutations are not usually denoted in Old Welsh orthography, however. Cf.\ e.g.\ \mw{ir bloidin} `the year' (Comp 20), a feminine noun after the article that does not show the soft mutation. However, there are rare examples where the mutations are shown, e.g.\ \mw{hendat} `grandfather' gl.\ \textit{auus} Ox 2, where the mutation of \mw{tat} > \mw{dat} is shown after \mw{hen}.' Own translation}}{falileyev_llawlyfr_2016}{9}
Essentially, this means that \gls{ow} has etymological spelling of consonants rather than following pronunciation exactly. Etymological spelling lends itself to slip-ups that may give insights in how \gls{ow} phonemes contrasted with each other. For stops, two types of slip-ups are conceivable: one may either spell an etymological (pre-apocope) /b, d, g/ with \graph{p, t, c}, or one may spell a pre-apocope /p, t, k/ with \graph{b, d, g}. Both types of unconventional \gls{ow} spelling are found, albeit rarely. The former type of mistake may be considered hypercorrectly conservative, because it incorrectly identifies word-medial or final /b, d, g/ as pre-apocope /p, t, k/, even when they descend from the much rarer /bb, dd, gg/, \ie voiced stops that have remained unlenited. The latter type of spelling slip-up may be considered innovative in that it has moved to represent lenition orthographically. I have collected all examples of both types from \textcite{falileyev_etymological_2000}'s glossary. They are discussed in sections \todo{ref! and ref!}, respectively.





% It is nevertheless worthwhile to reconsider whether spellings such as \mw{hendat, ceintiru, aperth} are really exceptions, or whether lenited voiceless stops and unlenited voiced stops were indeed kept separately. It may be that exceptions to orthographical non-lenition conform to a certain pattern rather than being random exceptions. All departures of stop spelling from their pre-apocope phonetic value are given in Tables \ref{owvoicedstops} and \ref{owvoicelessstops}.

\section{The /b, d, g/ with \graph{p, t, c}-type}
\label{bdgwithptc}
An example of the spelling discussed in this section is the gloss \ow{dir arpeteticion ceintiru} gl.\ \lat{miseris patruelibus} (Ovid 38a). Here,  \ow{ceintiru} (\gmow{cefnder(w)}) has a medial \mw{t} for /d/. Notably, it has \mw{t} for pre-apocope /\gls{x}d/, not /\gls{l}t/%
\footnote{\textcite[s.v.\ \mw{cefnder, cefnderw}]{bevan_geiriadur_2014} gives the following etymology: `\textwelsh{\mw{cein, ceifn (caifn)} > \textit{cefn+derw} ‘gwir, sicr’, Crn.\ \textit{kanderu}, Llyd.\ \textit{kenderf}, cf.\ Gwydd.\ \textit{derbbrathir}, \&c.; \textit{cefnderw̯} > \textit{cefnder}, cf.\ \textit{syberw̯, syber}}'}. 
The spelling consistent with \gls{ow} conventions \emph{and} with a pre-apocope consonant inventory would be with a medial \mw{d}, since the \mw{t} in \ow{ceintiru} is not the lenited form of a \ow{t}, but a radical \mw{d}. However, a speaker of \gls{ow} would have recognized the word as consisting of \mw{caifn} and \mw{derw}, in such composite words, the second element is typically lenited in Welsh, but not here. This exceptional case was not identified as such, however, and medial \ow{d} has been reanalysed as /\gls{l}t/. A similar example is the gloss \ow{aperth} gl.\ \lat{victima} (Ovid 41b), \gmow[sacrificing]{aberth}. This word descends from \gpc{*ad-ber-t-}, meaning the medial \ow{b} descends from a voiced stop, which was kept unlenited in this phonological context. Its spelling with \ow{p} may therefore be considered similarly hypercorrect to \ow{ceintiru}. A full list of this type of hypercorrect spelling is found in Table \ref{owvoicelessstops}. 

\begin{table}[h]
  \centering
    \begin{tabular}{llll}
    \toprule
    \tch{Gloss} & \tch{Modern Welsh} & \tch{Stop value} & \tch{Etymology} \\
    \midrule
    \ow{a\al{p}er, a\al{p}erou} & \mow{aber, aberau} & \graph{p} for /bb/ & \gpc{*ad-ber-} \\
    \ow{a\al{p}erth, a\al{p}erthou} & \mow{aberth, aberthau} & \graph{p} for /bb/ & \gpc{*ad-ber-t-} \\
    \ow{bri\al{c}er, bri\al{c}eriauc} & \mow{brigerog} & \graph{c} for /g/ & \gpie{*bhre\^g} \\
	\ow{cein\al{t}iru} & \mow{cefnder(w)} & \graph{t} for /d/ & \mow{cefn+derw} \\
    \ow{\al{cu}eeticc} & \mow{gwe\"edig} & \graph{cu} for /(g)\cu/ & \gpie{*\cu eg-} \\
    \ow{di\al{p}rotant} & \mow{difrodant} & \graph{p} for /bb/  & \mow{di-+brawd}\tablefootnote{With provection following \mw{di-}.} \\
    \ow{rum\al{p}} & \mow{rhwmb} & \graph{p} for /b/ & \glat{r(h)ombus} \\
    \ow{sum\al{p}l} & \mow{swmbwl} & \graph{p} for /b/ & \gvlat{*stum'blus} \\
    \bottomrule
    \end{tabular}%
  \caption{\gls{ow} words representing an historically voiced stop with a voiceless stop. }
  \label{owvoicelessstops}%
\end{table}%


\subsection{Non-word-initial \xD\ merged with \lT}
It should be noted that Table~\ref{owvoicelessstops} contains all instances of word-internal or word-final non-lenited voiced stops. This means that every single non-word-initial unlenited etymogical /b, d, g/ in \gls{ow} is indeed represented with a voiceless stop. Strictly speaking, then, the spellings given in Table \ref{owvoicelessstops} (with the exception of \mw{cueeticc}, which has a word-initial unlenited voiced stop) are not irregularities synchronically in \gls{ow}, because word-medial and word-final \xD\ were not pronounced differently. These spellings are irregularities in a diachronic sense, however.

Alliterative patterns found in poetry composed in the \gls{ow} period also attest to the equivalence of voiced geminates and voiceless stops. Consider the following line of \mw{Moliant Cadwallon}:
\mwcc{\mw{Moliant Cadwallon}, l.~43 in \textcite[191]{koch_cunedda_2013}}{Porth Ysgewin kyffin aber,}{In Portskewett, on the estuary where the borders meet,}
Here, radical \mw{p} in \mw{Porth} alliterates with \mw{b} < \pbr{*bb} in \mw{aber} through \gls{cyfl}. This line is especially interesting because it dates from a period when mutated and unmutated consonants could alliterate, if they were based on the same \gls{archphon}\footnote{Although in this case, \mw{Porth} may in fact have been lenited as an adverbial subclause denoting place.}. The poem contains several archaisms in its orthography and is held to be consistent with the seventh-century events it refers to~\autocite[186--187]{koch_cunedda_2013}.

Evidence from \gls{ow} therefore provides no counterevidence to the proposition that voiced geminates had already merged with non-word-initial lenited voiceless stops. 
I discuss the merger of these series in Chapter~\todo{Refer to chapter on word segmentation and word-medial/final merger}.
This merger between \lT\ and \xD\ non-word-initially accounts for every instance found in \tref{owvoicelessstops}, except for \ow{cueeticc}.



\subsection{Word-initial voiced stops}
An exceptional spelling is found in \ow{cueeticc} `woven', found in  the following gloss: \ow{orcueeticc cors gl.\ ex papyro textili} `out of the woven reed'. This word goes against the grain of the hypothesis of this thesis, because \graph{c} represents an unlenited /g/ here. If lenited /k/ and unlenited /g/ had merged word-initially by this point, then \ow{cueeticc} could be regarded as a hypercorrection in an attempt to not write lenition where this was heard. However, my hypothesis is that these sounds were phonologically distinct. 

The clue may lie in Breton. According to \textcite[64]{falchun_systeme_1951}, lenited voiceless stops and unlenited voiced stops were only distinct where the previous word ended in a vowel. \gow{cueeticc} is preceded by \mw{or} (\gmow{o'r}), so the same limitation may have been in place in \Gls{ow}. This cannot be stated with certainty on the basis of one example alone, however. Similarly, voiced and voiceless stops are sometimes confused in the vicinity of resonants, as will be discussed next. 

Another relevant point in explaining this form is that the word-initial  /gʷ/ goes back to \gpc{*w} rather than either /gʷ/ or /g/. \gpc{*w} only turned into \pbr{gʷ} in the \gls{pbr} period. Moreover, \gls{mob} /gʷ/ has /w/ as its lenited counterpart, whereas \gls{mob} /g/ has /γ/. These considerations make it possible that spelling /gʷ/ with \graph{cu} served to distinguish the initial consonant from a regular /g/, but this line of reasoning is similarly difficult to verify due to the scarcity of examples.

\subsection{Evidence from Old Breton}
Lenition is not represented as a rule in Old Breton, just like in Old Welsh. 
However, this rule is adhered to less strictly in Old Breton:
\tqt{\begin{french}Cependant la lénition de \textit{p, t, b} est parfois notée ; on le précise dans ce cas, car il n'y a ici que des cas d'esp\`ece%
\footnote{However, the lenition of \textit{p, t, b} is sometimes indicated; one specifies it in this case, since there are only these cases\todo[inline]{ask for help with French}}
\end{french}}{fleuriot_dictionary_1985}{18}
This means that an exhaustive account of how voiced geminates were represented in \gls{ob}, like what we see in \tref{owvoicelessstops}, would be bound to show some spellings with \graph{b, d, g}. 
However, this would not mean much in the face of haphazard orthographical lenition.
Nevertheless, the examples found in \tref{obvoicelessstops} indicate that the same word-medial and final merger of \lT\ and \xD\ had taken place:
\begin{table}[h]
  \centering
    \begin{tabular}{llll}
    \toprule
    \tch{Gloss} & \tch{Modern Breton} & \tch{Stop value} & \tch{Etymology} \\
    \midrule
\ob{ace\al{t}er} & N/A & \graph{t} for /d/ & \glat{abecedarium} \\
\ob{cri\al{t}im} & \mob{kridi, kredi} & \graph{t} for /d/ & \gpc{*kred-dhe}\\
\ob{do\al{t}ietue} & N/A & \graph{t} for /d/ & \gpc{*do-di-atau} \\\bottomrule
    \end{tabular}%
  \caption{Some \gls{ob} words representing an historically voiced stop with a voiceless stop. }
  \label{obvoicelessstops}%
\end{table}%

\section{The /p, t, k/ with \graph{b, d, g}-type}
\label{ptkwithbdg}
The word \mw{hendat} provides an example of the type of spelling discussed in this section: the fact that lenition is not only written, but is in fact written using the same consonant as its unlenited voiced counterpart \mw{d} may suggest that lenited voiceless stops and unlenited voiced stops had merged.  The \mw{d} is in a favourable sandhi position, between a voiced \mw{n} and an obviously voiced vowel. This might make the phonetic value [d], but its phonological value may well have been /\gls{l}t/ still. In this case, spelling \mw{hendat} with \mw{d} would be phonetically correct, but phonologically misleading. Another option is that orthographical representation with \graph{nt} was not possible here, even if the writer acknowledged the stop as being originally voiceless: in the \gls{ow} period, word-medial /nt/ became /nh/, with perhaps /nθ/ as an intermediary. \gls{ow} orthography may have lagged behind this phonological change here, leading scribes to associate the cluster \graph{nt} with /nh/ or /nθ/, neither of which should be read in \mw{hendat}\footnote{Note, however, that \gow{Hanther} `half' and \mw{pimphet} `fifth' are attested, showing \graph{nth} and \graph{mph} for clusters undergoing the sound shift of \textit{NT -> Nh} word-medially.}. Choosing \graph{nd} would consequently be the only spelling left. 

As we can see, however, the case of \mw{hendat} is not unique in its complexities. Similar consonant clusters containing a resonant and a stop tend to write pre-apocope voiceless stops with \graph{b, d, g}. Table~\ref{owvoicedstops} lists all words containing a pre-apocope lenited voiceless stop that is represented by a voiced stop. With the exception of \mw{gubennid} and \mw{guetid}, all stops with this type of spelling are in a consonant cluster with a resonant. This makes it possible that all these words were spelled with a \graph{b, d, g} as a result of similar conditions to \mw{hendat}, but it may also imply that the presence of resonants influenced the phonetic properties of lenited voiceless stops non-word-initially. I will explore the latter possibility in subsection\todo{ref!}.

\begin{table}[h]
  \centering
    \begin{tabular}{llll}
    \toprule
    \tch{Gloss} & \tch{Modern Welsh} & \tch{Stop value\tablefootnote{The phonemes given under `stop value' represent their presumed value before phonemicisation of lenition.}} & \tch{Etymology} \\
    \midrule
    \ow{cin\al{d}raid} & \mow{cyn + traeth} & \graph{d} for /t/ & \glat{contractus}\tablefootnote{Medial \ow{d} may represent /θ/, after \pbr{*ntr > θr}. However, the orthographical retention of \ow{n} would be unexpected in this case.} \\
    % \textit{\al{d}i} & \textit{i} & \graph{d} for /t/ & Clt.~*\textit{to-}\tablefootnote{This word's initial consonant underwent the following evolution: /t/ > /d/ > /ð/ > /\zero/, with perhaps /t/ divided into \lT\ and \xD\ in a three-way stop system. These developments are part of an irregular reduction of clitics, so \textit{d} may have stood for /d/ or /ð/ by this point, making this word irrelevant.} \\
    \ow{dissun\al{cg}netic} & \mow{disugnedig} & \graph{cg} for /k/ & \mow{sugn} < \gpc{*seuk-n-} \\
    \ow{gu\al{b}ennid} & \mow{gobennydd} & \graph{b} for /p/ & \mow{go+penn+ydd} \\
    \ow{gueti\al{d}} & \mow{*[dy]wedyd} & \graph{d} for /t/ & \mow{yd} < \gpc{*-et(i)} \\
    \ow{hen\al{d}at} & \mow{hendad} & \graph{d} for /t/ & \mow{hen+tad} \\
    \ow{mo\al{d}reped} & \mow{modryb(o)edd} & \graph{d} for /t/ & \gpc{*mātrVkʷī} \\
    \ow{scri\al{b}l} & \mow{ysgrubl} & \graph{b} for /p/ & \glat{scrūpulum} \\
    \ow{sebe\al{d}lauc} & \mow{sefydlog} & \graph{d} for /t/ & \gpc{*sabetlo-} \\
    \bottomrule
    \end{tabular}%
  \caption{\Gls{ow} words representing an historically voiceless stop with a voiced stop. }
  \label{owvoicedstops}%
\end{table}%



\subsection{Word-initial and non-word-initial stop phonetics differed\todo{nicer title}}

A speaker of \gls{ow} would have been aware of the compound nature of \mw{hendat}, so why did he still consider lenited \mw{t} here to be equal to \mw{d} in \mw{hendat} but presumably not in *\mw{hen tat}? If \mw{hen tat} were to be written as two words, lenition would definitely not have been represented because there are no \gls{ow} instances of word-initial lenition. The example of \mw{hendat} `grandfather' combined with the writing of lenited voiceless stops with \graph{p, t, c} word-initially is still consistent with the pattern found in the Black Book of Chirk, which also represents lenition of \mw{t} word-medially. This pattern is also consistent with what is found in le Bourg Blanc Breton\todo{Remove references to other parts of research, or merely link to chapters}\footnote{See subsection \ref{harveylenition}}. 

The word \mw{hendat} /hendat/ `grandfather' is a compound word, and it differs very little from the hypothetical word for `old father': \mw{hen tat} /hen \dd at/. Absence of grammatical lenition across word boundaries implies that lenition of voiceless stops produced /\bd, \dd, \gd/ word-initially while it produced /b, d, g/ word-medially. It is an open question whether word-internal lenition in compounds was  productive in the \gls{ow} period, i.e.\ whether lenition as a result of internal sandhi indeed went through a separate phonological process from word-initial morphonological lenition. If this was the case, \gls{ow} /t/ would lenite into two different phonemes depending on whether the lenited phoneme would be word-initial, or word-medial in a compound. 

\subsection{Sonorant consonants influenced pronunciation of word-medial stops}
Historically voiceless stops written with \graph{b, d, g} in \gls{ow} tended to occur right before or after a resonant. This fact has a parallel in how Celtic *\textit{kladjos} entered Latin in the form of \textit{gladius}, how Celtic *\textit{Pritanī} yielded English \textit{Britain} by the time of Caesar, and how \textit{Craupius} got corrupted into \textit{Graupius}. \textcite[\S 25]{koch_*cothairche_1990} interprets these three corruptions as the logical result of a liquid robbing the stop of its distinctive feature of aspiration.

A similar process appears to be going on in \gls{ow}: because lenited voiceless stops could be written like unlenited voiced stops next to resonants, it stands to reason that they did this because it was pronounced as such. Because resonants are known to prevent phonetic realization of aspiration, their presence demonstrates that  aspiration was absent from these consonants. This, in turn, imples that the difference between a lenited voiceless stop and an unlenited voiced stop lay in aspiration.

Lenited voiceless stops and unlenited voiced stops were kept separate word-initially, but merged elsewhere. If this was indeed the case, then \gls{ow} evidence on the orthography of word-internal stops may imply that stops phonetically similar to word-initial lenited voiceless stops were written as such, and stops phonetically similar to word-initial unlenited voiced stops could receive a similar treatment. After all, there was no distinction in stop series to maintain. Since what is written like unlenited voiced stops is found next to resonants, they must have been unaspirated. It follows that word-initial unlenited voiced stops must have been unaspirated. Conversely, lenited voiceless stops must have been aspirated  word-initially, because the distribution of the same graphemes word-medially implies that they were phonetically aspirated word-medially. This line of reasoning relies on the assumption that the orthographical departures from the \gls{ow} norm shown in Table~\ref{owvoicedstops} are a result of analogy, namely writing non-word-initial stops in analogy with how they are written word-initially, with the word-initial spelling as the analogical base.

\section{Results in perspective}
One way to make sense of both `innovative' and `hypercorrect' spellings as found in \mw{hendat}, and \mw{ceintiru, aperth}, respectively, is to consider them as what they are: exceptions. Word-initially, the radical voiceless stop is easily recoverable as a grammatical alternate of its lenited counterpart, but lenited voiceless stops are also separated from unlenited voiced stops word-medially. In the latter case, an unlenited voiceless stop is not available as an alternate. An example of this is found in the form \mw{petguar} (\gls{mp} 22b) `four', with /pedwar/ as its phonological form. Here, intervocalic \mw{t} correctly identifies a (lenited) historical \mw{t}. Word-medially and finally, etymologically correct spellings make up the vast majority of the cases, and word-initially, I am not aware of a single case where grammatical lenition was represented at all. These results are remarkable. Absence of evidence showing that \xD\ and \lT\ were conflated implies that they indeed remained separate, but absence of evidence is not evidence of absence.

The evidence of Section~\ref{bdgwithptc} is generally in accordance with later Middle Welsh evidence, and with how Late Old British phonology is generally reconstructed\footnote{\Textcite[31]{schrijver_old_2011} notes that `[lenited voiceless stops] were contrasted with the reflexes of the unlenited voiced stops [\dots]. In word-initial position, the contrast between short and long voiced stops is maintained'.}. Section~\ref{ptkwithbdg}, however, shows that the vicinity of resonants influenced the choice of whether to write lenited voiceless stops like voiceless stops, or as voiced stops. In order to understand the phonology behind this tendency, the subsections below draw some parallels.
\subsection{A note on Breton}
According to Falc'hun, length difference between \xD\ and \lT\ is maintained after vowels, but not after consonants. This is similar, but not identical to how word-medial \lT\ is written with the same letters that represent word-initial \xD\ within a consonant cluster.
\subsection{The presence of resonants and syllable length}
The presence of either vowels or resonants necessitated the phonetic process of lenition. Lenited stop consonants not next to resonants behaved normally, and were by definition surrounded by vowels (on both sides pre-apocope). The existence of either a vowel only or a vowel and a resonant may have influenced syllable length or weight. We know SE Welsh dialects maintain a difference between \xT\ and \xD\ through neighbouring syllable length. It is possible that a similar phonetic concomitant existed in \gls{ow}, and that the difference in behaviour of stops in clusters with a resonant had a different influence on neighbouring vowel length than those in no such cluster with a resonant.
\subsection{Clusters /tn/ and /tl/ in Welsh}
In some dialects, words such as \mw{bodlon, anadl, and cenedl} lose medial /d/. Apparently, the rule is /dl/ -> /ðl/, with a geographically retricted final rule of /ðl/ to /l/:

\tqt{
\begin{welsh}Yn fwyaf neilltuol, fodd bynnag, pan fyddai /t/ Frythoneg yn rhagflaenu /n/ neu /l/, gallai'r ffrwydrolen ddatblygu nid yn unig yn /d/ Gymraeg yn unol \^a'r disgwyl, ond hefyd yn /ð/. Er enghraifft, rhoes */witno/- Y Frythoneg y ffurfiau Cymraeg Diweddar /gwɨdɨn/ a /gwiðin/, ac aeth */siːtlo/- yn /hidil/ ac yn /hiðil/ O safbwynt ffonolegol, rhesymol yw tybied bod amrywiadau o'r fath yn cynrychioli dau gam yn natblygiad /t/ wreiddiol: /t/ > /d/ > /ð/. Atgyfnerthir y casgliad hwn ac ychwanegu dimensiwn arall ato gan ystyriaethau tafodieithol: Nodwedd gyffredinol a gwlad eang yw lleisio /t/, either dim ond mewn rhan o'r wlad --- y deheudir --- y ceir tystiolaeth gadarn am fodolaeth ffurfiau-/ð/ fel /gwiðin/ os yw'r dosbarthiad hwn yn adlewyrchu sefyllfa hanesyddol, yr oedd ffrithio'r /d/ nid yn unig yn ddiweddarach na lleisio /t/, ond yr oedd hefyd yn ddaearyddol gyfyngedig. Ond yr ystyriaeth ganolog i hanes y nodwedd ddaearyddol amrywiol hon yw ai yn y Frythoneg a datblygodd ynteu yn y Gymraeg.\end{welsh}}{thomas_o_1995}{220}

This matter may be related to the problem of how to pronounce aspirated consonants in the vicinity of resonants. the difficulty in pronouncing a cluster such as [tʰl] may have caused aspiration to have been lost in these contexts. This may in turn have caused postvocalic /\gls{l}tl/ to be reinterpreted as /\gls{x}dl/, which in turn was prone to lenition, given how it still stood in a leniting environment between a vowel and a resonant. Such a reinterpretation only makes sense when lenited voiceless stops had some degree of aspiration.

Middle Welsh also has an irregular tendency to lose distinction between \gls{T} and \gls{D} when it stands close to a resonant:
\tqt{Nicht regelhaft, d.h.\ nur in einzelnen F\"allen erscheinen sogar statt der etymologisch zu erwartenden intervokalischen Konsonantengruppen */dr/, */dv/, */dw/, */gj/, */gw/, */rbr/ und */rbVr sowie */rdr/ Clusters mit den aspirierten Obstruenten (d.h.\ Schreibung \graph{p}, \graph{t} und \graph{c} f\"ur /b/, /d/ und /g/); vielmehr treten in diesen Kontexten tats\"achlich /p/, /t/, /k/ auf, was daran erkennbar ist, dass das entsprechende Wort im Mittelkymrischen mit \graph{p}, \graph{pp}, \graph{t}, \graph{tt}, \graph{c}, \graph{k}, \graph{cc} oder \graph{ck} geschrieben wird und/oder dass das entsprechende Wort in der modernen Sprache lebendig ist und dort einen aspirierten Obstruenten aufweist. Hinzuzuf\"ugen ist noch, dass zwischen den involvierten Konsonanten meistens, aber nicht immer eine Morphemgrenze vorliegt. Beispiele \textit{Cattraeth} (R 1435.5), \textit{Kattraeth} (Phillimore 1886: 126) `Catraeth' (ein Ortsname) < lat.\ \textit{Cataracta} (so bei Beda Venerabilis, vgl.\ Jackson 1969:83); \textit{ac atuyd} (PKM 2.9), modern \textit{agatfydd} `vielleicht'; \textit{ymysgyttwaf} (Ms.-14c. Peniarth 5: 96r, 148.26) `ich werde mich sch\"utteln'; \textit{ettwa} (RM 60.23) neben \textit{edwaeth} (R 1173.36), modern \textit{eto} `noch, wiederum'; \textit{benffyccyaw} (Ms.-14c. Harley 958: 39v.20) `ausleihen', modern \textit{benthycio}; \textit{dryckwas} (Haycock 1994: 326.56c) `schlechter Diener'; \textit{dirprwy} (R 1143.6), modern \textit{dirprwyo} `freikaufen, ersetzen'; \textit{darpar} (LlB 1.14), modern \textit{darpar} `vorbereiten'; \textit{ertrei} (CO 16.447), modern \textit{ertrai} `erstes abebben der Flut'.
}{schumacher_mittel-_2011}{115}
\subsection{Lenited /b/ and /m/ near resonants}
Lenited /b/ and /m/ (spelled /β/ and /μ/, respectively by \textcite{russell_rowynniauc_2003}) merged in the later stages of the \gls{ow} period. This merger happened earlier in some environments than others. One environment is adjacent to a resonant. Evidence for this early merger is found in spellings such as \gow{Cobreidau} for MoW \mw{cyfreithiau} < \mw{*kom-rekt-} \autocite[39--41]{russell_rowynniauc_2003}. Apparently, adjacent resonants caused loss of the nasal concomitant found in /μ/ \footnote{Or the vowel perceding /μ/, because \textcite[27]{russell_rowynniauc_2003} assumes that /μ/ was typically realised as a bilabial fricative whose preceding vowel was nasalised: [\~{-}β].}.


\subsection{Perhaps also relevant}
\begin{itemize}
    \item Irish \textit{NT -> D}
    \item Another potentially relevant matter is the disappearance of syllable-initial \textit{dl} in Welsh OIr.\ \textit{dligid} vs.\ W.\ \mw{dylu}, but cf.\ Breton \mw{dle}.
\end{itemize}

