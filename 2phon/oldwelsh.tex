\chapter{Evidence from Old Welsh}
\label{oldwelsh}
This chapter discusses the orthography of stop consonants in \gls{ow}, and how orthographical inconsistencies give insight into the phonetic variables underpinning the various stop phonemes in \gls{ow}.  \Gls{ow} does not write lenition word-initially, so lenited \mw{p, t, c} are represented by \ow{p, t, c}. Word-medially, lenition is not found either, but note: 
\tqt{\begin{welsh}Ni ddynodir y treigladau dechreuol yn orgraff Hen Gymraeg fel arfer, fodd bynnag. Gw.\ e.e.\ \textit{ir bloidin} ‘y flwyddyn’ (Comp 20), enw benywaidd ar ôl y fannod heb ddangos y treiglad meddal. Fodd bynnag, ceir enghreifftiau prin lle y dangosir treigladau, e.e.\ \textit{hendat} ‘tad-cu, taid’ gl.\ \textit{auus} Ox 2, lle y dangosir y treiglad \textit{tat} > \textit{dat} ar ôl \textit{hen}.\end{welsh}\footnote{The initial mutations are not usually denoted in Old Welsh orthography, however. Cf.\ e.g.\ \mw{ir bloidin} `the year' (Comp 20), a feminine noun after the article that does not show the soft mutation. However, there are rare examples where the mutations are shown, e.g.\ \mw{hendat} `grandfather' gl.\ \textit{auus} Ox 2, where the mutation of \mw{tat} > \mw{dat} is shown after \mw{hen}. }}{falileyev_llawlyfr_2016}{9}
This means that \gls{ow} has etymological pre-lenition spelling of consonants rather than following pronunciation exactly. Etymological spelling always lends itself to slip-ups, and these slip-ups may give insights in how \gls{ow} phonemes contrasted with each other. For stops, two types of slip-ups are conceivable: one may either spell an etymological (pre-apocope) /b, d, ɡ/ with \ow{p, t, c}, or one may spell a pre-apocope /p, t, k/ with \ow{b, d, g}. I will show that both types of unconventional \gls{ow} spelling are found, albeit rarely. The former type of mistake may be considered hypercorrectly conservative, because it incorrectly identifies word-medial or final /b, d, ɡ/ as pre-apocope /p, t, k/, even when they descend from the much rarer geminates /bb, dd, ɡɡ/, \ie voiced stops that have remained unlenited. The latter type of spelling slip-up may be considered innovative in that it has moved to represent lenition orthographically. I have collected all examples of both types from \textcite{falileyev_etymological_2000}'s glossary. They are discussed in Section~\ref{bdgwithptc} and Section~\ref{ptkwithbdg}, respectively.





% It is nevertheless worthwhile to reconsider whether spellings such as \mw{hendat, ceintiru, aperth} are really exceptions, or whether lenited voiceless stops and unlenited voiced stops were indeed kept separately. It may be that exceptions to orthographical non-lenition conform to a certain pattern rather than being random exceptions. All departures of stop spelling from their pre-apocope phonetic value are given in Tables \ref{owvoicedstops} and \ref{owvoicelessstops}.

\section{The /b, d, ɡ/ with \ow{p, t, c}-type}
\label{bdgwithptc}
An example of the spelling discussed in this section is the following gloss:
\mwcc[]{\acrshort{ovid} 38a}{\ow{dir arpeteticion ceintiru} \textup{gl.~}\lat{miseris patruelibus}}{to the wretched nephews}
Here,  \ow{ceintiru} (\gmow{cefnder(w)}) has a medial \mw{t} for /d/, which is historically /\gls{x}d/, not /\gls{l}t/%
\footnote{\Textcite[s.v.\ \mw{cefnder, cefnderw}]{bevan_geiriadur_2014} gives the following etymology: `\textwelsh{\mw{cein, ceifn (caifn)} > \textit{cefn+derw} ‘gwir, sicr’, Crn.\ \textit{kanderu}, Llyd.\ \textit{kenderf}, cf.\ Gwydd.\ \textit{derbbrathir}, \&c.; \textit{cefnderw̯} > \textit{cefnder}, cf.\ \textit{syberw̯, syber}}'}. 
An orthography consistent with \gls{ow} conventions, and also having a pre-apocope consonant inventory would write a medial \ow{d} here, since the \mw{t} in \ow{ceintiru} is not the lenited form of a \ow{t}, but a radical \mw{d}. However, this exceptional situation, where a word-medial /d/ descends from a non-lenited pre-apocope /d/, rather than a lenited /t/  was not identified as such here, and medial \ow{d} has been reanalysed as /\gls{l}t/.

A similar example is the gloss \ow{aperth} gl.\ \lat{victima} (\acrshort{ovid} 41b), \gmow[sacrificing]{aberth}. This word descends from \gpc{*ad-ber-t-}, meaning the medial \ow{b} descends from a voiced stop, which was kept unlenited in this phonological context. Its spelling with \ow{p} may therefore be considered hypercorrect similarly to \ow{ceintiru}. A full list of this type of hypercorrect spelling is found in Table~\ref{owvoicelessstops}. 

\begin{table}[h]
  \centering
    \begin{tabular}{llll}
    \toprule
    \tch{Gloss} & \tch{Modern Welsh} & \tch{Stop value} & \tch{Etymology} \\
    \midrule
    \ow{a\al{p}er, a\al{p}erou} & \mow{aber, aberau} & \ow{p} for /bb/ & \gpc{*ad-ber-} \\
    \ow{a\al{p}erth, a\al{p}erthou} & \mow{aberth, aberthau} & \ow{p} for /bb/ & \gpc{*ad-ber-t-} \\
    \ow{bri\al{c}er, bri\al{c}eriauc} & \mow{brigerog} & \ow{c} for /ɡ/ & \gpie{*bhre\^g} \\
	\ow{cein\al{t}iru} & \mow{cefnder(w)} & \ow{t} for /d/ & \mow{cefn+derw} \\
    \ow{\al{cu}eeticc} & \mow{gweëdig} & \ow{cu} for /(ɡ)\cu/ & \gpie{*\cu eg-} \\
    \ow{di\al{p}rotant} & \mow{dibrodant} & \ow{p} for /bb/  & \mow{di-+brawd}\tablefootnote{With non-lenition following \mw{di-}~\autocite{Lin_OW85}.} \\
    \ow{rum\al{p}} & \mow{rhwmb} & \ow{p} for /b/ & \glat{r(h)ombus} \\
    \ow{sum\al{p}l} & \mow{swmbwl} & \ow{p} for /b/ & \gvlat{*stum'blus} \\
    \bottomrule
    \end{tabular}%
  \caption{\gls{ow} words representing an historically voiced stop with a voiceless stop. }
  \label{owvoicelessstops}%
\end{table}%


It should be noted that Table~\ref{owvoicelessstops} contains all instances of word-internal or word-final non-lenited voiced stops. This means that every single non-word-initial unlenited etymogical /b, d, ɡ/ in \gls{ow} is indeed represented with a voiceless stop. Strictly speaking, then, the spellings given in Table \ref{owvoicelessstops} (with the exception of \mw{cueeticc}, which has a word-initial unlenited voiced stop) are not irregularities at all in \gls{ow}, because word-medial and word-final \xD\ were not pronounced differently from \lT. These spellings are only irregularities in a diachronic sense; not in a synchronic sense.

Alliterative patterns found in poetry composed in the \gls{ow} period also attest to the equivalence of voiced geminates and voiceless stops. Consider the following line of \mw{Moliant Cadwallon}:
\mwcc{\mw{Moliant Cadwallon}, l.~43 in \textcite[191]{koch_cunedda_2013}}{Porth Ysgewin kyffin aber,}{In Portskewett, on the estuary where the borders meet,}
Here, radical \mw{p} in \mw{Porth} alliterates with \mw{b} < \pbr{*bb} in \mw{aber} through \gls{cyfl}\footnote{See Subsection~\ref{sec:from-allit-text} for more details on \gls{cyfl}.}. This line is especially interesting because it dates from a period when mutated and unmutated consonants could alliterate, if they were based on the same \gls{archphon}, but could not alliterate with a different \gls{archphon}\footnote{In this particular case, \mw{Porth} may have been lenited as an adverbial subclause denoting place, but irrespectively of whether \mw{aber} alliterates with /\gls{l}t/ or /\gls{x}t/, it alliterates with \gls{archphon} /t/ and not /d/.}. The poem contains several archaisms in its orthography and is held to be consistent with the seventh-century events it refers to~\autocite[186--187]{koch_cunedda_2013}.

Evidence from \gls{ow} therefore provides no counterevidence to the proposition that voiced geminates had already merged with word-medial lenited voiceless stops\footnote{I discuss what it means to be a word in Chapter~\ref{cha:some-phon-issu}.}. 
This merger between \lT\ and \xD\ non-word-initially accounts for every instance found in \tref{owvoicelessstops}, except for \ow{cueeticc}, because the spelling of \xD\ with \ow{c} is word-initial in this instance. It is found in the following gloss:
\mwcc[]{\acrshort{mc} 8 b.a.}{\ow{orcueeticc cors} \textup{gl.~}\lat{ex papyro textili}}{out of the woven reed.}
This word goes against the grain of the hypothesis of this thesis, because \ow{c} represents an unlenited /ɡ/ here. If lenited /k/ and unlenited /ɡ/ had merged word-initially by this point, then \ow{cueeticc} could be regarded as a hypercorrection in an attempt to not write lenition where this was heard. However, my hypothesis is that these sounds were distinct phonemes. 

\Textcite{Sch_Two01} also notes that word-medial \xD\ had already merged with \lT\ in \gls{ow}. He argues for this based on the same word-medial examples given in Table~\ref{owvoicelessstops}, and he uses the case of \ow{cueeticc} to argue that \lT\ and \xD\ had also fallen together word-initially. My position is that this one word cannot by itself prove  \xD\ and \lT\ merged word-initially also. The evidence of Chapter~\ref{cha:prov-mwbe-y} as well as Part~\ref{part:orthography} provide evidence that they were kept apart both phonologically and orthographically until well after the \gls{ow} period. Yet even the data from Table~\ref{owvoicelessstops} itself may be employed to argue that a word-internal merger of \xD/\lT\ does not necessarily imply the same word-initially. Etymological \xD\ is historically rare word-medially, but it is common word-initially\footnote{See Section~\ref{sec:martinet}.}. Yet what we see in Table~\ref{owvoicelessstops} is that every single instance of word-medial \xD\ is represented with \ow{p, t, c}. If representation of word-initial \xD\ with \ow{p, t, c} were anywhere near the 100\% we see for word-medial \xD, the word-medial instances would drown in a sea of word-initial instances. 

This is not what we see, and \ow{cueeticc} is exceptional. It is therefore best to explain it individually. One possible reason expected \ow{gu} is not written may be because of how \ow{g} lenited in \gls{ow}. Voiced stops /b d ɡ/ lenited to their corresponding spirants, so the phoneme of /ɡ/ was /ɣ/. This rule had one exception: the lenited phoneme of /ɡʷ/ was /w/. Because lenition was not written, /ɡʷ > w/ was represented with \ow{gu} in \gls{ow}, but /w/ which had never been /ɡʷ/ was also represented with \ow{gu}, hence spellings like \gow[four]{petguar} /pedwar/. Thus, the scribe may have avoided a spelling with \ow{gu} in order to avoid reading with /w/, and he may have considered /w/ to be a further departure from correct /\gls{x}ɡʷ/ than /\gls{l}kʷ/.

An \gls{ow} consonant written word-initially could be read as both its radical and its  lenited form, but non-word-initial consonants were to be read as the lenited form. Based on the gaps the scribe of \ow{orcueeticc cors} wrote, he seemed not to think that a word boundary existed between \ow{or} and \ow{cueeticc}. Perhaps the scribe considered \ow{cu} to be the proper orthography, because it is found word-medially. This raises the question how \ow{cu} could be considered word-medial even though the position allows /\gls{l}k/ to be distinguished from /\gls{x}ɡ/. Here, a clue may lie in Breton. According to \textcite[64]{falchun_systeme_1951}, lenited voiceless stops and unlenited voiced stops were only distinct where the previous word ended in a vowel\footnote{See Section~\ref{sec:falchun}.}. \gow{cueeticc} is preceded by \mw{or} (\gmow{o'r}), so the same limitation may have been in place in \gls{ow}.





% \subsection{Evidence from Old Breton}
% \todo[inline]{where to put this? Before cueeticc, surely}
% Lenition is not represented as a rule in Old Breton, just like in Old Welsh. 
% However, this rule is adhered to less strictly in Old Breton:
% \tqt{\begin{french}Cependant la lénition de \textit{p, t, b} est parfois notée ; on le précise dans ce cas, car il n'y a ici que des cas d'esp\`ece%
% \footnote{However, the lenition of \textit{p, t, b} is sometimes indicated; one specifies it in this case, since there are only these cases\todo[inline]{ask for help with French}}
% \end{french}}{fleuriot_dictionary_1985}{18}
% This means that an exhaustive account of how voiced geminates were represented in \gls{ob}, like what we see in \tref{owvoicelessstops}, would be bound to show some spellings with \ow{b, d, g}. 
% However, this would not mean much in the face of haphazard orthographical lenition.
% Nevertheless, the examples found in \tref{obvoicelessstops} indicate that the same word-medial and final merger of \lT\ and \xD\ had taken place:
% \begin{table}[h]
%   \centering
%     \begin{tabular}{llll}
%     \toprule
%     \tch{Gloss} & \tch{Modern Breton} & \tch{Stop value} & \tch{Etymology} \\
%     \midrule
% \ob{ace\al{t}er} & N/A & \ow{t} for /d/ & \glat{abecedarium} \\
% \ob{cri\al{t}im} & \mob{kridi, kredi} & \ow{t} for /d/ & \gpc{*kred-dhe}\\
% \ob{do\al{t}ietue} & N/A & \ow{t} for /d/ & \gpc{*do-di-atau} \\\bottomrule
%     \end{tabular}%
%   \caption{Some \gls{ob} words representing an historically voiced stop with a voiceless stop. }
%   \label{obvoicelessstops}%
% \end{table}%

\section{The /p, t, k/ with \ow{b, d, g}-type}
\label{ptkwithbdg}
The word \ow{hendat} mentioned in the opening of this chapter provides an example of the type of spelling discussed in this section. Here, lenition is not only written, but is in fact written using the same consonant as its unlenited voiced counterpart \ow{d} This is evidence that word-medial \xT\ had merged with \xD.  Alternatively, orthographical representation with \ow{nt} was not possible here, even if the writer acknowledged the stop as being originally voiceless: in the \gls{ow} period, word-medial /nt/ became /nh/, with perhaps /nθ/ as an intermediary. \Gls{ow} orthography may have lagged behind this phonological change here, leading scribes to associate the cluster \ow{nt} with /nh/ or /nθ/, neither of which should be read in \ow{hendat}\footnote{Note, however, that \gow{Hanther} `half' and \ow{pimphet} `fifth' are attested, showing \ow{nth} and \ow{mph} for clusters undergoing the sound shift of \textit{NT -> Nh} word-medially.}. Choosing \ow{nd} would consequently be the only spelling left. However, the case of \ow{hendat} is not unique in its complexities. Similar consonant clusters containing a resonant and a stop tend to write pre-apocope voiceless stops with \ow{b, d, g}. Table~\ref{owvoicedstops} lists all words containing a pre-apocope lenited voiceless stop that is represented by a voiced stop.

\begin{table}[h]
  \centering
    \begin{tabular}{llll}
    \toprule
    \tch{Gloss} & \tch{Modern Welsh} & \tch{Stop value\tablefootnote{The phonemes given under `stop value' represent their presumed value before phonemicisation of lenition.}} & \tch{Etymology} \\
    \midrule
    \ow{cin\al{d}raid} & \mow{cyn + traeth} & \ow{d} for /t/ & \glat{contractus}\tablefootnote{Medial \ow{d} may represent /θ/, after \pbr{*ntr > θr}. However, the orthographical retention of \ow{n} would be unexpected in this case.} \\
    % \textit{\al{d}i} & \textit{i} & \ow{d} for /t/ & Clt.~*\textit{to-}\tablefootnote{This word's initial consonant underwent the following evolution: /t/ > /d/ > /ð/ > /\zero/, with perhaps /t/ divided into \lT\ and \xD\ in a three-way stop system. These developments are part of an irregular reduction of clitics, so \textit{d} may have stood for /d/ or /ð/ by this point, making this word irrelevant.} \\
    \ow{dissun\al{cg}netic} & \mow{disugnedig} & \ow{cg} for /k/ & \mow{sugn} < \gpc{*seuk-n-} \\
    \ow{gu\al{b}ennid} & \mow{gobennydd} & \ow{b} for /p/ & \mow{go+penn+ydd} \\
    % \ow{gueti\al{d}} & \mow{*[dy]wedyd} & \ow{d} for /t/ & \mow{yd} < \gpc{*-et(i)} \\
    \ow{hen\al{d}at} & \mow{hendad} & \ow{d} for /t/ & \mow{hen+tad} \\
    \ow{mo\al{d}reped} & \mow{modryb(o)edd} & \ow{d} for /t/ & \gpc{*mātrVkʷī} \\
    \ow{scri\al{b}l} & \mow{ysgrubl} & \ow{b} for /p/ & \glat{scrūpulum} \\
    \ow{sebe\al{d}lauc} & \mow{sefydlog} & \ow{d} for /t/ & \gpc{*sabetlo-} \\
    \bottomrule
    \end{tabular}%
  \caption{\Gls{ow} words representing an historically voiceless stop with a voiced stop. }
  \label{owvoicedstops}%
\end{table}%

 With the exception of \ow{gubennid}, all stops with this type of spelling are in a consonant cluster with a resonant. This makes it possible that all these words were spelled with a \ow{b, d, g} as a result of similar conditions to \ow{hendat}, but it may also imply that the presence of resonants influenced the phonetic properties of lenited voiceless stops non-word-initially. 


\subsection{Word-initial and non-word-initial stop phonetics}
\label{sec:word-initial-non}
A speaker of \gls{ow} would have been aware of the compound nature of \ow{hendat}, so why did he still consider lenited \ow{t} here to be equal to \ow{d} in \ow{hendat} but presumably not in *\ow[old father]{hen tat}? If \ow{hen tat} were to be meant  as two words, lenition would definitely not have been represented because there are no \gls{ow} instances of word-initial lenition\footnote{The results given in Part~\ref{part:orthography} show that orthographical word-internal lenition seen in \ow[grandfather]{hendat}, as well as representation of lenited voiceless stops with \ow{p, t, c} word-initially is typical of early thirteenth-century orthography.}.  The word \ow[grandfather]{hendat} /hendat/ is historically  a compound word, and it differs very little from the hypothetical word for `old father': \ow{hen tat} /hen \gls{l}tat/, but it does not contain morphophonemic lenition on a synchronic level. Absence of grammatical lenition within word boundaries implies that lenition of voiceless stops produced \lT word-initially while \lT\  merged with \xD\ word-medially. It seems, then, that \gls{ow} /t/ would lenite into two different phonemes depending on whether the lenited phoneme would be word-initial, or word-medial in a historically compound expression. This is only true diachronically, however, because by the \gls{ow} period, \ow{hendat} had become an independent lexeme from either \gow{hen} or \gow{tat}. 

% \subsection{Sonorant consonants influenced pronunciation of word-medial stops}
% \label{sec:sonor-cons-infl}
Historically voiceless stops written with \ow{b, d, g} in \gls{ow} tended to occur right before or after a resonant. This fact has a parallel in how \gpc{kladjos} entered Latin in the form of \lat{gladius}, how \gpc{Pritanī} yielded \glat{Britannia} by the time of Caesar, and how the Scottish mountain named \lat{Craupius} got corrupted into \glat{Graupius}. \textcite[§~25]{koch_*cothairche_1990} interprets these three corruptions as the logical result of a liquid robbing the stop of its distinctive feature of aspiration\todo{Perhaps a footnote referring to literature on laryngeal realism}. A similar process appears to be going on in \gls{ow}: because \lT\  could be written like \xD\ word-medially next to resonants, it stands to reason that they did this because they were phonetically realised more like \xD\. Because resonants are known to prevent phonetic realization of aspiration, their presence demonstrates that  aspiration was absent from these consonants. This, in turn, imples that the difference between a lenited voiceless stop and an unlenited voiced stop lay in aspiration\footnote{The fact that resonants appear both before and after the word-medial stops written like \xD\ makes it unclear whether  aspiration in \gls{ow} was realised as pre-aspiration oras  post-aspiration. A third option may be that it was realised as pre-aspiration in some environments, and as post-aspiration in others.}.

Lenited voiceless stops and unlenited voiced stops were kept separate word-initially, but merged elsewhere. Thus, \gls{ow} evidence on the orthography of word-internal stops implies that stops phonetically similar to word-initial lenited voiceless stops were written as such, and stops phonetically similar to word-initial unlenited voiced stops could receive a similar treatment. After all, there was no distinction in stop series to maintain. Since what is written like unlenited voiced stops is found next to resonants, they must have been unaspirated. It follows that word-initial unlenited voiced stops must have been unaspirated. Conversely, lenited voiceless stops must have been aspirated  word-initially, because the distribution of the same graphemes word-medially implies that they were phonetically aspirated word-medially.

This line of reasoning relies on the assumption that the orthographical departures from the \gls{ow} norm shown in Table~\ref{owvoicedstops} are a result of analogy, namely writing non-word-initial stops in analogy with how they are written word-initially, with the word-initial spelling as the analogical base. The analogy is laid out in Figure~\ref{fig:analogyow}.

\begin{figure}[h]
  \centering
  \begin{tabular}{>{\small}c}
Word-initial \\
{[}+aspiration] \\
\ow{p t c} \\
\end{tabular}
:
\begin{tabular}{>{\small}c}
Word-initial \\
{[}-aspiration] \\
\ow{b d g} \\
\end{tabular}
=
\begin{tabular}{>{\small}c}
Word-medial \\
{[}+aspiration] \\
\ow{p t c} \\
\end{tabular}
: x\\[3ex]
x =
\begin{tabular}{>{\small}c}
Word-medial \\
{[}-aspiration] \\
\ow{b d g} \\
\end{tabular}
<
\begin{tabular}{>{\small}c}
Word-medial \\
{[}-aspiration] \\
\ow{p t c} \\
\end{tabular}

\caption{Analogy in \gls{ow} stop orthography schematised.}
\label{fig:analogyow}
\end{figure}


\section{Results in perspective}
\label{sec:results-perspective}
It should be noted that the number of \gls{ow} spellings with word-medial \ow{b, d, g} representing a stop is low both in absolute terms and in relative terms compared to the size of the \gls{ow} corpus as a whole, so they must be considered as what they are: exceptions. Word-initially, the radical voiceless stop is easily recoverable as a grammatical alternate of its lenited counterpart, but word-medial  stops have no comparable  paradigmatic alternation with another series of stops. An example of this is found in the form \ow[four]{petguar} (\gls{mp} 22b), with /pedwar/ as its phonological form. Here, intervocalic \ow{t} represents  a lenited historical \ow{t}, even though a speaker of \gls{ow} had no way to know this. Yet etymologically correct orthography makes up for the vast majority of  cases, and word-initially, there is not  a single case where grammatical lenition was represented at all. Absence of conflation between \xD\ and \lT\  implies that they indeed remained separate, even if absence of evidence is not evidence of absence.

The evidence of Section~\ref{bdgwithptc} is generally in accordance with later Middle Welsh evidence, and with how Late Old British phonology is generally reconstructed\footnote{\Textcite[31]{schrijver_old_2011} notes that `[lenited voiceless stops] were contrasted with the reflexes of the unlenited voiced stops […]. In word-initial position, the contrast between short and long voiced stops is maintained'.}. Section~\ref{ptkwithbdg}, however, shows that the vicinity of resonants influenced the choice of whether to write lenited voiceless stops like voiceless stops, or as voiced stops.



%In order to understand the phonology behind this tendency, the subsections below draw some parallels.
% \subsection{A note on Breton}
% According to Falc'hun, length difference between \xD\ and \lT\ is maintained after vowels, but not after consonants. This is similar, but not identical to how word-medial \lT\ is written with the same letters that represent word-initial \xD\ within a consonant cluster.

% \subsection{Clusters /tn/ and /tl/ in Welsh}
% In some dialects, words such as \mw{bodlon, anadl, and cenedl} lose medial /d/. Apparently, the rule is /dl/ -> /ðl/, with a geographically retricted final rule of /ðl/ to /l/:

% \tqt{
% \begin{welsh}Yn fwyaf neilltuol, fodd bynnag, pan fyddai /t/ Frythoneg yn rhagflaenu /n/ neu /l/, gallai'r ffrwydrolen ddatblygu nid yn unig yn /d/ Gymraeg yn unol \^a'r disgwyl, ond hefyd yn /ð/. Er enghraifft, rhoes */witno/- Y Frythoneg y ffurfiau Cymraeg Diweddar /gwɨdɨn/ a /gwiðin/, ac aeth */siːtlo/- yn /hidil/ ac yn /hiðil/ O safbwynt ffonolegol, rhesymol yw tybied bod amrywiadau o'r fath yn cynrychioli dau gam yn natblygiad /t/ wreiddiol: /t/ > /d/ > /ð/. Atgyfnerthir y casgliad hwn ac ychwanegu dimensiwn arall ato gan ystyriaethau tafodieithol: Nodwedd gyffredinol a gwlad eang yw lleisio /t/, either dim ond mewn rhan o'r wlad --- y deheudir --- y ceir tystiolaeth gadarn am fodolaeth ffurfiau-/ð/ fel /gwiðin/ os yw'r dosbarthiad hwn yn adlewyrchu sefyllfa hanesyddol, yr oedd ffrithio'r /d/ nid yn unig yn ddiweddarach na lleisio /t/, ond yr oedd hefyd yn ddaearyddol gyfyngedig. Ond yr ystyriaeth ganolog i hanes y nodwedd ddaearyddol amrywiol hon yw ai yn y Frythoneg a datblygodd ynteu yn y Gymraeg.\end{welsh}}{thomas_o_1995}{220}

% This matter may be related to the problem of how to pronounce aspirated consonants in the vicinity of resonants. the difficulty in pronouncing a cluster such as [tʰl] may have caused aspiration to have been lost in these contexts. This may in turn have caused postvocalic /\gls{l}tl/ to be reinterpreted as /\gls{x}dl/, which in turn was prone to lenition, given how it still stood in a leniting environment between a vowel and a resonant. Such a reinterpretation only makes sense when lenited voiceless stops had some degree of aspiration.

% Middle Welsh also has an irregular tendency to lose distinction between \gls{T} and \gls{D} when it stands close to a resonant:
% \tqt{Nicht regelhaft, d.h.\ nur in einzelnen F\"allen erscheinen sogar statt der etymologisch zu erwartenden intervokalischen Konsonantengruppen */dr/, */dv/, */dw/, */gj/, */gw/, */rbr/ und */rbVr sowie */rdr/ Clusters mit den aspirierten Obstruenten (d.h.\ Schreibung \graph{p}, \graph{t} und \graph{c} f\"ur /b/, /d/ und /g/); vielmehr treten in diesen Kontexten tats\"achlich /p/, /t/, /k/ auf, was daran erkennbar ist, dass das entsprechende Wort im Mittelkymrischen mit \graph{p}, \graph{pp}, \graph{t}, \graph{tt}, \graph{c}, \graph{k}, \graph{cc} oder \graph{ck} geschrieben wird und/oder dass das entsprechende Wort in der modernen Sprache lebendig ist und dort einen aspirierten Obstruenten aufweist. Hinzuzuf\"ugen ist noch, dass zwischen den involvierten Konsonanten meistens, aber nicht immer eine Morphemgrenze vorliegt. Beispiele \textit{Cattraeth} (R 1435.5), \textit{Kattraeth} (Phillimore 1886: 126) `Catraeth' (ein Ortsname) < lat.\ \textit{Cataracta} (so bei Beda Venerabilis, vgl.\ Jackson 1969:83); \textit{ac atuyd} (PKM 2.9), modern \textit{agatfydd} `vielleicht'; \textit{ymysgyttwaf} (Ms.-14c. Peniarth 5: 96r, 148.26) `ich werde mich sch\"utteln'; \textit{ettwa} (RM 60.23) neben \textit{edwaeth} (R 1173.36), modern \textit{eto} `noch, wiederum'; \textit{benffyccyaw} (Ms.-14c. Harley 958: 39v.20) `ausleihen', modern \textit{benthycio}; \textit{dryckwas} (Haycock 1994: 326.56c) `schlechter Diener'; \textit{dirprwy} (R 1143.6), modern \textit{dirprwyo} `freikaufen, ersetzen'; \textit{darpar} (LlB 1.14), modern \textit{darpar} `vorbereiten'; \textit{ertrei} (CO 16.447), modern \textit{ertrai} `erstes abebben der Flut'.
% }{schumacher_mittel-_2011}{115}
% \subsection{Lenited /b/ and /m/ near resonants}
% Lenited /b/ and /m/ (spelled /β/ and /μ/, respectively by \textcite{russell_rowynniauc_2003}) merged in the later stages of the \gls{ow} period. This merger happened earlier in some environments than others. One environment is adjacent to a resonant. Evidence for this early merger is found in spellings such as \gow{Cobreidau} for MoW \mw{cyfreithiau} < \mw{*kom-rekt-} \autocite[39--41]{russell_rowynniauc_2003}. Apparently, adjacent resonants caused loss of the nasal concomitant found in /μ/ \footnote{Or the vowel perceding /μ/, because \textcite[27]{russell_rowynniauc_2003} assumes that /μ/ was typically realised as a bilabial fricative whose preceding vowel was nasalised: [\~{-}β].}.


% \subsection{Perhaps also relevant}
% \begin{itemize}
%     \item Irish \textit{NT -> D}
%     \item Another potentially relevant matter is the disappearance of syllable-initial \textit{dl} in Welsh OIr.\ \textit{dligid} vs.\ W.\ \mw{dylu}, but cf.\ Breton \mw{dle}.
% \end{itemize}


%%% Local Variables:
%%% mode: latex
%%% TeX-master: "../main"
%%% End:
