\chapter{Conclusion --- phonology}
\label{cha:conclusion-phonology}

The aim of Part~\ref{part:phonology-phonetics} is to establish the phonological variables that served to maintain a phonemic three-way distinction between \xT, \lT, and \xD\ in Welsh. In Chapter~\ref{cha:introduction-phonology} I argue that such a distinction must be posited, and introduce earlier scholarship on this three-way distinction. The earlier scholarship generally agrees that a three-way stop series must be reconstructed for some of the earlier stages of Brittonic. What is not agreed upon is which phonological variables kept this three-way stop distinction apart or whether the distinction ever survived into the stage where Brittonic lenition was phonemic. It is also not agreed upon when this three-way stop distinction collapsed into the two-way stop distinction found in \gls{mow}. The chapter ends with the hypothesis that \xT\ was originally long and aspirated, \lT\ was short and aspirated, and \xD\ was long and unaspirated. The following chapters explore this hypothesis; their findings are summarised here.

Chapter~\ref{oldwelsh} considers \gls{ow} orthography. Generally speaking, \gls{ow} orthography follows pre-lenition phonology, \ie \gls{T} is represented with \ow{p, t, c} irrespectively of whether it represents \lT\ or \xT, and the same goes for \gls{D} and \ow{b, d, g}. In Chapter~\ref{oldwelsh} I gather the exceptions to this rule, \ie instances where the orthograpy of a stop consonant differs from its etymological value in \gls{ow}. These exceptional spellings occur in two classes: words where etymological  \gls{D} is written with \ow{p, t, c}, and words where etymological \gls{T} is written with \ow{b, d, g}.

Etymological \gls{D} is consistently written as \ow{p, t, c} when it represents a word-medial voiced geminate. This confirms the hypothesis by \textcite{martinet_celtic_1952} that voiced geminates merged with lenited voiceless stops in an early stage due to the rarity of the former class\footnote{See Section~\ref{sec:martinet}}. Because it is the voiced geminates which were rare and the lenited voiceless stops which were common, it is reasonable to assume that voiced geminates merged to the value of lenited voiceless stops, and not vice versa.

When etymological \gls{T} is written with \ow{b, d, g}, this is frequently done in the vicinity of \gls{rescon}s, \ie \ow{m, n, l, r}. Having established that word-medial \xD\ and \lT\ had already merged in \gls{ow}, the instances of \ow{b, d, g} representing stops cannot constitute different phonemes. I propose that this inconsistency with which the same phonemes are written in word-medial position may be understood in analogy with word-initial contrasts. That is, word-initial \lT\ and \xD\ differed from one another; yet word-medial \gls{D} was the result of a merger between the two, and this medial \gls{D} usually stood phonetically closer to initial \lT, but sometimes closer to initial \xD. Whenever word-medial \gls{D} stood phonetically closer to \xD\ than to \lT, it had a higher chance of being written with \ow{b, d, g} than otherwise. \Gls{rescon}s are known cross-linguistically to swallow the aspiration of adjacent consonants (See Section~\ref{sec:from-brittonic-welsh} and \textcite{koch_*cothairche_1990}). Thus, assuming that word-initial \lT\ was aspirated, while \xD\ was unaspirated, and assuming that word-medial \gls{D} was usually pronounced as \lT, adjacent \gls{rescon}s and their swallowing effect provided a context within which medial \gls{D} became comparatively more similar to \xD\ and less similar to \lT.

The \gls{ow} evidence in Chapter~\ref{oldwelsh} partly confirms the three-way stop distinction hypothesised. The analogy proposed to account for writing medial \gls{D} with \ow{b, d, g} occasionally implies that initial \lT\ was aspirated, and that \xD\ was not. However, \gls{ow} does not give a complete picture of the three-way distinction; there is no evidence either way whether length served a role to distinguish \xT\ and \xD\ from \lT.

Chapter~\ref{cha:prov-mwbe-y} does provide evidence for phonemic length of stops. The cynghanedd of the eleventh- to thirteenth-century Poets of the Princes gives insight into the sandhi patterns of the Welsh of this period. These sandhi patterns reveal the existence of phonemic stop length: long stops served to indicate radical phonemes and short stops served to indicate lenited phonemes. Evidence for phonemic aspiration is also found: \xT\ was aspirated and \xD\ was not.

It is unclear from this chapter, however, whether \lT\ was still aspirated. It is clear that \lT\ was aspirated whenever it was lengthened through sandhi, but this does not necessarily imply that \lT\ was aspirated in all sandhi positions. Section~\ref{sec:my-reconstruction} posits the need for a sound law where voicing, \ie loss of aspiration redundantly marked lenition in addition to short length. This sound law must have occurred before the merger of \lT\ and \xD, so it is quite possible that redundant voicing of lenition was already active at the time of the Poetry of the Princes.

The  chronology of lenition of stops from \gls{pic} to \gls{mow} is illustrated in Figure~\ref{fig:pctomowstops}. It is a reworking of Figure~\ref{fig:pctolbbbstops} in that it considers \gls{mow} rather than Léon Breton. It is a relative chronology, not an absolute chronology. That is, it gives the sound laws and the order in which the sound laws occurred needed to derive a \gls{mow} stop phonology, but it gives no dates as to when they occurred. The next step is to establish the dates of these sound laws.

\begin{figure}[h]
  %%% Modification of fig fig:pctolbbbstops
  \centering
  \begin{tikzpicture}
    %%% NODES
    \node(topl){};
    \node[right= 2.5cm of topl](pcxt){\gls{hash}\xT};
    \node[right= 5cm of topl](pclt){\gls{hash}\lT};
    \node[right= 7.5cm of topl](pcxd){\gls{hash}\xD};
    \node[below= 1cm of topl,align=left,anchor=west](1l){%
      Phonetic\\lenition};
    \node(1xt) at (1l-|pcxt){{[}pː tː kː]};
    \node(1lt) at (1l-|pclt){{[}p t k]};
    \node(1xd) at (1l-|pcxd){{[bː dː ɡː]}};
    \node[below=3cm of topl,anchor=west](2l){%
      Apocope};
    \node(2xt) at (2l-|pcxt){/pː tː kː/};
    \node(2lt) at (2l-|pclt){/p t k/};
    \node(2xd) at (2l-|pcxd){/bː dː ɡː/};
    \node[below=5cm of topl,align=left,anchor=west](3l){%
      Redundant\\voicing\\of \lT};
    \node(3xt) at (3l-|pcxt){/pː tː kː/};
    \node(3lt) at (3l-|pclt){/b d ɡ/};
    \node(3xd) at (3l-|pcxd){/bː dː ɡː/};
    \node[below=7cm of topl,align=left,anchor=west](4l){%
      Loss of \\phonemic \\length};
    \node(4t) at (4l-|pcxt){/p t k/};
    \node(3d)  at ($(3lt)!0.5!(3xd)$){};%%dummynode
    \node(4d) at (4l-|3d){/b d ɡ/};
    %%% ARROWS
    \foreach \x in {xt,lt,xd}{%
      \draw[->] (1\x) -- (2\x);
      \draw[->] (2\x) -- (3\x);
    }
    \draw[->] (3xt) -- (4t);
    \draw[->] (3lt) -- (4d);
    \draw[->] (3xd) -- (4d);
    %%% OPPOSITIONS
    \draw[<->,dashed, bend left = 50]
    (2xt) to node[midway,above]{±length} (2lt);
    \draw[<->,dashed, bend left = 50]
    (2lt) to node[midway,above]{±\gls{vot}\vphantom{g}} (2xd);
    \draw[<->,dashed, bend left = 50]
    (3xt) to node[midway,above]{±\gls{vot}\vphantom{g}} (3lt);
    \draw[<->,dashed, bend left = 50]
    (3lt) to node[midway,above]{±length} (3xd);
    \draw[<->,dashed, bend left = 50]
    (4t) to node[midway,above]{±\gls{vot}\vphantom{g}} (4d);
    %%% EXPLANATION
    \node[%
    below=9cm of topl,
    font=\footnotesize,
    align=left,
    text width=8cm,
    anchor=west] (5l){%
      Dashed lines indicate the phonological variable
      distinguishing pairs of stop series in a given period.
    };
  \end{tikzpicture}
  \caption{From \gls{pic} to \gls{mow} word-initial stops.}
  \label{fig:pctomowstops}
\end{figure}


\section{Towards an absolute chronology}
\label{sec:towards-more-precise}
Part~\ref{part:phonology-phonetics} gives some clues for the date of redundant voicing of lenited stops. Here I discuss these clues. The date of the loss of phonemic length is the topic of Part~\ref{part:orthography}, especially Chapter~\ref{cha:orth-concl}. It establishes the thirteenth century as the date when \lT\ merged with \xD, which implies the loss of phonemic length. This loss of phonemic length must have occurred after redundant voicing, because otherwise \lT\ would merge back with \xT.

The date of \gls{apoc} is argued by \textcite[§~182]{jackson_language_1953} to be the mid-sixth century; this date may be thought of as the full phonemicisation of lenition. In Figure~\ref{fig:pctomowstops}, this phonemicisation precedes redundant voicing of \lT, but there is no \lat{a priori} reason why this should be the case. Nevertheless, there is reason to think phonemicisation of lenition indeed preceded this redundant voicing. 

One reason to assume redundant voicing followed \gls{apoc} is the evidence given in Chapter~\ref{oldwelsh}, which implies that word-medial \gls{D} and word-initial \lT\ were still aspirated in \gls{ow}, well after \gls{apoc}. Conversely, the \oi{Coirthech}-stratum of loanwords from Brittonic into Goidelic provides evidence from the fifth century, shortly before \gls{apoc}, that word-medial \lT\ was no more voiced than \xT\footnote{It is harder to infer anything from the \oi{Pádraig}-stratum of loanwords from Brittonic to Goidelic: after Goidelic \lT\ became fricatives, a short aspirated Brittonic \lT\ no longer had an obvious Goidelic counterpart; it is thus unclear what would happen if such a sound were borrowed by Goidelic speakers.}. 

Redundant voicing or lack thereof is difficult to observe precisely because it is redundant. Still, there are some developments preceding the merger of \lT\ with \xD\ where redundant voicing may be observed. For example, even the earliest manuscripts containing \mw[]{Brut y Brenhinedd} discussed in Chapter~\ref{cha:indep-comp-mwbr} contain instances where \gls{morphophonlen} was petrified written with \mw{b, d, g}. In these manuscripts, lenited \mw[with]{can} is written as \mw[with]{y gan} even though lenition of \mw{c} is not otherwise represented.

When \gls{morphophonlen} was petrified, length distinctions must have been lost just like what \textcite{falchun_systeme_1951} observes for Le Bourg Blanc Breton\footnote{See Section~\ref{sec:falchun}.}. Then, if \gls{petr} descending from \gls{morphophonlen}\footnote{As opposed to clitic reduction.} was to yield \gmw{b, d, g}, this must have entailed fossilisation of redundant voicing. This implies that redundant voicing operated for enough time to yield these spellings with \mw{b, d, g} in these instances of \gls{petr} before consonantal length was lost more generally. On the other hand, manuscripts \gls{sA} and \gls{sC}, the earliest law texts discussed in Chapter~\ref{cha:welsh-laws}, do have instances of \mw[with]{y kan}. This may mean that these manuscripts still contain an orthographical stratum preceding redundant voicing, but it may also mean that lenition of \mw{c} was simply not petrified yet in this stratum.

The first instances where \gls{morphophonlen} is written with \mw{b, d, g} is the twelfth-century \mw{Braint Teilo}\footnote{See Example~\ref{ex:braintteiloardir} in Section~\ref{sec:two-exampl-mowm}.}. This instance implies that redundant voicing operated by the twelfth century.  

Another pointer to the date of redundant voicing is the fact that voicing of \lT\ is ultimately shared between all Brittonic dialects. It would therefore be attractive to posit a sound law of redundant voicing before the Brittonic dialects split up. This split is hard to date precisely. It may be dated from the earliest Brittonic migrations to the end of the \gls{ow} period, when the differences between the Brittonic dialects first became pronounced.

These considerations allow for some triangulation, but the date of redundant voicing as a sound law remains imprecise except that it must have been active within the \gls{ow} period  by the twelfth century. This uncertainty is  unsurprising, because this voicing rule operated as a synchronic rule fairly late in the phonological computation: iIt must have followed sandhi rules in the phonological computation, given how a \gls{doubconclus} of \gls{D}+\lT\ yields \xT, and it was initially redundant with short consonant length. There is no reason for an orthographical system to develop a notation for such a redundant feature.

% \begin{itemize}
% \item I borrow Carlyle's term for the voicing of short consonants; 
% \item This voicing is initially redundant: first it is shortness which marks lenition, then it is shortness as well as voice;
% \item This redundancy is observed to be productive in Léon Breton, but it may also be posited to have operated as a sound law in Welsh historically
% \item This voicing chiefly constitutes loss of aspiration in Welsh, because Welsh is an aspirating language rather than a voicing language. This constitutes merely a quantitative difference as argued in Section~\ref{sec:voice-aspiration};
% \item It is widely observed that word-medial \gls{D} patterns with \lT\ rather than with \xD, so it stands to reason that redundant voicing is applied to word-medial \gls{D} in the same period as initial \lT;
% \item It is unclear when this occurred in Welsh, but at any rate it occurred before loss of distinctive length in the 13th century. When length was lost, voicedness became the only marker of lenition, as it is now in \gls{mow};
% \item The previous chapters give some pointers to the earliest and latest possible dates for redundant voicing. Redundant voicing must have occurred after the following:
%   \begin{itemize}
%   \item The \oi{Coirthech}-stratum of Brittonic loanwords into Goidelic
%   \item The \gls{ow} evidence for aspiration in medial \gls{D} given in Chapter~\ref{oldwelsh}.
%   \end{itemize}
  
% \item Redundant voicing must have occurred before the following:
%   \begin{itemize}
%   \item When lenition in \lT\ becomes petrified, it is reanalysed as \xD. It is difficult to conceive how \lT\ could become \xD\ before redundant voicing: in such a scenario petrification of lenition would entail both voicing and loss of distinctive length, while petrification of lenition after redundant voicing would only entail loss of distinctive length.
%     \begin{itemize}
%     \item Here it should be noted that petrified lenition found in
%       even the earliest translations of \mw{Brut y Brenhinedd}
%       (discussed in Chapter~\ref{cha:indep-comp-mwbr}) represent
%       petrified lenition of \lT\ consistently with \mw{p, t, c}.
%     \item The law manuscripts have some instances of petrified
%       lenition not being represented, \eg \mw[with]{y kan} is found
%       in \gls{sA} and \gls{sC}. This may mean that the earliest
%       written versions of the Book of Iorwerth prece redundant
%       voicing, but redundant voicing may precede even this version if
%       accounting for orthographical conservatism. Alternatively again, 
%       lenition in \mw[]{y kan} may simply not have petrified yet when
%       it was first written.
%     \end{itemize}

%   \item ?? The evidential value of the \oi{Pádraig}-stratum of loanwords is doubtful, because after fricativisation of Goidelic \lT\, a short aspirated Brittonic \lT\ does not have an obvious Goidelic counterpart. It is thus unclear what we would expect to happen in a scenario where Brittonic \lT\ before redundant voicing was borrowed into Goidelic following Goidelic fricativisation of \lT.
%   \end{itemize}
  
% \item The cynghanedd offers little evidence either way: the gemination caused by doubling constitutes lengthening of \lT, and because the resulting consonant is long redundant voicing does not apply.
% \item Voicing of \lT\ is shared between all Brittonic dialects. It would therefore be attractive to posit the sound law of redundant voicing before the Brittonic dialects split up. The alternative would be to posit independent innovation of redundant voicing in the Brittonic dialects.
% \end{itemize}


\section{The chronology of lenition and Insular Celtic}
\label{sec:cons-other-rese-1}
The chronology of the phonology of lenited voiceless stops proposed in Part~\ref{part:phonology-phonetics} has consequences for thinking on Insular Celtic. \Textcite{matasovic_insular_2008} considers the divergent ways in which lenited voiceless stops are realised in Brittonic and Goidelic as evidence that Insular Celtic must be considered  a linguistic area, and that there is no such thing as a \gls{pic} node on the Celtic family tree:

\tqt{in British, the voiceless stops become voiced between vowels, while in Goidelic they become voiceless fricatives.
  What is common to IC developments is that in both cases lenition applied across
  word boundaries. It is as if both languages at the same time developed a rule prohibiting the occurrence of voiceless stops between vowels; such a rule could initially have developed in bilingual communities, and subsequently spread to monolingual speakers of both languages.  After the phonetic lenition of stops, and the subsequent apocope of final vowels, the results of word-initial lenition were
  grammaticalized, producing the system of consonant mutations.  This development had to be independent in British and Goidelic, because it presupposes
  earlier independent lenition, but there had to be some sort of causal connection.
  This conclusion cannot be avoided, because consonant mutations are typologically so rare that it would be extremely improbable that they developed in two
  neighbouring languages at approximately the same time, yet completely accidentally. The most likely explanation is that consonant mutations, as a type of
  morphophonemic rule, first developed in bilingual communities speaking early
  forms of British and Goidelic. The rules turned out differently in the two languages, because their phonological systems were already significantly different
  from each other.
}{matasovic_insular_2008}{97--98}
Matasović erroneously assumes that phonetic lenition of Brittonnic  voiceless stops must immediately have constituted voicing --- a phonological merger with unlenited voiced stops.

The phonological evidence of this thesis invalidates this argument. Before \gls{apoc}, \lT\ only differed from \xT\ in that it was shorter; not in degree of voicing. Only after \gls{apoc} and redundant voicing, Brittonnic  \lT\ was more voiced than \xT, but even then length drove the phonemic distinction between \xT\ and \lT. The phonetics of such lenition can be reconciled with an early pre-phonemic Goidelic lenition. That is, voicing would not stand in the way if one were to attempt to derive Goidelic lenition from Brittonic lenition even as late as post-apocope Brittonic. Consequently, there is no need to believe  that phonetic lenition of voiceless stops cannot have been a feature of a \gls{pic} proto-language that formed a node in the Celtic family tree. Hence, the chronology of lenition does not provide evidence against a \gls{pic} node on the Celtic family tree.

This does not necessarily imply that the chronology of lenition provides positive evidence for such a \gls{pic} node, however. In order to argue for such a node, one would first have to argue that developments in the chronology of lenition were not shared with Continental Celtic; even then lenition may be thought of as an areal feature. \Textcite{martinet_celtic_1952} demonstrates how lenition is also found in \eg Romance, but has remained sub-phonemic in most dialects\footnote{\Textcite{Oft_Lenition85} analyses Gran Canarian Spanish as having developed morphophonemic mutations similar to what is found in Insular Celtic.}. Lenition is indeed an areal feature: not an areal feature restricted to Insular Celtic, but a rather larger western European areal feature including  Romance languages.



% \todo[inline]{Structure of the conclusion
% \begin{itemize}
% \item What was the goal of Part~\ref{part:phonology-phonetics}?
% \item Does the hypothesis given in the introduction hold up?
% \item Per chapter: summarise what was found
% \item What are the implications of the findings, and more specifically:  
%   \begin{itemize}
%   \item Qualify the scope of what Chapter~\ref{oldwelsh} and Chapter~\ref{cha:prov-mwbe-y} really say about Welsh phonology
%   \item Repeat and rework Figure~\ref{fig:pctolbbbstops} for Welsh including dates and the final step of loss of distinctive length. 
%   \end{itemize}
% \item What issues merit further research?
%   \begin{itemize}
%   \item Most important: dating redundant voicing (see below)
%   \end{itemize}
% \item What other research topics may be revisited in the wake of Part~\ref{part:phonology-phonetics}?
%   \begin{itemize}
%   \item Most important: Matasovic and \gls{pic}
%   \end{itemize}
% \item Lookahead to Part~\ref{part:orthography}
% \end{itemize}
% }

% \begin{figure}[h]
%   \centering
%   \begin{tikzpicture}
%     \draw[->, thick] (1,3) to node[above]{\gls{vot}} (4,3);
%     \draw[->, thick] (1,3) to node[left]{length} (1,0);
%     \node at (1.5,0.5) {\xD};
%     \node at (3.5,0.5) {\xT};
%     \node at (3.5,2.5) {\lT};
%   \end{tikzpicture}
%   \caption{The two dimensions distinguishing three stop series}
%   \label{fig:twodimthreestop}
% \end{figure}

% \section{On length}
% Schrijver considers aspiration as only an epiphenomenon of the other phonological properties, as shown in Table~\ref{stopsystemschrijver}. Nevertheless, Schrijver  is convinced that they were phonetically aspirated: `MoW evidence shows that these were probably strongly aspirated (except after \textit{s}), the aspiration having been lost in Cornish and Breton under Late Latin influence'~\autocite*[31]{schrijver_old_2011}. For arguments, see \textcite[§~25]{koch_*cothairche_1990}.

% \begin{table}[h]
%   \centering
%   \begin{tabular}{{@{}lll@{}}}\toprule
%     \xT & \xD & \lT \\\midrule
%     {[}-voice] & [+voice] & [+voice] \\
%     {[}+long] & [+long] & [-long] \\\midrule
%     p{[}ʰː{]} & bː & b \\
%     t{[}ʰː{]} & dː & d \\
%     k{[}ʰː{]} & ɡː & ɡ\\\bottomrule
%   \end{tabular}
%   \caption{Common Brittonic stop system according to Schrijver \autocite*[33]{schrijver_old_2011}}
%   \label{stopsystemschrijver}
% \end{table}

% Phonemic consonant length is not unique to stops in Middle Welsh: \gls{rescon}s also had an opposition between /nː/ and /n/, /rː/ and /r/, and /lː/ and /l/. This opposition disappeared by the end of the \gls{mw} period~\autocite[127]{schumacher_mittel-_2011}. These long voiced \gls{rescon}s were found in e.g.\ the \gls{mw} word for `heart' sometimes spelled \mw{callon}, or in \mw{penn} `head'. Both length distinctions were neutralised by the end of the \gls{mw} period. The phonemic status of these long voiced \gls{rescon}s did not only contrast with their short voiced counterparts /n, l, r/, but also with voiceless/aspirated counterparts /n̥, ɬ, r̥/. The three-way contrast system exactly mirrors that of stops in this sense.




% The use of length in keeping apart voiceless radicals from their lenited counterparts has consequences for how they are connected to the rise of the spirant mutation from geminates in British. If a radical voiceless stop undergoes the spirant mutation, it turns into its corresponding fricative: i.e.\ /p, t, k/ become /f, θ, x/. \textcite{schrijver_spirantization_1999} tackles the issue of spirantization, and gives some relevant points.  Schrijver follows \textcite{greene_gemination_1956} in dating spirantization after apocope. Radical \textit{D} and lenited \textit{T} were only kept separate word-initially following apocope, so only word-initial spirantization is relevant here. Spirantization occurred across word boundaries in the following contexts: following pre-apocope \textit{*-s},  \textit{*-T, *-N, *-r,} and \textit{*-l}~\autocite[3]{schrijver_spirantization_1999}. After all of these consonants, a following voiceless stop would come to stand in a phonetically longer consonant cluster. Following apocope, the exact final consonant preceding the consonant to be spirantized (or proto-spirant consonant) became obscured, leaving only length as a marker for proto-spirant consonants. This matter presents a problem: if being long was a feature of proto-spirant consonants, then how could length be used to differentiate between radical and lenited voiceless stops? A phonemic three-way length distinction in stop consonants is unlikely at best, and was therefore unstable. This fact explains why the longest stop series, the proto-spirants, developed into spirants\footnote{If gemination also arose in Goidelic at any point, the phonemic system reorganised itself in a near-identical way: one voiceless stop series became a series of spirants, only this time the shortest (lenis) voiceless stops developed into spirants.}.  






% \section{From my seminar}
% Beth bynnag, hoffwn i gyrraedd rhyw fath o derfyniad nawr. Felly, dyna'r syniadau o sut roedd y \textit{stop systems} yn gweithio yn yr Hen Amser eto \emph{The Proto-Brittonic stop system}. Dyna Koch a Schrijver, a'u systemau tan nawr. Mae system Koch yn defnyddio anadliad caled a llais i gyferbynnu'r tair cyfres. Dyw e ddim yn s\^on am hyd. Mae Schrijver yn cop\"io hyd o Lydaweg, ac mae hynny'n cydweithio gyda llais i gyferbynnu'r tair cyfres. Mae'n s\^on am anadliad caled, ond dim ond mewn dimensiwn seinegol. Dyw anadliad caled ddim yn chwarae r\^ol i gyferbynnu dim byd, felly dyw anadliad caled ddim yn ffonemaidd. Mae hynny'n rhyfedd i fi; dyw hynny ddim yn esboniad economaidd. Hefyd, mae pawb yn glynu wrth lais, er bod hi mor hawdd dadlau bod cyferbyniad llais-dilais yn Llydaweg yn arloesiad sy'n dod o Ladin. Dyw e ddim yna yn y Gymraeg...

% Mae'r camsillafiadau yn yr Hen Gymraeg yn awgrymu i fi fod y gyfres ddi-lais dreigledig yn anadlog (\textit{aspirated}). Mae'r gynghanedd yn awgrymu i fi fod [hyd] yn chwarae r\^ol i gyferbynnu \xD\ a \lT, ble mae \textit{d} gysefin yn hir, a \textit{t} dreigledig yn fyr. Os felly, dyn ni'n cyrraedd y system ganlynol: \emph{Me}. Mae'r system yma yn seiliedig ar dystiolaeth seinegol Gymraeg yn unig, ac mae'n anwybyddu arloesiadau Llydaweg. Dyw llais ddim yna, mae hyd yn cyferbynnu cysefin a threigledig, ac mae anadliad caled yn cyferbynnu y ffrwydrolion di-lais a'r rhai lleisiol.

% Mae'r esboniad yma yn elegant fel yna. Os mai hon oedd y system yn yr amser yna, byddwn i'n rhagweld iddi hi ddiflannu hefyd. Yr unig arloesiad erbyn hyn ydy bod cyferbyniad meintiol wedi diflannu. Trwy hanes y Gymraeg, mae sawl system feintiol wedi diflannu: llafariaid: /a:\=a/ -> /au:a/, a seiniau soniarus: e.e.\ mae \textit{l} hir yn \textit{calon} wedi diflannu, dim ond olion ar \^ol o \textit{n} hir a byr: \textit{penn:hen}, a.y.y.b.

% Os oes amser, gallaf drafod:
% \begin{itemize}
% \item  treiglad llaes: [tːʰ : tʰ] > [θ : tːʰ], oherwydd bod cyferbyniad hyd newydd yn dod mewn gydag apocope. Mae hyn yn esbonio'r treiglad llaes fel \textit{chain shift}
% \item \textit{Irish lenition} efallai fod hwn yn esbonio sut oedd \textit{t} treigledig yn troi yn /θ/ > /h/ yn Wyddeleg.
% \end{itemize}


% In my view, phonetic voice did not play a phonological role in distinguishing the three series of stops. This is not a radical thought, because even nowadays what we call voiceless and voiced stops in Welsh, are actually realised as an opposition between aspirated and non-aspirated.

% Schrijver proposes that length kept lenited voiceless stops and unlenited voiced stops apart. This is what we find in MoB and it also mirrors \gls{rescon}s, which originally had quantitative oppositions between lenited and unlenited. Schrijver's symbols also imply that he considers lenited voiceless stops unaspirated, and their radical counterparts aspirated. However, he does not consider aspiration to be phonologically distinctive. 

% This does not make sense. For one: aspiration must have been more than a phonemic feature, because some consonant series do have it, and others do not. Moreover, Schrijver brings in voice as phonemic, but this is nowadays only found in Breton, and thought to be Latin influence. A similar case for aspiration and length cannot be made, so let's use these as building blocks.

% Now, early cynghanedd (tentatively) shows that two lenited voiceless stops may alliterate with a single unlenited voiceless stop. The same is not the case with two unlenited voiced stops. Assuming that two short consonants in succession are equivalent to a single long consonant, this indeed implies that length played a role in distinguishing lenited from unlenited voiceless stops, while it did not play a role in distinguishing unlenited voiceless and voiced stops. Moreover, aspiration must have been present in these doubled lenited voiceless stops in order for them to be equivalent to their radical counterparts.   

% The cynghanedd also shows that two lenited voiceless stops do not alliterate with a single unlenited voiced stop. If they did, that would have implied that length was the only element distinguishing lenited voiceless stops and radical voiced stops. The fact that they do not alliterate implicates an element in addition to length must have been present in disambiguating the two series. This suggests that Schrijver's model is incomplete.

% Old Welsh evidence shows that lenited voiceless stops may gave been aspirated, while unlenited voiced stops were not. If we assume that non-initial stops were written with initial stops as their analogical base, then my proposed stop system explains why word-medial /b, d, g/ (from both \lT\ and \xD) are written with \graph{b, d, g} next to \gls{rescon}s sometimes, but barely ever intervocalically. Presence of a \gls{rescon} consonant next to an aspirated stop is known to `eat' the aspiration. Whenever \graph{b d g} are written word-initially in Old Welsh, we are sure they do not cause aspiration, and we may be sure that these word-medial consonants do not cause aspiration either given their usual proximity to \gls{rescon}s. It follows, then, that aspiration was principally there wherever \graph{p t c} were written, even if they were lenited. Otherwise, there would be no explanation for the distribution of \graph{b d g} to represent stops intervocalically.

% Note that the difference between lenited voiceless stops and unlenited voiced stops was only maintained word-initially. Elsewhere, intervocalic voiced geminates had merged with regular intervocalic voiceless stops by the OBr.\ period. Evidence for this is found in Old Welsh words having a historical unlenited voiced stop non-initially consistently being written with \graph{p t c} .


% \begin{sidewaysfigure}[h]
%   \newcommand{\asp}[1][+]{[\textsc{#1asp}]}
%   \newcommand{\lng}[1][+]{[\textsc{#1long}]}
%   \centering
%   \begin{tikzpicture}[
%     ->,
%     align=left,
%     condit/.style={font=\footnotesize,above,near end, sloped}]
%     \node(pb){Archiphoneme};
%     \node[right=of pb](phot){Phonetic lenition \&\\First spirantization};
%     \node[right=of phot](apoc){Apocope};
%     \node[right=of apoc](spir){Second \\spirantization};
%     \node[right=of spir](merge){Merger\\\xD/\lT};
%     %% Voiceless stops
%     \node[below=35mm of pb](t1){/pʰ/\\/tʰ/\\/kʰ/};
%     \node[above=5mm of t1](t){\gls{T}\asp};
%     \node[yshift=9mm]at (phot|-t1)(t2r){[pːʰ]\\{[tːʰ]}\\{[kːʰ]}};
%     \node[yshift=-9mm]at (phot|-t1)(t2l){{[pʰ]}\\{[tʰ]}\\{[kʰ]}};
%     \node(t3r) at (apoc|-t2r){/pːʰ/\\{/tːʰ/}\\{/kːʰ/}};
%     \node(t3l) at (apoc|-t2l){/pʰ/\\{/tʰ/}\\{/kʰ/}};
%     \node(t4rs)[yshift=18mm] at (spir|-t3r){/f/\\/θ/\\/x/};
%     \node(t4ri) at (spir|-t3r){};
%     \node(t4l) at (spir|-t3l){};
%     \node(t5ri) at (merge|-t4ri){{/pʰ/}\\{/tʰ/}\\{/kʰ/}};
%     \node(t5rs) at (merge|-t4rs){};
%     %% Voiced stops
%     \node[below=75mm of pb](d1){/p/\\/t/\\/k/};
%     \node[above =5mm of d1](d){\gls{D}\asp[-]};
%     \node[yshift=9mm] at (phot|-d1)(d2r){[pː]\\{[tː]}\\{[kː]}};
%     \node[yshift=-9mm]at (phot|-d1)(d2l){[β]\\{[ð]}\\{[ɣ]}};
%     \node at (apoc|-d2r)(d3r){/pː/\\{/tː/}\\{/kː/}};
%     \node at (apoc|-d2l) (d3l){/β/\\{/ð/}\\{/ɣ/}};
%     \node at (spir|-d3r)(d4r){};
%     \node at (spir|-d3l)(d4l){};
%     \node at (merge|-d4l)(d5l){};
%     %% Merged stops
%     \node  at ($(t4l)!0.5!(d4r)$)(td4){};
%     \node at (merge|-td4)(td5){/p/\\/t/\\/k/};
%     %% Arrows
%     \draw[dashed] (t1) to node[condit] {/¬V\_V,R}(t2r);
%     \draw[dashed] (t1) to node[condit] {/V\_V,R}(t2l);
%     \draw[dashed] (d1) to node[condit] {/¬V\_V,R}(d2r);
%     \draw[dashed] (d1) to node[condit] {/V\_V,R}(d2l);
%     \draw (t2r)to node[above] {\lng}(t3r);
%     \draw (t2l)to node[above] {\lng[-]}(t3l);
%     \draw (d2r)to node[above] {\lng}(d3r);
%     \draw (d2l)to (d3l);
%     \draw[dashed] (t3r) to node[condit] {/¬\gls{hash}\_}(t4rs);
%     \draw[dashed] (t3r) to node[condit] {/\gls{hash}\_}(t4ri);
%     \draw(t4ri)--(t5ri);
%     \draw (t3l) to (td5);
%     \draw(d3r)--(td5);
%     \draw (d3l)--(d5l);
%     \draw[dashed] (d2r)to node[condit] {/¬\gls{hash}\_}(t3l);
%     \draw(t4rs)--(t5rs);
%   \end{tikzpicture}
%   \caption{Summary of posited phonological developments (Apocope before second spirantization)}
%   \label{fig:phondevelopconc}
% \end{sidewaysfigure}

% \begin{sidewaysfigure}[h]
%   \newcommand{\asp}[1][+]{[\textsc{#1asp}]}
%   \newcommand{\lng}[1][+]{[\textsc{#1long}]}
%   \centering
%   \begin{tikzpicture}[
%     ->,
%     align=left,
%     condit/.style={font=\footnotesize,above,near end, sloped}]
%     \node(pb){Archiphoneme};
%     \node[right=of pb](phot){Phonetic lenition \&\\1\textsuperscript{st}  spirantization};
%     \node[right=of phot](spir){1\textsuperscript{st} merger \lT/\xD\ \&\\2\textsuperscript{nd} spirantization};
%     \node[right=of spir](apoc){Apocope};
%     \node[right=of apoc](merge){2\textsuperscript{nd} merger \xD/\lT};
%     %% Voiceless stops
%     \node[below=35mm of pb](t1){/pʰ/\\/tʰ/\\/kʰ/};
%     \node[above=5mm of t1](t){\gls{T}\asp};
%     \node[yshift=9mm]at (phot|-t1)(t2r){[pːʰ]\\{[tːʰ]}\\{[kːʰ]}};
%     \node[yshift=-9mm]at (phot|-t1)(t2l){{[pʰ]}\\{[tʰ]}\\{[kʰ]}};
%     \node(t3r) at (apoc|-t2r){/pːʰ/\\{/tːʰ/}\\{/kːʰ/}};
%     \node(t3l) at (apoc|-t2l){/pʰ/\\{/tʰ/}\\{/kʰ/}};
%     \node(t4rs)[yshift=18mm] at (spir|-t2r){[f]\\{[θ]}\\{[x]}};
%     \node(t3rs) at (apoc|-t4rs){/f/\\/θ/\\/x/};
%     \node(t4ri) at (spir|-t3r){};
%     \node(t4l) at (spir|-t3l){};
%     \node(t5ri) at (merge|-t4ri){{/pʰ/}\\{/tʰ/}\\{/kʰ/}};
%     \node(t5rs) at (merge|-t4rs){};
%     %% Voiced stops
%     \node[below=75mm of pb](d1){/p/\\/t/\\/k/};
%     \node[above =5mm of d1](d){\gls{D}\asp[-]};
%     \node[yshift=9mm] at (phot|-d1)(d2r){[pː]\\{[tː]}\\{[kː]}};
%     \node[yshift=-9mm]at (phot|-d1)(d2l){[β]\\{[ð]}\\{[ɣ]}};
%     \node at (apoc|-d2r)(d3r){/pː/\\{/tː/}\\{/kː/}};
%     \node at (apoc|-d2l) (d3l){/β/\\{/ð/}\\{/ɣ/}};
%     \node at (spir|-d3r)(d4r){};
%     \node at (spir|-d3l)(d4l){};
%     \node at (merge|-d4l)(d5l){};
%     %% Merged stops
%     \node  at ($(t4l)!0.5!(d4r)$)(td4){};
%     \node at (merge|-td4)(td5){/p/\\/t/\\/k/};
%     %% Arrows
%     \draw[dashed] (t1) to node[condit] {/¬V\_V,R}(t2r);
%     \draw[dashed] (t1) to node[condit] {/V\_V,R}(t2l);
%     \draw[dashed] (d1) to node[condit] {/¬V\_V,R}(d2r);
%     \draw[dashed] (d1) to node[condit] {/V\_V,R}(d2l);
%     \draw[dashed] (t2r)to node[condit, near start] {/\gls{hash}\_} node[above] {\lng}(t3r);
%     \draw (t2l)to node[above] {\lng[-]}(t3l);
%     \draw[dashed] (d2r)to node[condit, near start] {/\gls{hash}\_} node[above] {\lng}(d3r);
%     \draw (d2l)to (d3l);
%     \draw[dashed] (t2r) to node[condit,midway] {/¬\gls{hash}\_}(t4rs);
%     \draw(t3r)--(t5ri);
%     \draw (t3l) to (td5);
%     \draw(d3r)--(td5);
%     \draw (d3l)--(d5l);
%     \draw[dashed] (d2r)to node[condit,midway] {/¬\gls{hash}\_}(t4l);
%     \draw(t4rs)--(t3rs);
%     \draw(t3rs)--(t5rs);
%   \end{tikzpicture}
%   \caption{Summary of posited phonological developments (Apocope after second spirantization)}
%   \label{fig:phondevelopconc2}
% \end{sidewaysfigure}

% \begin{sidewaysfigure}[h]
%   \newcommand{\asp}[1][+]{[\textsc{#1asp}]}
%   \newcommand{\lng}[1][+]{[\textsc{#1long}]}
%   \centering
%   \begin{tikzpicture}[
%     ->,
%     align=left,
%     condit/.style={font=\footnotesize,above,pos=0.8, sloped}]
%     \node(pb){Archiphoneme\\/kʷ/ > /p/};
%     \node[right=of pb](phot){Phonetic lenition \&\\1\textsuperscript{st}  spirantization};
%     \node[right=of phot](spir){1\textsuperscript{st} merger \lT/\xD\ \&\\2\textsuperscript{nd} spirantization \&\\
%       Apocope};
%     \node[above=of spir](apoc){Proto-Brittonnic};
%     \node[below right=of apoc](merge){2\textsuperscript{nd} merger \xD/\lT};
%     \node[above=of merge]{Welsh};
%     \node at ($(pb)!0.5!(phot)$)(pic){};
%     \node at (pic|-apoc){Proto-Insular-Celtic};
%     %% Voiceless stops
%     \node[below=35mm of pb](t1){/pʰ/\\/tʰ/\\/kʰ/};
%     \node[above=5mm of t1](t){\gls{T}\asp};
%     \node[yshift=9mm]at (phot|-t1)(t2r){[pːʰ]\\{[tːʰ]}\\{[kːʰ]}};
%     \node[yshift=-9mm]at (phot|-t1)(t2l){{[pʰ]}\\{[tʰ]}\\{[kʰ]}};
%     \node(t3r) at (apoc|-t2r){/pːʰ/\\{/tːʰ/}\\{/kːʰ/}};
%     \node(t3l) at (apoc|-t2l){/pʰ/\\{/tʰ/}\\{/kʰ/}};
%     \node(t4rs)[yshift=20mm] at (spir|-t2r){};
%     \node(t3rs) at (apoc|-t4rs){/f/\\/θ/\\/x/};
%     \node(t4ri) at (spir|-t3r){};
%     \node(t4l) at (spir|-t3l){};
%     \node(t5ri) at (merge|-t4ri){{/pʰ/}\\{/tʰ/}\\{/kʰ/}};
%     \node(t5rs) at (merge|-t4rs){/f/\\/θ/\\/x/};
%     %% Voiced stops
%     \node[below=75mm of pb](d1){/p/\\/t/\\/k/};
%     \node[above =5mm of d1](d){\gls{D}\asp[-]};
%     \node[yshift=9mm] at (phot|-d1)(d2r){[pː]\\{[tː]}\\{[kː]}};
%     \node[yshift=-9mm]at (phot|-d1)(d2l){[β]\\{[ð]}\\{[ɣ]}};
%     \node at (apoc|-d2r)(d3r){/pː/\\{/tː/}\\{/kː/}};
%     \node at (apoc|-d2l) (d3l){/β/\\{/ð/}\\{/ɣ/}};
%     \node at (spir|-d3r)(d4r){};
%     \node at (spir|-d3l)(d4l){};
%     \node at (merge|-d4l)(d5l){/β/\\{/ð/}\\{/ɣ/}};
%     %% Merged stops
%     \node  at ($(t4l)!0.5!(d4r)$)(td4){};
%     \node at (merge|-td4)(td5){/p/\\/t/\\/k/};
%     %% Arrows
%     \draw[dashed] (t1) to node[condit] {/¬V\_V,R}(t2r);
%     \draw[dashed] (t1) to node[condit] {/V\_V,R}(t2l);
%     \draw[dashed] (d1) to node[condit] {/¬V\_V,R}(d2r);
%     \draw[dashed] (d1) to node[condit] {/V\_V,R}(d2l);
%     \draw[dashed] (t2r)to node[condit] {/\gls{hash}\_} node[above,pos=.4] {\lng}(t3r);
%     \draw (t2l)to node[above, pos=.4] {\lng[-]}(t3l);
%     \draw[dashed] (d2r)to node[condit] {/\gls{hash}\_} node[above,pos=.4] {\lng}(d3r);
%     \draw (d2l)to (d3l);
%     \draw[dashed] (t2r) to node[condit] {/¬\gls{hash}\_}(t3rs);
%     \draw(t3r)--(t5ri);
%     \draw (t3l) to (td5);
%     \draw(d3r)--(td5);
%     \draw (d3l)--(d5l);
%     \draw[dashed] (d2r)to node[condit] {/¬\gls{hash}\_}(t3l);
%     \draw(t3rs)--(t5rs);
%   \end{tikzpicture}
%   \caption{Summary of posited phonological developments (Agnostic)}
%   \label{fig:phondevelopconc3}
% \end{sidewaysfigure}



%%% Local Variables:
%%% mode: latex
%%% TeX-master: "../main"
%%% End:
