\chapter{Conclusion --- phonology}
\label{cha:conclusion-phonology}

The aim of Part~\ref{part:phonology-phonetics} was to establish the phonological variables that served to maintain a phonemic three-way distinction between \xT, \lT, and \xD\ in Welsh. In Chapter~\ref{cha:introduction-phonology} I argue why such a distinction must be posited, and introduce earlier scholarship on this three-way distinction. The chapter ends with the hypothesis that \xT\ was originally long and aspirated, \lT\ was short and aspirated, and \xD\ was long and unaspirated. The following chapters explore this hypothesis; their findings are summarised here.

Chapter~\ref{oldwelsh} does X

Chapter~\ref{cha:prov-mwbe-y} does Y

Does the hypothesis hold up?

%% A chart like Figure 1.1 could go about here, because it could lead on nicely to avenues for further research on the date of redundant voicing

\section{Avenues for further research}
\label{sec:aven-furth-rese}

e.g. redundant voicing and its date is unsolved, but there are pointers; the date of loss of phonemic stop length is discussed in detail in Part~\ref{part:orthography}, especially Chapter~\ref{cha:orth-concl}.

\section{Consequences for other research topics}
\label{sec:cons-other-rese-1}

Matasovic

\todo[inline]{Structure of the conclusion
\begin{itemize}
\item What was the goal of Part~\ref{part:phonology-phonetics}?
\item Does the hypothesis given in the introduction hold up?
\item Per chapter: summarise what was found
\item What are the implications of the findings, and more specifically:  
  \begin{itemize}
  \item Qualify the scope of what Chapter~\ref{oldwelsh} and Chapter~\ref{cha:prov-mwbe-y} really say about Welsh phonology
  \item Repeat and rework Figure~\ref{fig:pctolbbbstops} for Welsh including dates and the final step of loss of distinctive length. 
  \end{itemize}
\item What issues merit further research?
  \begin{itemize}
  \item Most important: dating redundant voicing (see below)
  \end{itemize}
\item What other research topics may be revisited in the wake of Part~\ref{part:phonology-phonetics}?
  \begin{itemize}
  \item Most important: Matasovic and \gls{pic}
  \end{itemize}
\item Lookahead to Part~\ref{part:orthography}
\end{itemize}
}

\begin{figure}[h]
  \centering
  \begin{tikzpicture}
    \draw[->, thick] (1,3) to node[above]{\gls{vot}} (4,3);
    \draw[->, thick] (1,3) to node[left]{length} (1,0);
    \node at (1.5,0.5) {\xD};
    \node at (3.5,0.5) {\xT};
    \node at (3.5,2.5) {\lT};
  \end{tikzpicture}
  \caption{The two dimensions distinguishing three stop series}
  \label{fig:twodimthreestop}
\end{figure}

\section{On length}
Schrijver considers aspiration as only an epiphenomenon of the other phonological properties, as shown in Table~\ref{stopsystemschrijver}. Nevertheless, Schrijver  is convinced that they were phonetically aspirated: `MoW evidence shows that these were probably strongly aspirated (except after \textit{s}), the aspiration having been lost in Cornish and Breton under Late Latin influence'~\autocite*[31]{schrijver_old_2011}. For arguments, see \textcite[§~25]{koch_*cothairche_1990}.

\begin{table}[h]
  \centering
  \begin{tabular}{{@{}lll@{}}}\toprule
    \xT & \xD & \lT \\\midrule
    {[}-voice] & [+voice] & [+voice] \\
    {[}+long] & [+long] & [-long] \\\midrule
    p{[}ʰː{]} & bː & b \\
    t{[}ʰː{]} & dː & d \\
    k{[}ʰː{]} & ɡː & ɡ\\\bottomrule
  \end{tabular}
  \caption{Common Brittonic stop system according to Schrijver \autocite*[33]{schrijver_old_2011}}
  \label{stopsystemschrijver}
\end{table}

Phonemic consonant length is not unique to stops in Middle Welsh: resonants also had an opposition between /nː/ and /n/, /rː/ and /r/, and /lː/ and /l/. This opposition disappeared by the end of the \gls{mw} period~\autocite[127]{schumacher_mittel-_2011}. These long voiced resonants were found in e.g.\ the \gls{mw} word for `heart' sometimes spelled \mw{callon}, or in \mw{penn} `head'. Both length distinctions were neutralised by the end of the \gls{mw} period. The phonemic status of these long voiced resonants did not only contrast with their short voiced counterparts /n, l, r/, but also with voiceless/aspirated counterparts /n̥, ɬ, r̥/. The three-way contrast system exactly mirrors that of stops in this sense.


\begin{table}[h]
  \centering
  \begin{tabular}{{@{}lll@{}}}\toprule
    \xT & \xD & \lT \\\midrule
    {[}+asp] & [-asp] & [+asp] \\
    {[}+long] & [+long] & [-long] \\\midrule
    pʰː & bː~[b̥ː] & pʰ \\
    tʰː & dː~[d̥ː] & tʰ \\
    kʰː & ɡː~[ɡ̥ː] & kʰ\\\bottomrule
  \end{tabular}
  \caption{Common Brittonic stop system reconstructed on the basis of the cynghanedd}
  \label{stopsystemme}
\end{table}\

% The use of length in keeping apart voiceless radicals from their lenited counterparts has consequences for how they are connected to the rise of the spirant mutation from geminates in British. If a radical voiceless stop undergoes the spirant mutation, it turns into its corresponding fricative: i.e.\ /p, t, k/ become /f, θ, x/. \textcite{schrijver_spirantization_1999} tackles the issue of spirantization, and gives some relevant points.  Schrijver follows \textcite{greene_gemination_1956} in dating spirantization after apocope. Radical \textit{D} and lenited \textit{T} were only kept separate word-initially following apocope, so only word-initial spirantization is relevant here. Spirantization occurred across word boundaries in the following contexts: following pre-apocope \textit{*-s},  \textit{*-T, *-N, *-r,} and \textit{*-l}~\autocite[3]{schrijver_spirantization_1999}. After all of these consonants, a following voiceless stop would come to stand in a phonetically longer consonant cluster. Following apocope, the exact final consonant preceding the consonant to be spirantized (or proto-spirant consonant) became obscured, leaving only length as a marker for proto-spirant consonants. This matter presents a problem: if being long was a feature of proto-spirant consonants, then how could length be used to differentiate between radical and lenited voiceless stops? A phonemic three-way length distinction in stop consonants is unlikely at best, and was therefore unstable. This fact explains why the longest stop series, the proto-spirants, developed into spirants\footnote{If gemination also arose in Goidelic at any point, the phonemic system reorganised itself in a near-identical way: one voiceless stop series became a series of spirants, only this time the shortest (lenis) voiceless stops developed into spirants.}.  


\section{Why lenited voiceless stops were aspirated}
The first reason is that we have already established on the basis of cynghanedd that it is principally the feature [long] keeping lenited and unlenited voiceless stops apart. Given how the opposition between radical and lenited consonants only phonemicised after apocope, there was only limited time for this contrast to develop much further phonetically. It is simpler and therefore preferable to propose as few features as possible distinguishing fortis and lenis up until the point where the phonemicise. 




Perhaps a third reason may be posited: the realisation of \lT\ as /f, θ, x/ in Goidelic. Postulating concomitant of aspiration in the original stop would be sensible in theorising how the fricatives emerged. Aspirated voiceless stops similarly developed into fricatives in the phonological history of Greek, for example. Also, Old Irish \lT\ > /θ/ developed into \gls{mir} /h/. This ultimate development in Irish may make more sense in light of aspiration.

\section{On the date of redundant voicing}
\label{sec:date-redund-voic}

\begin{itemize}
\item I borrow Carlyle's term for the voicing of short consonants; 
\item This voicing is initially redundant: first it is shortness which marks lenition, then it is shortness as well as voice;
\item This redundancy is observed to be productive in Léon Breton, but it may also be posited to have operated as a sound law in Welsh historically
\item This voicing chiefly constitutes loss of aspiration in Welsh, because Welsh is an aspirating language rather than a voicing language. This constitutes merely a quantitative difference as argued in Section~\ref{sec:voice-aspiration};
\item It is widely observed that word-medial \gls{D} patterns with \lT\ rather than with \xD, so it stands to reason that redundant voicing is applied to word-medial \gls{D} in the same period as initial \lT;
\item It is unclear when this occurred in Welsh, but at any rate it occurred before loss of distinctive length in the 13th century. When length was lost, voicedness became the only marker of lenition, as it is now in \gls{mow};
\item The previous chapters give some pointers to the earliest and latest possible dates for redundant voicing. Redundant voicing must have occurred after the following:
  \begin{itemize}
  \item The \oi{Coirthech}-stratum of Brittonic loanwords into Goidelic
  \item The \gls{ow} evidence for aspiration in medial \gls{D} given in Chapter~\ref{oldwelsh}.
  \end{itemize}
  
\item Redundant voicing must have occurred before the following:
  \begin{itemize}
  \item When lenition in \lT\ becomes petrified, it is reanalysed as \xD. It is difficult to conceive how \lT\ could become \xD\ before redundant voicing: in such a scenario petrification of lenition would entail both voicing and loss of distinctive length, while petrification of lenition after redundant voicing would only entail loss of distinctive length.
    \begin{itemize}
    \item Here it should be noted that petrified lenition found in
      even the earliest translations of \mw{Brut y Brenhinedd}
      (discussed in Chapter~\ref{cha:indep-comp-mwbr}) represent
      petrified lenition of \lT\ consistently with \mw{p, t, c}.
    \item The law manuscripts have some instances of petrified
      lenition not being represented, \eg \mw[with]{y kan} is found
      in \gls{sA} and \gls{sC}. This may mean that the earliest
      written versions of the Book of Iorwerth prece redundant
      voicing, but redundant voicing may precede even this version if
      accounting for orthographical conservatism. Alternatively again, 
      lenition in \mw[]{y kan} may simply not have petrified yet when
      it was first written.
    \end{itemize}

  \item ?? The evidential value of the \oi{Pádraig}-stratum of loanwords is doubtful, because after fricativisation of Goidelic \lT\, a short aspirated Brittonic \lT\ does not have an obvious Goidelic counterpart. It is thus unclear what we would expect to happen in a scenario where Brittonic \lT\ before redundant voicing was borrowed into Goidelic following Goidelic fricativisation of \lT.
  \end{itemize}
  
\item The cynghanedd offers little evidence either way: the gemination caused by doubling constitutes lengthening of \lT, and because the resulting consonant is long redundant voicing does not apply.
\item Voicing of \lT\ is shared between all Brittonic dialects. It would therefore be attractive to posit the sound law of redundant voicing before the Brittonic dialects split up. The alternative would be to posit independent innovation of redundant voicing in the Brittonic dialects.
\end{itemize}
These considerations allow for some triangulation, but the picture is not very consistent except that redundant voicing may have occurred in the \gls{ow} period.

\subsection{The presence of resonants and syllable length}
The presence of either vowels or resonants necessitated the phonetic process of lenition. Lenited stop consonants not next to resonants behaved normally, and were by definition surrounded by vowels (on both sides pre-apocope). The existence of either a vowel only or a vowel and a resonant may have influenced syllable length or weight. We know SE Welsh dialects maintain a difference between \xT\ and \xD\ through neighbouring syllable length only. It is possible that a similar phonetic concomitant existed in \gls{ow}, and that the difference in behaviour of stops in clusters with a resonant had a different influence on neighbouring vowel length than those in no such cluster with a resonant.

\section{From my seminar}
Beth bynnag, hoffwn i gyrraedd rhyw fath o derfyniad nawr. Felly, dyna'r syniadau o sut roedd y \textit{stop systems} yn gweithio yn yr Hen Amser eto \emph{The Proto-Brittonic stop system}. Dyna Koch a Schrijver, a'u systemau tan nawr. Mae system Koch yn defnyddio anadliad caled a llais i gyferbynnu'r tair cyfres. Dyw e ddim yn s\^on am hyd. Mae Schrijver yn cop\"io hyd o Lydaweg, ac mae hynny'n cydweithio gyda llais i gyferbynnu'r tair cyfres. Mae'n s\^on am anadliad caled, ond dim ond mewn dimensiwn seinegol. Dyw anadliad caled ddim yn chwarae r\^ol i gyferbynnu dim byd, felly dyw anadliad caled ddim yn ffonemaidd. Mae hynny'n rhyfedd i fi; dyw hynny ddim yn esboniad economaidd. Hefyd, mae pawb yn glynu wrth lais, er bod hi mor hawdd dadlau bod cyferbyniad llais-dilais yn Llydaweg yn arloesiad sy'n dod o Ladin. Dyw e ddim yna yn y Gymraeg...

Mae'r camsillafiadau yn yr Hen Gymraeg yn awgrymu i fi fod y gyfres ddi-lais dreigledig yn anadlog (\textit{aspirated}). Mae'r gynghanedd yn awgrymu i fi fod [hyd] yn chwarae r\^ol i gyferbynnu \xD\ a \lT, ble mae \textit{d} gysefin yn hir, a \textit{t} dreigledig yn fyr. Os felly, dyn ni'n cyrraedd y system ganlynol: \emph{Me}. Mae'r system yma yn seiliedig ar dystiolaeth seinegol Gymraeg yn unig, ac mae'n anwybyddu arloesiadau Llydaweg. Dyw llais ddim yna, mae hyd yn cyferbynnu cysefin a threigledig, ac mae anadliad caled yn cyferbynnu y ffrwydrolion di-lais a'r rhai lleisiol.

Mae'r esboniad yma yn elegant fel yna. Os mai hon oedd y system yn yr amser yna, byddwn i'n rhagweld iddi hi ddiflannu hefyd. Yr unig arloesiad erbyn hyn ydy bod cyferbyniad meintiol wedi diflannu. Trwy hanes y Gymraeg, mae sawl system feintiol wedi diflannu: llafariaid: /a:\=a/ -> /au:a/, a seiniau soniarus: e.e.\ mae \textit{l} hir yn \textit{calon} wedi diflannu, dim ond olion ar \^ol o \textit{n} hir a byr: \textit{penn:hen}, a.y.y.b.

Os oes amser, gallaf drafod:
\begin{itemize}
\item  treiglad llaes: [tːʰ : tʰ] > [θ : tːʰ], oherwydd bod cyferbyniad hyd newydd yn dod mewn gydag apocope. Mae hyn yn esbonio'r treiglad llaes fel \textit{chain shift}
\item \textit{Irish lenition} efallai fod hwn yn esbonio sut oedd \textit{t} treigledig yn troi yn /θ/ > /h/ yn Wyddeleg.
\end{itemize}


% In my view, phonetic voice did not play a phonological role in distinguishing the three series of stops. This is not a radical thought, because even nowadays what we call voiceless and voiced stops in Welsh, are actually realised as an opposition between aspirated and non-aspirated.

% Schrijver proposes that length kept lenited voiceless stops and unlenited voiced stops apart. This is what we find in MoB and it also mirrors resonants, which originally had quantitative oppositions between lenited and unlenited. Schrijver's symbols also imply that he considers lenited voiceless stops unaspirated, and their radical counterparts aspirated. However, he does not consider aspiration to be phonologically distinctive. 

% This does not make sense. For one: aspiration must have been more than a phonemic feature, because some consonant series do have it, and others do not. Moreover, Schrijver brings in voice as phonemic, but this is nowadays only found in Breton, and thought to be Latin influence. A similar case for aspiration and length cannot be made, so let's use these as building blocks.

% Now, early cynghanedd (tentatively) shows that two lenited voiceless stops may alliterate with a single unlenited voiceless stop. The same is not the case with two unlenited voiced stops. Assuming that two short consonants in succession are equivalent to a single long consonant, this indeed implies that length played a role in distinguishing lenited from unlenited voiceless stops, while it did not play a role in distinguishing unlenited voiceless and voiced stops. Moreover, aspiration must have been present in these doubled lenited voiceless stops in order for them to be equivalent to their radical counterparts.   

% The cynghanedd also shows that two lenited voiceless stops do not alliterate with a single unlenited voiced stop. If they did, that would have implied that length was the only element distinguishing lenited voiceless stops and radical voiced stops. The fact that they do not alliterate implicates an element in addition to length must have been present in disambiguating the two series. This suggests that Schrijver's model is incomplete.

% Old Welsh evidence shows that lenited voiceless stops may gave been aspirated, while unlenited voiced stops were not. If we assume that non-initial stops were written with initial stops as their analogical base, then my proposed stop system explains why word-medial /b, d, g/ (from both \lT\ and \xD) are written with \graph{b, d, g} next to resonants sometimes, but barely ever intervocalically. Presence of a resonant consonant next to an aspirated stop is known to `eat' the aspiration. Whenever \graph{b d g} are written word-initially in Old Welsh, we are sure they do not cause aspiration, and we may be sure that these word-medial consonants do not cause aspiration either given their usual proximity to resonants. It follows, then, that aspiration was principally there wherever \graph{p t c} were written, even if they were lenited. Otherwise, there would be no explanation for the distribution of \graph{b d g} to represent stops intervocalically.

% Note that the difference between lenited voiceless stops and unlenited voiced stops was only maintained word-initially. Elsewhere, intervocalic voiced geminates had merged with regular intervocalic voiceless stops by the OBr.\ period. Evidence for this is found in Old Welsh words having a historical unlenited voiced stop non-initially consistently being written with \graph{p t c} .


\begin{sidewaysfigure}[h]
  \newcommand{\asp}[1][+]{[\textsc{#1asp}]}
  \newcommand{\lng}[1][+]{[\textsc{#1long}]}
  \centering
  \begin{tikzpicture}[
    ->,
    align=left,
    condit/.style={font=\footnotesize,above,near end, sloped}]
    \node(pb){Archiphoneme};
    \node[right=of pb](phot){Phonetic lenition \&\\First spirantization};
    \node[right=of phot](apoc){Apocope};
    \node[right=of apoc](spir){Second \\spirantization};
    \node[right=of spir](merge){Merger\\\xD/\lT};
    %% Voiceless stops
    \node[below=35mm of pb](t1){/pʰ/\\/tʰ/\\/kʰ/};
    \node[above=5mm of t1](t){\gls{T}\asp};
    \node[yshift=9mm]at (phot|-t1)(t2r){[pːʰ]\\{[tːʰ]}\\{[kːʰ]}};
    \node[yshift=-9mm]at (phot|-t1)(t2l){{[pʰ]}\\{[tʰ]}\\{[kʰ]}};
    \node(t3r) at (apoc|-t2r){/pːʰ/\\{/tːʰ/}\\{/kːʰ/}};
    \node(t3l) at (apoc|-t2l){/pʰ/\\{/tʰ/}\\{/kʰ/}};
    \node(t4rs)[yshift=18mm] at (spir|-t3r){/f/\\/θ/\\/x/};
    \node(t4ri) at (spir|-t3r){};
    \node(t4l) at (spir|-t3l){};
    \node(t5ri) at (merge|-t4ri){{/pʰ/}\\{/tʰ/}\\{/kʰ/}};
    \node(t5rs) at (merge|-t4rs){};
    %% Voiced stops
    \node[below=75mm of pb](d1){/p/\\/t/\\/k/};
    \node[above =5mm of d1](d){\gls{D}\asp[-]};
    \node[yshift=9mm] at (phot|-d1)(d2r){[pː]\\{[tː]}\\{[kː]}};
    \node[yshift=-9mm]at (phot|-d1)(d2l){[β]\\{[ð]}\\{[ɣ]}};
    \node at (apoc|-d2r)(d3r){/pː/\\{/tː/}\\{/kː/}};
    \node at (apoc|-d2l) (d3l){/β/\\{/ð/}\\{/ɣ/}};
    \node at (spir|-d3r)(d4r){};
    \node at (spir|-d3l)(d4l){};
    \node at (merge|-d4l)(d5l){};
    %% Merged stops
    \node  at ($(t4l)!0.5!(d4r)$)(td4){};
    \node at (merge|-td4)(td5){/p/\\/t/\\/k/};
    %% Arrows
    \draw[dashed] (t1) to node[condit] {/¬V\_V,R}(t2r);
    \draw[dashed] (t1) to node[condit] {/V\_V,R}(t2l);
    \draw[dashed] (d1) to node[condit] {/¬V\_V,R}(d2r);
    \draw[dashed] (d1) to node[condit] {/V\_V,R}(d2l);
    \draw (t2r)to node[above] {\lng}(t3r);
    \draw (t2l)to node[above] {\lng[-]}(t3l);
    \draw (d2r)to node[above] {\lng}(d3r);
    \draw (d2l)to (d3l);
    \draw[dashed] (t3r) to node[condit] {/¬\#\_}(t4rs);
    \draw[dashed] (t3r) to node[condit] {/\#\_}(t4ri);
    \draw(t4ri)--(t5ri);
    \draw (t3l) to (td5);
    \draw(d3r)--(td5);
    \draw (d3l)--(d5l);
    \draw[dashed] (d2r)to node[condit] {/¬\#\_}(t3l);
    \draw(t4rs)--(t5rs);
  \end{tikzpicture}
  \caption{Summary of posited phonological developments (Apocope before second spirantization)}
  \label{fig:phondevelopconc}
\end{sidewaysfigure}

\begin{sidewaysfigure}[h]
  \newcommand{\asp}[1][+]{[\textsc{#1asp}]}
  \newcommand{\lng}[1][+]{[\textsc{#1long}]}
  \centering
  \begin{tikzpicture}[
    ->,
    align=left,
    condit/.style={font=\footnotesize,above,near end, sloped}]
    \node(pb){Archiphoneme};
    \node[right=of pb](phot){Phonetic lenition \&\\1\textsuperscript{st}  spirantization};
    \node[right=of phot](spir){1\textsuperscript{st} merger \lT/\xD\ \&\\2\textsuperscript{nd} spirantization};
    \node[right=of spir](apoc){Apocope};
    \node[right=of apoc](merge){2\textsuperscript{nd} merger \xD/\lT};
    %% Voiceless stops
    \node[below=35mm of pb](t1){/pʰ/\\/tʰ/\\/kʰ/};
    \node[above=5mm of t1](t){\gls{T}\asp};
    \node[yshift=9mm]at (phot|-t1)(t2r){[pːʰ]\\{[tːʰ]}\\{[kːʰ]}};
    \node[yshift=-9mm]at (phot|-t1)(t2l){{[pʰ]}\\{[tʰ]}\\{[kʰ]}};
    \node(t3r) at (apoc|-t2r){/pːʰ/\\{/tːʰ/}\\{/kːʰ/}};
    \node(t3l) at (apoc|-t2l){/pʰ/\\{/tʰ/}\\{/kʰ/}};
    \node(t4rs)[yshift=18mm] at (spir|-t2r){[f]\\{[θ]}\\{[x]}};
    \node(t3rs) at (apoc|-t4rs){/f/\\/θ/\\/x/};
    \node(t4ri) at (spir|-t3r){};
    \node(t4l) at (spir|-t3l){};
    \node(t5ri) at (merge|-t4ri){{/pʰ/}\\{/tʰ/}\\{/kʰ/}};
    \node(t5rs) at (merge|-t4rs){};
    %% Voiced stops
    \node[below=75mm of pb](d1){/p/\\/t/\\/k/};
    \node[above =5mm of d1](d){\gls{D}\asp[-]};
    \node[yshift=9mm] at (phot|-d1)(d2r){[pː]\\{[tː]}\\{[kː]}};
    \node[yshift=-9mm]at (phot|-d1)(d2l){[β]\\{[ð]}\\{[ɣ]}};
    \node at (apoc|-d2r)(d3r){/pː/\\{/tː/}\\{/kː/}};
    \node at (apoc|-d2l) (d3l){/β/\\{/ð/}\\{/ɣ/}};
    \node at (spir|-d3r)(d4r){};
    \node at (spir|-d3l)(d4l){};
    \node at (merge|-d4l)(d5l){};
    %% Merged stops
    \node  at ($(t4l)!0.5!(d4r)$)(td4){};
    \node at (merge|-td4)(td5){/p/\\/t/\\/k/};
    %% Arrows
    \draw[dashed] (t1) to node[condit] {/¬V\_V,R}(t2r);
    \draw[dashed] (t1) to node[condit] {/V\_V,R}(t2l);
    \draw[dashed] (d1) to node[condit] {/¬V\_V,R}(d2r);
    \draw[dashed] (d1) to node[condit] {/V\_V,R}(d2l);
    \draw[dashed] (t2r)to node[condit, near start] {/\#\_} node[above] {\lng}(t3r);
    \draw (t2l)to node[above] {\lng[-]}(t3l);
    \draw[dashed] (d2r)to node[condit, near start] {/\#\_} node[above] {\lng}(d3r);
    \draw (d2l)to (d3l);
    \draw[dashed] (t2r) to node[condit,midway] {/¬\#\_}(t4rs);
    \draw(t3r)--(t5ri);
    \draw (t3l) to (td5);
    \draw(d3r)--(td5);
    \draw (d3l)--(d5l);
    \draw[dashed] (d2r)to node[condit,midway] {/¬\#\_}(t4l);
    \draw(t4rs)--(t3rs);
    \draw(t3rs)--(t5rs);
  \end{tikzpicture}
  \caption{Summary of posited phonological developments (Apocope after second spirantization)}
  \label{fig:phondevelopconc2}
\end{sidewaysfigure}

\begin{sidewaysfigure}[h]
  \newcommand{\asp}[1][+]{[\textsc{#1asp}]}
  \newcommand{\lng}[1][+]{[\textsc{#1long}]}
  \centering
  \begin{tikzpicture}[
    ->,
    align=left,
    condit/.style={font=\footnotesize,above,pos=0.8, sloped}]
    \node(pb){Archiphoneme\\/kʷ/ > /p/};
    \node[right=of pb](phot){Phonetic lenition \&\\1\textsuperscript{st}  spirantization};
    \node[right=of phot](spir){1\textsuperscript{st} merger \lT/\xD\ \&\\2\textsuperscript{nd} spirantization \&\\
      Apocope};
    \node[above=of spir](apoc){Proto-Brittonnic};
    \node[below right=of apoc](merge){2\textsuperscript{nd} merger \xD/\lT};
    \node[above=of merge]{Welsh};
    \node at ($(pb)!0.5!(phot)$)(pic){};
    \node at (pic|-apoc){Proto-Insular-Celtic};
    %% Voiceless stops
    \node[below=35mm of pb](t1){/pʰ/\\/tʰ/\\/kʰ/};
    \node[above=5mm of t1](t){\gls{T}\asp};
    \node[yshift=9mm]at (phot|-t1)(t2r){[pːʰ]\\{[tːʰ]}\\{[kːʰ]}};
    \node[yshift=-9mm]at (phot|-t1)(t2l){{[pʰ]}\\{[tʰ]}\\{[kʰ]}};
    \node(t3r) at (apoc|-t2r){/pːʰ/\\{/tːʰ/}\\{/kːʰ/}};
    \node(t3l) at (apoc|-t2l){/pʰ/\\{/tʰ/}\\{/kʰ/}};
    \node(t4rs)[yshift=20mm] at (spir|-t2r){};
    \node(t3rs) at (apoc|-t4rs){/f/\\/θ/\\/x/};
    \node(t4ri) at (spir|-t3r){};
    \node(t4l) at (spir|-t3l){};
    \node(t5ri) at (merge|-t4ri){{/pʰ/}\\{/tʰ/}\\{/kʰ/}};
    \node(t5rs) at (merge|-t4rs){/f/\\/θ/\\/x/};
    %% Voiced stops
    \node[below=75mm of pb](d1){/p/\\/t/\\/k/};
    \node[above =5mm of d1](d){\gls{D}\asp[-]};
    \node[yshift=9mm] at (phot|-d1)(d2r){[pː]\\{[tː]}\\{[kː]}};
    \node[yshift=-9mm]at (phot|-d1)(d2l){[β]\\{[ð]}\\{[ɣ]}};
    \node at (apoc|-d2r)(d3r){/pː/\\{/tː/}\\{/kː/}};
    \node at (apoc|-d2l) (d3l){/β/\\{/ð/}\\{/ɣ/}};
    \node at (spir|-d3r)(d4r){};
    \node at (spir|-d3l)(d4l){};
    \node at (merge|-d4l)(d5l){/β/\\{/ð/}\\{/ɣ/}};
    %% Merged stops
    \node  at ($(t4l)!0.5!(d4r)$)(td4){};
    \node at (merge|-td4)(td5){/p/\\/t/\\/k/};
    %% Arrows
    \draw[dashed] (t1) to node[condit] {/¬V\_V,R}(t2r);
    \draw[dashed] (t1) to node[condit] {/V\_V,R}(t2l);
    \draw[dashed] (d1) to node[condit] {/¬V\_V,R}(d2r);
    \draw[dashed] (d1) to node[condit] {/V\_V,R}(d2l);
    \draw[dashed] (t2r)to node[condit] {/\#\_} node[above,pos=.4] {\lng}(t3r);
    \draw (t2l)to node[above, pos=.4] {\lng[-]}(t3l);
    \draw[dashed] (d2r)to node[condit] {/\#\_} node[above,pos=.4] {\lng}(d3r);
    \draw (d2l)to (d3l);
    \draw[dashed] (t2r) to node[condit] {/¬\#\_}(t3rs);
    \draw(t3r)--(t5ri);
    \draw (t3l) to (td5);
    \draw(d3r)--(td5);
    \draw (d3l)--(d5l);
    \draw[dashed] (d2r)to node[condit] {/¬\#\_}(t3l);
    \draw(t3rs)--(t5rs);
  \end{tikzpicture}
  \caption{Summary of posited phonological developments (Agnostic)}
  \label{fig:phondevelopconc3}
\end{sidewaysfigure}

\section{Consequences for other research}
\label{sec:cons-other-rese}
The early phonetics and phonology of lenition of voiceless stops proposed in this thesis have consequences for other arguments made. Some examples of how exactly this works out are given here.

Matasović considers the divergent ways in which lenited voiceless stops are realised in Brittonnic and Goidelic evidence that \gls{pic} must be considered  \emph{Sprachbund} rather than a node on the Celtic family tree:

\tqt{in British, the voiceless stops become voiced between vowels, while in Goidelic they become voiceless fricatives.
  What is common to IC developments is that in both cases lenition applied across
  word boundaries. It is as if both languages at the same time developed a rule prohibiting the occurrence of voiceless stops between vowels; such a rule could initially have developed in bilingual communities, and subsequently spread to monolingual speakers of both languages.  After the phonetic lenition of stops, and the subsequent apocope of final vowels, the results of word-initial lenition were
  grammaticalized, producing the system of consonant mutations.  This development had to be independent in British and Goidelic, because it presupposes
  earlier independent lenition, but there had to be some sort of causal connection.
  This conclusion cannot be avoided, because consonant mutations are typologically so rare 18 that it would be extremely improbable that they developed in two
  neighbouring languages at approximately the same time, yet completely accidentally. The most likely explanation is that consonant mutations, as a type of
  morphophonemic rule, first developed in bilingual communities speaking early
  forms of British and Goidelic. The rules turned out differently in the two languages, because their phonological systems were already significantly different
  from each other.
}{matasovic_insular_2008}{97--98}
Matasović erroneously assumes that phonetic lenition of Brittonnic  voiceless stops must immediately have constituted voicing or deaspiration --- a phonological merger with unlenited voiced stops.

The phonological evidence of this thesis invalidates this argument. Brittonnic intervocalic voiceless stops were initially lenited on a phonetic level only, with length distinction. The phonetics of such lenition can be reconciled with an early pre-phonemic Goidelic lenition.

As such, there is no need to believe  that phonetic lenition of voiceless stops cannot have been a feature of \gls{pic} on the basis of lenited stops.

%%% Local Variables:
%%% mode: latex
%%% TeX-master: "../main"
%%% End:
