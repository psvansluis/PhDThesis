\chapter{Early Middle Welsh phonology overview}


\section{On length}
Schrijver considers aspiration as only an epiphenomenon of the other phonological properties, as shown in Table~\ref{stopsystemschrijver}. Nevertheless, Schrijver  is convinced that they were phonetically aspirated: `MoW evidence shows that these were probably strongly aspirated (except after \textit{s}), the aspiration having been lost in Cornish and Breton under Late Latin influence'~\autocite*[31]{schrijver_old_2011}. For arguments, see \textcite[\S 25]{koch_*cothairche_1990}.

\begin{table}[h]
\centering
\begin{tabular}{{@{}lll@{}}}\toprule
\xT & \xD & \lT \\\midrule
{[}-voice] & [+voice] & [+voice] \\
{[}+long] & [+long] & [-long] \\\midrule
p{[}ʰː{]} & bː & b \\
t{[}ʰː{]} & dː & d \\
k{[}ʰː{]} & ɡː & ɡ\\\bottomrule
\end{tabular}
\caption{Common Brittonic stop system according to Schrijver \autocite*[33]{schrijver_old_2011}}
\label{stopsystemschrijver}
\end{table}

Phonemic consonant length is not unique to stops in Middle Welsh: resonants also had an opposition between /nː/ and /n/, /rː/ and /r/, and /lː/ and /l/. This opposition disappeared by the end of the \gls{mw} period~\parencite[127]{schumacher_mittel-_2011}. These long voiced resonants were found in e.g.\ the \gls{mw} word for `heart' sometimes spelled \mw{callon}, or in \mw{penn} `head'. Both length distinctions were neutralised by the end of the \gls{mw} period. The phonemic status of these long voiced resonants did not only contrast with their short voiced counterparts /n, l, r/, but also with voiceless/aspirated counterparts /n̥, ɬ, r̥/. The three-way contrast system exactly mirrors that of stops in this sense.


\begin{table}[h]
\centering
\begin{tabular}{{@{}lll@{}}}\toprule
\xT & \xD & \lT \\\midrule
{[}+asp] & [-asp] & [+asp] \\
{[}+long] & [+long] & [-long] \\\midrule
pʰː & bː~[b̥ː] & pʰ \\
tʰː & dː~[d̥ː] & tʰ \\
kʰː & ɡː~[ɡ̥ː] & kʰ\\\bottomrule
\end{tabular}
\caption{Common Brittonic stop system reconstructed on the basis of the cynghanedd}
\label{stopsystemme}
\end{table}\

% The use of length in keeping apart voiceless radicals from their lenited counterparts has consequences for how they are connected to the rise of the spirant mutation from geminates in British. If a radical voiceless stop undergoes the spirant mutation, it turns into its corresponding fricative: i.e.\ /p, t, k/ become /f, θ, x/. \textcite{schrijver_spirantization_1999} tackles the issue of spirantization, and gives some relevant points.  Schrijver follows \textcite{greene_gemination_1956} in dating spirantization after apocope. Radical \textit{D} and lenited \textit{T} were only kept separate word-initially following apocope, so only word-initial spirantization is relevant here. Spirantization occurred across word boundaries in the following contexts: following pre-apocope \textit{*-s},  \textit{*-T, *-N, *-r,} and \textit{*-l}~\autocite[3]{schrijver_spirantization_1999}. After all of these consonants, a following voiceless stop would come to stand in a phonetically longer consonant cluster. Following apocope, the exact final consonant preceding the consonant to be spirantized (or proto-spirant consonant) became obscured, leaving only length as a marker for proto-spirant consonants. This matter presents a problem: if being long was a feature of proto-spirant consonants, then how could length be used to differentiate between radical and lenited voiceless stops? A phonemic three-way length distinction in stop consonants is unlikely at best, and was therefore unstable. This fact explains why the longest stop series, the proto-spirants, developed into spirants\footnote{If gemination also arose in Goidelic at any point, the phonemic system reorganised itself in a near-identical way: one voiceless stop series became a series of spirants, only this time the shortest (lenis) voiceless stops developed into spirants.}.  


\section{Why lenited voiceless stops were aspirated}
The first reason is that we have already established on the basis of cynghanedd that it is principally the feature [long] keeping lenited and unlenited voiceless stops apart. Given how the opposition between radical and lenited consonants only phonemicised after apocope, there was only limited time for this contrast to develop much further phonetically. It is simpler and therefore preferable to propose as few features as possible distinguishing fortis and lenis up until the point where the phonemicise. 




Perhaps a third reason may be posited: the realisation of \lT\ as /f, θ, x/ in Goidelic. Postulating concomitant of aspiration in the original stop would be sensible in theorising how the fricatives emerged. Aspirated voiceless stops similarly developed into fricatives in the phonological history of Greek, for example. Also, Old Irish \lT\ > /θ/ developed into \gls{mi} /h/. This ultimate development in Irish may make more sense in light of aspiration.


\section{From my seminar}
Beth bynnag, hoffwn i gyrraedd rhyw fath o derfyniad nawr. Felly, dyna'r syniadau o sut roedd y \textit{stop systems} yn gweithio yn yr Hen Amser eto \mn{The Proto-Brittonic stop system}. Dyna Koch a Schrijver, a'u systemau tan nawr. Mae system Koch yn defnyddio anadliad caled a llais i gyferbynnu'r tair cyfres. Dyw e ddim yn s\^on am hyd. Mae Schrijver yn cop\"io hyd o Lydaweg, ac mae hynny'n cydweithio gyda llais i gyferbynnu'r tair cyfres. Mae'n s\^on am anadliad caled, ond dim ond mewn dimensiwn seinegol. Dyw anadliad caled ddim yn chwarae r\^ol i gyferbynnu dim byd, felly dyw anadliad caled ddim yn ffonemaidd. Mae hynny'n rhyfedd i fi; dyw hynny ddim yn esboniad economaidd. Hefyd, mae pawb yn glynu wrth lais, er bod hi mor hawdd dadlau bod cyferbyniad llais-dilais yn Llydaweg yn arloesiad sy'n dod o Ladin. Dyw e ddim yna yn y Gymraeg...

Mae'r camsillafiadau yn yr Hen Gymraeg yn awgrymu i fi fod y gyfres ddi-lais dreigledig yn anadlog (\textit{aspirated}). Mae'r gynghanedd yn awgrymu i fi fod [hyd] yn chwarae r\^ol i gyferbynnu \xD\ a \lT, ble mae \textit{d} gysefin yn hir, a \textit{t} dreigledig yn fyr. Os felly, dyn ni'n cyrraedd y system ganlynol: \mn{Me}. Mae'r system yma yn seiliedig ar dystiolaeth seinegol Gymraeg yn unig, ac mae'n anwybyddu arloesiadau Llydaweg. Dyw llais ddim yna, mae hyd yn cyferbynnu cysefin a threigledig, ac mae anadliad caled yn cyferbynnu y ffrwydrolion di-lais a'r rhai lleisiol.

Mae'r esboniad yma yn elegant fel yna. Os mai hon oedd y system yn yr amser yna, byddwn i'n rhagweld iddi hi ddiflannu hefyd. Yr unig arloesiad erbyn hyn ydy bod cyferbyniad meintiol wedi diflannu. Trwy hanes y Gymraeg, mae sawl system feintiol wedi diflannu: llafariaid: /a:\=a/ -> /au:a/, a seiniau soniarus: e.e.\ mae \textit{l} hir yn \textit{calon} wedi diflannu, dim ond olion ar \^ol o \textit{n} hir a byr: \textit{penn:hen}, a.y.y.b.

Os oes amser, gallaf drafod:
\begin{itemize}
    \item  treiglad llaes: [tːʰ : tʰ] > [θ : tːʰ], oherwydd bod cyferbyniad hyd newydd yn dod mewn gydag apocope. Mae hyn yn esbonio'r treiglad llaes fel \textit{chain shift}
    \item \textit{Irish lenition} efallai fod hwn yn esbonio sut oedd \textit{t} treigledig yn troi yn /θ/ > /h/ yn Wyddeleg.
\end{itemize}


% In my view, phonetic voice did not play a phonological role in distinguishing the three series of stops. This is not a radical thought, because even nowadays what we call voiceless and voiced stops in Welsh, are actually realised as an opposition between aspirated and non-aspirated.

% Schrijver proposes that length kept lenited voiceless stops and unlenited voiced stops apart. This is what we find in MoB and it also mirrors resonants, which originally had quantitative oppositions between lenited and unlenited. Schrijver's symbols also imply that he considers lenited voiceless stops unaspirated, and their radical counterparts aspirated. However, he does not consider aspiration to be phonologically distinctive. 

% This does not make sense. For one: aspiration must have been more than a phonemic feature, because some consonant series do have it, and others do not. Moreover, Schrijver brings in voice as phonemic, but this is nowadays only found in Breton, and thought to be Latin influence. A similar case for aspiration and length cannot be made, so let's use these as building blocks.

% Now, early cynghanedd (tentatively) shows that two lenited voiceless stops may alliterate with a single unlenited voiceless stop. The same is not the case with two unlenited voiced stops. Assuming that two short consonants in succession are equivalent to a single long consonant, this indeed implies that length played a role in distinguishing lenited from unlenited voiceless stops, while it did not play a role in distinguishing unlenited voiceless and voiced stops. Moreover, aspiration must have been present in these doubled lenited voiceless stops in order for them to be equivalent to their radical counterparts.   

% The cynghanedd also shows that two lenited voiceless stops do not alliterate with a single unlenited voiced stop. If they did, that would have implied that length was the only element distinguishing lenited voiceless stops and radical voiced stops. The fact that they do not alliterate implicates an element in addition to length must have been present in disambiguating the two series. This suggests that Schrijver's model is incomplete.

% Old Welsh evidence shows that lenited voiceless stops may gave been aspirated, while unlenited voiced stops were not. If we assume that non-initial stops were written with initial stops as their analogical base, then my proposed stop system explains why word-medial /b, d, g/ (from both \lT\ and \xD) are written with \graph{b, d, g} next to resonants sometimes, but barely ever intervocalically. Presence of a resonant consonant next to an aspirated stop is known to `eat' the aspiration. Whenever \graph{b d g} are written word-initially in Old Welsh, we are sure they do not cause aspiration, and we may be sure that these word-medial consonants do not cause aspiration either given their usual proximity to resonants. It follows, then, that aspiration was principally there wherever \graph{p t c} were written, even if they were lenited. Otherwise, there would be no explanation for the distribution of \graph{b d g} to represent stops intervocalically.

% Note that the difference between lenited voiceless stops and unlenited voiced stops was only maintained word-initially. Elsewhere, intervocalic voiced geminates had merged with regular intervocalic voiceless stops by the OBr.\ period. Evidence for this is found in Old Welsh words having a historical unlenited voiced stop non-initially consistently being written with \graph{p t c} .