\chapter{Comparative evidence from Breton}
\label{cha:comp-evid-from}

\section{Falc'hun}
\textcite[63]{falchun_systeme_1951} noted that in Le Bourg Blanc Breton, word-initial lenited voiceless stops and unlenited voiced stops are kept separate by length.
% \tqt{\begin{french}
% Elles [les occlusives sonores initiales] sont toujours fortes, \`{a} moins qu'elles no proviennent de la mutation de \textit{p, t, k}. Pour plus de clart\'{e}, nous les transcrirons par \textit{bb, dd, gg}, et r\'{e}serverons les signes \textit{b, d, g} pour les occlusives intervocaliques, ou initiales r\'{e}sultant de la mutation \textit{p, t, k}.
% \end{french}}{falchun_systeme_1951}{63}
In Falc'hun's notation, Breton therefore has \mob{bb, dd, gg} for word-initial non-lenited voiced stops, while word-medial voiced stops and lenited word-initial voiceless stops are represented by \mob{b, d, g}. In other words, word-medial lenited voiceless stops are not kept separate from word-medial voiced radicals, signalling that they indeed have merged. The consequence of non-merger in initial positions is that some minimal pairs may be discerned:
\tqt{\begin{french}
Prenons les deux suivantes, qui ont \'{e}t\'{e} exp\'{e}riment\'{e}es, avec bien d'autres du m\^{e}me genre, sur des auditoires bretonnants du L\'{e}on, du Tr\'{e}guier ou de la Cornouaille:

1. \textit{(t\^{o}rr\`{e}d \'{e}\'{o} \'{e} g\={\c{a}}r)}; 2. \textit{(t\^{o}rr\`{e}d \'{e}\'{o} \'{e} gg\={\c{a}}r)}. Les bretonnants traduisent sans h\'{e}sitation: %
\begin{enumerate}
    \item << Sa charette \`{a} lui est cass\'{e}e >>, \textit{torret eo e garr.}
    \item << Sa jambe \`{a} elle est cass\'{e}e >>, \textit{torret eo he gar.}
\end{enumerate}
\end{french}}{falchun_systeme_1951}{64}

The distinct pronunciation of these consonants is restricted to postvocalic contexts
\todo{This restriction is similar to how lenition of voiceless stops is sometimes written in consonant clusters in \gls{ow} orthography, but hardly ever intervocalically, see Chapter~\ref{oldwelsh}.}:
\tqt{
  \begin{french}
    Toutefois, la distinction n'est possible qu'apr\`{e}s voyelle. Le \textit{d} initial de \textit{an dud}, << les gens >>, de \textit{tud} ne diff\`{e}re rien de celui de \textit{an dour}, << l'eau >>.  
    Et si une occlusive sonore provenant de la mutation de \textit{p, t, k}, se trouve \`{a} l'initiale absolue, elle se prononce forte, ainsi le \textit{b} de \textit{breman}, << \`{a} pr\'{e}sent >>, qui provient du \textit{p} de \textit{pred}, << moment >>.
   Le \textit{b} sera doux dans \textit{abred, (abr\c{ē}d}, << de bonne heure >>, ais fort dans \textit{ha breman (a bbr\`{\c{ē}}m\~{a})}, << et \`{a} pr\'{e}sent >>, parce que \textit{ha}, r\'{e}duction de \textit{hag}, n'adoucit pas la consonne suivante [\dots].
 \end{french}}{falchun_systeme_1951}{64} 


% In Breton, it seems, the distinct pronunciation of lenited voiceless stops and unlenited voiced stops word-initially is based on the phonotactic prerequisite that the relevant consonant should follow a vowel. Notably, this excludes the Breton definite article \textit{an}.

% This phonotactic constraint may give a way to test Harvey's position that length distinction in word-initial stops may disambiguate between lenited voiceless and unlenited voiced: if Early Middle Welsh similarly shows that these stop series are disambiguated after vowels, but not after consonants, then this might point to common ancestry, thereby ruling out analogical formation.
Falc'hun gives the following measurements on word-initial stop length:

\tqt{\begin{french}
Des enregistrements ont permis de v\'{e}rifier les dur\'ees des occlusives sonores \`a lint\'erieur de la phrase, mais au d\'ebut du mot: 

\begin{tabular}{rr}
    \textit{bb} 8,6 centisecondes (13 ex.) & \textit{b} 6,8 centisecondes (10 ex.) \\
    \textit{dd} 9,5 centisecondes (12 ex.) & \textit{b[d]} 5,6 centisecondes (11 ex.) \\
    \textit{gg} 8,5 centisecondes (10 ex.) & \textit{g} 5,2 centisecondes (14 ex.) \\
\end{tabular}%

A l'intervocalique dans le mot, apr\`es voyelle accentu\'ee, la dur\'ee moyenne des occlusives sourdes \textit{p, t, k} a \'et\'e de 10,80; celle des sonores \textit{b, d, g}, de 5,64.
\end{french}}{falchun_systeme_1951}{65}

% Question: duration of what exactly, in terms of phonetics? Answer: most likely, the duration the airway remains closed before the release of the air signifying the stop.

% Question: does this phenomenon of lengthening voiced stops occur purely as a measure to disambiguate length only where ambiguity might otherwise arise (most notably after \textit{e} `his' or `her', depending on the presence or absence of lenition), or is there a quantitative opposition between long and short voiced stops word-initially throughout Le Bourg Blanc Breton? The former situation would argue for Harvey's position that the opposition was analogically reintroduced, since using lenition exclusively to disambiguate echoes later grammatical innovations such as syntactic lenition more than it does inherited patterns. Falc'hun implies this position in his quote two paragraphs below.

% \tqt{\begin{french}Les occlusives sonores fortes ont tendance \`a s'assourdir: on entend fr\'equemment\textit{ (va t\={\c{u}}\'e) va Doue} << mon Dieu! >> pour\textit{ (va dd\={\c{u}}\'e)}. M\^eme quand elles sont enti\`erement sonores, leur explosion peut \^etre suivie d'un souffle sourd, qui dure jusqu'\`a 5 centisecondes (cf. infra p. 159). Leur hauteur explosive est toujours plus grande que celle de \textit{b, d, g}. \end{french}}{falchun_systeme_1951}{65}

% The above quote seems to imply a relationship between fortition and distinguishing unlenited voiced stops and lenited voiceless stops.
According to Falc'hun, the ability to distinguish lenited voiceless stops from unlenited voiced stops is dependent on understanding lenition as a system, which explains why it only occurs word-initially: \tqt{\begin{french}
Cette opposition entre deux séries d'occlusives sonores, l'une forte et l'autre douce, ne joue dans la langue qu'un rôle négligeable. Son existence fait cependant mieux comprendre la logique du mécanisme des mutations tel qu'il sera écrit plus loin. 
% Sans elle, dans un syst\`eme consonantique o\`u toute consonne est forte ou douce, on ne saurait o\`u classer \textit{b, d, g,} qui sont des douces, puisque provenant de l'adoucissement de \textit{p, t, k,} et qui seraient en m\^eme temps des fortes, puisque s'adoucissant elles-m\^emes en \textit{v, z, h}.
\end{french}}{falchun_systeme_1951}{65}

\subsection{Jackson on the Plougrescant dialect}
\label{sec:jacks-plougr-dial}
Falc'hun's observations on the dialect of Le Bourg Blanc are far from universal, compare the following quote on the Plougrescant dialect:
\tqt{Since initial consonants are always short (lenis) there is no initial distinction of fortis-lenis, and therefore no system such as that described by Falc'hu for Le Bourg Blanc. He gives \textit{he bbaz} <<her stick>> and \textit{he ggar} <<her leg>>, from \textit{baz} and \textit{gar}, versus \textit{e baz} <<his cough>> and \textit{e garr} <<his cart>>, from \textit{paz} and \textit{karr} [\dots]. Similarly, \textit{an hini nneta, an hini llousa}, and \textit{an hini rruz} when masculine, but \textit{an hini neta, an hini lousa,} and \textit{an hini ruz} when feminine, and compare \textit{he lleur} <<her threshing-floor>> versus \textit{e leur} <<his threshing-floor>>, etc. [\dots] At Plougrescant, \textit{he baz} <<her stick>> and \textit{e baz} <<his cough>> are both [\textit{i \textsuperscript{l}ba̤·s}]; and there is no difference between \textit{an hini nevez, an hini lousañ,} and \textit{an hini ruz} whether masculine or feminine. [̣\dots] In Falc'hun's system the distinction must be phonemic, sonce it carries with it a distinction in meaning which the speakers recognise, but at Plougrescant there is certainly no difference and therefore no phonemic difference.}{Jac_Phonology61}{332}


\subsection{Bothorel}
\label{sec:bothorel}

\todo[inline]{discuss \textcite{Bot_Etude82}}

\subsection{Carlyle}
\label{sec:carlyle}

More evidence on Breton phonology is given by \textcite{carlyle_syllabic_1988}, \todo[inline]{summarize her position, or at least note that she confirms Falc'hun}

\subsection{Kennard}
\label{sec:kennard}

\todo[inline]{discuss \textcite{KL_Mutation17}}

\subsection{Jackson}
According to \textcite[§132]{jackson_language_1953}, evidence from Breton phonology suggests that the opposition between lenited and unlenited may originally have been one of quantity (\ie duration) rather than quality (\ie voiced vs voiceless, stop vs fricative): 
\tqt{Thus, for example, the lenited \textit{b} in \textit{e baz} ``his cough'', from \textit{paz} (W. \textit{ei bas}, from \textit{pas}), and the lenited \textit{l} in \textit{e leur,} ``his floor'', from \textit{leur} (W. \textit{ei lawr}, from \textit{llawr}), have approximately only half the articulatory duration of the non-lenited \textit{p} in \textit{paz} or the non-lenited \textit{b} in \textit{bac'h} ``hook'' (W. \textit{bach}), and the non-lenited \textit{l} in \textit{leur}.}{jackson_language_1953}{§132} On the basis of these findings in Breton, Jackson proposes that Common Celtic consonants may have been comparatively long in absolute initial position, while short when placed between vowels.

Schrijver repeats this view: 
\tqt{As a result of phonemic lenition, \textit{*p, *t, *k} became short \textit{*b, *d, *g}. These were contrasted with the reflexes of the unlenited voiced stops, which were now phonemically long (and, presumably, tense), merging with the rare old geminate voiced stops: \textit{*bː, *dː, *gː}. In word-initial position, the contrast between short and long voiced stops is maintained (but according to Harvey (1984) analogically reintroduced) in MoB dialects. The unlenited voiceless stops were undoubtedly long too, \textit{*pː, *tː, *kː}, but they did not contrast with short voiceless stops. MoW evidence shows that these were probably strongly aspirated (except after \textit{s}), the aspiration having been lost in Cornish and Breton under Late Latin influence. The lenited voiced stops became the voiced fricatives \textit{*β, *ð, *γ}.}{schrijver_old_2011}{31}
Schrijver dates this development to the Proto-British period. Schrijver assumes that aspiration was the original distinguishing feature between unlenited and lenited voiceless stops. He also states that phonemic lenition caused voiceless stops to have a lenited (short) counterpart which does not merge with unlenited voiced stops word-initially. 


%%% Local Variables:
%%% mode: latex
%%% TeX-master: "../main"
%%% End:
