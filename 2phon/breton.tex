\chapter{Comparative evidence from Breton}
\label{cha:comp-evid-from}


\subsection{Jackson on the Plougrescant dialect}
\label{sec:jacks-plougr-dial}

\subsection{Bothorel}
\label{sec:bothorel}

\todo[inline]{discuss \textcite{Bot_Etude82}}

\subsection{Carlyle}
\label{sec:carlyle}

More evidence on Breton phonology is given by \textcite{carlyle_syllabic_1988}, \todo[inline]{summarize her position, or at least note that she confirms Falc'hun}

\subsection{Kennard}
\label{sec:kennard}

\todo[inline]{discuss \textcite{KL_Mutation17}}

\subsection{Jackson}
According to \textcite[§132]{jackson_language_1953}, evidence from Breton phonology suggests that the opposition between lenited and unlenited may originally have been one of quantity (\ie duration) rather than quality (\ie voiced vs voiceless, stop vs fricative): 
\tqt{Thus, for example, the lenited \textit{b} in \textit{e baz} ``his cough'', from \textit{paz} (W. \textit{ei bas}, from \textit{pas}), and the lenited \textit{l} in \textit{e leur,} ``his floor'', from \textit{leur} (W. \textit{ei lawr}, from \textit{llawr}), have approximately only half the articulatory duration of the non-lenited \textit{p} in \textit{paz} or the non-lenited \textit{b} in \textit{bac'h} ``hook'' (W. \textit{bach}), and the non-lenited \textit{l} in \textit{leur}.}{jackson_language_1953}{§132} On the basis of these findings in Breton, Jackson proposes that Common Celtic consonants may have been comparatively long in absolute initial position, while short when placed between vowels.

Schrijver repeats this view: 
\tqt{As a result of phonemic lenition, \textit{*p, *t, *k} became short \textit{*b, *d, *g}. These were contrasted with the reflexes of the unlenited voiced stops, which were now phonemically long (and, presumably, tense), merging with the rare old geminate voiced stops: \textit{*bː, *dː, *gː}. In word-initial position, the contrast between short and long voiced stops is maintained (but according to Harvey (1984) analogically reintroduced) in MoB dialects. The unlenited voiceless stops were undoubtedly long too, \textit{*pː, *tː, *kː}, but they did not contrast with short voiceless stops. MoW evidence shows that these were probably strongly aspirated (except after \textit{s}), the aspiration having been lost in Cornish and Breton under Late Latin influence. The lenited voiced stops became the voiced fricatives \textit{*β, *ð, *γ}.}{schrijver_old_2011}{31}
Schrijver dates this development to the Proto-British period. Schrijver assumes that aspiration was the original distinguishing feature between unlenited and lenited voiceless stops. He also states that phonemic lenition caused voiceless stops to have a lenited (short) counterpart which does not merge with unlenited voiced stops word-initially. 


%%% Local Variables:
%%% mode: latex
%%% TeX-master: "../main"
%%% End:
