% \maketitle
% \tableofcontents
% {\tiny\listoftodos}




\chapter{Provection in \mw{Beirdd y Tywysogion}}





 \section{Introduction}
% \todo[inline]{Perhaps mention examples like \mw{atgas, atborth, datganiad}, whose first consonant is provected, but whose second consonant is not (this might be in written/formal Welsh only, with spoken \mw{DD}). This is the cynghanedd in reverse, where in such a cluster the first would rhyme with a voiced stop while the second would rhyme with a voiceless stop. Other points to mention: Iolo Goch: \mw{heb bwyd, heb bai etc.}, also final devoicing creeping in (e.g. Engl. lw., and \mw{wyt (ti) <- wyd (ti)}).}
Provection is a sandhi phenomenon found in \gls{mw} whereby \mw{b, d, g} become \mw{p, t, c}. It is primarily found word-internally, and may occur when two consonants combine, \eg \mw{d + d} becomes \mw{tt} in \mw{letty} `lodging' < \mw{lled + dy} < \mw{ty} `house' and \mw{d + h} becomes \mw{tt} in \mw{atteb} `reply' < \mw{ad + heb}. Provection is also found before equative and superlative endings of adjectives, and in some subjunctive forms of verbs~\parencite[\S 17]{evans_grammar_1964}\footnote{In the case of \mw{lletty} and similar cases where two stop consonants combine, it must be noted that the medial \mw{t} may be simply a result of non-lenition, which could precede any sandhi process. Cases where the second element starts with \xD\ are rare, but exist in fixed expressions that are not close compounds. They provect inconsistently, \eg \gmow[good health, cheers]{iechyd da} with medial /d/, but \gmow[God's blessing]{Rhad Duw} with medial /t/, as shown by \mw{RhâTv̂w }~\autocite[s.v.\ \emph{rhad}]{bevan_geiriadur_2014}.}. These \gls{mw} spellings show that voiced stops could provect word-internally, but they do not tell us about word-external provection. I shall argue that \gls{mw} alliterative patterns may give insight into word-external sandhi.

Alliteration patterns may give insight into how voiceless stops contrasted with voiced stops, and how lenited stops contrasted with radical stops. In the \gls{mow} cynghanedd, voiceless stops may  alliterate with voiced stops in three circumstances. A voiceless stop --- \gls{T} --- may alliterate with its voiced counterpart if the preceding word also ends with a voiced stop (\gls{D}+\gls{D}). A voiceless stop may also alliterate with an \mw{h}, if it is preceded by its voiced counterpart (\gls{D}+\mw{h}). Finally, a voiceless stop followed by \mw{r} may alliterate with a voiceless stop followed by \mw{rh} (\gls{T}\mw{r} with \gls{D}+\mw{rh}). For a full account, see section \ref{rules}.

The existence of multiple phonotactic environments within which voiced stops may alliterate with voiceless stops is problematic, because these rules imply different variables by which voiceless stops contrast with voiced stops. The difference between a single voiced stop and two voiced stops in succession is in length, so if a (single) short voiced stop does not alliterate with its voiceless counterpart, but a double (lengthened) voiceless stop does, then  length is implicated as the distinguishing variable between voiced and voiceless stops. Similarly, if a voiced stop followed by \mw{h} corresponds with its voiceless counterpart, some feature of \mw{h} is implicated as distinguishing voiced from voiceless stops, \ie aspiration would be the variable distinguishing them. The fact that both rules exist does not make sense, so it stands to reason that at least one rule would not exist, and any rule remaining would give insight in what phonetic variables contrasted \mw{p, t, k}, and \mw{b, d, g}. At the same time, however, the same existence of multiple types of provection is to be hypothesized. Both provection by doubling and provection by aspiration are attested word-internally in \gls{mw}~\parencite[\S 17]{evans_grammar_1964}.


\subsection{From alliteration by \textit{cyflythyriaeth} to phonetic alliteration}
The \gls{mw} period saw a change in how alliteration operated: in the earliest poems, radical and lenited consonants could alliterate, but they could no longer alliterate by the end of the \gls{mw} period~\parencite[339]{rowland_early_1990}. However, there are instances of a mutated consonant alliterating with a radical consonant, but where the radical of the former does not equal the latter, \eg \mw{m < b} alliterating with radical \mw{m}, or \mw{g < c} alliterating with radical \mw{g}. This newer phonetic alliteration is already as old as the \mw{Hengerdd}, about which Rowland writes: 
\tqt{there may be an overlapping pattern g--g < c--c, cf.\ \textit{ar cloyn a gerynt gwraged}, or simply alliteration of the two words with c as their radical. Some examples seem to have a mixture in themselves of alliterating radical, mutated and actual sounds, occasionally with the radical omitted.}{rowland_early_1990}{341} 
This pattern demonstrates that even where mutated and radical consonants are held to alliterate, alliteration based on synchronic phonological grounds may also be found. Both systems thus exist side by side in some of the oldest Welsh poetry. 

A single-word term for the type of alliteration in which radical and mutated consonants may alliterate is `\gls{cyfl}'. \Textcite{koch_gododdin_1997} explains this term, and makes the following comments on its prehistory in the context of the \mw{Gododdin}:
    \tqt{\textit{Cyflythyraeth} is the type of consonance in which radical and mutated forms of consonants alliterate, as is highly common throughout the elegies [\dots] As Jackson first recognised and I later argued, \textit{cyflythyraeth} is most naturally explained as a reanalysis of the Old Celtic, pre-apocope system. That original simpler system was based on the harmony of identical phonemes. After the Neo-Brittonic syllable losses, what had previously been purely allophonic variants became distinct phonemes. Old verse came to imply a new and more complex rule (likewise in both Irish and Brittonic) according to which the newly contrasting radical and mutated consonants harmonised. }{koch_gododdin_1997}{cxlii}

Alliteration of lenited and unlenited consonants  invalidates the idea that alliteration schemes provide clues into Early Welsh phonetics, because accepting alliteration as providing insight into phonetics obviously presupposes that the alliterating sounds sounded similar, rather than being similar on a lexico-semantic level. It is therefore important to know the extent to which \gls{cyfl} still operated in the time of the Poets of the Princes. Koch acknowledges that some traces of \gls{cyfl} existed as late as the corpus under consideration, but also notes that it was very infrequent:
    \tqt{\textit{Cyflythyraeth} is found in poetry of the Old Welsh period, such as the saga \textit{englynion, Armes Prydein}, and even (though far more infrequently than in \textit{\"E Godoδin}) in the \textit{awdlau} of the \textit{Gogynfardd} Meilyr Brydydd (\textit{fl.\ c.} 1081--1137). Therefore, the mere presence or absence of the phenomenon does not imply a narrower date range than Archaic to Late Old Welsh (\textit{saec.} VI--XII\textsuperscript{1}). However, frequency of \textit{cyflythyraeth}, specific subtypes of it, and the relative infrequency of the newer type of alliteration (\ie radical \textit{b-, d-, g-} answering the lenitions of \textit{p-, t-, c-}) probably varied over time.
    }{koch_gododdin_1997}{cxlii}

\Textcite{jones_meddwl_2005} gives a more methodical treatment of the question to what extent \gls{cyfl} was still employed. In the poem \mw{Marwnad Gruffudd ap Cynan} (\acrshort{CBTI} 3), Jones notes that phonemic initial alliteration with the first word of the line is found 58 times, while alliteration between mutations and the radical is found 26 times. 
However, even some of this minority of lines showing  \gls{cyfl} raise doubt. Jones, for instance, counts the following line as an example of \gls{cyfl}~\parencite[162]{jones_meddwl_2005}:
\mwcc[gwaneikynhor]{\acrshort{CBTI}~3,~l.~25}{\al{G}\w anei yg \al{k}ynhor eissor Medra\w d}{He struck a leader like Medrawd}
Here, he possibly assumes that the initial consonants /ɡ/ of \mw{g\w anei} and /ŋ̊/ of \mw{kynnor} alliterate. However, although both sounds nowadays exist as mutated forms of \gls{archphon} /k/, the \gls{archphon}s in Example~\ref{gwaneikynhor}  are /ɡ/ and /k/, respectively. There can be no question of \gls{cyfl} when the alliterating sounds have different radical consonants, because the whole principle behind \gls{cyfl} is that it is based on correspondence of radical consonants. This therefore looks like a hypercorrect usage of \gls{cyfl}. Alternatively, the only line-internal ornamentation used here is probably end-rhyme between \mw{kynhor} and \mw{eissor}, cf.\ \textcite[xlvii]{andrews_welsh_2007}, or we are dealing with a \mw{cynghanedd lusg bengoll}.

% Within the middle of a line, alliteration is found 54 times between the same phonemes, and 20 times with mutated versions of the same consonant. With the word of the line, the ratio is 63 to 22. Between the first and the last word of a line, the ratio is 37 to 14. Phonemic alliteration around a syllable is found 33 times, as opposed to twice for alliteration of mutated consonants

More significantly, this hypercorrectly archaic example imitating \gls{cyfl} shows that this system as found in the \mw{Hengerdd} was no longer understood and no longer productive. Instead, the rule was apparently reanalysed more broadly: any \gls{archphon} may alliterate with any other \gls{archphon} as long as they share a phoneme between them. This differs from old-style \gls{cyfl} in that formerly only alliteration of phonemes sharing an \gls{archphon} was allowed, or alliteration phonologically similar phonemes, e.g. alliteration of \mw{d} and \mw{t}. What was not allowed, however, is alliteration of all phonemes falling under two different \gls{archphon}s.

The literature clearly suggests \gls{cyfl} was on its way out. Still, there is no reliable way to know whether an individual instance of a \gls{doubconclus} ending with \lT\ alliterates with \xT\ because of provection as a result of sandhi or as a result of \gls{cyfl}. This is to be kept in mind when reviewing these instances.

A comparable reinterpretation of rules of poetry is found in Old Irish metrics, and may serve as a useful comparandum in understanding the development of \gls{cyfl}. \textcite[173]{sims-williams_dating_2016} notes the following: `what happened was that when older, now lost, pre-syncope poems with trisyllabic cadences were recited after syncope, they gave rise to new metres [\dots] which rhymed disyllabic cadences containing medial consonant clusters, irrespective now of whether or not all such clusters had arisen by syncope.' Here, too, linguistic change (syncope instead of phonemicisation of lenition) causes a formerly straightforward rule of poetry to become opaque, and the rule now applies in a wider contexts.




\subsection{The later rules}
\label{rules}
This section discusses the rules of the cynghanedd that are relevant for this research. The rules as discussed here apply chiefly to the system as it was agreed upon in subsequent centuries. The goal at present is to see how they hold up when backtracking to the Poetry of the Princes.
They are summarily laid out by \textcite{roberts_anghenion_1973}:

% \tqt{\begin{welsh}\label{roberts}
% Ond pan fo dwy \textbf{b,} dwy \textbf{d} a dwy \textbf{g} yn digwydd gyda'i gilydd heb lafariad neu lafariaid rhyngddynt, maent yn caledu, hynny yw, mae dwy \textbf{b} yn ffurfio \textbf{p,} dwy \textbf{d} yn ffurfio \textbf{t,} a dwy \textbf{g} yn ffurfio \textbf{c,} felly, ni chaiff dwy \textbf{b} ateb un \textbf{b}, ac ymlaen. \dots

% Yn yr un modd, mae \textbf{h} pan fo'n dilyn \textbf{b, d,} ac \textbf{g} yn syth eto'n eu caledu'n \textbf{p, t, }ac \textbf{c,} a chaiff \textbf{b/h} ateb \textbf{p}, ac ymlaen. \dots

% Mae'r gytsain \textbf{rh} yn caledu'r tair cytsain \textbf{b, d, g} hefyd, pan ddilyn y cytseiniaid hyn yn syth. \dots

% Fe all unrhyw un 'or cyfuniadau hyn ateb ei gilydd, hynny yw, gall \textbf{b/h} ateb \textbf{b/b, d/h} ateb \textbf{d/d, d/rh} ateb \textbf{d/dr,} ac ymlaen, gan mai'r un sain sydd iddynt ar \^ol caledu.
% \end{welsh}}
% {roberts_anghenion_1973}{47--48}
\tqt{When two \mw{b}'s, \mw{d}'s, and \mw{g}'s appear with each other without any vowels between them, they provect, that is, two \mw{b}'s form \mw{p}, two \mw{d}'s form \mw{t}, and two \mw{g}'s form \mw{c}, therefore, two \mw{b}'s may not correspond to one \mw{b}, et cetera. \dots\ In the same manner, \mw{h}, when it immediately follows \mw{b, d,} and \mw{g}, provects to \mw{p, t,} and \mw{c}, and may \mw{b/h} may correspond to \mw{p}, et cetera. \dots\ The consonant \mw{rh} also provects the consonants \mw{b, d, g}, when it immediately follows these consonants. \dots\ Any one of these combinations may correspond to each other, that is, \mw{b/h} may corespond to \mw{b/b}, \mw{d/h} to \mw{d/d}, \mw{d/rh} to \mw{d/dr}, et cetera, since they make the same sound after provecting.}{roberts_anghenion_1973}{pp.\ 47--8, my own translation}

Beyond these basic rules, \textcite{morris-jones_cerdd_1925} notes that this provection changes the value of voiced stops for the purpose of alliteration, but not for rhyme.
% \tqt{\begin{welsh}
% Wrth y ddwy enghraifft olaf [Ag ar dyfiad | \textit{t}af\u{a}w(\textit{d} | \textit{h}\'{o}y\cw.; Ag ar dyfiad | y t\'{a}(\textit{d} | \textit{h}\'{a}el.] fe welir y gall cytsain fud ddwbl ddechreu'n feddal a diweddu'n galed : y mae'r \textit{d} yn \textit{tad} yn dechreu'n feddal, a'r gair yn odli \^{a} \textit{dyfiad} : ond yr \textit{h} yn caledu ei diwedd, sef y ffrwydrad, nes rhoi \textit{t}ael i gytseinio \^{a} \textit{t}ad. Fe ddigwydd hyn yn aml mewn cynghanedd ddisgynedig fel y drydedd groes uchod [\textit{Tr}as mawr h\'a(\textit{d} | \textit{Rh}ys amhr\'ed\u{u}dd.], lle mae ``mawr ha\textit{d}'' yn cytseinio ag ``Amhre\textit{d}udd,'' ond diwedd y \textit{d} yn \textit{had} yn cysylltu \^{a}'r \textit{Rh} i gyfateb i'r \textit{Tr} ar y dechreu. Yn wir, fe ellir dywedyd mai dyna'r ynganiad ym mhob cyswllt ewinog ; yn yn groes gyntaf uchod [\textit{T}udur Ll\'wy(\textit{d} | \textit{h}yder y ll\'ew.], er enghraifft, y mae'r \textit{d} yn Llwy\textit{d} yn cau'n feddal, yna daw anadliad caled a bair iddi agor (neu ffrwydro) 'n galed. A'r un modd, lle bo dwy \textit{d}, y mae'r gytsain yn cau fel \textit{d} ac yn agor fel \textit{t}.
% \end{welsh}}{morris-jones_cerdd_1925}{\S 391}
When a word ends with a voiced stop which is provected by the first consonant of the next word, it may still rhyme with another voiced stop. This means that such a resulting consonant cluster may be thought of as beginning voiced and ending voiceless. In other words, a \gls{provconclus} rhymes with -\gls{D}, even though it alliterates with \gls{T}-. A single \gls{provconclus} may even alliterate with \gls{T}- and rhyme with -\gls{D}~\autocite[\S 391]{morris-jones_cerdd_1925}. 
In the Poetry of the Princes, this type of rhyme is often found, \eg in Example \ref{gymryddy}, where \mw{gymryd} `take' rhymes with \mw{wynuyd} `joy', even though \mw{gymryd} is followed by a \mw{d}:     
    \mwcc[gymryddy]{\acrshort{CBTI}~11,~l.~60}{Dy gymri gymry\al{d d}y wyn, dy wynuy\al{d};}{Your taking of sorrow is your wish, your joy;}%gynnif ddylif ddedlid gymain. /Dy gymri gymryd dy wyn, dy wynfyd; /Dygannan gennyd am bryd, am brain. /

Provection occurs when two successive consonants are pronounced in quick succession. If there is a prosodic pause between the \gls{doubcon} and the consonant preceding it, then it might not double after all. Morris-Jones states that a c\ae sura may, but does not have to prevent consonant doubling~\parencite[\S 392]{morris-jones_cerdd_1925}. In the \gls{cbt}, the exact location of the c\ae sura is often hard to pinpoint, and may depend on our understanding of \gls{mw} metrics~\parencite[241]{daniel_cyfuniadau_2003}. It is for this reason that I have not excluded these instances from the corpus, except where a line break breaks up a consonant cluster. Nevertheless, whenever a c\ae sura stands within a \gls{provconclus}, the example will be considered dubious. A c\ae sura appears about halfway a line in the \mw{cynghanedd groes} or \mw{braidd gyffwrdd}, \ie between the part where the consonants forming the cynghanedd first appear, and the part where their corresponding consonants appear~\autocite[\S 233]{morris-jones_cerdd_1925}. In a \mw{cynghanedd draws}, the c\ae sura  appears after the first half-line containing corresponding consonants, but before the middle consonants that have no corresponding consonants~\autocite[\S 255]{morris-jones_cerdd_1925}. A line in the \mw{cynghanedd sain} has two c\ae suras: one between the two rhyming words, and one between the two alliterating words. In addition, there is the \mw{gair cyrch}, which is separated from its line by a c\ae sura.
% In such cases, the voiced stop only optionally alliterates with its voiceless counterpart~\parencite[\S 392]{morris-jones_cerdd_1925}.
% \tqt{\begin{welsh}
% Eithr nid yw dwy feddal yn caledu \textit{o angenrheidrwydd} yn yr orffwysfa. Er enghraifft : 
%
% \textit{D}eled i oed, | \textit{d}euliw dydd. 
%
% \textit{D}own y ddwywlad | \textit{d}an ddolef. [\dots]
%
% Nid rhaid i ni yma chwaith gymryd bod toriad rhwng y geiriau ; nid gorffen y \textit{d} gyntaf a dechreu un arall a wn\^ai'r datgeiniad, eithr parhad o'r gyntaf yw'r ail, ond bod hamdden yn yr orffwysfa i'w gorffen fel \textit{d}, ac nid fel \textit{t}. Fe geir yr un peth ar ddechreu'r ail gyfres gyfatebol mewn traws, fel hyn: 
%
% \textit{D}ydd yn nos, | pand (\textit{d}iddawn wyd? [\dots]
% \end{welsh}}{morris-jones_cerdd_1925}{\S 392}
An example from the \mw{Beirdd y Tywysogion} is given below, and is found as Example \ref{maeystawddeheu}: \mwcc{\acrshort{CBTIV}~9,~ll.~37--8}{Kedwis kygha\w s ma\w s maeysta\w \al{d --- D}eheu,~| \al{D}uw o nef a'e gwara\w d,}{The joyful pleader saved majesty --- of the South,~| God from heaven saved him,}% ai a ddywawd. /Cedwis cyngaws maws maeystawd - QDeau, /Duw o'r nef a'i gwarawd, /Gw_^r gwrthaw gwrthrych
In Example \ref{maeystawddeheu}, \mw{Deheu} `the South' is a \mw{gair cyrch}, and it alliterates with \mw{Duw} `God' in the next line. This alliteration implies that the \mw{d} in \mw{Deheu} has retained its quality as a voiced stop, despite being preceded by a word ending in \mw{d}. The c\ae sura before the \mw{gair cyrch} allows for this. Instances of provection where a c\ae sura separates a voiceless stop and a \gls{provcon} in the corpus are included, but these examples are to be considered less dependable in drawing conclusions on the working of provection in \gls{mw}.

The rules of the cynghanedd had not fully developed yet in the Poetry of the Princes. Therefore not every example showing consonance may be considered a proper \mw{cynghanedd groes}, \mw{draws}, \mw{sain} or even \mw{braidd gyffwrdd}. This complicates this research somewhat, as it is not always obvious with which consonant a \gls{provconclus} is supposed to alliterate, if at all. Moreover, there are a few lines which do not employ alliteration as a form of line-internal ornamentation at all. Instead, they have rhyme only. These issues make some examples unreliable, as it is not always obvious whether what we interpret as alliteration was meant to alliterate, or whether it is by chance that these two consonants appear together.

\subsection{The corpus}
I have taken the 11\textsuperscript{th}- to 13\textsuperscript{th}-century \mw{Beirdd y Tywysogion}, or the Poets of the Princes as the source of sandhi patterns. I found them using \textcite{parry_owen_concordans_????}'s concordance, which contains all poems in \gls{mow} orthography. This corpus precedes the formalisation of the rules of the cynghanedd in the 14\textsuperscript{th} century, but postdates the \mw{Cynfeirdd}, in whose work \gls{cyfl} was common. The edition of these texts used is the seven-volume \gls{cbt}. When citing \gls{mw} from these works, their source will be referred to by \acrshort{cbt}, followed by their respective volume number, and then by the poem and line numbers as given in the edition.  

From this corpus, I have collected all instances of two successive voiced stops, or a voiced stop followed by \mw{h/rh}, where they alliterate with a voiceless stop. I have also collected all counterexamples, comprising all instances where these consonant clusters alliterate with a voiced stop. In \gls{mow}, these counterexamples would be considered incorrect according to the rules of the cynghanedd. I have not included instances where two coalescing consonants are separated by a line break, but I have included other instances where they are separated by a c\ae sura. I have also excluded instances where the alliterating voiceless stop is also a cluster of a voiced stop followed by either another voiced stop or an \mw{h}, for the obvious reason that such an example would only be able to yield information through circular reasoning. 

\section{Alliteration of  \textit{D}+\textit{D} and \textit{T}}
\label{ddt}
Alliteration of two successive voiced stops with a voiceless stop is fairly common in later \gls{mw}. The implication of the correspondence between \mw{d+d} and \mw{t} is that voiceless and voiced stops are at least in part kept separate by length of articulation. A later example that illustrates this rule is given below, in which the consonants in \mw{Tai nawplad} are repeated in \mw{-d deunawplas}:
\mwcc[folddeunawplas]{Iolo Goch, \mw{Llys Owain Glynd\^wr}, l.\ 37}{\al{T}ai nawplad fol\al{d d}eunawplas,}{nine-plated buildings on the scale of eighteen mansions,}

Examples showing a \gls{doubconclus} alliterating with a voiceless stop may be subdivided into two sub-categories: those showing correspondence of an unlenited voiceless stop and a doubled lenited voiceless stop, and those showing correspondence of an unlenited voiceless stop with a doubled unlenited voiced stop. The former category is more common: \gls{D}+\lT=\gls{T} is found sixteen times, as opposed to nine times for \gls{D}+\xD=\gls{T}.

\subsection{Doubling of radical \xD}
The following lines % Examples \ref{bidduw}, \ref{tauavdda}, \ref{trimuddremynt}, \ref{tauawtdiwyt}, \ref{tafawddiwyd}, \ref{droetdy}, \ref{gorucgwassanaetheu}, \ref{rhaggalar}, and \ref{tyghetdewi} 
 show a \gls{doubconclus} ending with an unlenited voiced stop alliterating with a voiceless stop:
\begin{mwl}
\mwc[gorucgwassanaetheu]{\acrshort{CBTII}~14,~l.~85}{\al{K}etwyr a'm goru\al{c g}\w assanaetheu:}{Warriors did me services:}% /Diarchar arial a dan dalau. /Cedwyr a'm gorug Qgwasanaethau: /Nid y_^nt hyll dihyll na hau dihau. /Cyni
% If the line above is a \mw{cynghanedd draws}, \mw{Ketwyr} alliterates with \mw{-c g\w assanaetheu}, and it is an example. If it is a \mw{cynghanedd groes wreiddgoll}, \mw{goruc} alliterates with \mw{-c g\w assanaetheu}, and it is a counterexample.
\mwc[tyghetdewi]{\acrshort{CBTII}~26,~l.~164}{Vrth glywed dahed \al{t}yghe\al{d D}ewi}{When hearing how good the destiny of Dewi was}% gweryd i gyd a_^ hi /Wrth glywed da_"ed tynged QDewi /A'i fuchedd wirionedd, wirion ynni. /A e_^l ym med
% Alternatively, the words \mw{glywed, dahed, Dewi} also form a full \mw{cynghanedd sain} without the need for the \mw{d} in \mw{Dewi} to be provected by doubling.
% \begin{mwl}
\mwc[rhaggalar]{\acrshort{CBTIII}~8,~l.~63}{O thyrr \al{c}alon rha\al{g g}alar}{If a heart breaks because of mourning}%Ac nid byw fy llyw llawhir, /O thyr calon rhag Qgalar /Y fau a fydd dau hanner. /Pei byw llary Lleisiawn
\mwc[bidduw]{\acrshort{CBTV}~13,~ll.~5--6}{Gloewdid a rydid gan eu rad --- boed \al{t}eu,~| A bi\al{d D}uw yn ganhyad,}{Brightness and freedom with their grace --- may be yours,~| And let God permit,}%  oywdid a rhydid gan eu rhad - boed tau, /A bid QDuw yn ganiad, /Hywel, hoywal pob eirchiad, /Hil Cynan h
\mwc[tauavdda]{\acrshort{CBTVI}~6,~l.~29}{O gysga\w d \al{t}auav\al{d, d}a yr avr --- y medreis}{Through the shelter of my tongue, good was the hour, --- that I could}% dawl, cadw fi i'th gysgawd. /O gysgawd tafawd, Qda yr awr - y medrais /Ymadrawdd o Dduw mawr: /Diwyn dy
% This is an example of \mw{cynghanedd sain}, where \mw{gysga\w d} rhymes with \mw{tauavd}, which in turn alliterates with \mw{-d da}. Significantly, \mw{tauavd} and \mw{da} have different radical consonants, which means this alliteration may not be explained as alliteration of mutated forms of the same radical. 
% An especially notable characteristic of this example is that a \mw{gorffwysfa} may have stood in between \mw{da} and its preceding \mw{-d}, but this does not seem to have prevented doubling.
\mwc[trimuddremynt]{\acrshort{CBTVI}~12,~l.~27}{Bart \w um itt, \al{t}rimu\al{d d}remynt,\footnote{It is debatable whether \mw{dremynt} is the lenited form of \mw{tremynt}, or whether its radical starts with a \mw{d}. \Textcite[s.v.\ ``tremynt\textsuperscript{1}, tremyn\textsuperscript{2}, dremynt'']{bevan_geiriadur_2014} gives both possibilities, and connects it to \mw{trem, drem}~\parencite[s.v.\ ``trem, drem'']{bevan_geiriadur_2014}, which gives the following etymology: `Llyd.\ C.\ drem, Llyd.\ Diw.\ \textit{dremm}: < Clt.\ *\textit{driksmā} < IE.\ *\textit{dr̥ksmā}, o’r un gwr.\ \textit{*derk}- ‘gweld’ ag a welir yn \textit{drych}, Gr.\ δέρκομαι ‘edrych’.' If this etymology is correct, then \mw{dremynt} historically had a radical \mw{d}, so both lenited \mw{t} and unlenited \mw{d} are valid readings for \mw{dremynt}, but unlenited \mw{d} is the etymologically correct reading.}}{I was a bard to you, perfect sight}% /I rwng Rhos ac Epynt, /Bardd fu_^m it, trimud Qdremynt, /A chedymddaith ganwaith gynt. /Cyntaf achubaf
% Here \mw{dremynt} alliterates with \mw{trimud}. 
\mwc[tauawtdiwyt]{\acrshort{CBTVI}~24,~l.~66}{A gna\w t vot \al{t}aua\w \al{t d}iwyt ida\w.}{And it is usual that he has a faithful tongue.}% yw o'i elyn hylithr wylaw /A gnawd fod tafawd  Qdiwyd iddaw. /Gnawd ysgafael hael yn hwylaw - o'i du /A
% This is an example of a \mw{cynghanedd sain}, where \mw{gna\w t} rhymes with \mw{taua\w t}, which in turn alliterates with \mw{-t diwyt}. The \mw{d} in \mw{diwyt} is a radical voiced stop, so there is no question of a lenited consonant corresponding to its radical counterpart, 
\mwc[tafawddiwyd]{\acrshort{CBTVII}~26,~l.~16}{Nyd anna\w d \al{t}afa\w \al{d d}iwyd ita\w!}{It is not uncommon that he has a faithful tongue!}%i feirdd, fyrddoedd wallaw, /Nid annawd tafawd Qdiwyd iddaw! /Arf torfoedd, terrwyn achubaw, /Arwydd coe
% This is an example of a \mw{cynghanedd sain}, where \mw{anna\w d} rhymes with \mw{tafa\w d}, which in turn alliterates with \mw{-d diwyd}, 
\mwc[droetdy]{\acrshort{CBTVII}~33,~l.~31}{Tynn droe\al{t d}y ued\w l, \al{t}ra uedych --- dy b\w yll,}{Withdraw the foot of your mind, whilst you possess --- your prudence,}%o o ffyrdd didro hyd tra'i ceffych, /Tyn droed Qdy feddwl, tra feddych - dy bwyll, /O blith maglau twyll
\end{mwl}

% At any rate, \mw{dy} starts with its radical initial consonant which means the correspondence between \mw{dy} and \mw{tra} cannot be construed as alliteration of a radical and mutated consonant.
Some examples must be considered doubtful, because hard-and-fast rules had not been established for the cynghanedd. Example~\ref{tyghetdewi} potentially lends itself to a different analysis of alliteration: \mw{Dewi} may be thought to alliterate with \mw{dahed} instead of \mw{tyghed}. Examples~\ref{gorucgwassanaetheu} and \ref{droetdy} are similarly doubtful because the alliteration pattern shown in the examples seem to be the product of chance. In the latter case, two possible analyses of the cynghanedd ar possible. If \mw{-d dy} alliterates with \mw{tra}, then it is a proper \mw{cynghanedd groes wreiddgoll}, except that the \mw{r} in \mw{tra} is unaccounted for, although resonants may occasionally appear without a corresponding consonant~\parencite[203--07]{jones_meddwl_2005}. Alternatively, \mw{droet}  alliterates with \mw{tra}, and the consonant of \mw{-d dy} is unaccounted for. In Example~\ref{tauavdda}, a c\ae sura may prevent consonant doubling, because a new phrase starts with \mw{da}. In total, four examples attesting to \gls{D}+\xD=\xT\ may be considered doubtful.

Otherwise, the validity of these examples does not depend on presupposing non-orthographical lenition, and given how they alliterate with words whose radical consonant is different, their alliteration is never due to \gls{cyfl}. As such, these are in theory the most straightforward examples attesting to the alliteration of \gls{D}+\gls{D} and \gls{T}.  We shall see that this is not the case with the following examples.


\subsection{Doubling of lenited \gls{T}}
The lines in this subsection show a \gls{doubconclus} ending with a lenited voiceless stop alliterating with a radical voiceless stop. Notably, none of these examples represent lenition orthographically. Because lenition is never written in the original \gls{mw} in these instances, it has been inserted by the editors of the \gls{cbt}. 
\begin{mwl}
\mwc[kygauawckeigyeu]{\acrshort{CBTI}~27,~l.~81}{Gwedy \al{k}ygaua\w \al{c k}eigyeu --- y dyndid,}{After entangled branches --- the courage,}%  wedi gwasanaeth y penaethau, /Gwedi cynghafawg Qgeingiau - y ddyndid, /Gwedi llid ymlid yn ymladdau, /Go
% Here, \mw{keigyeu} should be read as \mw{geigyeu}, because it is lenited following a preposed adjective. This lenition is not written, however, because it follows \mw{-c}, and therefore alliterates with \mw{kygaua\w c}.
\mwc[cristyoniccroes]{\acrshort{CBTI}~33,~l.~53}{Y gymryt creuyd \al{C}ristyoni\al{c c}roes,}{To take the faith of the Christian cross,}%  ei ddwyn ar foes /I gymryd crefydd Cristionig  Qgroes, /Anwogawn woglyd o aneirioes, /Can ni w_^yr ein p
% Here, \mw{croes} should be lenited following a preposed adjective. It is not shown as lenited following \mw{-c}, and therefore suggests alliteration with \mw{Cristyonic}.
\mwc[tremidtreiddiaw]{\acrshort{CBTII}~24,~l.~22}{A llafn tra llafn, a llid tra llid \al{t}remi\al{d t}reiddiaw,}{And blade over blade, and wrath over wrath piercing battle}% aw, /A llafn tra llafn, a llid tra llid tremid Qdreiddiaw, /A gw_^r tra gw_^r, a thw_^r tra thw_^r yn tr
% Note that \mw{treiddiaw}, which is written with its radical initial consonant, should be lenited to \mw{dreiddiaw} according to the Modern Welsh respelling, but this is not written. 
\mwc[rhadtad]{\acrshort{CBTII}~24,~l.~25}{Rhyn wyn wenwyn, rha\al{d t}ad \al{t}erwyn, torf ymandaw}{Rigid white poison, gracious fierce father, Lord of a host}% gwaith daith derfyniaw. /Rhyn wyn wenwyn, rhad Qdad terwyn, torf ymandaw /Rhod fod falchrydd rhad fad fe
% Here, \mw{tad} is lenited in the Modern Welsh respelling, although this is not written in Middle Welsh.
\mwc[hebperiglawr]{\acrshort{CBTIV}~17,~l.~28}{Rac yn \al{p}erigla\w\ he\al{b p}erigla\w r;}{Before endangering us without a priest}%  rhydd iolawr - heb daw /Rhag ein periglaw heb beriglawr; /Gwddam y'n gorllwg Gw_^r llwyrddwg llawr, /G
% The word \mw{perigla\w r} is lenited after \mw{heb} here, but the resulting clash of two \mw{b}'s still alliterates with \mw{p} in \mw{perigla\w}. Interestingly, orthography corroborates this point here, since lenition of \mw{perigla\w r} is not written.
\mwc[termudteyrnon]{\acrshort{CBTV}~4,~l.~24}{Am olud \al{t}ermu\al{d T}eyrnon;}{For the wealth of silent Teyrnon;}%  Im o'm dawn y'm daw cyflawddon /Am olud termud QDey_"rnon; /Fy nhafawd yn frawd ar Frython /O Fo_^r Udd
% \mw{Teyrnon} is given as \mw{Deyrnon} in Modern Welsh.
\mwc[eurdudterrwyndrud]{\acrshort{CBTV}~4,~l.~51}{Yn eurdu\al{d t}errwyndrud \al{t}irion,}{In the golden people's fierce gentle hero,}%  orion - clod, /Yn cludaw anoethion, /Yn eurdud Qderwynddrud tirion, /Yn eurdwr ar eurdorchogion /Yn eurd
% Note that lenition in \mw{terrwyndrud} is not shown here, but the Modern Welsh respelling shows \mw{derwynddrud}.
\mwc[eurgreidteyrnet]{\acrshort{CBTV}~6,~ll.~5--6}{Ardwy cad, argrad eurgrei\al{d --- t}eyrnet,~| \al{T}eyrngert ordyfneid,}{A defence in battle, fear of valiantly fighting reign~| Familiarity with a sovereign's song,}%  nnygn drechiaid, /Ardwy cad, argrad eurgraid - Qdey_"rnedd, /Tey_"rngerdd orddyfnaid, /Llew trylew tryle
% Here, \mw{-d teyrnet} alliterates with \mw{Teyrngert} in the next line. It is written lenited in the Modern Welsh respelling.
\mwc[enwawckyueilyawc]{\acrshort{CBTV}~14,~l.~8}{Enwa\w \al{c K}yueilya\w c, ys \al{c}of aele.}{Famous Cyfeiliog, sad is his memory.}%  ei belre, /Heb dymyr Tudyr, tud Elise, /Enwawg QGyfeiliawg, ys cof aele. /Heb ferch Gynfelyn, gelyn gne
% Here \mw{Kyueilya\w c} is lenited following a preposed adjective, but lenition is not shown. It alliterates with \mw{cof aele}.
\mwc[duccroc]{\acrshort{CBTV}~15,~ll.~9--10}{Archaf arch y Bedyr o berthynas --- \al{C}rist~| A du\al{c C}roc yn urtas,}{I request a request to Pedr who is property --- of Christ~| who carried a Cross with dignity,}%  Archaf arch i Bedr o berthynas - Crist /A ddug QGrog yn urddas, /Trwy eiriawl teg ymiawl Tomas /A Phylip
% Here, \mw{Crist} alliterates with \mw{-c Croc} in the next line. The Modern Welsh respelling lenites \mw{Croc}, presumably as object lenition, which may not yet have existed in this period~\parencite[55-57]{van_sluis_development_2014}.
\mwc[gadtrom]{\acrshort{CBTV}~23,~l.~153}{Eil ga\al{d t}rom y'n \al{t}remynassant,}{In a second heavy battle they raided us,}%  borthi, /Burthiaist wy_^r yn nifant. /Ail gad  Qdrom y'n tremynasant, /Udd addien, uch Dygen Dyfnant; /A
% Middle Welsh \mw{trom} should be read as lenited here, following feminine \mw{gad}, but lenition is not written presumably as a result of following \mw{-d}. This also makes it alliterate with \mw{tremynassant}.
\mwc[yttreul]{\acrshort{CBTVI}~5,~l.~43}{Gwae\w\ dur yn y\al{t t}reul gur \al{t}reis,}{[Like] a steel spear where he spends his pain of violence,}% ei hil, /Hael freenin Cemais; /Gwayw dur yn yd Qdraul gur trais, /Gwae niw gw_^yl fal y'i gwelais! /Gwel
% Note that verbal particle \mw{yt} causes lenition here, so \mw{treul} should be read as \mw{dreul}, and it alliterates with \mw{trais}.
\mwc[eudunedtrwyted]{\acrshort{CBTVI}~10,~l.~6}{Eudune\al{d t}rwyted, \al{t}ra yn gatter.}{A vow's journey, while we must be permitted}% ffydd, i grefydd, i gryfder - myned /Eidduned  Qdrwydded, tra yn gater. /I achubaw nawdd ni wader - ydd
% Here, \mw{trwyted} alliterates with \mw{tra}, and is given as \mw{drwydded} in the Modern Welsh respelling.
\mwc[hebporth]{\acrshort{CBTVI}~12,~ll.~41--2}{O dywedeis-y eir ar wekry --- he\al{b p}orth~| \al{P}arth eurgolofyn Kymry,}{If I spoke a word on feebleness --- without a gate~| The place of golden support of Wales,}% oreuraw gair. /O dywedais-i air ar wecry - heb borth /Parth eurgolofn Cymru, /Diwygaf, honnaf hynny, /D
% The words \mw{-b porth} and \mw{Parth} alliterate here. \mw{Heb} causes lenition to \mw{porth}, although this is not shown orthographically.
\mwc[hebplyc]{\acrshort{CBTVI}~24,~l.~4}{M\w yneir o'm \al{p}legyt he\al{b p}lyc arna\w.}{A gentle word from my side without distorting it.}% wn a'm daw - i gennyd, /Mwynair o'm plegyd heb blyg arnaw. /Mur, Modur, Llywiadur, rhag llaw /Y mae i'm
%  Here, \mw{p} in \mw{plegyt} alliterates with \mw{heb plyc} (\gls{mow}: \mw{heb blyg}). Even though \mw{heb} causes lenition to \mw{plyc}, the succession of two \mw{b}'s make for a sound that may alliterate with \mw{p}. It is relevant here, however, that the second \mw{b} is in fact a lenited \mw{p}, and may have alliterated due to lenited and unlenited variants of the same consonant still being able to alliterate in earlier Middle Welsh poetry. 
\mwc[hebpenn]{\acrshort{CBTVII}~36,~l.~86}{Gadael \al{p}enn arnaf he\al{b p}enn arna\w:}{To leave a chieftain on me without a head on him:}%oedd im, am fy nhwyllaw, /Gadael pen arnaf heb ben arnaw: /Pen pan las, ni bu gas gymraw, /Pen pan las,
% Although lenition is not written here, the second \mw{penn} is lenited following \mw{heb}, yielding \mw{-b benn}, which corresponds with the first \mw{penn}.
% \mwcc{\acrshort{CBTVII}~44,~l.~13}{}{}%/Oedd coeth, digrawnddoeth, digrif. /Tros nad  Qdigrif im am was - haelddifai, /Hwyl ddifefl gyweithas,
\end{mwl}

Table \ref{reasonlenitionexddt} lists the reasons why lenition should be inserted in the lines given above.  The reason why lenition should be read in these cases varies, and so does the certainty whether lenition should be inserted. In all of these cases, emending lenition depends on our present-day understanding of \gls{mw} lenition rules. 
\begin{table}[h]
\centering
\begin{tabular}{@{}llll@{}}
\toprule
\textbf{\textbf{Example}} & \textbf{\textbf{Source}} & \textbf{\textbf{Lenited word}} & \textbf{\textbf{Reason for lenition}} \\ \midrule
\ref{kygauawckeigyeu} & \acrshort{CBTI}~27,~l.~81 & \mw{keigyeu} & Preposed adjective \\
\ref{cristyoniccroes} & \acrshort{CBTI}~33,~l.~53 & \mw{croes} & Preposed adjective \\
\ref{tremidtreiddiaw} & \acrshort{CBTII}~24,~l.~22 & \mw{treiddiaw} & Preposed adjective \\
\ref{rhadtad} & \acrshort{CBTII}~24,~l.~25 & \mw{tad} & Preposed adjective \\
\ref{hebperiglawr} & \acrshort{CBTIV}~17,~l.~28 & \mw{perigla\w r} &  \mw{heb} \\
\ref{termudteyrnon} & \acrshort{CBTV}~4,~l.~24 & \mw{Teyrnon} & Preposed adjective \\
\ref{eurdudterrwyndrud} & \acrshort{CBTV}~4,~l.~51 & \mw{terrwyndrud} & Preposed adjective \\
\ref{eurgreidteyrnet} & \acrshort{CBTV}~6,~ll.~5--6 & \mw{teyrnet} & Preposed adjective \\
\ref{enwawckyueilyawc} & \acrshort{CBTV}~14,~l.~8 & \mw{Kyueilya\w c} & Preposed adjective \\
\ref{duccroc} & \acrshort{CBTV}~15,~ll.~9--10 & \mw{Croc} & Object lenition \\
\ref{gadtrom} & \acrshort{CBTV}~23,~l.~153 & \mw{trom} & Feminine adjective \\
\ref{yttreul} & \acrshort{CBTVI}~5,~l.~43 & \mw{treul} &  Particle \mw{yt} \\
\ref{eudunedtrwyted} & \acrshort{CBTVI}~10,~l.~6 & \mw{trwyted} & Preposed adjective \\
\ref{hebporth} & \acrshort{CBTVI}~12,~ll.~41--2 & \mw{porth} &  \mw{heb} \\
\ref{hebplyc} & \acrshort{CBTVI}~24,~l.~4 & \mw{plyc} &  \mw{heb} \\
\ref{hebpenn} & \acrshort{CBTVII}~36,~l.~86 & \mw{penn} &  \mw{heb} \\ \bottomrule
\end{tabular}
\caption{Reason for lenition of lenited voiceless stops in examples showing \gls{D}+\gls{D}=\gls{T}.}
\label{reasonlenitionexddt}
\end{table}

Lenition should certainly be inserted where the words to be lenited follow \mw{heb} `without', follow verbal particle \mw{yt}, or serve as an adjective following a feminine noun. We know lenition occurred in these contexts~\parencite[\S\S 20, 22, 23]{evans_grammar_1964}. 
In most cases, however, a preposed adjective is responsible for lenition. There is no doubt that lenition following preposed adjectives occurred in \gls{mw}~\parencite[\S 20]{evans_grammar_1964}, but emending lenition depends on the correct reading of the syntax of the line. The \gls{cbt} tends to contain notoriously ambiguous syntax, and nouns may be used as adjectives and vice versa~\parencite{daniel_cyfuniadau_2003}\footnote{When a preposed adjective is used as a noun, it similarly lenites. Alternatively, one could consider such nouns to be a preposed genitive rather than a preposed adjective. However, preposed genitives also lenite, so the question whether such nouns should be considered a preposed adjective or a preposed genitive is irrelevant to the issue of lenition, so `preposed adjective' is the term used throughout for ease's sake.}. 
The most doubtful example is Example~\ref{duccroc}, which is speculatively lenited in the \gls{mow} standardised text; presumably because object lenition is held to have caused lenition following \mw{duc} `brought'. However, object lenition following this type of verb is a comparatively late \gls{mw} phenomenon~\parencite[55-57]{van_sluis_development_2014}. 

The fact that lenition is not shown in these examples makes them doubtful cases when arguing that consonant doubling could produce a voiceless stop, since their validity as an example is predicated on our understanding of \gls{mw} grammar and emending accordingly rather than on reading the text as it is presented to us. On the other hand, the very fact that lenition here is consistently not written when the lenited consonant is preceded by the same consonant confirms the assumption that this preceding consonant indeed undid  lenition of a lenited voiceless stop. This makes these lines particularly strong examples showing that provection by doubling operated across word boundaries.

In Example~\ref{eurgreidteyrnet}, the doubled consonant is the first of a \mw{gair cyrch}, meaning it only doubled optionally. Together with Example~\ref{duccroc}, this means that two examples must be considered doubtful examples.

\subsection{Non-doubling}
Here, I will present instances of non-doubling stops, or counterexamples to the assertion that \gls{D}+\gls{D} could produce \gls{T}. They show a correspondence of \gls{D}+\gls{D} to \gls{D} instead. These instances may be subdivided into examples where lenited voiceless stops alliterate with lenited voiceless stops, and unlenited voiced stops alliterate with unlenited voiced stops. The doubling \gls{D} and the alliterating \gls{D} all have the same radical consonants, except in a handful of instances, which are discussed below.


\subsection{Non-doubling of radical \textit{D}}
Only two instances of non-doubling show correspondence between lenited voiceless stops and unlenited voiced stops, of which one is doubled. In both cases, the doubling consonant is an \xD.
\begin{mwl}
\mwc[alltuddrud]{\acrshort{CBTIV}~18,~l.~27}{Neud wyf alltu\al{d d}rud a \al{d}reigyl dy wendud}{I am a foolish outlaw that inhabits your blessed country}% liaf /O bechu a bechws Addaf. /Neud wyf alltud Qdrud a dreigl Dy wendud /A'th wendorf amdanaf; /Neud bei
% \mwcc{\acrshort{CBTV}~1,~l.~142}{}{}%  ein tud rwyf hollawl, /Ac ein Taid, ac ein Tad Qdwywawl, /Y Gw_^r a orug wrawl - dey_"rnllin /Breienin b
\mwc[kymrawtdreic]{\acrshort{CBTV}~28,~ll.~29--30}{Mawr a gwyn kymr\w yn yr dwyn kymra\w \al{t --- d}reic~| \al{D}rwy ganyat y Drinda\w t,}{Large pain and mourning because of a friend --- of the lord~| Through the messenger of the Trinity}%  mawr. /Mawr a gw_^yn cymrwyn er dwyn cymrawd - Qdraig /Drwy ganiad y Drindawd, /Ba_^r anwar llachar, lla
\end{mwl}
In Example \ref{alltuddrud}, radical \mw{drud} `foolish' alliterates with lenited \mw{dreigyl} `inhabits'. In Example \ref{kymrawtdreic},  radical \mw{dreic} alliterates with lenited \mw{drwy} in the next line, or alternatively with \mw{Drindod}. Note, however, \mw{drwy} is not lenited as the result of any morphological process here. Rather, a lenited version exists side by side with radical \mw{trwy}. When lenition is a petrified property of a word, the phonetics may work out differently than they do with grammatical lenition\footnote{Cf.\  \mw{garauuys} and \mw{ga/rauuys} `Lent' in ll.~9, 9--10, respectively in the \gls{bbch}, where word-initial lenition of voiceless stops is written only for these words, perhaps because lenition petrified here.}. 

There are two instances where a doubling \xD\ alliterates with \lT, but also with another radical \gls{D}. 
\begin{mwl}
\mwc[gwenycgwynn]{\acrshort{CBTI}~9,~l.~153}{Dygoglat \al{g}\w eny\al{c g}\w ynn \al{G}ygrea\w dyr vynyt,}{White waves strike against Cyngreawdr mountain}%  /O Abermenai mynych dyllydd. /Dyogladd gwenyg  Qgwyn Gyngreawdr fynydd, /Morfa Rhianedd Maelgwn rebydd.
\mwc[daddaeoni]{\acrshort{CBTV}~19,~l.~27}{\al{D}etyf dy \al{d}a\al{d, d}aeoni a'th glyn,}{Custom of your father, goodness adheres to you,}%  laf /Hyd aeth haul o'i gylchyn. /Deddf dy dad, Qdaioni a'th ly_^n, /Dadeni haeloni Heilyn. /W_^yr Ywain
\end{mwl}
In Example \ref{gwenycgwynn}, radical \mw{-c gwynn} `white' alliterates with \mw{gwenyc} `waves', but also with the lenited place name \mw{Gygrea\w dyr}. In Example \ref{daddaeoni}, radical \mw{-d daeoni} `kindness' alliterates with  lenited \mw{dad} `father', but also with radical \mw{detyf} `act'. The latter is not a convincing example of non-doubling, given the strong  prosodic boundary between \mw{-d} and \mw{daeoni}.

There are 72 instances where a radical voiced stop preceded by a radical voiced stop alliterates with another radical voiced stop. This number contrasts with the meagre nine examples of a doubled radical voiced stop alliterating with a radical voiceless stop.
\begin{mwl}
\mwc[gwerennicgurhid]{\acrshort{CBTI}~1,~l.~43}{Goruir menic, mur \al{g}werenni\al{c, g}urhid gormant,}{Noblemen from a place, of lively defence, of excess manliness,}%  ddon feddwaint, /Gorwyr mennig, mur gwyrennig, Qgwryd gormant, /Terrwyn am dir, rhi rhaith cywir o hil M
\mwc{\acrshort{CBTI}~3,~l.~58}{Ke\al{d d}oeth ef ny\al{d a}eth yn warthega\w c:}{Although he came, he did not go possessing cattle:}%Efrawg. /Dybu brenin Lloegr yn lluyddawg, /Cyd doeth ef nid aeth yn warthegawg: /Ni yn Eryri yn rhei_"a
\mwc{\acrshort{CBTI}~3,~l.~95}{\al{G}\w ladoet ouyneic, drei\al{c G}\w yndodyt,}{Lands of hope, dragon of Gwyndodydd,}%  na meirch gweilydd, /Gwladoedd ofynaig, draig  QGwyndodydd, /Gw_^r a roddai gad cyn dybu ei ddydd. /Gru
\mwc{\acrshort{CBTI}~4,~l.~3}{\al{G}wledi\al{c g}\w ladoruod goruchel wenrod,}{A prince of land-gaining of the highest heavens,}%  i, /I'm Harglwydd uchaf archaf weddi. /Gwledig Qgwladorfod goruchel wenrod, /Gwrda, gwna gymod gryngod a
% \end{mwl}
% Note that this line has correspondence between lenited \mw{wenrod} and all the other words starting with a radical \mw{g}, although it also forms a \mw{cynghanedd sain} without \mw{wenrod} alliterating.
% % \mwc{\acrshort{CBTI}~6,~l.~6}{}{}%dain adanad, prydyddion borthiad, /Boed cyfoed dy rad a_^'th wlad a'th wawd. /Ethyw dy ergryd yn eithaf
% \begin{mwl}
\mwc{\acrshort{CBTI}~7,~l.~5}{Caru \al{d}yn, ny\al{d d}ilys ogoned,}{To love a man, not true glory,}%g foes, /Corf eirioes eurfyged. /Caru dyn, nid dilys ogoned, /Can dyddaw i fraw frwyn dynged. /Cerais u
\mwc{\acrshort{CBTI}~7,~l.~126}{Urwysc fer fraeth, ureis\al{c g}ra\w n uaeth \al{g}rewys;}{A vigorous steadfast swift [horse], large grain-storage for mares;}%  w gwalchfrowys, /Frwysg ffe_^r ffraeth, fraisg Qgrawnfaeth grewys; /Lliaws gwinau ffadw ffrawdd tywys, /
\mwc{\acrshort{CBTI}~9,~l.~50}{Ra\al{c g}orwyr Yago \al{g}wyar drablut.}{Before the great-grandson of Iago [was] bloody battle.}%  udd; /Go-rys-ymedd glyw o'i glywed yno, /Rhag  Qgorwyr Iago gwyar drabludd. /Gwalchmai y'm gelwir, ga_^l
\mwc{\acrshort{CBTI}~9,~l.~89}{Pell nad huna\w \al{c g}\w enn (\al{g}ogwn pahyr!)}{It is long that a fair girl is not slumbering (I know how long!)}%  ilging porffor pwyllad fyfyr. /Pell nad hunawg Qgwen (gogwn pahyr!) /Pan ddyfrig trawd blawd blaen efell
\mwc{\acrshort{CBTI}~9,~l.~121}{`Arglodi\al{c G}\w endyd \al{g}wynn yssym,' met pa\w b,}{`Fair Venedotians are renowned to us,' said everyone,}%  lles ne_^r Cynan cynweddiawn ffaw. /`Arglodig  QGwyndyd gwyn ysym', medd pawb, /Prydain allweddawr oll y
\mwc[geddy]{\acrshort{CBTI}~11,~l.~58}{Dyganre \al{d}y ge\al{d d}y gyfwyrein,\footnote{Manuscript reading for \mw{dyganre}: \mw{dyganred}. Accepting the manuscript reading invalidates this line as a counterexample, because it would make this line an irrelevant \mw{cynghanedd sain}, rather than a \mw{cynghanedd braidd gyffwrdd}. However, the word \mw{dyganred} is not found elsewhere in \gls{mw}. Another possibility is that this line should be read as a \mw{cynghanedd sain}, with the \mw{d-} following \mw{dyganre} also attached to the end of this word to indicate rhyme with \mw{ged} rather than consonant doubling.}}{Your gift goes with stirring,}%Alaf geinrydd, elw ddyganrain, /Dyganre dy ged dy gyfwyrain, /Dy gynnif ddylif ddedlid gymain. /Dy gymr
\mwc[gymryddy2]{\acrshort{CBTI}~11,~l.~60}{\al{D}y gymri gymry\al{d d}y wyn, dy wynuyd;}{Your taking of sorrow is your wish, your joy;}%gynnif ddylif ddedlid gymain. /Dy gymri gymryd dy wyn, dy wynfyd; /Dygannan gennyd am bryd, am brain. /
\mwc{\acrshort{CBTI}~13,~l.~5}{Kennhyn can yn \al{D}uw, neu\al{d d}e,}{Who is with us with our God, is consuming}%ru /Ac afar gyfwyre. /Cennyn can ein Duw, neud de, /Cyfryw atreg cof atre. /Cyfdaerant i ro_^n, a rhin
\mwc{\acrshort{CBTI}~13,~l.~19}{Gna\w d \al{g}\w rei\al{c g}well genthi a \w o gwaeth iti:}{Usually a woman prefers what is bad for her:}%  e /Yn farch dewr, yn farch dyre. /Gnawd gwraig Qgwell genthi a fo gwaeth iddi: /Fy addas nid debre. /Dil
\mwc{\acrshort{CBTI}~19,~l.~13}{Ys bwyf yn oe\al{d d}yn \al{d}engmlwydd --- ar hugain}{I should be the age of a thirty years old man}%byddawd oes barawd barthlwydd. /Ys bwyf yn oed dyn dengmlwydd - ar hugain /Rhag deulin fy Arglwydd
% \mwc{\acrshort{CBTI}~21,~l.~24}{}{}%ddef y pechu; /Ac Efa ac ef a'i gwybu: /Gwypid Duw, gwybod a ddarfu. /Gorau im o'm Rhe_^n rhagfeddu, /
% \mwc{\acrshort{CBTI}~24,~l.~9}{}{}%west Duw bu deugeinpryd. /Am dduw Merchyr Brad dybu bryd - Iddas, /Bredychu ein Hysbryd, /A Difiau y'n
% \mwc{\acrshort{CBTI}~27,~l.~41}{}{}% a_^ myned myn y'th giglau. /Pen cred fu weled Dy weli_"au /Pan yth archollwyd o'th archollau. /Can bor
\mwc{\acrshort{CBTI}~28,~l.~9}{Pobloet by\al{d d}yfryd, eu \al{d}ifra\w d --- a wneir,\footnote{Variant reading for \mw{Pobloet}: \mw{pobyl}; manuscript reading for \mw{dyfryd}: \mw{hyfryd}, another manuscript has \mw{hyfryt}. These variant readings invalidate it as a counterexample here, but may serve as counterexamples to the idea that \gls{T} may alliterate with \gls{D}+\mw{h}.}}{Peoples of the dismal world, their devastation --- is done,}%ddaw ataw, pobl fraw, i'r Frawd; /Pobloedd byd dyfryd, eu difrawd - a wneir, /Ar ni wne_^l Ei gardawd.
\mwc[goludawcgolychwn]{\acrshort{CBTI}~33,~l.~6}{Du\w\ g\w yl \al{g}oluda\w \al{c g}olych\w n Di}{Prosperous gracious God, we pray to You}%  ein da, ein doeth weini. /Duw gw_^yl goludawg, Qgolychwn Di /Nad elom i'n llog, yn lle cyni /Yn llwgr ll
\mwc[dirnaddedwyt]{\acrshort{CBTII}~1,~ll.~87--8}{Mor ya\w n ym o'm da\w n ac o'm dirna\al{d --- d}edwyt~| Goffau \al{d}ouyt o'm newyt nad,}{As well for me from my ability as from my comprehension --- fortunate~| to remember the lord of my new song,}%harfor. /Mor iawn im o'm dawn ac o'm dirnad - dedwydd /Goffa_"u dofydd o'm newydd na_^d, /Can rhoddes
\mwc{\acrshort{CBTII}~1,~l.~178}{Kedwi\al{d D}u\w\ \al{d}e\w rdoeth kyuoeth Caduann.}{Let wise and brave God protect the kingdom of Cadfan.}%n, /Cadwedig fy ngwawd i'w logawd lan: /Cedwid Duw dewrddoeth gyfoeth Cadfan
\mwc[gyrchyatdeivyr]{\acrshort{CBTII}~3,~ll.~3--4}{Aruaeth y\w\ gennyf, arueu gyrchya\al{t --- D}eivyr,~| Dodi \al{d}\w fyr y'th var\w nat.}{I have the intention, weapon-battler of [against] the English,~| To water your elegy.}% roddiad, /Arfaeth yw gennyf, arfau gyrchiad - Deifr, /Dodi dwfr i'th farwnad. /Ni hepgoraf-i rwyf, hyd
\mwc{\acrshort{CBTII}~4,~l.~32}{Ual y deruyd \al{g}\w yd ra\al{c g}\w ynt.\footnote{Variant reading for \mw{rhag}: \mw{ra}.}}{As the forest perishes before wind.}% y deryw, y derynt, /Fal y derfydd gwy_^dd rhag Qgwynt. /Gwen ysgor Feiriawn, Feirionnydd - hygar /A'i hy
\mwc{\acrshort{CBTII}~5,~l.~6}{O blegy\al{t D}u\w, a'e \al{d}ywa\w t}{Because of God, who said it}%aw_^yr. /Morudd, meidrawl ei ddefawd /O blegid Duw, a'i dywawd /Maint yr arwyddon a fydd /Pymthegfed dy
\mwc{\acrshort{CBTII}~5,~l.~8}{Pymthecue\al{t d}yd kynn \al{D}yd Bra\w t.\footnote{Variant reading for \mw{Pymthecuet dyd}: \mw{pymthecuettyd}.}}{The fifteenth day before the Day of Judgment.}%i dywawd /Maint yr arwyddon a fydd /Pymthegfed dydd cyn Dydd Brawd. /Y pedwerydd dydd ar ddeg /Yd gyrch
\mwc{\acrshort{CBTII}~5,~l.~17}{Deudecue\al{t d}yd, \al{D}u\w\ diga\w n\footnote{Variant readings for  \mw{Deudecuet dyd}: \mw{deudecvettyd} (\gls{wbr} ff.\ 11\textsuperscript{v}--12\textsuperscript{r}),  \mw{deudecuettyd} (\gls{wbr} f.\ 61\textsuperscript{v--r}); variant reading for \mw{Duw}: \mw{duc} (\gls{wbr} ff.\ 11\textsuperscript{v}--12\textsuperscript{r}).
}}{The twelfth day, God makes}%'w syllu, /Nad ym men yd fu yd fi. /Deuddegfed dydd, Duw ddigawn /Anifeilaid mo_^r mawrddawn: /Y daw po
\mwc[decuetdaw]{\acrshort{CBTII}~5,~l.~29}{Nawuet g\w edy \al{d}ecue\al{t d}a\w,}{The ninth comes after the tenth,}%ffrawdd, ffrydiau ta_^n. /Nawfed gwedi degfed daw, /Duw ei hun yn ei luniaw: /Ufeliar ta_^n trwy ysgyr
\mwc{\acrshort{CBTII}~5,~l.~33}{Wythue\al{t d}yd \al{d}ybyd dyar ---\footnote{Variant readings: \mw{Wythuettyd herwyd arwyd} (\gls{wbr} ff.\ 11\textsuperscript{v}--12\textsuperscript{r}); for \mw{dybyd}: \mw{defnyd} (\gls{wbr} f.\ 61\textsuperscript{v--r}).}}{The eighth day, sadness will come}%ysgyr, /Ergyr o'r sy_^r yn syrthiaw. /Wythfed dydd dybydd dyar - /Deddfau diau diarchar - /Dygn ddango
\mwc{\acrshort{CBTII}~5,~l.~37}{Seithue\al{t d}yd, \al{d}yd darogan:\footnote{Variant readings: \mw{Seithuetdyd darogan;} (\gls{wbr} f.\ 61\textsuperscript{v--r}); \mw{Seithuettyd dyd darogann} (\gls{wbr} ff.\ 11\textsuperscript{v}--12\textsuperscript{r})}}{The seventh day, a day of prophecy;}%s erfyn, /Fal yd gry_^n dyn a daear. /Seithfed dydd, dydd darogan: /Main mwyaf oll a holltan; /Gwyrthau
\mwc{\acrshort{CBTII}~6,~l.~66}{Ac ny chwart y \al{g}\w r hi ra\al{c g}ortin.}{And her husband does not laugh at violence.}% yn chwerthin, /Ac ni chwardd ei gw_^r hi rhag  Qgorddin. /Gorddin mawr a'm dawr, a'm daerawd, /A hiraeth
\mwc{\acrshort{CBTII}~7,~l.~2}{Gorawenus \al{g}ly\w\ ra\al{c g}le\w\ arglwyt.}{Joyful is the host before the bold lord.}% r haf, amsathr gorw_^ydd, /Gorawenus glyw rhag Qglew arglwydd. /Gorewynnawg ton tynhegl ebrwydd, /Gorwis
\mwc{\acrshort{CBTII}~9,~l.~2}{Myn yd gar \al{g}wylde\al{c g}\w eled gwylann.}{The place where the fair modest one would wish to see a seagull.}% r wenglaer o du gwenlan, /Myn yd ga_^r gwyldeg Qgweled gwylan. /Yd garwn-i fyned (cyni'm cared yn rhwy)
% \mwc{\acrshort{CBTII}~14,~l.~141}{Menestyr, g\w elut-dy g\w yth g\w eith Llid\w m dir;}{}%gloyw golau, gwrddlew babir. /Menestr, gwelud-di w_^yth gwaith Llidwm dir; /Y gwy_^r a barchaf, wynt a
% This example is questionable, because it operates under the assumption that \mw{dy} alliterates with \mw{dir}. However, \mw{dy} is not a full syllable for the purpose of metrics, \ie pronouncing \mw{dy} as a separate syllable produces too many syllables, so this would be avoided.
% \mwc{\acrshort{CBTII}~14,~l.~143}{}{}%r a barchaf, wynt a berchir. /Menestr, gwelud di galchdo_"ed - cyngrain /Yng nghylchyn Owain, gylchwy
\mwc{\acrshort{CBTII}~16,~l.~48}{\al{G}wr ra\al{c g}werin Dyssillyaw.}{A man before the people of Tysilio}% w; /Gwy_^r orfod gwrddglod gludaw, /Gw_^r rhag Qgwerin Dysiliaw. /Ysef a'u herly, arlwy garthan - ddyn,
\mwc{\acrshort{CBTII}~23,~l.~17}{Rhwyfiadur \al{D}ygen, rha\al{d d}igain,}{Chieftain of Dygen, fair grace,}%glyd am glod orwyrain, /Rhwyfiadur Dygen, rhad digain, /Rhwyf Arfon, Iorferth fab Owain
\mwc{\acrshort{CBTII}~26,~l.~75}{Ac o blei\al{d D}ouyt \al{d}iheuuart wyf,}{And by the side of the Lord I am a true poet,}% d gaffwyf-i barch cyn nis archwyf, /Ac o blaid QDofydd diheufardd wyf, /Ac ar nawdd Dewi y dihangwyf, /A
\mwc[gorucgwr]{\acrshort{CBTII}~26,~l.~124}{A Dewi a'e \al{g}oru\al{c, g}\w r bieifyt,}{And Dewi made it, the man who owns [it],}% e_"stawd Dydd Brawd dybydd; /A Dewi a'i gorug, Qgw_^r bieifydd, /Magna fab yn fyw a'i farw ddeuddydd; /A
\mwc[orucgwyrthuawr]{\acrshort{CBTII}~26,~l.~178}{A \al{g}\w yrtheu a oru\al{c g}\w yrthua\w r Dewi,}{And powerful Dewi performed miracles,}% dd, pob celfydd geilw Dewi. /A gwyrthau a orug Qgwyrthfawr Ddewi, /Bu obaith canwaith cyn no'i eni. /Dan
\mwc{\acrshort{CBTII}~26,~l.~255}{Dichones ra\al{c g}ormes \al{g}ormant greirieu}{He made splendid relics in the face of oppression}% u. /Dewi differwys ei eglwysau, /Dichones rhag Qgormes gormant greiriau /A ffynnawn Ddewi a'i ffynhonnau
\mwc[gymryddewi]{\acrshort{CBTII}~26,~l.~287}{Y gymry\al{d D}ewi \al{d}y gymrodet}{To receive Dewi through agreement}% , /I Frefi, ar Ddewi dda ei fuchedd, /I gymryd QDewi di gymrodedd /Yn bennaf, yn decaf o'r tey_"rnedd. /
\mwc{\acrshort{CBTII}~26,~l.~291}{Drwy eiriole\al{d D}ewi, a \al{D}uw a uet,}{Through mediation of Dewi, and God rules,}% /Cyfodwn, archwn arch ddiomedd /Drwy eirioled  QDewi, a Duw a fedd, /Gwaeanad gwenwlad gwedi maswedd, /D
\mwc{\acrshort{CBTIII}~3,~l.~19}{Ma\al{b B}rochuael \al{b}ronn hael ha\w l orned,}{The son of Brochfael, generous breast with his right of fear,}%g fawredd, /Maboliaeth arfoliaeth waredd, /Mab QBrochfael bron hael hawl ornedd, /Gorpu nef yn Eifionydd
\mwc{\acrshort{CBTIII}~3,~l.~76}{Yn oe\al{t d}ewr \al{d}eg ml\w yd ar hugeint,}{At the great age of thirty,}%t! /Pan fo pawb, pan fwyf heb henaint, /Yn oed Qdewr deng mlwydd ar hugaint, /Pan dde_^l Brawd rhag bron
\mwc{\acrshort{CBTIII}~3,~l.~241}{A'm rodwy \al{G}wledic \al{g}\w leityadon}{And may the King of kings give me}%dd Duw can fod yn wirion: /A'm rhoddwy Gwledig Qgwleidiaddon /Drefred gwlad wared worchorddion
\mwc[goruynawcgorprwy]{\acrshort{CBTIII}~5,~l.~114}{Eil Gwynn \al{G}oruyna\w \al{c, g}orprwy enwir!}{A second to Gwynn Gorfynnog, he must conquer wickedness!}%yll, gorddyfn-di gywir, /Ail Gwyn Gorfynnawg,  Qgorpwy enwir! /Manyled meinwen mal ydd iolir, /Mal ydd a
% \mwc{\acrshort{CBTIII}~7,~l.~23}{}{}%, /Llafn gwyar a ga_^r o gydwaith, /Llaw esgud Qdan ysgwyd galchfraith, /Llyw Powys, peues ddiobaith; /H
\mwc[daerawddarvu]{\acrshort{CBTIII}~7,~l.~30}{Can \al{d}aera\w \al{d, d}arvu gedymdeith!}{Since he came [to die], a friend perished!}%. /Can deryw, darfuam o'i laith, /Can daerawd, Qdarfu gydymdaith! /Oedd beirddgar barddglwm dilediaith,
\mwc{\acrshort{CBTIII}~10,~l.~66}{Gyrth yn \al{g}\w an ra\al{c g}\w aeduriw,}{Striking rough before a bloody wound,}%en am geinwiw - garthan, /Gyrth yn gwa_^n rhag Qgwaedfriw, /Hydr eu gwir o'r gw_^r ni ddiw, /Hil Gwriae
\mwc[erchwynawcgwledic]{\acrshort{CBTIII}~15,~ll.~13--14}{Mochnant diheuchwant erchwyna\w \al{c, --- g}wledic~| \al{G}wlad Urochfael Ysgithra\w c,}{Lord of Mochnant, whose defence is truly needed~| The realm of Brochfael the Tusked,}%ochnant. /Mochnant ddiheuchwant erchwynnawg, - Qgwledig /Gwlad Frochfael Ysgithrawg, /Dyfnfedd a orchudd
\mwc{\acrshort{CBTIII}~16,~l.~217}{Ysgryd \al{g}ryd ra\al{c g}reid Eborthun,}{A frightening shout before the zeal of Eborthun,}%, /Cledr cedyrn, cad eiddun, /Ysgryd gryd rhag Qgraid Eborthun, /Ysgrud wlydd ar wledd ym Melltun. /Ys g
% \mwc{\acrshort{CBTIII}~16,~l.~192}{}{}%/Dyglud glod mal y clyw llawer, /Dychyrch cad, Qdyran rhad rhif se_^r! /Dychymell Prydain o'r pryder - y
\mwc{\acrshort{CBTIII}~20,~l.~13}{Gwry\al{d d}iogel \al{d}iogan --- fysgyad}{A brave, safe, faultless --- rusher}%wyneb, /Ar wy_^r wawr, w_'rach no neb. /Gwryd  Qdiogel diogan - ffysgiad /Yn ffysgiaw biw garthan, /Aerf
\mwc[gwledicgwlad]{\acrshort{CBTIII}~21,~l.~173}{Ym buchet \al{g}\w ledi\al{c g}\w lad orchorton,}{During the life of a ruler of a country's host,}%w aches, buches beirddion, /Ym muchedd gwledig Qgwlad orchorddon, /Gorddyfnws uddudd budd a berthon, /Go
\mwc[gwledicgwladoet]{\acrshort{CBTIII}~24,~l.~109}{Ym buchet \al{g}\w ledi\al{c, g}\w ladoet berchenn,}{In the life of a king, a posessor of lands.}%m-ni feddw fedd y Drefwen /Ym muchedd gwledig, Qgwladoedd berchen, /Madawg mur cyhoedd, niferoedd nen, /
\mwc{\acrshort{CBTIII}~26,~l.~10}{Yg gorwyt \al{g}lasure ra\al{c g}lasuer:}{On the slope of a green hill before a steel spear:}%s ei hoedl ei hyder /Yng ngorwydd glasfre rhag Qglasfer: /Yn nhrymglais, yn nhrais, yn nhrymder, /Yn nhr
\mwc{\acrshort{CBTIII}~29,~l.~5}{Llas g\w as g\w a\w \al{d d}igart, \al{d}igabyl y gwynaw,}{A servant of splendid poetry was killed, it is faultless to mourn him,}%/I dan llasar glas llas llew. /Llas gwas gwawd Qdigardd, digabl ei gwynaw, /Od is llaw llys Bennardd, /B
\mwc{\acrshort{CBTIV}~4,~l.~286}{O diffryd \al{G}wyndyd ra\al{c g}wander,}{From defending the Venedotians from weakness,}% o ddeddf, o ddyfnder, /O ddiffryd Gwyndyd rhag Qgwander, /O warthrudd wrthyn, Gwyn Gwarther, /O dyrru ar
\mwc[maeystawddeheu]{\acrshort{CBTIV}~9,~ll.~37--8}{Kedwis kygha\w s ma\w s maeysta\w \al{d --- D}eheu,~| \al{D}uw o nef a'e gwara\w d,}{The joyful pleader saved majesty --- of the South,~| God from heaven saved him,}% ai a ddywawd. /Cedwis cyngaws maws maeystawd - QDeau, /Duw o'r nef a'i gwarawd, /Gw_^r gwrthaw gwrthrych
% \mwc{\acrshort{CBTIV}~15,~l.~1}{}{}%                                Cyn ni bai amod Qdyfod - i'm herbyn /A Duw gwyn yn gwybod, /Oedd iawnach
\mwc{\acrshort{CBTIV}~16,~l.~153}{Berni\al{t D}u\w\ an \al{d}\w yn y wennbleit}{Let God judge our bringing to blessed company}% iaid; /Pan farner tri nifer trwy naid, /Bernid QDuw ein dwyn i wenblaid /Y nefoedd yn nefod ynaid. /Nid
\mwc[gorllwcgwr]{\acrshort{CBTIV}~17,~l.~29}{G\w tam y'n \al{g}orll\w \al{c G}\w r llwyrdd\w c lla\w r,\footnote{Variant reading for \mw{G\w r llwyrdd\w c}: \mw{g\w rll \w yd\w c}.}}{We know that the floor-sustaining man will expect us,}% in periglaw heb beriglawr; /Gwddam y'n gorllwg QGw_^r llwyrddwg llawr, /Gw_^r hydrfau angau yng nghynghe
\mwc[keluytyeiddaear]{\acrshort{CBTV}~6,~ll.~37--8}{Pan oruc Keli keluytyei\al{d --- d}aear,~| \al{D}ynyadon edeifnyeid,}{When God made the experts --- of the world,~| Leaders of men,}%  aw gwrthrychiaid. /Pan orug Celi celfyddiaid - Qdaear, /Dyniaddon edeifniaid, /Ef gorau un gwron ym mhla
\mwc[dyuodybid]{\acrshort{CBTV}~8,~l.~24}{Cabyl neu glod o'm \al{d}yuo\al{d d}ybid.}{Slander or praise may come from my coming.}%  de ddewis o'r rhydid, /Cabl neu glod o'm dyfod Qdybid. /Dy-m-doddyw edliw ac edlid /Am haelon haelder gy
\mwc{\acrshort{CBTV}~8,~l.~37}{Neud llei \al{d}youod, neu\al{d d}youid --- ked,}{A gift is less, a present is bad,}%  dr fy llawrgerdd odid. /Neud llai dyofod, neud Qdyofid - ced, /Cadwgawn rhyddelid, /Mab Llywarch, ddihaf
% \mwc{\acrshort{CBTV}~14,~l.~19}{}{}%  annwg yn eu harfle. /Llawer uchenaid i'm rhaid Qdyre, /Llwybrant o'm nwyfiant uch no'r nwyfre. /Neur arw
% \mwc{\acrshort{CBTV}~15,~l.~3}{}{}%  /Credwn i hwn fal y credwn i Ionas. /Dur ynad  Qdeddf rad rhyswynas - Dofydd, /Dof wyf it yn wanas. /Dyw
% \mwc{\acrshort{CBTV}~18,~l.~3}{}{}%  /Deddf ddyfnddawn cyfiawn, cyweithas, /Dedfryd Qdydd gwynfyd gwyn wanas - dragon, /Draig Prydain a'i hur
% \mwc{\acrshort{CBTV}~14,~l.~8}{}{}%  ei belre, /Heb dymyr Tudyr, tud Elise, /Enwawg QGyfeiliawg, ys cof aele. /Heb ferch Gynfelyn, gelyn gne
\mwc{\acrshort{CBTV}~14,~l.~22}{\al{G}wis\al{c g}wyndeil gwyeil gwet adarre,\footnote{Variant reading: \mw{g\w isc wieil gwynnyeil g\w yd adarre,}}}{Clothing of bright-leaved twigs like a flock of birds,}%  fre. /Neur arwedd dyfredd yn eu dyfrlle /Gwisg Qgwynddail gwyail gwedd adarre, /Neud adnau cogau, coed n
\mwc[madawggwrddfeirch]{\acrshort{CBTV}~17,~l.~15}{\al{G}wrfab Madaw\al{g  g}wrddfeirch arab}{Manly lad of Madog, brave fine horse}%  - cad, /Cadr ddragon gyfundab, /Gwrfab Madawg  Qgwrddfeirch arab, /Gwryd Pyr, byr y bu fab. /Mabolaeth f
\mwc{\acrshort{CBTV}~23,~l.~25}{\al{D}yuryd ynn, ueirt byd, bo\al{d d}aear --- arna\w\footnote{Variant reading for \mw{Dyuryd}: \mw{dybryt}.}}{Sad for us, bards of the world, that there is earth --- on him}%  n ufel drwy fa_^r. /Dyfryd in, feirdd byd, bod Qdaear - arnaw /Ac arnan ei alar. /Ef ein llyw cyn llid g
% \end{mwl}
% This example is doubtful, as the line forms a full \mw{cynghanedd sain} with \mw{Dyfryd, byd, daear}, and the  alliteration between \mw{Dyfryd} and \mw{daear} may be merely accidental.
% \begin{mwl}
\mwc{\acrshort{CBTV}~25,~l.~43}{\al{D}ywa\w \al{d D}erwyton dadeni haelon}{Bards told about the new birth of the generous}%  fod breienin /O Gymru werin o gamhwri; /Dywawd Qderwyddon dadeni haelon /O hil eryron o Eryri. /O wyron
\mwc[gloddy]{\acrshort{CBTV}~26,~ll.~121--22}{Dyued rwyf, dy glwyf, dy glo\al{d, --- d}y gynnygyn,~| \al{D}y gynnif ys hynod;}{Dyfed's lord, your plague, your honour, --- your adversary,~| Your fighting is impressive;}%  ydd Ddyfed. /Dyfed rwyf, dy glwyf, dy glod, -  Qdy gynnygn, /Dy gynnif ys hynod; /Dy gleddyf rhyglywsam
\mwc{\acrshort{CBTV}~26,~l.~139}{Cannhya\al{d D}uw, ys \al{d}iheu y uod}{Permission of God, it is certain to exist}%  fod - ar bawb, /Ar bobloedd anghydfod. /Caniad QDuw, ys diau ei fod /I'th gannerth cyn darmerth darfod.
\mwc[debycglut]{\acrshort{CBTVI}~4,~l.~21}{\al{G}wladoed ny deby\al{c (g}lut etli\w\ --- ym byt)\footnote{Three variant readings: \mw{Gwlat oedd nitebic glvd edliw ymbryd} (\gls{nlw} 17114B); \mw{Gwlad oedd nitebic glydedliw ym bryd} (\gls{nlw} Peniarth 240B); \mw{Gwlat oedd nitebic glyd edliw ym byd.} (\gls{nlw} 2021B)}}{The lands they do not suppose (a sticky taunt --- in the world)}% edd arlleng pymthengwlad. /Gwladoedd ni debyg (Qglud edliw - ym myd) /Madawg oedd cyn heddiw, /Esgor o'i
% \mwcc{\acrshort{CBTVI}~14,~l.~16}{}{}% rch osethrid - borthfor, /Diachor ysgor ysgwyd Qdurnid, /Cadr wychawg farchawg, mynawg mwynlid, /Cadelli
\mwc{\acrshort{CBTVI}~15,~l.~40}{Ne\w\ dar\w o\al{t d}awn \al{D}uw yn y diwed,}{Or the end of God's grace in the end,}% onid trech celwydd no gwirionedd /Neu ddarfod  Qdawn Duw yn y diwedd, /Ys mi a feflawr o'r gyngheusedd:
\mwc{\acrshort{CBTVI}~18,~l.~33}{Prydua\w r Lywelyn pry\al{d d}yn \al{d}adyein,}{Splendid Llywelyn, the appearance of a splendid man,}% rhyfyr prifwyr Prydain, /Prydfawr Lywelyn pryd Qdyn dadiain, /Prydus ddiesgus esgar ddilain, /Esgynnu ar
% \mwcc{\acrshort{CBTVI}~19,~l.~26}{}{}% wa_^r, ca_^r cerddorion, /Llawhir, llary mynud Qdraig mwyndud Mo_^n, /Gorau un cyfaillt o'r cyfeillon -
\mwc{\acrshort{CBTVI}~24,~l.~19}{Mab Du\w\ \al{d}ewissei\al{t d}ewissa\w\ --- a wnaf,}{Son of God, the choice's choosing --- I will do,}% A fo mynws byd yn ei bydiaw. /Mab Duw dewisaid Qdewisaw - a wnaf, /Ef dewisaf Naf, nawdd i ganthaw! /Di
\mwc[waisggosgedd]{\acrshort{CBTVI}~25,~l.~19}{Oedd breis\al{g w}ais\al{g g}osgedd ei fyddin,\footnote{In one variant reading, \mw{gosgedd} is left out.}}{Large and splendid was the appearance of his army,}% egr, /Yn llygru Swydd Erbin /Oedd braisg waisg Qgosgedd ei fyddin, /Oedd brwysg rhwysg rhag y godorin, /
% \end{mwl}
% This is a doubtful example, because \mw{gosgedd} only alliterates with another word if \mw{waisg} takes its preceding \mw{-g} as an initial consonant.
% % \mwcc{\acrshort{CBTVI}~27,~l.~63}{}{}% arw i gyd. /Gorofn ysy arnaf hyd na'm gweryd - Qdyn, /Wedi Llywelyn, llywiawdr Gwyndyd. /Gwarafun ni wna
% \begin{mwl}
\mwc[degcyfreithiau]{\acrshort{CBTVI}~29,~l.~64}{Gwedi llaryda\al{d, D}uw \al{d}eg cyfreithiau,}{After a generous father, the God of the just commandments,}% ddef pum archoll o archollau. /Gwedi llarydad, QDuw deg gyfreithiau, /Llofrudd-dab Gruffudd, draul budd
% Here, \mw{cyfreithiau} is lenited following preposed \mw{deg}, if it is taken to mean `fair'. Alternatively, however, one could consider \mw{deg} to mean `10', in which lenition needs not be read here. The latter option would invalidate it as a counterexample. 
\mwc[gwledicgwychyr]{\acrshort{CBTVII}~1,~l.~5}{\al{G}wledi\al{c g}wychyr, hydric, gwychna\w s --- yw Ywein,}{A fierce lord, persistent, brave --- is Owain,}%foledd y Berfeddwlad. /Gwledig gwychr, hydrig, Qgwychnaws - yw Owain, /Cledd lith brain, clod liaws, /O
% \end{mwl}
% Here, \mw{Gwledic} and \mw{-c gwychyr} may be thought of as alliterating, but alternatively this is accidental, and \mw{gwychyr} only alliterates with \mw{gwychna\w s}. In the latter case, we would be dealing with alliteration of two \gls{doubconclus}s, which would constitute a research exception.
% % \mwcc{\acrshort{CBTVII}~13,~l.~24}{}{}%lew Maredudd, oedd gloywner - esgud /Yn ysgwyd Qdau hanner. /Oedd dau hanner be_^r, ba_^r gwythrudd - yn
% % \mwcc{\acrshort{CBTVII}~20,~l.~5}{}{}%awr hoeddl ddirfawr ddarfod. /Deryw Blegywryd, Qdeurudd arwydd - hoyw, /Neu'm doeth hoed o'i dramgwydd,
% \begin{mwl}
\mwc{\acrshort{CBTVII}~20,~l.~20}{G\w r coeth, g\w ery\al{t d}oeth \al{d}ethol.}{A pure man, chosen wise soil.}%wr cadarn yn cadwyd o_^l, /Gw_^r coeth, gweryd Qdoeth dethol. /Neud dethol Balch nef, neud aeth - Blegyw
% \mwcc{\acrshort{CBTVII}~20,~l.~21}{}{}%o_^l, /Gw_^r coeth, gweryd doeth dethol. /Neud Qdethol Balch nef, neud aeth - Blegywryd /O'i gywraint wa
% \mwcc{\acrshort{CBTVII}~25,~l.~11}{}{}%yth holed, /A thithau yn gwychaw pan goched -  Qdy ba_^r, /Ni cha_^i dy esgar esgor lludded! /Ciliaw ni
% \mwcc{\acrshort{CBTVII}~32,~l.~1}{}{}%                 Mab a'n rhodded, Mab mad aned Qdan ei freinau, /Mab gogoned, Mab i'n gwared, y Mab gora
\mwc{\acrshort{CBTVII}~56,~ll.~21--2}{Cad gannerth mawrwerth, mor wae --- y'm \al{d}igywyd!~| Ny'm ga\al{d d}eigyr y warae;}{Valuable battle assistance, great woe --- that befell me!~| Tears do not permit me to lie;}%th mawrwerth, mor wae - y'm digiwyd! /Ni'm gad Qdeigr i warae; /Gryd esbyd oesbell warchae: /Gruffudd ym
\end{mwl}
Examples~\ref{kymrawtdreic},   \ref{daddaeoni},  \ref{gwerennicgurhid},  \ref{gymryddy2}, \ref{goludawcgolychwn},  \ref{dirnaddedwyt},  \ref{gyrchyatdeivyr},  \ref{decuetdaw},  \ref{gorucgwr},  \ref{orucgwyrthuawr},  \ref{goruynawcgorprwy},  \ref{daerawddarvu},  \ref{erchwynawcgwledic},  \ref{gwledicgwlad},  \ref{gwledicgwladoet},  \ref{maeystawddeheu},  \ref{keluytyeiddaear},  \ref{madawggwrddfeirch}, \ref{gloddy}, and \ref{degcyfreithiau} are all doubtful on account of a c\ae sura breaking up the doubling, making it optional. Additionally, \ref{daddaeoni}, \ref{geddy}, \ref{gymryddewi}, \ref{waisggosgedd}, and \ref{gwledicgwychyr} are dubious examples because it is not completely evident whether the alliteration as marked is intentional. This means that a total of twenty-four of the examples found above is dubious.

\subsection{Non-doubling of lenited \textit{T}}
The examples given here comprise twelve instances of a non-doubling {\lT} alliterating with another {\lT}. In ten of those twelve instances, lenition is represented orthographically, and only Examples \ref{aruoniccaer} and \ref{drindawttraethu} fail to represent lenition orthographically. This ratio contrasts starkly with the examples attesting to \gls{D}+\gls{D}=\gls{T}, because lenition is never represented orthographically in those cases (see Table \ref{reasonlenitionexddt}).
\begin{mwl}
\mwc{\acrshort{CBTI}~8,~l.~64}{Ar barabyl perwa\w d, ar \al{d}raethaw\al{d d}ras,}{Through a speech of pure song, through a declaration about lineage,}%rawr ffosawd, /Ar barabl perwawd, ar draethawd dras, /A'm cyhudd am udd iawn esillydd Idwal /Nid e_^l h
\mwc[aruoniccaer]{\acrshort{CBTI}~9,~l.~140}{Yn Aruoni\al{c c}aer \al{g}er Hiryell beu.}{In the castle in Arfon near the land of Hiriell}%  wallofied ei win o'i wen addaf udd /Yn Arfonig Qgaer ger Hiriell bau. /Derllesid i'm llaw llad i'm godda
\mwc[undawddrindawd]{\acrshort{CBTI}~14,~l.~54}{Yn unda\w \al{d D}rinda\w d yn \al{d}ragywyt.}{In unity the Trinity is everlasting.}%: /Credwn eu cydfod, ac nid adfydd, /Yn undawd Drindawd yn dragywydd. /Tragywyddawl Dduw tra gynanwyf 
\mwc{\acrshort{CBTI}~16,~l.~6}{Keinwa\w t o'm taua\w t ar \al{d}raetha\w \al{t d}r\w n}{Fine poetry from my tongue in splendid song}%raint canon, /Ceinwawd o'm tafawd ar draethawd drw_^n. /Llathraid fy marddair wedi Myrddin, /Llethrid a
\mwc[drindawttraethu]{\acrshort{CBTI}~21,~l.~27}{Prydesta\w t o'r \al{D}rinda\w \al{t t}raethu,\footnote{Variant reading for \mw{Drinda\w t}: \mw{trindawd}.}}{a poem of the Trinity's adress,}%th o brifiaith brydu; /Prydestawd o'r Drindawd draethu, /Traeth folawd teilyngdawd dalu. /Cymhennaf yw
\mwc{\acrshort{CBTII}~23,~l.~15}{Oer, \al{g}ywas\al{g g}ywisg pridd a main}{Cold, compressed and thin covering of earth}% in, /Bychan budd i fedydd a fain, /Oer, gywasg Qgywisg pridd a main /Llandutglyd am glod orwyrain, /Rhwy
\mwc[gaduarchawcgeduerchyr]{\acrshort{CBTIII}~12,~l.~41}{Ef oet \al{g}aduarcha\w \al{c g}eduerchyr --- a'm clwyf,\footnote{Variant reading for \mw{geduerchyr}: \mw{get verchy}}}{He was a warhorse, a grace-giver --- and my grief,}%Ni tholir gwawd gwedi ef. /Ef oedd gadfarchawg Qgedferchyr - a'm clwyf, /Cleddyfrudd nid etgyr, /Gweilch
\mwc{\acrshort{CBTIV}~6,~l.~202}{Keinuyged am \al{d}refre\al{d d}ryfrwyd}{As praiseworthy about an adorned abode}% Yn cyfladd ym mhlymnwyd; /Ceinfyged am drefred Qdryfrwyd /Cerdd gan gyrdd am gylch ei aelwyd. /Ceffid eu
\mwc{\acrshort{CBTVI}~14,~l.~49}{Wedy ca\al{t d}ramawr a gawr \al{d}remit}{After a very big battle and a shout of battle}% g, /Llawer march a gwisg a wesgerid. /Wedi cad Qdramawr a gawr dremid /A chyfarfod pybl pen yn erfid /Rh
\mwc[drindoddrwy1]{\acrshort{CBTVII}~40,~l.~41}{I foli'r \al{D}rindo\al{d d}rwy gardodau,}{To praise the trinity through benevolence,}%n troso i bob trosedd gorau, /I foli'r Drindod Qdrwy gardodau, /I garu Mab Duw yn ei wiw gaerau, /I gadw
\mwc[drindoddrwy2]{\acrshort{CBTVII}~40,~l.~113}{Archwn i'r \al{D}rindo\al{d  d}rwy wir ufudd-dod}{I pray to the Trinity through true humility}%derbyniaid i'n heneidiau. /Archwn i'r Drindod  Qdrwy wir ufudd-dod /Roi in lwyr gymod o'n pechodau, /A
\mwc[gadawcgadarnn]{\acrshort{CBTVII}~44,~l.~24}{\al{G}ada\w \al{c, g}adarnn ureinhya\w c uro.}{Gadawg, a powerful nobleman of the land.}%ro! /Llawr gwerthfawr Llan a'i gortho /Gadawg, Qgadarn freiniawg fro. /Ym mro Dewdwr gw_^r gwrdd yn arga
% \mwc{\acrshort{CBTVII}~51,~l.~21}{}{}%goch gatgun eiddun oedd. /Oedd rhaid ystyriaid Qdarfod cyffro - mawr /Am ymherodr Cymro; /Ys terfyn byd,
\end{mwl}
Of all examples given here, Examples~\ref{gaduarchawcgeduerchyr}, \ref{drindoddrwy1}, \ref{drindoddrwy2} and
\ref{gadawcgadarnn} should be considered doubtful on account of their c\ae sura, or because the alliterations indicated may be unintentional. This makes for a total of four dubious examples in this category.

\section{Alliteration of \textit{D}+\mw{h} and \textit{T}}
\label{dht}
Alliteration of a voiced stop followed by \mw{h} with a voiceless stop is also found in Welsh. In \gls{mow}, a voiced stop followed by \mw{h} may not correspond to a single voiced stop. If it is found in \gls{mw}, then aspiration is implicated as a key variable in distinguishing voiced from voiceless stops. This type of provection is similar to that discussed in section \ref{drhtr} in that it creates an \gls{aspconclus}. A \gls{mow} example of provection by aspiration is found in R.\ Williams-Parry's poem \mw{Hedd Wyn}, where \mw{tyner} alliterates with the final two consonants (or consonant clusters) of \mw{lleuad heno} in the line below: \mwcc[lleuadheno]{R.\ Williams-Parry, \mw{Hedd Wyn}, l.\ 9}{\al{T}yner yw'r lleua\al{d h}eno --- tros fawnog}{The moon is gentle tonight --- over the peat bog}

\subsection{Aspiration}
\begin{mwl}
\mwc{\acrshort{CBTIII}~3,~l.~75}{pan vo pa\w p, \al{p}an v\w yf he\al{b h}eneint,}{When all are, when I am without old age,}
\mwc{\acrshort{CBTIII}~15,~l.~28}{O'e golli, \al{t}ewi ny\al{d h}a\w t}{From losing it, it is not easy to keep silent}
\mwc{\acrshort{CBTIV}~9,~l.~173}{\al{T}i hebof, ny\al{d h}ebu oet teu,}{You without me, speaking was not yours,}
\mwc[gelwithwnn]{\acrshort{CBTVI}~14,~ll.~13--14}{Ny'ch traethaf-i gelwyd, nyt ef gelwi\al{t --- h}wnn~| Cledyf ysgwnn \al{t}wnn, twyll y gwndit;}{I will not tell you a lie, he was not called --- this~| A ready broken sword, deceitful his song;}
\mwc{\acrshort{CBTVII}~26,~l.~25}{Rwyl Carre\al{c}oua! \al{C}auas ynda\w\footnote{\mw{Carrecoua} is \mw{Carreg Hofa} in  \gls{mow} orthography. This example is particularly notable since \mw{-g Hofa} is part of a proper noun, that is, a place name, and was therefore probably thought of as a single word with a medial \mw{-c-} in the \gls{aspconclus}. This is further corroborated by its spelling as a single word \mw{Carrecoua} here.}}{The hall of Carreg Hofa! He received in it}
\mwc{\acrshort{CBTVII}~40,~l.~33}{A byd \al{t}eg a bywy\al{d h}ir}{And a fair world and a long life}
\mwc[cheissethocket]{\acrshort{CBTVII}~42,~l.~39}{Na cheisse\al{t h}ocket, \al{t}r\w y gam guhudet,}{He must not attempt deceit, through false accusation,}
\end{mwl}
% Alliteration between \mw{-t hocket} and \mw{trwy} is not confirmed by any other consonants here. Alternatively, the only ornament in this line is a \mw{cynghanedd sain} out of order comprising \mw{hocket, gam, guhudet}.
% \begin{verbatim}
% CBM  III 15  28  ed gw_^r eurged aergawdd, /O'i golli, tewi nid $hawdd. /Nid hawdd im wybod, neud anobaith - cyrdd, /Cer
% CBM  IV  9  173  a_^r gwrdd, gorddyfn ei faddau! /Ti hebof, nid $hebu oedd tau, /Mi hebod, ni hebaf finnau. /Hir y'th ard
% CBM  III 3   75   wynt fy ngheraint! /Pan fo pawb, pan fwyf heb $henaint, /Yn oed dewr deng mlwydd ar hugaint, /Pan dde_^
% GYC  VII 40  33   /Duw i holl ddynion daear, /A byd teg a bywyd $hir /A gawn o charwn y Gw_^r. /Y Gw_^r a rannawdd inni r
% GYC  VII 42  39  /Llewenydd llonydd ffydd a ffynna. /Na cheised $hoced, trwy gam guhudded, /Nef yn ei theced, can ni thyc
% LlG  VII 26  25  _^r dros fylchau yn falch arnaw. /Rhwyl Carreg $Hofa! Cafas ynddaw /Rhawd Saeson lladron i eu lludwaw! /
% PhB  VI  14  13  id. /Ni'ch traethaf-i gelwydd, nid ef gelwid - $hwn /Cleddyf ysgwn twn, twyll ei gwndid; /Diseirch meir
Examples~\ref{gelwithwnn} and \ref{cheissethocket} are doubtful here. In the first, a c\ae sura precedes \mw{\mbox{-t} --- hwnn}, which  alliterates with \mw{twnn} on the next line. In the second, it is unclear whether alliteration between \mw{\mbox{-t} hocket} and \mw{trwy} intentional, because it is not confirmed by any other consonants. This makes for two doubtful instances.
% \end{verbatim}
\subsection{Non-aspiration}
% \mwcc{\acrshort{CBTI}~1,~l.~32}{Ny dav metic hid orphen bid hid y nottvy.}{}
% \mwcc{\acrshort{CBTI}~3,~l.~92}{N'at y eneid hael yn waeleta\w c}{} Is this example actually a counterexample?
% \mwcc{\acrshort{CBTI}~3,~l.~108}{Pan gedwis y wyneb heb gewilyt}{}
% \mwcc{\acrshort{CBTI}~7,~l.~64}{Gloed grut hyged hygar.}{}
% \mwcc{\acrshort{CBTI}~9,~l.~118}{Ada\w d y'm g\w rthra\w d g\w rthred hotyaw.}{}
% \mwcc{\acrshort{CBTI}~11,~l.~34}{Gollewin wledic, wlad teithia\w c --- hael,}{}
% This one depends on whether the c\ae sura between the main line and the \mw{gair cyrch} breaks up consonant clusters.
% \mwcc{\acrshort{CBTI}~16,~l.~30}{Neut amd\w ll awyr, neut h\w yr hinon.}{}
% \mwcc{\acrshort{CBTII}~3,~l.~11}{Diryeit ny ha\w d beit hed\w ch;}{}
% \mwcc{\acrshort{CBTII}~3,~l.~65}{Llochyat gwlat hygat, byd hygar --- wrthyf;}{} This involves word-final alliteration (rhyme). How does that factor play into here?
% \mwcc{\acrshort{CBTII}~4,~l.~19}{Etlit hil g\w ythlit g\w aethlon:}{}
% \mwcc{\acrshort{CBTII}~26,~l.~216}{G\w rthebed hyged y hygleu --- gerta\w r}{} Note that this example has \mw{g\w rthebet} as a variant reading.
% \mwcc{\acrshort{CBTII}~31,~l.~24}{Hiruod heb gymmod yn y gamwet}{}
% \mwcc{\acrshort{CBTII}~31,~l.~35}{Hil Mada\w c hydyr uyna\w c uonhet,}{}
% \mwcc{\acrshort{CBTIII}~3,~l.~67}{Anheb y'r bleit a blyc heint,}{} What kind of cynghanedd is this anyway?
% \mwcc{\acrshort{CBTIII}~8,~l.~59}{Gwae yw y byd hyd y gwn}{}
% \mwcc{\acrshort{CBTIII}~13,~l.~59}{Ohona\w d handid uy gobeith,}{}
% \mwcc{\acrshort{CBTIII}~15,~l.~29}{Nyd ha\w t ym wybod, neud anobeith --- kyrt,}{}
% \mwcc{\acrshort{CBTIII}~27,~l.~8}{Hoed hydyr am hoetylwydyr wodrut.}{}
% \mwcc{\acrshort{CBTIV}~2,~l.~45}{}{}
% \mwcc{\acrshort{CBTIV}~4,~l.~120}{}{}
% \mwcc{\acrshort{CBTIV}~6,~l.~19}{}{}
\mwcc[torchawchael]{\acrshort{CBTIV}~6,~l.~90}{T\w r\w f tonn, torcha\w \al{c h}ael, tr\w m oet y \al{g}lywael,}{A loud wave, noble torque-wearing, the battle was intense,}
This counterexample depends on \mw{\mbox{-c} hael} alliterating with \mw{glywael}, but there is an \mw{l} breaking up the cynghanedd. This \mw{l} breaking up the cynghanedd may be a feature of the early cynghanedd, in which voiced non-stops could occasionally `intrude' on the cynghanedd\footnote{Cf.\ \mw{y bydd hallt gan y blaidd hen}, where the \mw{l} in \mw{blaidd} does not correspond to any consonant in the first half-line~\parencite[203--07]{jones_meddwl_2005}.}.
% \mwcc{\acrshort{CBTV}~1,~l.~149}{}{}
% \mwcc{\acrshort{CBTV}~1,~l.~2}{}{}
% \mwcc{\acrshort{CBTV}~2,~l.~39}{}{}
\mwcc{\acrshort{CBTV}~7,~l.~8}{A'e Their Ra\al{c}ynys, re\al{c h}offeint.}{And her three adjacent islands, a gift they love.}
This one may be irrelevant, as the \mw{c} in \mw{rec} is a final /ɡ/, rather than starting the next word with /k/. This makes /ɡ/  the more appropriate reading on the basis that the consonant correspondences that make the cynghanedd are all found in the word \mw{rec} (which shows consonance with \mw{Rac}) rather than \mw{g- hoffeint} (which does not share consonant phonemes with \mw{ynys}). This makes this consonant correspondence similar to the rhyme of word-final consonants found in the \mw{cynghanedd sain} than alliteration\footnote{Two similarly doubtful counterexamples are found in the \gls{cbt}, but are not included in full for this very reason: \mw{dragon, draig hirlwys} (\acrshort{CBTV}~17,~l.~37), and \mw{tud hwy, alltudion} (\acrshort{CBTVI}~31,~l.~41).}. 
% \mwcc{\acrshort{CBTV}~8,~l.~33}{}{}
% \mwcc{\acrshort{CBTV}~17,~l.~37}{}{}%This one may be relevan if V,7,8 turns out to be relevant
% \begin{mwl}
% \mwc{\acrshort{CBTV}~19,~l.~31}{}{}
% \mwc{\acrshort{CBTV}~24,~l.~64}{}{}
% \mwc{\acrshort{CBTV}~26,~l.~60}{}{}
% \mwc{\acrshort{CBTVI}~3,~l.~32}{}{}
% \mwc{\acrshort{CBTVI}~24,~l.~60}{}{}
% \mwc{\acrshort{CBTVI}~26,~l.~6}{}{}
% \mwc{\acrshort{CBTVI}~26,~l.~20}{}{}
% \mwc{\acrshort{CBTVI}~29,~l.~31}{}{}
% \end{mwl}
% \mwcc{\acrshort{CBTVI}~31,~l.~41}{}{} idem to V17,37 and V7,8
% \mwcc{\acrshort{CBTVI}~33,~l.~39}{}{}
% \mwcc{\acrshort{CBTVI}~35,~l.~40}{}{}
% \mwcc{\acrshort{CBTVII}~24,~l.~83}{}{}
% \mwcc{\acrshort{CBTVII}~25,~l.~75}{}{}
% \mwcc{\acrshort{CBTVII}~27,~l.~9}{}{}
\mwcc[ghyt-hoet]{\acrshort{CBTVII}~30,~l.~7}{Llifa\w d \w yg keu\al{d}awd \w y ghy\al{t-h}oet dyd:}{My day of co-longing sharpens my heart:}
Here, \mw{keudawd} and \mw{ghyt-hoet} show full consonant correspondence, except that medial \mw{d} in \mw{keudawd} corresponds with \mw{d} followed by \mw{h}. This counterexample is unique in the coalescence between a voiced stop and \mw{h} occurs word-medially, rather than around a word boundary. This has the obvious consequence that this coalescence follows internal sandhi rules rather than external sandhi rules\footnote{A three-way stop system distinguishing \xD\ from \lT\ is not thought to have existed word-internally, so the word-internal sandhi system is unlikely to have had the same effects on stops as the word-external one.}.  Assuming no \gls{cyfl} operated here, this counterexample indeed implies this, but this cannot be stated with certainty based on Example~\ref{ghyt-hoet} alone, and it challenges \textcite[\S 17b]{evans_grammar_1964}, which states that \mw{h} may cause word-internal provection in compound words. %\todo{Any work on internal/external sandhi? Is the BBCh relevant?}
% \mwcc{\acrshort{CBTVII}~7,~l.~15}{}{}
% \mwcc{\acrshort{CBTVII}~40,~l.~80}{}{}
% \mwcc{\acrshort{CBTVII}~41,~l.~9}{}{}
% \mwcc{\acrshort{CBTVII}~41,~l.~21}{}{}
% \mwcc{\acrshort{CBTVII}~49,~l.~5}{}{}

All in all, no dependable counterexamples remain to the proposition that \gls{D}+\mw{h}=\xT.
\section{Alliteration of \textit{D}+\mw{rh} and \textit{T}+\mw{r}}
\label{drhtr}
Alliteration between \gls{D}+\mw{rh} with \gls{T}\mw{r} implies that the key difference voiced and voiceless stops is the same property that distinguishes fortis \mw{rh} from lenis \mw{r}. Fortis \mw{rh} differs from lenis \mw{r} in that `it is completely voiceless and strongly aspirated'~\parencite[50]{jones_distinctive_1984}. Also, there is something to be said for viewing \mw{rh} as a co-articulation of /r/ and /h/, because `/\rh/ does not occur in those dialects of Welsh which lack /h/, but /r/ occurs where all +/h/ dialects have /\rh/'~\parencite[50]{jones_distinctive_1984}. Naturally, then, the existence of the type of alliteration discussed in this section implies that voice and aspiration were the primary variables distinguishing voiceless from voiced stops in the \gls{mw} period, similarly to the type of alliteration discussed in section \ref{dht}. A more modern example of this type of alliteration is found in the following line by Dafydd ap Gwilym, in which \mw{-d rhwydd} alliterates with \mw{trai}: \mwcc{Dafydd ap Gwilym, \mw{Pererindod Merch}, l.\ 15}{Crist Arglwydd, boe\al{d rh}wydd, bid \al{tr}ai,}{Lord Christ, let it be gentle, at low water,} 

In \gls{mw} orthography, the difference between \mw{rh} and \mw{r} is not written. In order to make the examples more understandable, I have therefore marked voiceless \mw{\rh}'s (\gmow{rh}) with a circle underneath to distinguish them from voiced \mw{r}'s (\gmow{r}) in this section. The information on the quality of the \mw{r}'s is taken from the editions in \gls{mow} orthography supplied in the \gls{cbt}.
\subsection{Aspiration}
\begin{mwl}
\mwc{\acrshort{CBTI}~16,~l.~33}{Kan wyt \al{Tr}i, ny\al{t \rh }eit itt amgen,}{Since you are Three, there is no need to be different,}%  pob defnydd, froydd Frenin; /Can wyd Tri, nid  Qrhaid it amgen, /Can wyd Dau (pell goddau) ac Un
\mwc[cadrac]{\acrshort{CBTIII}~10,~l.~60}{\al{Tar}yf \rh ac ca\al{d, \rh }ac K\w cwlluyrryaun.}{Excitement before a battle, before the tribe of Cwcwll Fyr.}%f ysgwn ys gnawd yng nghamawn, /Tarf rhag cad, Qrhag Cwcwllfyriawn. /Undegfed awen, undegfed - awydd /A
\mwc[ysgwydrac]{\acrshort{CBTIII}~16,~l.~147}{\al{Tr}eis ar y ysgwy\al{d \rh }ac ysgor --- Dinteirw,}{Violence on his shield before the stronghold --- of Dinteirw,}%fore yn rhagre rhag io_^r, /Trais ar ei ysgwyd Qrhag ysgor - Dinteirw, /A gwy_^r meirw rhag mur co_^r: 
\mwc[nydreit]{\acrshort{CBTV}~22,~l.~13}{Ny\al{d \rh }eit \al{tr}a dilyn pell ofyn pwy,\footnote{Manuscript  reading for \mw{reit}: \mw{treid}.}}{There is no need to go  far to ask who,}%  , /Milgi_"aidd eu gwy_^r ym mhob tramwy. /Nid  Qrhaid tra dilyn pell ofyn pwy, /Py geidw y gorddwfr rhag
% \mwcc{\acrshort{CBTV}~26,~ll.~117--18}{\Rh ys ua\al{b \Rh }ys, dilys dylyed --- \al{Pr}ydein,~| Prydytyon eituned;}{}%  , /Rhyn wryd, llyw rhyd, llawrodded, /Rhys fab QRhys, dilys dylyed - Prydain, /Prydyddion eidduned; /Rhy
% \mwcc{\acrshort{CBTV}~26,~ll.~129--30}{Dy ysgwyd \rh wygwy\al{d \rh }agod --- \al{tr}ychanweith~| Trychangwyth gyuaruod,}{}%  /Can wy_^r Lloegr ys llwyr adnabod. /Dy ysgwyd Qrhwygwyd rhagod - trychanwaith /Trychangwyth gyfarfod; /
% The two examples above do not technically have \mw{-d/b rh-} alliterating with \mw{t/pr} within their own cynghanedd, since the \mw{gair cyrch} only has the corresponding consonants. Nevertheless, both examples appear closely in succession and the \mw{gair cyrch} connecting two lines is not unheard of.
\mwc{\acrshort{CBTVI}~4,~l.~1}{Neu\al{t \rh }eit am Vada\w c \al{tr}engi --- kiwdodoed!}{Inescapable because of Madog [is] the perishing --- of cities!}% Neud Qrhaid am Fadawg trengi - ciwdodoedd! /Gwalch cadoedd, ca
\mwc[ryredreith]{\acrshort{CBTVI}~10,~l.~67}{Gwae ny gred \al{Tr}ined kyn \rh yre\al{d \rh }eith:}{Woe who does not believe in the Trinity before rushing to judgment:}% ab Mair maith. /Gwae ni gred Trined cyn rhyred Qrhaith: /Gnawd aelwyd ddiffydd a fo diffaith! /Pan fynnw
% \end{mwl}
% Doubtful example, because the line is a full \mw{cynghanedd sain} even without \mw{Trined} alliterating with \mw{-d \rh eith}.
% \begin{mwl}
\mwc{\acrshort{CBTVI}~24,~l.~12}{Na ma\w reir \al{tr}aha ny\al{t \rh }eit \w rtha\w\footnote{Variant reading for \mw{traha}: \mw{draha}.}}{He does not need to boast pride either}% byd, er bod iddaw - dda, /Na mawrair traha nid Qrhaid wrthaw: /Mab Duw a'i dyrydd yn eiddaw /A Mab Duw a
\mwc[kedrwyf]{\acrshort{CBTVII}~4,~l.~8}{\al{Tr}eul ke\al{d, \rh }wyf Dyfed, \rh ac dy ofyn.}{A spender of gifts, Lord of Dyfed, because of your fear.}%foroedd, fa_^r dwfn, /Traul ced, rhwyf Dyfed,  Qrhag d'ofn. /Diofnawg farchawg, feirch ffraeth, /Yw hwn,
% \end{mwl}
% Here \mw{Treul, -d \rh wyf}, and \mw{-d \rh ac} may be thought of as alliterating, but this alliteration is only secondary to the \mw{cynghanedd sain}.
% \begin{mwl}
\mwc[notedracdaw]{\acrshort{CBTVII}~27,~ll.~27--8}{Agheu ny'n edeu un note\al{d --- \rh }acda\w:~| \al{Tr}wy benn an treissya\w\ y\w\ an tristed.\footnote{Variant reading for \mw{Trwy}: \mw{drwy}.}}{Death does not leave us one refuge --- before him: ~| Our sadness is for us to be vanquished through our leader['s death]}%_^r gorau a aned. /Angau ni'n edau un nodded - Qrhagddaw: /Trwy ben ein treisiaw yw ein tristed. /Trist
%  \mw{-d \rh acda\w} alliterates with \mw{Trwy}, but also rhymes with \mw{treissya\w}.
\mwc[bytrac]{\acrshort{CBTVII}~51,~l.~9}{Gwae'r by\al{t \rh }ac \al{tr}istyt treisdwyn kythrudd --- llew:\footnote{Variant reading for \mw{tristyt}: \mw{trymfryd}; variant reading for \mw{treisdwyn}: \mw{traisgwydd}.}}{Woe the world for sadness of violence-bringing affliction --- lion:}%reulyd, /Llas Arthur, benadur byd. /Gwae'r byd Qrhag tristyd treistwyn gythrudd - llew: /Llywelyn ap Gru
\end{mwl}
Examples~\ref{cadrac}, \ref{ysgwydrac}, \ref{nydreit}, \ref{ryredreith}, \ref{kedrwyf}, and \ref{bytrac} are doubtful, because the alliterations marked may be unintentional. Example~\ref{notedracdaw} is doubtful because a c\ae sura separates the aspirating consonant from the \mw{\rh}. A total of seven examples are dubious.
\subsection{Non-aspiration}     
\begin{mwl}
\mwc{\acrshort{CBTI}~11,~l.~20}{A\al{c o r}odria\w \al{c \Rh }wyf lliwydoet\footnote{In \gls{mow} orthography, this looks like an example, of aspiration because \mw{Ac} `And' ends with a \mw{c}. However, this is only a orthographical clue to distinguish it from \mw{ag} `than' \parencite[\S 386]{morris-jones_cerdd_1925}. This has no bearing on the phonology of the word. This example does not show correspondence between consonants by rhyme but by cynghanedd-style alliteration, since \mw{Ac} and \mw{rodriawg} do not rhyme.}}{And from the royal sphere a Lord of troops}%  ysgwthr cedyrn y cadwennoedd, /Ac o rodri_"awg QRhwyf lliwedoedd /Rhodri rhad gymryd y byd fydoedd! /Bid
\mwc[habadrotyad]{\acrshort{CBTII}~1,~l.~57}{A'e haba\al{d r}otya\al{d \rh }ad rydyryt;}{And his abbot-giver is giving grace;}% hael orfawr arhawl orfydd; /A'i habad roddiad  Qrhad ryddyrydd; /Atan rhyddyran o'i lan luosydd, /Rhyddy
% This may be considered a counterexample, since \mw{dr} corresponds with \mw{d\rh}, but this is more telling about oppositions between two \mw{r}'s than it is about voiceless stops, so it is slightly irrelevant.
\mwc[ysgwydrwyd]{\acrshort{CBTIII}~16,~l.~221}{Ar ysgwy\al{d r}wy\al{d \rh }odwyt ual Run,}{Before an ensnarer of a shield like Rhun at a ford,}%frain, /Ysgafaeth i faidd cun /Ar ysgwyd rwyd  Qrhodwydd fal Rhun, /Ysgawl torf rhag trefred Alun, /Ysgo
% Like with example \ref{habadrotyad}, this line may be considered a counterexampple, but is more telling about the oppositions between different \mw{r}'s than the opposition between stops.
\mwc{\acrshort{CBTIII}~20,~l.~3}{Ha\w l wa\w l wasta\al{d r}einya\al{d \rh }eid,}{A giver of hardship, his claim is bright and perpetual,}%ilcant Maelgwn raid, /Hawl wawl wastad reiniad Qrhaid, /Hael wael wared ged gydnaid. /Cydnaid gynifiaid
\mwc[drychidrhac]{\acrshort{CBTIII}~21,~l.~85}{Eruyd a \al{dr}ychi\al{d \rh }ac y drachwres,}{A blow is struck before his fury,}%fn yr Aelwyd oedd aelaw tres. /Erfid a drychid Qrhag ei drachwres, /Arfau pendrychion cochion coches, /A
% \end{mwl}
% Doubftul, because \mw{rhag} is not required to make a full \mw{cynghanedd sain}.
% \begin{mwl}
\mwc{\acrshort{CBTIV}~4,~l.~233}{Euranre\al{c \rh }ede\al{c r}odolyon,}{Golden gift for wanderers who travel [to it],}% yddion, /Eurgolofn eurddiofn ddeon, /Euranrheg Qrhedeg rodolion, /Eurgor ddo_^r, ddinas cerddorion, /Eur
% Again, this example tells more about the opposition between \mw{r}'s than about the opposition between voiced and voiceless, similarly to examples \ref{habadrotyad} and \ref{ysgwydrwyd}.
\mwc{\acrshort{CBTIV}~4,~l.~253}{Yd wara\w \al{d \rh }eidgwna\w \al{d r}eidussyon;}{He delivered the needy poor;}% wyson, /Hyd Weryd, wryd orchorddon, /Yd warawd Qrheidnawd reidusion; /Hyd Gaer Gaint i gadw braint Bryth
% \mwcc{\acrshort{CBTIV}~6,~l.~256}{Gna\w d \rh oda\w c \rh ac marcha\w c midlan;}{}% s luman - archafad /Yn aergad i ar gan; /Gnawd Qrhodawg rhag marchawg midlan; /Gnawd cynheirf cynnwrf i
% Does \mw{Gnawd} alliterate with the first syllable of \mw{rhodawg} here? Otherwise, this line is irrelevant.
% \mwcc{\acrshort{CBTV}~20,~l.~3}{Yn gert}{}%  Lloegr, /Delis llew o Gymry, /Yn gerdd gydfod, Qrhod ryly, /Yn ged gyhydr, gad hydr hy. /Hy byddai Arthu
\mwc[nodetrhac]{\acrshort{CBTVI}~33,~ll.~59--60}{Kaffael ida\w\ nef o'e node\al{t --- \rh }ac dr\w c,~| \al{Dr}agywyd damunet.}{May he receive heaven as his refuge --- before evil,~| Eternal wish.}% 'i foddha_"ed; /Caffael iddaw nef o'i nodded - Qrhag drwg, /Dragywydd ddamuned. /Gwehenais-i ag ef, nid
% This one is only optionally a counterexample, because there is a c\ae sura between \mw{rhag} and the \mw{d} preceding it. 
\mwc[uabrhys]{\acrshort{CBTVII}~3,~l.~11}{Aeth \Rh ys ua\al{b \Rh }ys, \Rh os \al{br}eitya\w,}{Rhys, son of Rhys went, plundering Rhos,}%/Ethym heb udd llafnrudd, llew, /Aeth Rhys fab QRhys, Rhos breiddiaw, /I'w ddiwedd fanfedd Fynyw. /Ger M
\end{mwl}

% \mwcc{\acrshort{CBTVII}~24,~l.~3}{Kynytu canu, cany\al{d \rh }yuet --- \al{dr}eth}{}%rau doniau dinam fawredd, /Cynyddu canu, canyd Qrhyfedd - dreth /O draethawd gyfannedd /I foli fy rhi,
% Doubtful, because \mw{dreth} is a \mw{gair cyrch} to the line containing \mw{-d rhyfedd}.

Among all the cases of non-aspiration here, only Examples  \ref{drychidrhac}, \ref{nodetrhac}, and \ref{uabrhys} may potentially shed light on the phonetic properties of the differing \gls{mw} stop series. Other examples show a final voiced stop which once followed by a radical \mw{r}, and once followed by a lenited \mw{r}. However, opposition between voiced and voiceless stops word-finally is absent to very rare in Early \gls{mw}. The consequence of this lack of opposition is that it does matter which \mw{r} follows the stop, because there is no phonemic opposition to make.
Moreover, the three remaining examples are all dubious with regard to whether the alliteration is intentional or not. In Example \ref{drychidrhac}, \mw{drychid} may alliterate with \mw{drachwres} rather than \mw{-d \rh ac}. In Example  \ref{nodetrhac}, \mw{dr\w c} alternatively provides alliteration for \mw{Dragywyd}, although the vowel of  \mw{-d \rh ac}  also corresponds with \mw{Dragywyd}. In Example \ref{uabrhys}, a full \mw{cynghanedd sain} is found even when discounting correspondence between \mw{-b Rhys} and \mw{breiddiaw}. 
This leaves no straightforward counterexample to the proposition that \mw{-D \rh-} and \mw{Tr-} may alliterate.


\section{Results}
\label{excex}
Table \ref{examplescounterexamples} shows the amount of examples and counterexamples to each type of provection. The examples given in sections \ref{ddt}, \ref{dht}, and \ref{drhtr} each have some dubious ones in their midst. To account for this, the amount of the most dependable examples is given in parentheses. The results are discussed separately for \gls{doubconclus}s and \gls{aspconclus}s.

\begin{table}[h]
\centering
\begin{tabular}{@{}lrrrrrrrr@{}}
\toprule
& \multicolumn{2}{r}{\textbf{\gls{D}+\xD}} & \multicolumn{2}{c}{\textbf{\gls{D}+\lT}} & \multicolumn{2}{c}{\textbf{\gls{D}+\mw{h}}} & \multicolumn{2}{c}{\textbf{\gls{D}+\mw{rh}}} \\ \midrule
Alliterates with \gls{T}\mw{(r)} & 9 & (5)  & 16 & (14)  & 7 & (5)  & 10 & (3)  \\
Alliterates with \gls{D}\mw{(r)} & 76 & (52)  & 12 & (8)  & 3 & (0)  & 10 & (0)  \\ \bottomrule
\end{tabular}
\caption{Number of examples given in sections \ref{ddt}, \ref{dht}, and \ref{drhtr}, with the amount of dependable examples within parentheses.}
\label{examplescounterexamples}
\end{table}

\gls{D}+\mw{h} and \gls{D}+\mw{rh} alliterate with \mw{T(r)} comparatively more than \gls{D}+\gls{D}. Even though \gls{doubconclus}s are found, \gls{D}+\gls{D} is found comparatively often alliterating with a single \gls{D}. This implies that a doubled voiced stop did not sound similar to a single voiceless stop to the \gls{mw} ear. There are several examples showing that  an \mw{h} or \mw{rh} following a voiced stop made this sound equivalent to a voiceless stop, while no reliable counterexamples remain. This confirms the impression that aspiration played a role in distinguishing voiced and voiceless stops. 

The behaviour of \gls{doubcon}s must be considered separately for the \gls{D}+\xD-type and the \gls{D}+\lT-type. The former type is the only type where counterexamples outweigh examples. It seems proven, then, that this type of provection did not exist in the \gls{cbt}. Nevertheless, this leaves nine examples (five of which are dependable) that do attest to this type of provection. A possible mechanism that may account for this is the hypercorrect application of \gls{cyfl} in this period. An example containing a \gls{doubconclus} containing a \xD\ could be considered similar to Example~\ref{gwaneikynhor} on \pref{gwaneikynhor}. This finding raises another question: why is this type of provection considered valid in \gls{mow}?  And is this type of consonant doubling grounded in linguistic processes, or based on analogy with doubling \lT? There is reason to think that it is a learned feature of modern-day poets, and may not be grounded on speakers' perceptions of spoken Welsh~\autocite{jones_y_2015}.

\subsection{The doubling of {\lT}}
The doubling of \lT\ is a more complicated matter. For one, \lT\ may alliterate with \xT\ by \gls{cyfl} because they share an \gls{archphon}. This means that it cannot be stated with certainty whether a singular instance of \gls{D}+\lT=\gls{T} may be attributed to \gls{cyfl}, or because we are dealing with a \gls{doubconclus}. Similarly, counterexamples may be due to \gls{cyfl}, because each counterexample is in fact an instance of \gls{D}+\lT=\lT. It is nevertheless possible to infer what variables played a role in deciding alliteration on the basis of the relative frequency of \gls{doubcon}s and consonants which do not. 

If \gls{cyfl} were the \emph{only} factor deciding whether two consonants may alliterate, a \gls{doubconclus} ending in \lT\ it would be equally likely to alliterate with a single \lT\ or with \xT, so they would be found with  roughly equal frequency\footnote{One might argue that alliteration with \xT\ would be more likely, because radical words are more common than lenited words in a typical Welsh text. However, a typical half-line of poetry in the \gls{cbt} may be expected to show some degree of formulaic similarity to the other half-line. A \gls{doubconclus} containing \lT\ in one half-line is therefore likely to correspond with another \lT\ in the other half-line, because the half-lines themselves are likely to be similar. I shall assume that both factors cancel each other out.}.
By contrast, if pre-sandhi (that is, a doubled \lT\ would be equal in value to a regular \lT) phonological agreement were the only factor in alliteration between stops, then a \gls{doubcon} \lT\ would alliterate with \lT\ exclusively. We would thus find only counterexamples to provection by doubling. If a combination of \gls{cyfl} and pre-sandhi phonological alliteration exclusively played a role in deciding alliteration patterns, we would expect more counterexamples to doubling than examples.

None of these three scenarios may account for the distribution we find. What we find is a majority of examples showing \gls{D}+\lT=\xT, and a minority of counterexamples showing \gls{D}+\lT=\lT. We know for a fact that \gls{cyfl} still operated, but only rarely. The rarity of this phenomenon means that the first option, predicting a fifty-fifty distribution between examples and counterexamples, is not viable. The implication of all these considerations is that sandhi must have played a role in allowing \gls{D}+\lT\ to alliterate with \xT. All the other possibilities are exhausted by this point. 

\section{Conclusion}
Sandhi within the cynghanedd may bring insight into the Early \gls{mw} phonological system. I shall argue that they mark length and aspiration as phonemic. 

\subsection{Length}
\Gls{doubconclus}s are primarily found with lenited voiceless stops, but not with unlenited voiced stops. This implies that the pronunciation of \lT\ was such that increasing its length by doubling it was sufficient to produce the equivalent of \xT. This was not the case for \gls{D}+\xD. This situation may be accounted for under the following scenario: unlenited voiceless stops were long, while lenited voiceless stops were short. Unlenited voiced stops were also long. 

In this scenario, lengthening an unlenited voiced stop could only lengthen what is already long, giving the phoneme no properties that bring it closer to unlenited voiceless stops. Lengthening a lenited voiceless stop would add length, bringing it in line with a radical voiceless stop. This meant that length served to disambiguate radical from lenited consonants, so lengthening a lenited phoneme made it equivalent to its unlenited counterpart. 
\subsection{Aspiration}
If length were the only feature used to distinguish stop series, doubled lenited \textit{T} should also alliterate with radical \gls{D}, because they were both effectively long. Alliteration between radical \gls{D} and \gls{T} is not found, however. The fact that they do not alliterate shows that length by itself was not sufficient to distinguish stop series.  The fact that \gls{D}+\lT≠\xD\ implies that a variable beside length is found in one, but not both of these stop series. I shall argue that this variable was aspiration.

The examples attesting to \gls{D}+\mw{h}=\gls{T} and \gls{D}+\mw{rh}=\gls{T}\mw{r} confirm that aspiration was a feature of unlenited voiceless stops. Aspiration was an inherited feature of voiceless stops in general, so aspiration may or may not have existed in lenited voiceless stops in the immediate post-apocope period up to the \gls{cbt}. If aspiration was a feature of lenited voiceless stops, one would expect -\gls{D}+h=\lT, but not -\gls{D}+h=\xD. It is hard to confirm this on the basis of counterexamples to provection by aspiration in this work. Only Examples \ref{drychidrhac}, \ref{nodetrhac}, and \ref{uabrhys} corroborate this hypothesis, and only \ref{torchawchael} forms a counterexample. All other examples are invalid as they involve a non-initial consonant, where the opposition between lenited voiceless and radical voiced stops does not exist. 

If lenited voiceless stops were not aspirated, then aspiration distinguished radical voiceless stops from lenited voiceless stops along with length. The corollary of this position would be that length was the only variable keeping apart \lT\ and \xD, unless a third variable were posited. If this were the case, doubled \lT\ should alliterate with a single \xD. Since a doubled \lT\ does not alliterate with single \xD, while it does alliterate with \xT, longness was not only a necessary feature for this kind of transformation, but also a  sufficient feature. This implies that the feature of aspiration was indeed shared between \lT\ and \xT, but not with \xD.

\subsection{Implications}
The most obvious implication of this conclusion is that lenited voiceless stops and unlenited voiced stops had not yet merged by the time of the \gls{cbt}. Length served to distinguish radical from lenited stops, with radicals being long, and lenited stops being short. Aspiration served to distinguish phonemes under \gls{archphon}s /p t k/ from phonemes under \gls{archphon}s /b d g/, with the former series being aspirated, and the latter being unaspirated.

When doubled, lenited voiceless stops could be provected. This was not the case for unlenited voiced stops. This point may be key in understanding why the \gls{mow} cynghanedd has a general rule of provection allowing \gls{D}+\gls{D} to be equivalent to \gls{T}, irrespectively of whether this \gls{D} goes back to \lT, or \xD. Perhaps, this type of provection was patterned on examples with doubling \lT, and spread to \xD\ after these two series of phonemes had merged.