\chapter{Why \ei\ and \oes?}
\label{eioes}
The fact that limited postverbal lenition occurs, but only after these two endings means that one must account for what makes these two verbal forms so special. The short answer is that \ei\ and \oes\ most likely ended in a vowel before apocope. The etymologies and their development will be discussed in this chapter.
\section{The origin of \mw{-ei}}
This ending goes back to earlier \mw{-i}, as explained by Isaac: \tqt{Nimmt man an, daß die \"{a}ltere Endung der 3. Pers. Sing. des Impf \textit{-i} , z.B. \textit{keri, gocheli} usw. wirklich ererbt war
(lautgesetzlich < *\textit{-ii̯e-} < *\textit{-ei̯e-}: Iterativ wie laut Kury\l{}owicz, \textit{Infl. Cat.} 134--5 + sekund\"{a}re Endung \textit{-t}? Und vgl. abret. \textit{-i}, Fleuriot, \textit{Le Vieux Breton} (Paris, 1964), 304), dan wurde daraus in de oft vorkommenden Verben `gehen, kommen, machen' deren Wurzeln auf /γ/ auslauteten, im Altkymrischen *\textit{egi}, *\textit{degi}, *\textit{guregi} (<g> = /γ/). Nach dem Verschwinden des /γ/ wurden diese Formen regelm\"{a}ßig zu zweisilbigen \textit{ei} (in \textit{Canu Taliesin} V, Z.5 belegt), *\textit{dei}, und (umgebildet) *\textit{gwnei} und schließlich kontrahiert zu einsilbigen Formen.

Nun liegt die Vermutung nahe, daß zu diesen so kurzen (und deswegen vielleicht morphologisch unklaren) einsilbigen Formen der zugrundeliegende Vokal der Zurzeln wieder eingeschoben wurde, und sich so \textit{aei, daei, gwnaei}, die \"{u}blichen mkymrn. Formen ergaben. Von daher wurde offensichtlich eine neue Endung \textit{-ei} abgeleitet, die auf andere Verben \"{u}bertragen wurde.}{isaac_zwei_1994}{201-202}

In short, this moves the question. The etymology of \ei\ is earlier MW \mw{-i}, but where does \mw{-i} come from?

Rodway has the following to say about the third singular endings \mw{-ei}, \mw{-i}, and \mw{-iad}: 

\tqt{Three 3 sg. imperf. endings are attested in the corpora: \textit{-ei} (> Modern Welsh \textit{ai}, the only form that has survived until today), \textit{i} and \textit{-(i)ad} (see Evans, 1964, 121-22). The latter is cognate with the Old Irish 3 sg. imperf. ending \textit{-e/ad}, with *\textit{-to} added to the stem vowel (Lewis and Pedersen, 1961, 284; Evans, 1964, 122; Isaac, 1996, 377; Schrijver, 2011, 61; cf. McCone, 1994, 161). The Brittonic forms without /j/ could derive from Class IV *\textit{-a-to}. The forms with /j/ must be from *\textit{-je-to} (Schrijver's Class V), and in the light of well-attested yod-deletion in some Middle Welsh dialects, the Middle Welsh forms in \textit{-ad} could also have this derivation. \textit{-(i)ad} almost exclusively occurs with the verb \textit{bod} and its compounds in Middle Welsh (see, for instance, the examples listed in Lewis and Pedersen, 1961, 280; Evans, 1964, 122). That this was not always the case is shown by Old Breton \textit{bitat} `cut' and \textit{troeat} `turned, went around' (Fleuriot, 1964, 304; Schrijver, 2011, 61; cf. Koch, 1988, 36), and by the form \textit{pwyllatt} `thought' from a potentially historical Taliesin poem (Williams, 1968, II.26, note on p. 40; Koch, 1988, 36; Isaac, 1998a, 63). The only example included in the present survey is \textit{ny didoryat y Brytannyeit pa damwein y dygwydynt yndaw} `The Britons did not take heed of what accident they should fall into' (NLW 5266, 267; Lewis, 1942, 172). Henry Lewis explains that \textit{y Brytannyeit} is the object of the verb and \textit{pa damwein} the subject. This construction was later changed, so that the opposite would be true, cf. the Red Book of Hergest version of this sentence:\textit{ny didorynt py damwein y dygwydynt yndaw} (Lewis, 1942, 273-274). D. Simon Evans tentatively suggests emending \textit{didoryat} in NLW 5266 to \textit{didorynt} (Evans, 1964, 122), but this seems unwarranted. Note the verbal noun \textit{didorbot} (Lewis, 1942, 274), \textit{diddarbod} (GPC, 968). This verb is a compound of the defective verb \textit{doraf}, which has a verbal noun \textit{dorbod}, \textit{darbod} (GPC, 1076). Forms o this verb in which forms of the verb \textit{bod} are attached to the stem \textit{dor} are attested, e.g. 3 sg. future \textit{dorbi} (Jarman, 1982, 17.155, 17.214; cf. Schumacher, 2004, 270). Thus this verb, whatever its prehistory (see Schumacher, 2004, 267-70), had been brought into the orbit of the verb \textit{bod}, and the form \textit{didoryat} therefore conforms to the Middle Welsh pattern whereby \textit{-(i)ad} almost exclusively occurs with \textit{bod} and its compounds.

The origin of the other two 3 sg. imperf. endings is problematic. Eric Hamp posited \textit{-ei} < *\textit{-eset}, an old unstressed and enclitic form of the verb `to be' (Hamp, 1974), but this has been shown to be impossible on phonological grounds by Graham Isaac (1996, 373-74). The latter argues that the form \ei\ developed from a misanalysis of 3 sg. imperf. forms in \textit{-i} (of uncertain origin) of the irregular verbs \textit{myned} (stem *\textit{-ag}), \textit{dyfod} (stem *\textit{dag-} and \textit{gwneuthur} (stem *\textit{gurag-}), which would yield, following \textit{i}-affection and loss of lenited \textit{g}, \textit{e-i} `went', < Old Welsh *\textit{egi}; *\textit{de-i} `came'< */mw{degi}; *\textit{gwne-i} `made, did' < *\textit{guregi}, `into whih forms, for morphological clarity, the basic unaffected vowel of the stem was reintroduced, giving \textit{aei}, \textit{daei} (more usual \textit{doei}, \textit{gwnaei}, the M[iddle] W[elsh] forms' (Isaac, 1996, 376-77). These were then resegmented as \textit{a-ei}, \textit{da-ei}, \textit{gwna-ei} (Isaac, 1994, 201-2). This ending had begun to spread by the tenth century at least on the evidence of \textit{nacgenei} `that there was no need' from the Computus fragment (Falileyev, 2000, 2) -- cf. \textit{immis-line} `he besmeared himself' from the possibly ninth-century Martianus Capella glosses (Falileyev, 2000, 91), with wel-attested Old Welsh <e> for /ei/. Therefore wemight expect to find a high incidence of \textit{-i} in the early works being superseded by \ei\ in later ones. However, Peter Schrijver, while accepting Isaac's explanation as a possibility, further suggests that \ei\ could derive from Class IV present stems in \textit{-a-} + \textit{-i-}, while \textit{-i} always have been restricted in distribution. The statistics for the court poetry (taken from Rodway 2003b, 69) are as follows (the figures below the percentages represent the number of instances in each text and those in brackets represent forms confirmed by rhyme):

\begin{table}[H]
\centering
\label{my-label}
\begin{tabular}{lllll}
                          & \textbf{Group I} & \textbf{Group II} & \textbf{Group II} & \textbf{Group IV} \\
\multirow{2}{*}{\textit{-ei}} & 70\% (56\%)      & 93\% (87\%)       & 89\% (83\%)       & 100\% (100\%)     \\
                          & 21 (9)           & 103 (54)          & 17 (5)            & 20 (7)            \\
\multirow{2}{*}{\textit{-i}}  & 30\% (44\%)      & 7\% (13\%)        & 11\% (17\%)       & 0\% (0\%)         \\
                          & 9 (7)            & 8 (8)             & 2 (1)             &                  
\end{tabular}
\end{table}

It can be seen that \ei\ increases from 70\% in \textbf{Group I} to 100\% in \textbf{Group II}. If we compare the situation in undated poetry, we see that \textit{-i} is always in a minority in the sample which I examined (84\% \textit{-ei} in the Book of Aneirin; 79\% in the Book of Taliesin; 93\% in the saga \textit{englynion} --- Rodway, 2003b, 70). This sort of statistical analysis is only valid if we assume that `at one stage \textit{-i} was the only form taken by non-\textit{bod} verbs' (Rodway, 2003b, 70). This assumption is questionable, however, both in light of Schrijver's suggeston that the endings \ei\ and \textit{-i} could have arisen in different stem-classes, and in light of the evidence that \textit{-(i)ad} at one stage occurred with `non-\textit{bod} verbs.'

\textit{-i} is vanishingly rare in the prose. I have noted two examples of \textit{rody} `gave' in Llanstephan 1 and one of \textit{lunyeithi} in NLW 5266. In the latter case, the  scribe wrote \textit{lunyeith}, and <i> was added above the line (Lewis, 1942, 35, n. 3) It is quite plausible that all three examples are simply errors. Henry Lewis duly emends \textit{lunyeithi} to \textit{lunyeithei} (Lewis, 1942, 35). Simon Evans has only one example of \textit{-i} from a prose text, namely \textit{seui} `stood' in the \textit{Four Branches of the Mabinogi} (Williams, 1951, 92.9; Hughes, 2000, l. 579). The <e> in the stem here mitigates against restoring \textit{-ei,}, which would not have caused \textit{i}-affection, here. Thus we would expect *\textit{sau[e]i}. Thus this looks like a bona fide prose example of \textit{-i}. Note however that the Red Book of Hergest has regular 3 sg. pret. \textit{seuis} here, while the White Book of Rhydderch scribe initially wrote \textit{seuit}, with a \textit{punctum delens} subsequently added beneath the \textit{t} (Hughes, 2000, 75). While it is perfectly plausible that the Red Book copyist (or the copyist of his exemplar) replaced an obsolete form (i.e. \textit{seui} with a more familiar one, as envisaged by Ifor Willams (1951, 303), it is equally possible that the common exemplar of the White and Red Book versions contained \textit{seuis}, which was miscopied in the White Book as \textit{seuit}. The scribe, or a subsequent reader, noticed the error and deleted the \textit{t}, but omitted to replace it with \textit{s}.}{rodway_dating_2013}{65-66}

\section{The problem of \mw{oes}}

The verb \oes\ is a form of the verb \mw{bod} `to be'. It typically means `there is'. Schrijver connects the etymology of \oes\ to \mw{yssit} `there is' (not to be confused with \mw{yssyd} `which is'):
\tqt{There is strong evidence that the PIE connectors *\textit{de}, *\textit{k\textsuperscript{w}e} and *\textit{et(i)} played a major role in the syntax of Ir. (Watkins 1963; cf. VI(6)). Elsewhere I have discussed the British evidence for the main clause connector *\textit{et(i)} and \textit{k\textsuperscript{w}e} (1994:181-184):

MW \textit{nyt} `not' before a vowel < *\textit{ne-et(i)} (See VI.1 (7));

MW \textit{nyt} `is not' < *\textit{ne-et(i)-est(i)};

OW \textit{rit} < *\textit{ro-et(i)} (see VII.1.2 (2) above);

MW \textit{neut} before vowel <*\textit{nou̯(e)-et(i)}(see VII.1.2 (2) above);

OW \textit{immit cel} (MP) `he hides him(self)' < *\textit{ambi-et(i)-en kelet(i)} (see VI.1(7))

MW \textit{nac} `not' (in imperatives and answers), MB \textit{nac} `who not' < *\textit{ne-k} < *\textit{ne-k\textsuperscript{w}e}

The recognition that *\textit{et(i)} survives in MW in the form \textit{-t} and that this \textit{-t} obeys Wackernagel's law may help to explain two problematic MW verbal forms.

In MW, there are two forms expressing `there is'. One is \textit{yssit}, which occurs clause-initially (Evans 1976: 142): \textit{yssit ny cheffych} `there is that thou wilt not get' (WM 480.19). The other is \textit{oes}, which appears after a negation (\textit{nyt, nad}), the interrogative particle \textit{a}, and after \textit{ot} `if' (Morris Jones 1913: 350; Evans 1976: 144): \textit{Nit oes yndi nep ni'th adnappo} `there is no one in it that will not know thee' (PKM 4.3). This distribution strongly suggest that, in OIr. terms, \textit{yssit} is the old absolute and \textit{oes} the old conjunct form. If so, the formal difference between \textit{yssit} and \textit{nyt oes} must be attributed solely to the different position of a clitic (or clitics): after the verb in \textit{yssit}, and after the negation in \textit{nyt oes}. This is indeed what explains the attested forms. I suggest that the constituents of these forms are PCl. *\textit{esti} `is', *\textit{ed} (> *\textit{e}), the neuter pronoun, and the main clause particle *\textit{et} (< *\textit{eti} `and' by the early apocope of *\textit{I}; McCone 1978, Schrijver 1994: 159-65).

As to the syntax, *\textit{esti} + *\textit{e} means `it is to it, it has', cf. OIr. \textit{issid} `it was to him/it, he/it has' < *\textit{esti-de-e/en} (GOI 269; Evans (1976:142) already suggeted a similar syntagma, but he osited a non-eisting 3sgm. pronoun \textit{-it}). For the semantic development to `there is', cf. French \textit{il y a} `there is'. There is evidence for the meaning `has' in OW: cf. \textit{issit padiu itau gulat} gl. celsi thronus et cui regia caeli (Juv.) `he to whom (\textit{padiu}) is (\textit{itau}) the kingdom (\textit{gulat}, sc. \textit{regia caeli}), has \textit{issit} it (sc. \textit{celsi thronus})'.

On the formal side, we may reconstruct \textit{nyt oes} as *\textit{ne et e est(i)}. This developed into *\textit{n\=ed-e-iss} (early apocope of final *\textit{i}, \textit{st} > \textit{ss}; *\textit{iss} as per McCone 1995). Subsequently -\textit{e}-, which stood in hiatus before a front vowel, regularly merged with *\textit{\=ɛ} (Schrijver 1995: 388-89). In LPBr. *\textit{\=ɛ} became *\textit{oɨ}, which led to LPBr. *\textit{nɨd oɨɨs}, wich, with contraction, yielded the attested MW \textit{nyt oes}. One might have expected PBr. word-final *\textit{-ss} t have dropped off as a result of PBr. apocope. However, since *\textit{-ss} was 
clearly contained in MW \textit{nos} `night' < *\textit{nox(s)s} < *\textit{nok\textsuperscript{w}ts}, I follow Morris Jones' suggestion (1913: 191) that at the end of a stressed monosyllable *\textit{-ss} was retained as \textit{-s}. In all probability *\textit{iss} `there is' was stressed, while the copula (*\textit{n\=ed iss} > MW \textit{nyt} `is not') was not: this would then mirror the situation in OIr., where the substantiveverb receives normal stress while the copula is a clitic.

In the proto-form of \textit{yssit}, the clitics *\textit{et} and *\textit{e} followed clause-initial *\textit{esti}: *\textit{esti (e)t(i) e} *\textit{issit-t-e} (elision of *\textit{et} > *\textit{t} after a vowel as in Ir.: *\textit{ber\=u-t} > *\textit{ber\=u-h} > OIr. absolute \textit{biru} `I carry'; *\textit{es} > *\textit{is} as above) > LPBr. *\textit{ɨsɨd} > MW \textit{yssit}. The MW spelling with \textit{i} instead of \textit{y} probably reflects an archaic orthography. \textit{yssit} is a rare and archaic form already in MW, which was supplanted by \textit{y mae} (Morris Jones 1913: 350). That \textit{y} could be written as \textit{i} in older MW is beyond doubt: cf. e.g. the spelling \textit{yssit} for MW \textit{yssyð}, MoW \textit{sy} `who, which is' in the Black Book of Carmarthen (ed. Jarman 1982; eg. 5.8, 9.5, 12.1) Since MW manuscripts usually do not distinguish between /d/ and /ð/, a fossilized orthography \textit{yssit} /əsɨd/ `there is' would b useful in order to distinguish this form graphically from \textit{yssyd} /əsɨð/ `who, what is'. There is independent evidence that \textit{i} in \textit{yssit} indeed stands for /ɨ/: the analogically created 3pl. form \textit{yssydynt} `there are', which is formed by adding \textit{-ynt} t \textit{yssit}, is written with \textit{y} = /ə/ in E.g. WM 487.1-2.

In MW, both \textit{oes} and \textit{yssit} cause lenition (Evans 1976): 17; Morgan 1952: 281-83). According to the suggested reconstructions, lenition is regular after \textit{yssit} (which originally ended in *\textit{-e}, i.e. the neuter pronoun) but not after \textit{oes}. Lenition after \textit{oes} may well be analogical after \textit{yssit}.

The Breton cognate of MW \textit{oes} is found in OSWBr. \textit{ois}, MMoB \textit{eux, eus}. Vannetais sometimes has \textit{wes} (ALBB map 83). In Cornish we find \textit{es, vs, eus, ues}, which points to /œs/, corresponding to B \textit{eus}. The phonologically unique correspondence of W \textit{oe} with B \textit{eu} may probably exlaine on the basis of the unique LPBr. disyllabic form *\textit{oɨɨs} reconstructed above. In Vannetais, the latter was contracted to *\textit{oɨs}, much as in W, which further developed into \textit{wes} (with development of \textit{o-} into a consonantal labial glide \textit{w-}). In the other B dialects, *\textit{oɨɨ} fused into a rounded vowel /œ/ = \textit{eu}. The latter development may also be assumed for Cornish, athough it probably occured independently in view of the OSWBr. and Vannetais forms, which do not sow this particular contraction.

We may conclude that the MW pair \textit{yssit, nyt oes} reflect an ancient InsCl. syntagma invovling the verb *\textit{esti} > InsCl. *\textit{iss(i)} and two clitics, the main clause particle *\textit{et(i)} and the neuter pronoun *\textit{e(d)}. The phonological difference between the two MW forms ultimately reflects the diferent position of the clitics in accordance with Wakcernagel's Law. 

The 3sg. subjunctive counterpart of the present \textit{nyt oes} can still be found in OW: \textit{cennit boi loc guac} (Comp.) `although there is no empty space'. 

These Welsh forms can be connected with the verb `to have' in SWBr. The latter consists of the 3sg. of the verb `to be' preceded by an object pronounm e.g. 1sg. pres. MB \textit{a-m eux}, MCo. \textit{a-m bues} (with \textit{b-} taken from the root *\textit{b\textsuperscript{h}u-}). This form ultimately reflects *\textit{-(e)t-me iss(i)}, but formally the regular rele of *\textit{iss(i)}, viz. MB *\textit{es}, MCo. *\textit{ys}, was replaced by the reflex of *\textit{e iss(i)}, which contains the petrified 3sgn. pronoun.

The old form of the 3rd person present survives in MMoB \textit{deux, deus}, which is the exact formal counterpart of MW \textit{(ny-)t  oes}.

A LPBr. element *\textit{-de-} (> MB \textit{-de-}, MCo \textit{-ge-} appears only in the 3rd person forms and is positioned between the infixed object pronoun and the 3sg. form of `to be', e.g. 3sgm. prterite MB \textit{en de-uoe}, MCo. \textit{yn ge-ve} `he had'. The original locus of *\textit{-de-} may be traced to the consuetudinal present, the preterite, the subjunctive and the future, all of which contain reflexes of the verbal stem *\textit{b\textsuperscript{h}u-}. *\textit{-de-} may be reconstructed as the particle *\textit{(e)t(i)} > LPBr. *\textit{-d-} + petrified *\textit{e(d)}+ the lenited forms of the root *\textit{b\textsuperscript{h}u}: *\textit{et(i) e(d) b-} > *\textit{(e)deβ-}, on which was based e.g. the 3sgm. *\textit{ɨn deβ-} (with PBr. *\textit{ɨn} `him') > MB \textit{en deu-}, MCo. \textit{yn gev-}. As a result of a reanalysis, the element *\textit{-deβ-} spread to the 3rd person of those tenses which originally did not contain the root *\textit{b\textsuperscript{h}u-}, viz. the present and imperfect:

Present: MB \textit{deueux}, MoB \textit{deveus} `has/have' (beside old \textit{deux, deus}), MCo. \textit{gefes} `has/have';

Imperfect: MB \textit{deuoa}, MoB \textit{devoa} `had' (beside old \textit{doa}), MCo.. \textit{gevo}. The reason why LPBr. *\textit{-de-} appears only in 3rd person forms is doubtlessly because in these forms, and not in the first and second persons, the 3sgn. pronoun was patrified at an early stage.

Chronologically, the following development may be assumed.
\begin{enumerate}
\item Insular Celtic: the syntagma of ``3sg. of `to be' + min clause particle *\textit{et(i)} + oblique personal pronoun'' was used to express `there is to X, X has'. Both *\textit{et(i)} and the pronoun obey Wackernagel's Law":

*\textit{issi-(e)t(i) en} `there is to him'

*\textit{ne-(e)t(i) en iss(i)} `there is not to him'

The syntagma can be shifted back to Insular Celtic because it is present in all Insular Celtic languages: OIr. \textit{yssym} `there is to me' etc., MW \textit{oes, yssit}, and the SWBr. verb `to have'. 
\item British 1: first stage of petrification, connected with the loss of the category neuter *\textit{e} (< *\textit{ed}) becomes a fixed element in third person forms. These forms acquire the meaning `there is, exists':

*\textit{issi-d-e} *\textit{ne-d-e-iss} (> MW \textit{yssit, nyt oes})
\item British 2: on the basis of the form `there is', new third person forms of the paradigm of `to have' are created by adding the oblique personal pronouns to `there is':

*\textit{-ɨn d-e-iss} > MoB \textit{en deus} `he has'; *\textit{-sus d-e-iss} > (indirectly) MCo. a-s teves `they have'

*\textit{-ɨn d-e-bV-} > MoB \textit{en devoa} `he had'; *\textit{-sus d-e-bV-} > MCo. \textit{a-s tefe}

But e.g. 2sg. *\textit{-(e)t-tu̯e bV-} > *\textit{-θ βV- }> MB \textit{a-z foe} `you had' retains the original syntax.
\item British 3: as a result of the contraction of the old 3sgn. pres. conjunt *\textit{-e-iss} and its becoming opaque, the opportunity is created to spread it to the first and second persons: e.g. 2sg. pres. MB \textit{a-z eux}, MCo. \textit{a-th-ues} `you have'.
\end{enumerate}
}{schrijver_studies_1997}{172--176}

Lenition after \oes, Schrijver states, is most likely analogous to lenition after \mw{yssit}. If limited postverbal lenition is indeed due to a particular property of \oes\ and/of of postverbal lenition in general, and not due to phonotactic concerns, this begs the question how exactly postverbal lenition worked after \mw{yssit}. 

\subsection{\mw{yssit}}
The following examples constitute all instances of \mw{yssit} followed by a lenitable consonant:

\mwcc[yssitbrenhin]{Peniarth~9,~f38v,~l.~20}{a hỽynt a dugant idaỽ y [amws] du ymdeithic y llad yssit brenhin niniuent y arnaỽ.}{And they brought the black swift [steed] of killing on which the Ninevite king is (??)} Is example \ref{yssitbrenhin} even a proper example? Or does \mw{yssit} in fact mean what is more commonly written as \mw{yssyd}, meaning `that is'?
\mwcc[yssitgallu]{Peniarth~5,~f30v,~l.~28}{a chann yssit gallu vdunt y dywedut}{And although there is power to them to speak}
\mwcc[yssitle]{LlCH,~col.~836,~l.~26}{Oi a ỽr yssit le idaỽ y gỽynaỽ y neb yssyd yma}{Oh man, is there a place to complain to the one that is here?}
\mwcc[yssitbwyt]{LlCH,~col.~971,~l.~38}{Yssit bỽyt dryc ỽr; heb y diolỽch.}{``There is bad food, man''; against his will.}

Anyway, the examples all show \mw{yssit} followed by consonants other than voiceless stops, so one cannot tell from these examples what happened to these consonants. More sources are therefore needed. 

The examples also show an inconsistent pattern after consonants other than voiceless stops: examples \ref{yssitgallu} and \ref{yssitbwyt} show no lenition while \ref{yssitle} does show lenition.
