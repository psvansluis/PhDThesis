\chapter{Introduction}
\section{Voiceless stops after verbs}

This work deals with the system of postverbal lenition in Early Middle Welsh poetry. In particular, it deals with the type of lenition following verbs ending in \mw{–ei} and \mw{oes}\footnote{It should be noted that there is no known phonetic or phonological distinction between voiceless stops following \mw{s} and when their voiced counterparts follow \mw{s} in Welsh. Erratic behaviour of lenition following \oes\ is therefore problematic: if a scribe could not hear the distinction between lenited and unlenited consonants in this position, lack of lenition in writing is expected, but not exclusively following \oes. As such, the opposition in lenition between examples \ref{storch} and \ref{sbont} may be merely orthographical in nature.}. Lenition after these verbs is not dependent on the grammatical relationship to the following word, distinguishing it from lenition of the object or nominal predicate found in Modern Welsh and in parts of the Middle Welsh period. This type of lenition has undergone some changes in the Middle Welsh period.

A description of the lenition found after these verbs is given by \textcite{morgan_y_1952}. He analyses these verbal endings as causing lenition unconditionally, irrespectively of consonant type or grammatical function of the following word. Exceptions to this rule are explained as the result of imperfect representation of lenition in the orthography. \Textcite{van_development14}, however, finds that lenition following these verbs is not wholly unconditioned in the earliest Middle Welsh prose. Rather, these verbs only cause lenition to consonants other than voiceless stops. So far, this pattern has been found in the White Book recensions of \mw{Culhwch ac Olwen} and \mw{Pwyll Pendeuic Dyuet}. The following examples serve to illustrate the system:\todo{proper BibTeX citation for manuscripts}
\begin{mwl}
\mwc{WBR~2.2-4}{ac ual \y\ llathrei \w ynnet \y\ c\w n \y\  llathrei cochet \y\ clusteu}{And as white as the dogs shone, so red shone their ears.}
\mwc{WBR~481.22}{canẏt oes lestẏr ẏn ẏ bẏt a dalhẏo ẏ llẏn cadarn hỽnnỽ namẏn hi.}{Since there is no vessel in the world that may keep the strong drink except for this one.}
\mwc[storch]{WBR~483.23}{Nẏt oes torch ẏn ẏ bẏt a dalhẏo ẏ gẏnllẏuan namẏn torch canastẏr kanllaỽ}{There is no collar in the world that may hold the leash except for the collar of Canastr Canllaw.}
\end{mwl}
However, this system had already broken down by the time the White Book of Rhydderch itself was written. In the White Book recension of Branwen Uerch Lyr, lenition following these verbs is haphazard, and no regularity may be discerned in it~\parencite[42]{van_development14}. Furthermore, the scribe departs from this system on a handful of occasions, which suggests that he did not master this point of grammar himself, but merely copied this feature from an earlier text. These exceptions imply that this system where voiceless stops are not lenited disappeared at some point in the Middle Welsh period. To be more precise, it must have disappeared at some point in time between the date of composition of these tales and when the White Book of Rhydderch was composed. The date of composition of these tales is still debated, e.g.\ by Rodway~\parencite*{rodway_date_2005}, but the date the manuscript itself was written may be dated fairly safely to the middle of the fourteenth century~\parencite[228]{huws_medieval_2000}.

This type of lenition is not a typical orthographical aberration of this manuscript. Although the White Book recension of \mw{Culhwch ac Olwen} does not write lenition with 100\% accuracy, failure to represent lenition orthographically is not typically dependent on consonant type. Compare, for example, all instances of \mw{deu, dwy} `2' in the White Book recension of \mw{Culhwch ac Olwen}. As can be seen in Table \ref{deudwy}, only two examples fail to lenite, and these are one voiceless stop and one other type of consonant\footnote{Moreover, the only example suggesting non-lenition, \mw{deu par} is problematic, because the single word \mw{deupar} `two pairs' is also attested in Middle Welsh~\parencite[deubar, deupar]{bevan_geiriadur_2014}.}.

\begin{table}[h]
\centering

\begin{tabular}{ll|l|ll}
\textbf{col.} & \textbf{line} & \textbf{gloss}         & \textbf{len.} & \textbf{\mw{p, t, c}} \\ \hline
455          & 1             & \mw{deu par}          & no            & yes            \\
455          & 13            & \mw{deu uilgi}        & yes            & no            \\
455          & 21            & \mw{dỽẏ morwennaỽl}   & no            & no            \\
467          & 17            & \mw{deu geneu}        & yes            & yes            \\
467          & 35            & \mw{deu was}          & yes            & no            \\
468          & 12            & \mw{deu was}          & yes            & no            \\
469          & 13            & \mw{dỽẏ goẏs}         & yes            & yes            \\
478          & 5             & \mw{dỽẏ garant}       & yes            & yes            \\
480          & 27            & \mw{deu gẏtbreinaỽc}  & yes            & yes            \\
480          & 33            & \mw{deu gẏtbreinhaỽc} & yes            & yes            \\
484          & 9             & \mw{deu geneu}        & yes            & yes            \\
484          & 17            & \mw{deu geneu}        & yes            & yes           
\end{tabular}

\caption{Lenition after \mw{deu, dwy} `2' in the WBR recension of \mw{Culhwch ac Olwen}}
\label{deudwy}
\end{table}
\section{Voiceless stops as a mechanism for dating}
This dating for the disappearance of non-lenition of voiceless stops is not satisfactorily precise. A preciser date of the disappearance of this feature would allow texts to be dated on the basis of whether or not voiceless stops are lenited following verbs ending in \mw{–ei} and \mw{oes}. This work attempts to create such a tool. 

An obvious prerequisite for the creation of such a tool is that the texts used to date the disappearance of non-lenition of voiceless stops as after verbs ending in \mw{–ei} and \mw{oes} must be datable themselves. For Middle Welsh prose, this presents a serious problem, as Middle Welsh prose did not require named authors for the text to have any authority at all, and the name or even time or place of its authors was usually not recorded for this reason~\parencite[51]{rodway_date_2005}. The \emph{auctoritas} of Early Welsh poetry, by contrast, depended greatly on who composed the poem. Rowland~\parencite*[44]{rowland_y_2003} considers that a poem attributed to authors such as Taliesin or Myrddin may be more of a recognition that its contents are traditional than attribution of authorship in the modern sense of the word. As a result, saga poetry dealing with well-known characters and well-known stories are difficult to date on the basis of authorship.

Instead, the slightly later Poets of the Princes and Poets of the Nobility may be used. The subject matter of these poems is typically contemporary to its writers, and typically consists of praise or satyre poetry. Most poems within this genre have an identified author, and the receiver of the praise or satyre is oftenly identifiable as well. 
\section{Other linguistic tools used for dating}
This approach has been taken before, and several criteria which may serve to date texts on linguistic grounds have been advanced so far. Linguistic developments charted in this manner so far include: the spread of sandhi \mw{h}~\parencite{sims-williams_spread_2010}, the development of imperfect \mw{–i} into \mw{–ei} and the development of subjunctive \mw{–wy} into \mw{-o}~\parencite{rodway_two_2003}, the replacement of preterite ending \mw{–wys} by \mw{–awd}~\parencite{rodway_datable_1998}, the disappearance of absolute verbal endings~\parencite{rodway_absolute_2002}, the disappearance of a distinctive future tense~\parencite{isaac_old-_2004}, and the disappearance of plurals in \mw{–awr}~\parencite{nurmio_middle_2014}. In his monograph \textit{Dating Medieval Welsh Literature: Evidence from the Verbal System}, Rodway analyses many developments in the Medieval Welsh verbal system. On the basis of datable court poetry, he supplies the following list of developments, although sparseness of evidence makes some of them very approximate~\parencite*[166]{rodway_dating_2013}:

\begin{itemize}
\item Saec. XII\textsuperscript{2}
    \begin{itemize}
    \item 3 sg. pret. \mw{amwyth} 'attacked' no longer in use (?).
    \item First attestatiions of 3 sg. pret. \mw{-awdd} outside \mw{lladd}.
    \end{itemize}
\item Saec XIII\textsuperscript{1}
    \begin{itemize}
    \item 1 sg. pres. ind. \mw{-if} no longer in use.
    \item 3 pl. pres. ind. \mw{-ynt} no longer in use.
    \item First attestations of 3 pl. imperf. \mw{-eint}
    \item 1 sg. pret. \mw{ceintum} occurs (Gwynedd only?)
    \item 3 sg. pret. \mw{-as} no longer in use with verbs other than \mw{cael} 'to have' and \mw{gallu} 'to be able'
    \end{itemize}
\item Saec. XIII\textsuperscript{2}
    \begin{itemize}
    \item 2 sg. pres. ind \mw{-ydd} no longer in use.
    \item impers. pres. ind \mw{-awr} no longer in use.
    \item impers. pres. absolute forms no longer used.
    \item 3 sg. imperf. \mw{-i} no longer in use.
    \item 1 sg. pret. \mw{-t} no longer in use.
    \item 3 sg. pret. absolute forms no longer used.
    \item 3 sg. pres. \mw{s}-subjunctive forms no longer used.
    \item 3 sg. pres subj. \mw{-hwy} no longer used.
    \item Object pronouns can no longer be infixed after the first preverb of compound verbs.
    \end{itemize}
\item Saec. XIV\textsuperscript{1}
    \begin{itemize}
    \item 3 sg. pres relative \mw{-ydd} no longer used.
    \item First attestations of 1 sg. pret. \mw{clywais} 'I heard'. 
    \item 3 sg. pret. \mw{gallas} 'was able  no longer used.
    \item 3 sg. pret. \mw{-awdd} regular with verbs which previously took \mw{-wys.}
    \end{itemize}
\item Saec. XIV\textsuperscript{2}
    \begin{itemize}
    \item 3 sg. pres. ind. absolute forms no longer used.
    \item 3 sg. pret. \mw{-w(y)s} no longer used.
    \end{itemize}
\end{itemize}

Not all of these innovations may be dated to the exact same period, but the end of the thirteenth century in particular saw a significant change in the literary language used in poetry. Rodway suggests, one the basis of three variables, \tqt{the existence of a standard literary language in Wales that changed substantially towards the end of the thirteenth century, a time of great political upheaval which had far-reaching implications for the literary establishment.}{rodway_two_2003}{74} Already, then, we may be able to discern whether a Middle Welsh text was written before or after this political upheaval, as long as the text is composed in a literary register.
\section{Lexical manipulation in Middle Welsh poetry}
It is indeed important to note that different \textit{genres} of Middle Welsh texts differ linguistically. Many of these linguistic variables survive in poetry for longer than they do in prose. They do so because they are susceptible to conscious lexical manipulation. In other words, a medieval poet would want to use archaising language in order to lend authority to his poems. It is for this reason that poetry often looks more archaic than prose of the same age. 

This presents a problem for prose. Poetry is more often datable on external grounds than prose, and therefore so are its linguistic features. However, these same linguistic features are partly the result of conscious lexical manipulation. In general, prose is not as susceptible to conscious lexical manipulation, so a prose text and a poetry text composed in the same time frame need not have the same linguistic features. 

One method of correcting for this archaising effort made by Medieval Welsh poets is by quantifying how much older poetry appears compared to contemporary prose. The language of Medieval Welsh poetry generally looks more archaic, but this supposition is difficult to test in the absence of externally datable prose. However, some prose is datable, and Rodway~\parencite[167-168]{rodway_dating_2013} uses \mw{Llyfr Iorwerth}\todo{Which manuscript? Is the Black Book of Chirk contemporary with Iorwerth himself?}, which is named after a known person who lived in the early thirteenth century. When comparing the verbal forms of this manuscript with that of contemporary and near-contemporary court poets, he finds that the prose in \mw{Llyfr Iorwerth} is most similar to poets from about a generation later. This picture only becomes apparent after discounting the tendency of poetry to be filled with absolute forms, which in turn is caused by the relatively frequent use of verb-initial word order. 

As far as the choice of verbal forms  is concerned, then, poets seem to have been able to archaise their texts up to about 30 or 40 years in the early thirteenth century. Theoretically, then, this knowledge may be of aid in dating prose using poetry on the base of its verbal forms. We do not know, however, whether linguistic subsystems other than choice of verbal forms were equally susceptible to conscious archaisation.
\section{Constraints to linguistic manipulation}
Archaisation may be seen as a form of borrowing or language shift, although in this case not towards a different language or dialect, but rather towards a different chronolect. By employing an archaic register, one borrows elements from an earlier chronolect. Van Coetsem~\parencite*{van_coetsem_loan_1988} distinguishes source language agentivity from recipient language agentivity. Source language agentivity is when a speaker retains stable elements from his native speech when learning a new language. Recipient language agentivity is when unstable elements from a second language are incorporated in a speaker’s native speech. 

Van Coetsem~\parencite*{van_coetsem_loan_1988} places different linguistic subsystems in a stability hierarchy, which may be ordered from more stable to less stable. This yields the following order from most to least stable: lexicon, lexical phonology, morphology, syntax, and pronunciation. By way of illustration, this may mean the following: a native speaker of language A shifts to target language B, but elements in language A are transferred to language B. Within this scenario, the process of source language agentivity causes more stable elements to be transferred to language B. This means that this speaker would be more likely to transfer syntax and pronunciation than lexicon from language A into language B, for example.

If one considers archaisation a form of language shift towards an older chronolect, there is reason to think that not all linguistic subsystems are equally susceptible to archaisation. Speakers are able to consciously manipulate different linguistic subsystems to different degrees. One may therefore expect that a poet who attempted to shift to an archaic register to have more success in manipulating unstable features than stable features. The easier it is to subject a linguistic element to manipulation, the less similar this element is to contemporary speech. More stable linguistic elements are therefore preferable to unstable elements in determining the date of a text on linguistic grounds. \todo{find out more about sociolinguistic salience and what makes something salient. Notably Van Coetsem 1988, Thomason?. Any literature on purism in language?} 

In general, Van Coetsem's stability hierarchy orders linguistic subsystems from less to more abstract. I propose that the the early Middle Welsh system of limited postverbal lenition is more stable than archaic verbal endings\footnote{With \textquoteleft stable\textquoteright, I mean stable in Van Coetsem's sense, but not diachronically stable per se.}. Substituting one verbal ending for another is an essentially lexical replacement, since this change does call for a reanalysis of part of the syntactic or morphological structure of Welsh. Postverbal lenition, is a more abstract part of Middle Welsh morphology, and is dependent on and has ramifications for other elements in a Middle Welsh clause, namely the word following is to be lenited, and choice for lenition or lack thereof is based on its initial consonant. grammatical rule that dictates non-lenition of voiceless stops in a particular context is less As such, one may expect a smaller discrepancy between poetry and prose in the system of postverbal lenition. I propose, therefore, that postverbal lenition may serve as a mechanism that bridges the gap in how the language of prose is connected to that of poetry.

In Later Middle Welsh and in Early Modern Welsh, postverbal lenition was applied unconditionally, as seen in the Welsh translation of \textit{Vita Sancti Martini}, which was translated into Welsh in 1488. In this text, lenition following \ei\ is unconditioned~\parencite[42]{van_development14}. However, there is some evidence that medieval scribes were aware of a system ruling that  postverbal lenition should not be applied at all times even after the disappearance of the rule exempting voiceless stops. They did not, however, understand the exact rule governing lenition or non-lenition. What we see through a large part of the later Middle Welsh period, then, is haphazard lenition after \mw{-ei} and \mw{oes}~\parencite[38]{van_development14}. This is apparent in \mw{Branwen Uerch Lyr}, for example:
\mwcc[sbont]{WBR~52.10}{ti a ỽdost kẏnnedẏf ẏr auon. nẏ eill neb uẏnet drỽẏdi. nẏt oes bont arnei hitheu}{You know the peculiarity of the river: nobody can go through it, and there is no bridge over it.}
Example \ref{sbont} shows that lenition spread to voiceless stops, but non-lenition of other consonants is also found in the later Middle Welsh period. The following example comes from the Red Book of Hergest recension of \mw{Owein}:
\mwcc[imorwyn]{RBH~636.21}{Ac a welei morỽyn benngrech uelen a ractal eur am y phenn.}{And he saw a  curly yellow-haired maiden with a headband around her head.}

These examples are problematic. As far as I can see, lenition is completely haphazard and no rule may be formulated on when it occurs. This is not to say that there was no rule governing which consonants were to be lenited in this period. I propose that postverbal lenition occurred unconditionally following \ei\ and \oes\ in this period. This is the simplest way to account for the random distribution of lenition given the exceptional nature of excepting voiceless stops in Early Middle Welsh after verbal endings\footnote{This exact pattern of lenition and non-lenition does not happen in any other environment in Middle Welsh, as far as I know.}, and given how the contemporarily emerging object lenition was not dependent on the following consonant\footnote{See, for example, lenition following \mw{s}-preterites~\parencite[51-54]{van_development14}.}, and given the fact that Early Modern Welsh shows postverbal lenition consistently in writing~\parencite[42]{van_development14}.

These examples imply that even later medieval scribes were unable to reproduce the earlier system of non-lenition of voiceless stops, but they were still aware that not all consonants should be lenited in this context. The result of this is either failure to maintain a radical voiceless stop, or hypercorrection leading to failure to lenite a different type of consonant. I maintain that the lower degree of saliency of this feature compared to the choice of verbal endings caused the system to be less susceptible to conscious manipulation, and therefore led to both innovations (i.e.\ lenition of voiceless stops) and hypercorrections (i.e.\ non-lenition after other conconsants) in texts written following the disappearance of restricted lenition. However, the existence of hypercorrections, such as seen in example \ref{imorwyn} shows that scribes were not completely ignorant of this feature.

The relative lack of saliency of this grammatical feature compared to the choice for archaising or modernising verbal endings makes this feature less susceptible to conscious manipulation by scribes. I propose that it is for this reason that the phenomenon is so much less widespread, even though its date of disappearance may well coincide with the political upheaval of the end of the 13th century.

\section{Limits to using poetry as a source}
Lack of grammatical saliency of postverbal lenition may also pose a problem. Precisely because postverbal lenition was not a salient grammatical feature, and because it did not serve to make semantic distinctions, poets may have felt some degree of freedom when composing poetry. The existence of erratic lenition or lack thereof is indeed attested, as Lake notes:

\tqt{Y mae’r modd y defnyddiai’r beirdd yr iaith, a rheolau iaith, yn cynrychioli dull arall
sy’n caniatáu iddynt fesur o ryddid a hyblygrwydd. Carwn gynnig dwy ystyriaeth o dan
y pen hwn. Yn y lle cyntaf, y mae’n dra hysbys fod y beirdd yn arddel mwy nag un
patrwm wrth dreiglo. Byddant yn parchu cystrawen lafar naturiol eu cyfnod hwy a
byddant yn manteisio hefyd ar arferion cyfnodau cynharach y trwythid hwy ynddynt
yn yr ysgolion barddol. Dyma ‘enghraifft dda o’r beirdd yn cael “rhyddid” i ddewis yr
un a fynnent o ddwy gystrawen’, yn ôl T. J. Morgan. O dan rai amodau gwelir treiglo
goddrych y ferf lle bellach y disgwylid y gysefin a chedwir cysefin y gwrthrych lle y
disgwylid treiglad erbyn hyn.
(\dots)
Gallant fanteisio hefyd ar dreigladau rhanbarthol neu dafodieithol, a
dewis felly rhwng dau batrwm cyfoes yn hytrach na rhwng y cyfoes a’r hynafol.
(\dots)
Ac ar dro bydd y
beirdd yn manipiwleiddio’r rheolau at eu diben eu hunain. Cyffredin yw treiglo’r enw
priod sy’n dilyn enw benywaidd unigol mewn ymadroddion genidol. Gwelodd y beirdd
yn dda gymhwyso’r treiglad at ymadroddion genidol a gynhwysai enwau gwrywaidd.
Yn ail, ni ellir llai na sylwi ar yr amryfal gystrawennau hyn wrth astudio gwaith y
beirdd unigol er bod lle i gredu hefyd fod hyn yn amlycach erbyn y bymthegfed ganrif
a’r unfed ganrif ar bymtheg. Bydd y bardd yn arfer un dull mewn un gerdd ac yn arfer
dull amgen mewn cerdd i noddwr gwahanol mewn rhan arall o’r wlad, ac, o bosibl,
flynyddoedd yn ddiweddarach. Yn fynych, fodd bynnag, gwelir yr amrywio cystrawennol
hwn o fewn cwmpas un gerdd, ac yn fwy na hynny, yn yr un llinell neu yn yr
un cwpled fel pe bai’r bardd yn mynd allan o’i ffordd i gyhoeddi i’r byd a’r betws beth
yn union yr oedd yn ei wneud.}{lake_gwe_2004}{69}
Translation: 
\begin{quote}\onesp{
The way in which the bards used the language, and the rules of language is representative of another manner shich allows them a degree of freedom and suppleness. I would love to propose two considerations below here. In the first place, it is very obvious that the bards acknowledge more than one pattern of mutation. They will praise natural spoken construction of their period and they will also take advantage of manners of earlier periods in which they would be immersed in bardic schools. This is a `good example of the bards taking ``freedom'' to choose the one that they would wish out of two constructions', according to T.J. Morgan. Under some circumstances mutation of the subject is seen where otherwise one would expect the radical and the radical of the object is kept where a mutation would be expected by then. 
(\dots)
They could also take advantage of areal or dialectal mutations, and thus choose between two contemporary patterns instead of between the contemporary and the archaic.
(\dots)
And soon the bards will be manipulating the rules for their own sake. It is common to mutate the proper noun that follows a feminine singular noun within a genitive phrase that contains masculine names. Secondly, one cannot do less than take note of these various constructions when studying the work of singular bards although there is reason to believe also that this is more obvious now by the fifteenth century and the sixteenth century. The bard will use one scheme within one poem and will use an alternative within a poem for a different sponsor within another part of the country, ac possibly, years later. Oftenly, however, this variety of constructions is seen within one poem, and more than that, in the same line in the same couplet as if the bard would go out of his way to show the world and the church what exactly he would be doing.}
\end{quote}

