
It is often assumed that poetry has remained more or less unchanged during its transmission for metrical reasons (e.g. Williams, 1944, 23). This is not always the case, however. Thomas Owen Clancy Draws our attention to the practice of inserting modern words into late copies of Irish bardic poetry, with further emendation to preserve the correct number of syllables in the line (Clancy, 2013, 167, n. 3). Scholars of Old English have shown that copyists fairly frequently substituted words and phrases which were metrically acceptable into the poems that they were copying, with over 20\% of lines in some poems extant in more than one manuscript showing scribal interference (O'Brien O'Keeffe, 1990; Liuzza, 200, 292-93; Lapidge 2000, 36-40).
Rodway 2013 p19

Thomas Charles-Edwards has suggested another approach, namely that we can distinguish between 'fixed' texts produced by professional scribes and 'fluid' texts roduced by scholars of various types, who were prone to emend to a far greater degree. The\textit{Four Branches} and the Iorwerth redaction of the Medieval Welsh laws exemplify 'fixed' texts, whereas \textit{Peredur} and the Cyfnerth redaction of the laws are 'fluid' texts (Charles-Edwards, 2001). This is an attractive distinction, for it would mean that linguistic data from 'fixed' prose texts could be used with a degree of confidence. However, the paucity of independent manuscript witnesses or most medieval prose means that the 'fixedness'of a given text might be illusory -- in other words, if more early copies of the \textit{Four Branches}  survived, they might appear to be as 'fluid' as \textit{Peredur} (Russell, 2003a, 61-62; Rodway, 2005, 24). Another problem is that, even if we accept the distinction between 'fixed' and 'fluid' texts as a genuine one, a 'fluid' tradition might lie begind the 'fixed' version of, say, the \textit{Four Branches} which is visible to us in the extant manuscripts.
Rodway 2013 p21

\begin{itemize}
\item Saec. XII\textsuperscript{2}
    \begin{itemize}
    \item 3 sg. pret. \mw{amwyth} 'attacked' no longer in use (?).
    \item First attestatiions of 3 sg. pret. \mw{-awdd} outside \mw{lladd}.
    \end{itemize}
\item Saec XIII\textsuperscript{1}
    \begin{itemize}
    \item 1 sg. pres. ind. \mw{-if} no longer in use.
    \item 3 pl. pres. ind. \mw{-ynt} no longer in use.
    \item First attestations of 3 pl. imperf. \mw{-eint}
    \item 1 sg. pret. \mw{ceintum} occurs (Gwynedd only?)
    \item 3 sg. pret. \mw{-as} no longer in use with verbs other than \mw{cael} 'to have' and \mw{gallu} 'to be able'
    \end{itemize}
\item Saec. XIII\textsuperscript{2}
    \begin{itemize}
    \item 2 sg. pres. ind \mw{-ydd} no longer in use.
    \item impers. pres. ind \mw{-awr} no longer in use.
    \item impers. pres. absolute forms no longer used.
    \item 3 sg. imperf. \mw{-i} no longer in use.
    \item 1 sg. pret. \mw{-t} no longer in use.
    \item 3 sg. pret. absolute forms no longer used.
    \item 3 sg. pres. \mw{s}-subjunctive forms no longer used.
    \item 3 sg. pres subj. \mw{-hwy} no longer used.
    \item Object pronouns can no longer be infixed after the first preverb of compound verbs.
    \end{itemize}
\item Saec. XIV\textsuperscript{1}
    \begin{itemize}
    \item 3 sg. pres relative \mw{-ydd} no longer used.
    \item First attestations of 1 sg. pret. \mw{clywais} 'I heard'. 
    \item 3 sg. pret. \mw{gallas} 'was able  no longer used.
    \item 3 sg. pret. \mw{-awdd} regular with verbs which previously took \mw{-wys.}
    \end{itemize}
\item Saec. XIV\textsuperscript{2}
    \begin{itemize}
    \item 3 sg. pres. ind. absolute forms no longer used.
    \item 3 sg. pret. \mw{-w(y)s} no longer used.
    \end{itemize}
\end{itemize}
Rodway 2013, 166

Nonetheless, we are faced with a number of features which set the language of poetry apart from the language of contemporary prose, notably the use of absolute forms (also found in proverb collections but not, with one exception (see 112-114 abve), in narrative or descriptive prose in our corpus), of 3 sg. res. ind. relative \mw{-ydd}, and of 3 sg. pres. subj. \mw{-hwy}. If we compare Cotton Caligula A.iii with the work of the next generation of poets (\textbf{Group IV}), we see less difference in language. The following features are attested in the poetry but not in the prose: 3 sg. pres. ind. absolute \mw{-awdd}, \mw{-id}; 3 sg. pres. ind. relative \mw{-ydd}; impers. imperf. \mw{-ed}; \mw{cigleu} 'I heard'; 3 sg. pret. \mw{-awdd}; \mw{llas} 'was killed'. We have already dealt with \mw{cigleu} and \mw{-awdd}. The form \mw{llas} is likewise without significance, as it survived well beyond the thirteenth century. Thus, apart from the survival of absolute and relative form in 3 sg. pres. ind. and of impers. imperf. \mw{-ed}, the language of the \textbf{Group IV} poets is very similar to that of \mw{Llyfr Iorwerth}, as reserved in Cotton Caligula A.iii. Thus, a certain amount of conservatism is detectable in the poetry, with changes in some case lagging behind prose to the tune of thirty or forty years or so. Syntactic factors seem to have facilitated the survival of absolute forms in poetry at a time when verb-initial constructions in prose were exceptional. As regards the retention of obsolete fossils in the poetic language, I have seen little evidence for this, the best candidate, perhaps, being 3 sg. pres. ind. relative \mw{-ydd}, which was being used more or less 'correctly' at least as late as the second half of the thirteenth century (see Chapter 5 above)
Rodway 2013 168

Three 3 sg. imperf. endings are attested in the corpora: \mw{-ei} (> Modern Welsh \mw{ai}, the only form that has survived until today), \mw{i} and \mw{-(i)ad} (see Evans, 1964, 121-22). The latter is cognate with the Old Irish 3 sg. imperf. ending \textit{-e/ad}, with *\textit{-to} added to the stem vowel (Lewis and Pedersen, 1961, 284; Evans, 1964, 122; Isaac, 1996, 377; Schrijver, 2011, 61; cf. McCone, 1994, 161). The Brittonic forms without /j/ could derive from Class IV *\textit{-a-to}. The forms with /j/ must be from *\textit{-je-to} (Schrijver's Class V), and in the light of well-atsted yod-deletion in some Middle Welsh dialects, the Middle Welsh orms in \mw{-ad} could also have this derivation. \mw{-(i)ad} almost exclusively occurs with the verb \mw{bod} and its compounds in Middle Welsh (see, for instance, the examples listed in Lewis and Pedersen, 1961, 280; Evans, 1964, 122). That this was not always the case is shown by Old Breton \textit{bitat} 'cut' and \textit{troeat} 'turned, went around' (Fleuriot, 1964, 304; Schrijver, 2011, 61; cf. Koch, 1988, 36), and by the form \mw{pwyllatt} 'thought' from a potentially historical Taliesin poem (Williams, 1968, II.26, note on p. 40; Koch, 1988, 36; Isaac, 1998a, 63). The only example included in the present survey is \mw{ny didoryat y Brytannyeit pa damwein y dygwydynt yndaw} 'The Britons did not take heed of what accident they should fall into' (NLW 5266, 267; Lewis, 1942, 172). Henry Lewis explains that \mw{y Brytannyeit} is the object of the verb and \mw{pa damwein} the subject. This construction was later changed, so that the opposite would be true, cf. the Red Book of Hergest version of this sentence:\mw{ny didorynt py damwein y dygwydynt yndaw} (Lewis, 1942, 273-274). D. Simon Evans tentatively suggests emending \mw{didoryat} in NLW 5266 to \mw{didorynt} (Evans, 1964, 122), but this seems unwarranted. Note the verbal noun \mw{didorbot} (Lewis, 1942, 274), \mw{diddarbod} (GPC, 968). This verb is a compound of the defective verb \mw{doraf}, which has a verbal noun \mw{dorbod}, \mw{darbod} (GPC, 1076). Forms o this verb in which forms of the verb \mw{bod} are attached to the stem \mw{dor} are attested, e.g. 3 sg. future \mw{dorbi} (Jarman, 1982, 17.155, 17.214; cf. Schumacher, 2004, 270). Thus this verb, whatever its prehistory (see Schumacher, 2004, 267-70), had been brought into the orbit of the vrb \mw{bod}, and the orm \mw{didoryat} therefore conforms to the Middle Welsh pattern whereby \mw{-(i)ad} almost exclusively occurs with \mw{bod} and its compounds.

The origin of the other two 3 sg. imperf. endings is problematic. Eric Hamp posited \mw{-ei} < *\textit{-eset}, an old unstressed and enclitic form of the verb 'to be' (Hamp, 1974), but this has been shown to be impossible on phonological grounds by Graham Isaac (1996, 373-74). The latter argues that the form \ei\ developed from a misanalysis of 3 sg. imperf. forms in \mw{-i} (of uncertain origin) of the irregular verbs \mw{myned} (stem *\textit{-ag}), \mw{dyfod} (stem *\textit{dag-} and \mw{gwneuthur} (stem *\textit{gurag-}), which would yield, following \textit{i}-affection and loss of lenited \textit{g}, \mw{e-i} 'went', < Old Welsh *\mw{egi}; *\mw{de-i} 'came'< */mw{degi}; *\mw{gwne-i} 'made, did' < *\mw{guregi}, 'into whih forms, for morphological clarity, the basic unaffected vowel of the stem was reintroduced, giving \mw{aei}, \mw{daei} (more usual \mw{doei}, \mw{gwnaei}, the M[iddle] W[elsh] forms' (Isaac, 1996, 376-77). These were then resegmented as \mw{a-ei}, \mw{da-ei}, \mw{gwna-ei} (Isaac, 1994, 201-2). This ending had begun to spread by the tenth century at least on the evidence of \mw{nacgenei} 'that there wass no need' from the Computus fragment (Falileyev, 2000, 2) -- cf. \mw{immis-line} 'he besmeared himself' from the possibly ninth-century Martianus Capella glosses (Falileyev, 2000, 91), with wel-attested Old Welsh <e> for /ei/. Therefore wemight expect to find a high incidence of \mw{-i} in the early works being superseded by \ei\ in later ones/ owever, Peter Schrijver, while accepting Isaac's explanation as a possibility, further suggests that \ei\ could derive from Class IV present stems in \textit{-a-} + \textit{-i-}, while \mw{-i} always have been restricted in distribution. The statistics for the court poetry (taken from Rodway 2003b, 69) are as follows (the figures below the percentages represent the number of instances in each text and those in brackets represent forms confirmed by rhyme):

\begin{table}[h]
\centering
\caption{My caption}
\label{my-label}
\begin{tabular}{lllll}
                          & \textbf{Group I} & \textbf{Group II} & \textbf{Group II} & \textbf{Group IV} \\
\multirow{2}{*}{\mw{-ei}} & 70\% (56\%)      & 93\% (87\%)       & 89\% (83\%)       & 100\% (100\%)     \\
                          & 21 (9)           & 103 (54)          & 17 (5)            & 20 (7)            \\
\multirow{2}{*}{\mw{-i}}  & 30\% (44\%)      & 7\% (13\%)        & 11\% (17\%)       & 0\% (0\%)         \\
                          & 9 (7)            & 8 (8)             & 2 (1)             &                  
\end{tabular}
\end{table}

It can be seen that \ei\ increases from 70\% in \textbf{Group I} to 100\% in \textbf{Group II}. If we compare the situation in undated poetry, we see that \mw{-i} is always in a minority in the sample which I examined (84\% \mw{-ei} in the Book of Aneirin; 79\% in the Book of Taliesin; 93\% in the saga \mw{englynion} -- Rodway, 2003b, 70). This sort of statistical analysis is only valied if we assume that 'at one stage \mw{-i} was the only form taken by nn-\mw{bod} verbs' (Rodway, 2003b, 70). This assumption is questionable, however, both in light of Schrijver's suggeston that the endings \ei\ and \mw{-i} could have arisen in different stem-classes, and in light of the evidence that \mw{-(i)ad} at one stage occurred with 'non-\mw{bod} verbs.'

\mw{-i} is vanishingly rare in the prose. I have noted two examples of \mw{rody} 'gave' in Llanstephan 1 and one of \mw{lunyeithi} in NLW 5266. In the latter case, the  scribe wrote \mw{lunyeith}, and <i> was added above the line (Lewis, 1942, 35, n. 3) It is quite plausible that all three examples are simly errors. Henry Lewis duly emends \mw{lunyeithi} to \mw{lunyeithei} (Lewis, 1942, 35). Simon Evans has only one example of \mw{-i} from a prose text, namely \mw{seui} 'stood' in the \textit{Four Branches of the Mabinogi} (Williams, 1951, 92.9; Hughes, 2000, l. 579). The <e> in the stem here mitigates against restoring \mw{-ei,}, which would not have caused \textit{i}-affection, here. Thus we would expect *\mw{sau[e]i}. Thus this looks like a bona fide prose example of \mw{-i}. Note however that the Red Book of Hergest has regular 3 sg. pret. \mw{seuis} here, while the White Book of Rhydderch scribe initially wrote \mw{seuit}, with a \textit{punctum delens} subsequently added beneath the \mw{t} (Hughes, 2000, 75). While it is perfectly plausible that the Red Book copyist (or the cyist of his exemplar) replaced an obsolete form (i.e. \mw{seui} with a more familiar one, as envisaged by Ifor Willams (1951, 303), it is equally possible that the common exemplar of the White and Red Book versions contained \mw{seuis}, which was miscopied in the White Book as \mw{seuit}. The scribe, or a subsequent reader, noticed the error and deleted the \mw{t}, but omitted to replace it with \mw{s}. 
Rodway 2013 65-66

Rodway 2013 169 cautiously dates \mw{Culhwch ac Olwen} to the mid to late twelfth century.

I emphasize the need to assemble as much data as possible. Cumulative linguistic evidence can point to an early or late date, overcoming sparseness of evidence relating to one feature. 

\section{Koch}
 Koch (Cothairge) \S\S24--30:
 
 \S24: Koch on lenition of stops based on Jackon's LHEB and HPB:
 
 \begin{enumerate}
 \item Within a common stage of the Old Celtic ancestor of Goidelic and Brittonic an environmentally-conditioned phonetic opposition of fortes and lenes arose: thus
 
 Fortis \textit{K\textsuperscript{w}, K, T, B, D, G, D} opposed to lenis \textit{k\textsuperscript{w}, k, t, b, d ,g} (in which the symbols are largely algebraic, lacking a specific phonetic value).
 \item More-or-less simultaneously in the later fifth century AD, the lenis series undergoes the partially identical and partially opposed Goidelic and Brittonic lenitions: thus
 
 Later Primitive Irish unlenited \textit{K\textsuperscript{w}, K, T, B, G, D} opposed to lenited \textit{χ, χ\textsuperscript{w}, θ, β, γ, δ,} and Late British unlenited \textit{P, K, T, B, G, D} opposed to lenited \textit{b, g, d, β, γ, δ.}
 \end{enumerate}
 
 Koch proposes the following alternative explanation:
 
 \begin{enumerate}
 \item Old Celtic lenition: /\textit{k\textsuperscript{w}, k, t, b, g, d}/ develope the allophones fortis [k\textsuperscript{wh}, k\textsuperscript{h}, t\textsuperscript{h}, b, g, d], in absolute initial position and lenis [\gwd, \gd, \dd, β, γ, δ] in intervocal and some other positions.
 
 (Also within the prehistoric period, \textit{p}-Celtic /\textit{k\textsuperscript{w}}/ [k\textsuperscript{wh}, \gwd] became /p/ [p\textsuperscript{h}, \bd].)
 \item Late Primitive Irish Spirantisation: the voiceless lenes [\gwd, \gd, \dd] are further weakened to [χ\textsuperscript{w}, χ, θ] by losing their stop quality in the mid to later fifth century.
 \end{enumerate}
 
 
 Koch \S26: Contrary to the influential view of Jackson, I now regard the tendency of Welsh and Breton to voice the lenitions of /\textit{p, t, k}/ to be a substantially later process, based on the Neo-Brittonic, i.e.\ post-apocope/syncope distribution of vowels and consonants. In the spelling systems of ninth-century Old Welsh and Old Breton, the lenitions of /\textit{p, t, k}/ are written \textit{p, t, c} every bit as consistently as they had been during the Romano-Celtic horizon. It therefore is highly unlikely that they had as yet fallen together phonemically with radical initial /\textit{b-, g-, d-}/. For later Welsh, \textit{cynghanedd} demonstrates the convergence to have been complete before the fourteenth century. As Falc'hun has shown, these series never completely converged in Breton, at least not in some dialects, where the historical voiced fortes still contrast with the historical voiceless lenes in minimal pairs by a clinically measurable diference in duration: e.g. [\c{e} g:a:r] `her leg' vs. [\c{e} ga:r] `his car'.
 
 Koch \S27: The chronology of phonemic convergence, though hardly irrelevant to that of phonetic voicing, is nonetheless not the same issue nor guaranteed to have run the same course. The evidence of early English borrowings, such as \textit{Dinoot, Madoc,} and \textit{Cerdic}, implies that voicing occurred in post-apocope word-internal position quite early on, but not so in absolute final. This is hardly surprising considering that the lenited voiceless stops are usually voiceless today in absoolute final in Breton (being voiced there only in favorable sandhi). Even in Modern Welsh, \textit{-b, -g, -d} are frequently voiceless in sandhi. In short, there is no early evidence or sufficiently broad comparative evidence to suggest that voicing of intervocal /\textit{-p-, -k-, -t-}/ had occurred before apocope so as to result in voice final [-b, -g, -d] in Common Neo-Brittonic.
 
 Koch \S29: Many details remain to be worked out concerning the development of the various Brittonic consonant systems through the mediaeval and modern periods, but it is safe at present to proscribe the old axiom that \textit{p}-Celtic /\textit{p, t, k, b, g, d}/ gave Common Neo-Brittonic radical /\textit{p, t, k, b, g, d}/, lenited /\textit{b, g, d,  β, γ,  δ}/, in which the two series partially overlap as in Modern Welsh. In his more recent ponderous phonology, Jackson decisively replaced this defective construction by setting out a non-overlapping `Primitive Breton' stop system as fortis \textit{P, K, T, B, G, D} opposed to lenited \textit{b, g, d,  β, γ,  δ}. There, the essential distinction between fortis \textit{B-, G-, D-} and the lenes \textit{b-, g-, d-} is visualised as being not of voice, but of relative duration of the voiced series, as in Modern Breton. Such as reconstruction is doubtful for the reason that Breton has moved far in the direction of voicing historically voiceless sounds in these same (intervocal and some other) positions; this is the `new lenition' of Archaic Neo-Brittonic [s, θ, f, χ] to Modern Breton [z, z, v] and a sound resembling a weak [γ]. This development has no analogue in Welsh and is clearly not of Common Neo-Brittonic date. Further, it is not common in the languages of the world for the duration of consonant segments in absolute phrase-initial position to be phonologically significant. A Common Neo-Brittonic phonemic opposition of \textit{b-, g, d-} = /\textit{b:-, g:-, d:-}/ vs \textit{-p-, -c-, -t-} = /\textit{-b-, -g-, -d-}/ is accordingly unlikely, though a phonetic concomitant of duration was probably present. Anglo-Saxon borrowings of the \textit{Cerdic} type suggest that voicing in internal position was as old as the Migration Period, but it is likely that Brittonic speakers were still keeping the series distinct on the basis of \textbf{degree of voicing}. In support of this view one may adduce examples such as \textit{Landican} (Cheshire), Old English \textit{Archet} (now \textit{Orchard,} Dorset) corresponding to Welsh \textit{Argoed}, Breton \textit{Argoad}, or English \textit{Eccles} = Welsh \textit{eglwys}, in which the word-internal lenited voiceless stops are represented by voiceless sounds. On the other hand, radical \textit{b-, g-, d-} are never borrowed as voiceless.
 
 Koch \S30: To sum up, the probably phonetic reality of the Neo-Brittonic stop system in the immediate post-apocope period was a three-way opposition in which lenited [\bd, \gd, \dd] was distinct from radical [b-, g-, d-] by being voiceless and from radical [p\textsuperscript{h}-, k\textsuperscript{h}-, t\textsuperscript{h}-] by being unaspirated. This formulation requires no fifth-century phonological transformation to remake it from its Common Celtic precursor.
 
 \section{Rowland}
 
 From EWSP 338--339:
 
 A feature long noticed in early Irish and Welsh verse is the ability of mutated consonants to alliterate with their radical. It was first demonstrated for Welsh by Ifor Williams in \textit{Canu Aneirin} and has since been shown to be a feature of of Welsh poetry through the period of the Gogynfeirdd. Several theories have been put forward for the practice in Welsh and Irish. Ifor Williams suggests it is primarily an archaism dating from the period before initial mutations when poetic craft was established, as well as the extensive time before mutation was completed, when the radical and mutated forms showed less difference in pronunciation.
 
 \subsection{Schrijver}
Schrijver 1994 181-184: 

(1) It is a well-known fact that, beside clear traces of an absolute: conjunct opposition in verbal endings, Early Middle Welsh shows distinct traces of a system in which the negative or the first preverb of a verbal complex was followed by the spirant mutation in main clauses and by lenition in relative clauses (Strachan 1907; Evans 1976, 61-3), e.g.:

main clause: \textit{ny chenir} is not sung' (\textit{k}-)
\textit{ry chedwis} he kept' (\textit{k-})
rel. clause: \textit{ny bara} `who will not last' (\textit{p-})
\textit{ry gedwys} `who kept' (\textit{k}-)

The lenition in relative clauses may be compared with the lenition in Old Irish leniting relative clauses (caused by the particle *\textit{yo}), and, more importantly in view of the present discussion, the spirantization in main clauses may be compared with the absence of mutation after the negative and pretonic preverbs in Old Irish. Thus a MW form ny chan sings not' may be equated with OIr. ni cain `id.' According to the particle theory, the particle responsible for non-lenition in Irish (Cowgill's *(\textit{e})\textit{s}) is also responsible for the spirantization in Welsh.

Whereas this is a well-known fact, a different but similar correspondence between Irish and Welsh seems to have been almost completely ignored. In MW main clauses containing a verbal stem beginning with a vowel, the negation appears as \textit{nyt} (e.g. \textit{nyt af} `I do not go'; cf. e.g. Evans
1976, 173). In OIr., a pretonic preverb was followed by zero before a verb beginning with a vowel, the zero reflecting *-\textit{h}- < *-(\textit{e})\textit{s} according to Cowgill's particle theory. Thus we find MW -t- corresponding with OIr. -\textit{0}- ( < *-\textit{h}-). In Middle Breton, the negation ne appears as ned before forms of `to be' and `to go' starting with a vowel (\textit{ne d-oc'h ket} `you are not', \textit{ne d-in quet} `I shall not go'; Hemon 1975, 281). This corresponds
with OB \textit{nit, net}. The Breton evidence shows that the dental (MW \textit{t}, OB \textit{t}, MB \textit{d} < PC *\textit{t}) in this position goes back at least to Proto-British.

Similarly, the MW preverbal particle \textit{neu} ( + spirantization: \textit{neu cheint} `I have sung', BT. 19.1) appears as neut before vowels (\textit{neut athoed} `had gone', PKM. 9.20; see Evans 1976, 169-70). MW \textit{neu(t)} corresponds with OIr. \textit{no}-. Compare also the OW main clause forms \textit{rit pucsaun mi} `I should have desired', \textit{rit ercis} `he has required' (Evans 1976, 166), for the MW preverbal particle \textit{ry} which corresponds with OIr. \textit{ro}-. The upshot is that the particle, which in Irish has a reflex -0- before vowels, in British has a reflex -\textit{t}-.

The following objection to the comparison might be raised. In MW, the -\textit{t}- also occurs after the relative negative \textit{na} `that not' preceding verbs beginning with a vowel (\textit{nat erchis} `who did not require'; Evans 1976, 173). In OIr., the relative negative is \textit{nad}(-) < *\textit{ne-de}(-), \textit{nach}- < *\textit{ne-k\textsuperscript{w}e}- (Thurneysen 1946, 539), with the clitics *\textit{de} and *\textit{k\textsuperscript{w}e} (Watkins 1963, 25ff), not with Cowgill's *es. Consequently, it might be argued that since
MW relative \textit{nat} certainly is not the phonological correspondent of OIr.
relative \textit{nad}-, \textit{nach}-, there is no compelling reason to think that MW -\textit{t}- in
main clause \textit{nyt} must correspond with the Irish main clause particle.

I do not think that this objection has any value. In the first place, the
MB negative relative is \textit{nac} before vowels (Hemon 1975, 282; cf. OB
nac), which reflects *\textit{ne-k\textsuperscript{w}} < *\textit{ne-k\textsuperscript{w}e} and does correspond with OIr.
\textit{nach}-. It seems likely that since there is no trace of the clitic *\textit{de}
anywhere in the British verbal complex it has been lost, its territory being
taken over by either (non-relative) *\textit{t} or (relative) \textit{*k\textsuperscript{w}(e)}. Secondly, if
one asserts that British *\textit{t} originally appeared neither in main clauses nor
in subordinate clauses, its original locus is obscure. Surely a theory which
claims that *\textit{t} spread from a locus where the comparative British evidence
supports its antiquity, viz. main clauses, is to be preferred.

Hence I conclude that we have the following formal and functional
correspondence between Irish and British:

Ir. non-lenition of \textit{C}- : -\textit{0}- before \textit{V}- =

Br. spirantization of \textit{C}- : -\textit{t}- before \textit{V}-

Cowgill has established that the Pr.Ir. shape of the main clause particle
was *\textit{es}. In view of its British correspondent \textit{t} and the discussions in
sections 1-4 above, it may now be concluded that Pr.Ir. *\textit{es} reflects
*\textit{et} < *\textit{eti}. Obviously, the development *-\textit{t} > *-\textit{s} did not take place in
British (as was already suggested in section 6).

(2) As Watkins has shown (1963, 25ff), Pr.Ir. *\textit{de} and *\textit{k\textsuperscript{w}e}, both
occurring in the Irish verbal complex, were originally sentence connectives,
in complementary distribution (i.e. they cannot co-occur in one
verbal syntagm).

Cowgill's *\textit{es} is in complementary distribution with *\textit{de} and *\textit{k\textsuperscript{w}e}: we
generally do not find evidence of a reflex of *\textit{es-de}, *\textit{es-k\textsuperscript{w}e} or *\textit{de-es},
*\textit{k\textsuperscript{w}e-es}.

Seeming counter-examples to this distribution can be accounted for.
OIr. \textit{fritat}- `against' + 2 sg. inf. pron. class B seems to require * \textit{writ-de-h-tu},
with -\textit{h}- < *\textit{es}. However, it seems more likely that the non-lenited -\textit{t} of
the 2 sg. pronoun was taken from class A: *\textit{ro-h-tu} > OIr. \textit{rot}-. That this
is indeed the case is shown by e.g. the 3 pl. form of class B, \textit{frita}- < * \textit{writ-de-huh} < * \textit{writi-de-sons}, not **\textit{fritas}. < * \textit{writ-de-(e)s-suh} < * \textit{writi-de-essons}.
A similar explanation accounts for the unlenited pronoun in
\textit{nachit-beir} `who does not carry you' < *\textit{ne-k\textsuperscript{w}e-tu-beret(i) }rather than
*\textit{ne-k\textsuperscript{w}e-(e)s-tu- beret(i) }etc.

Another possible exception to the rule that *\textit{es} is in complementary
distribution with *\textit{de} is a number of present-tense conjunct forms of the
copula, which will be discussed in section 8.

Since the main clause particle is in complementary distribution with the
connectors *\textit{de} and *\textit{k\textsuperscript{w}e}, it seems likely that it is a connector too. This
would render the interpretation of the particle as copula (*\textit{esti}) or
anaphoric pronoun (*\textit{es}) difficult to maintain.

Hence the comparative evidence of Irish and British points to a main
clause particle *\textit{eti} ((1) above), and the distribution of this particle, which
is complementary with *\textit{de} and *\textit{k\textsuperscript{w}e}, points to the conclusion that the
particle (*\textit{eti}) is a connector ((2) above).

Both conclusions interlock perfectly. On the one hand, there is clear
evidence for a connector *\textit{eti} in PIE: cf. Lat. Umbr. \textit{et} `and', Goth. \textit{iþ}
'but, and', Gk `\textit{ἔτι}' further, moreover', Skt \textit{áti} (adv.) `beyond, very', and
probably Gaul. \textit{eti} `and, further?', \textit{etic} `and also?'<*\textit{eti-k\textsuperscript{w}e} (cf. e.g.
Pokorny 1949-59, 344). On the other hand, we may be able to explain
why *\textit{eti} occurs in main clauses. In Gothic, \textit{iþ} is always found at the
beginning of the sentence (see e.g. Streitberg 1919, 69; Braune and
Ebbinghaus 1973, 187, s.v. \textit{iþ}). In Latin et can be used in a very wide
range of contexts where it can co-occur with -\textit{que} and \textit{atque/ac}, but,
unlike -\textit{que} and \textit{atque/ac}, \textit{et} may be used to connect whole sentences or
items which do not belong closely together, and it may occur at the
beginning of a sentence (see e.g. Kühner and Stegmann ii, 1955, 3ff,
24ff). Thus, on the evidence of Gothic and Latin, we may conclude that
*\textit{eti }is a relatively `loose' connector (in comparison with other connectors)
and connects whole sentences, which are presented to the hearer as being
on a par, without the clause following *\textit{eti} being in any way subordinated
to the preceding sentence. Given the fact that *\textit{eti} for some reason
became productive and finally even obligatory in most Insular Celtic main
clauses, we can now at least explain why it came to be a marker of main
rather than subordinate clauses.


\textcite[37--38]{thomas_tafodiaith_1993}: Cynenir ffrwydrolion di-lais y dafodiaith hon --- fel rhai tafodieithoedd eraill y Gymraeg --- \^a chryn ynni. Maent felly'n ffrwydrolion di-lais, ffortis, anadlog.

[...] Seiniau lenis yw'r ffrwydrolion lleision /b, d, g/, felly, ac er bod anadliad iddynt, y mae'n lleisiol a llawer llai hyglyw nag anadliad y ffrwydrolion di-lais. Er mai fel seiniau lleisiol y rhestrir /b, d, g/, felly, ac er bod anadliad iddynt, y mae'n lleisiol a llawer llai hyglyw nag anadliad y ffrwydrolion di-lais. Er mai fel seiniau lleisiol y rhestrir /b, d, g/, tecach fyddai eu hystyried yn seiniau lenis yn syml, oblegid fel y nodir yn yr adran ar Amrywiadau allfonig, nid yw'r lleisio byth yn gryf a gallant, yn \^ol eu safle, fod yn rhannol leisiol yn unig. 

\textcite[37--38]{thomas_tafodiaith_1993}:
Yn y safleoedd dechreuol a diweddol, y ,ae'r ffrwydrolion di-lais yn anadlog iawn ond yr anadliad yn wannach ynghanol gair.

Un nodwedd ar y Gymraeg yw na chynhyrchir lleisio cryf wrth gynanu cytseiniad lleisiol; lenis ydynt yn fwy na lleisiol, a phan fo'r ffrwydrolen leisiol ar ddechrau neu ddiwedd gair, rhannol leisiol ydyw ar y gorau, gan nad yw'r tannau llais yn dechrau dirgrynu ar yr unpryd ag y dechreuir cynhyrchu ffrwydrolen ddechreuol, a bydd y lleisio'n darfod cyn y gorffennir cynanu ffrwydrolen ddiweddol. Gan hynny, y mae rhan gyntaf y ffrwydrolen ddiweddol [-b\bd]; e.e. [\bd bod\dd]; [\bd bɪr; pob\bd] [\bd dod\bd]; [\bd