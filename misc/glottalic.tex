\section{Proto-Indo-European evidence on voice and aspiration}
Proto-Indo-European linguistics saw a new turn with the Glottalic theory in the seventies. This theory states that what is traditionally identified as the voiced stop series was in fact a series of glottal consonants.  Table~\ref{bomhardcomparison} offers a comparison of the different phonological systems.

This model has several typological advantages over the traditional model~\autocite[377]{bomhard_glottalic_2016}. The first is that /b/ (in the old model) is extremely rare among the Proto-Indo-European vocabulary. This makes little sense typologically. If a language has different stop series contrasting on voice, and these series contain labials, then the voiceless variant is typically lacking rather than the voiced one. This contrasts with the Glottalic Model, where lack of  /p'/ is typologically precedented. Typology may similarly account for why traditional plain voiced stops were so rare in inflectional affixes and pronouns, as this is also found present-day natural languages. Also, Proto-Indo-European had some root structure constraint laws. For example, a root such as **\textit{bed} could not exist, because two voiced stops may not co-occur. This is typologically implausible. By contrast, languages with ejectives often show a constraint against co-occurrence of two ejective consonants.


\begin{table}[h]
\centering
\begin{tabular}{@{}lllllll@{}}
\toprule
\multicolumn{3}{c}{\textbf{Lehmann}} & \textbf{} & \multicolumn{3}{c}{\textbf{Gamkrelidze}} \\ \midrule
b & bʰ & p & = & p’ & bh/b & ph/p \\
d & dʰ & t & = & t’ & dh/d & th/t \\
g & gʷʰ & k & = & k’ & gh/g & kh/k \\
gʷ & gʷʰ & kʷ & = & k’\textsuperscript{u̯} & g\textsuperscript{u̯}h/g\textsuperscript{u̯} & k\textsuperscript{u̯}h/k\textsuperscript{u̯} \\ \bottomrule
\end{tabular}
\caption{Comparison of traditional (left) and glottalic stop systems (right) in Proto-Indo-European \autocite[376]{bomhard_glottalic_2016}}
\label{bomhardcomparison}
\end{table}

In addition to these typological considerations, some Germanic and Armenian consonant shifts that look awkward within the traditional model may be viewed as archaic instead.
\subsection{what this means for Celtic}

For Celtic, the Glottalic Model means that we may naturally arrive at a contrast between aspirated and unfeatured stops based on one simple step: the merger between what would traditionally be called the voiced aspirates and plain voiced stops simply involved the loss of secondary features in both series. \Textcite{bomhard_glottalic_2016}

\begin{figure}[h]
\newcommand{\lijn}[3]{%
\draw (#1,0) node[below] {*#3\vphantom{g}} -- (#1,1) node[above]{*#2\vphantom{g}};%
}
\begin{center}
\begin{tabular}{@{}c@{}}
\toprule
\begin{tikzpicture}
\lijn{0}{pʰ}{\zero}
\lijn{1}{tʰ}{tʰ}
\lijn{2}{kʰ}{kʰ}
\lijn{3}{kʷʰ}{kʷʰ}
\draw (4.5,0) node[below]{*d\vphantom{g}} --  (4,1) node[above]{*d\vphantom{g}};
\draw (4.5,0)  --  (5,1) node[above]{*t\vphantom{g}};
\draw (8,0) node[below]{*g\vphantom{g}} -- (6,1) node[above]{*g\vphantom{g}};
\draw (8,0) -- (7,1) node[above]{*k\vphantom{g}};
\draw (8,0) -- (8,1) node[above]{*gʷ\vphantom{g}};
\draw (8,0) -- (9,1) node[above]{*kʷ\vphantom{g}};
\draw (9.5,0) -- (9,1);
\draw (9.5,0) node[below]{\vphantom{g}*b} -- (10,1) node[above]{\vphantom{g}*b};
\end{tikzpicture}\\
\bottomrule
\end{tabular}
\end{center}
Notes:
\begin{enumerate}
\item The earlier dental and velar plain (unaspirated) voiceless stops (*\textit{t} and *\textit{k}) merge completely with
the plain voiced stops (*\textit{d} and *\textit{g}) in Pre-Proto-Celtic.
\item Next, the voiced labiovelar *\textit{gʷ} is delabialized and merges with *\textit{g}.
\item Then, the plain (unaspirated) labiovelar *\textit{kʷ} develops (A) into *\textit{b} initially and medially after consonants
and (B) into *\textit{g} initially before *\textit{u} and medially between vowels and before consonants.
\item Original *\textit{pʰ} is lost in all of the Celtic languages: *\textit{pʰ→*h→*Ø}. However, p has been reintroduced
into Old Irish through loanwords.
\end{enumerate}

    \caption{Proto-Indo-European to Proto-Celtic developments~\autocite[377]{bomhard_glottalic_2016}.}
    \label{bomhardpc}
\end{figure}
