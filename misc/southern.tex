
\chapter{Limited lenition as a Southern Welsh feature}
Van Sluis \parencite*{van_development14} found that postverbal non-lenition of voiceless stops is found in the White Book recensions of \mw{Culhwch ac Olwen} and \mw{Pwyll Pendeuic Dyuet}. Notable other texts researched were the White Book recension of \mw{Branwen Uerch Lyr} and the Red Book recension of \mw{Owein}. These texts do not show a similar pattern of postverbal non-lenition. Since this work focused on the chronology of postverbal lenition, the difference was perceived to be a chronological difference. 

Beyond this, there is the evidence from Breton showing that lenited voiceless stops were kept separate from unlenited voiced stops through length distinctions\footnote{see also: section \ref{kochvoiceless}}. Southern Welsh shares in some linguistic innovations with Southwestern British, i.e.\ Cornish and Breton. An example of a linguistic innovation shared between Southern Welsh and Southwestern British is double i-affection, which is not found in Northern Welsh\footnote{An example of this is Northern \mw{bedydyaỽ} `baptise' (RBH col.\ 810, l.\ 14), vs.\ Southern \mw{bẏdẏdaỽ} (WBR col. 452, l.\ 19); see also \cite[§~165]{jackson_language_1953}.}. A big difference between the innovation of secondary i-affection and the preservation of a difference between lenited voiceless stops and unlenited voiced stops is that the former is a shared innovation between S Welsh and SW Brittonic while the latter is a shared archaism. 

While it is true that the Red Book recension of \mw{Owein} is a much younger-looking text than the other texts mentioned, this is by no means proven for \mw{Branwen Uerch Lyr}\todo{Short overview on discussion on the origins of Branwen}. In addition, the White Book recensions of \mw{Manawydan Uab Llyr} and \mw{Math Uab Mathonwy} do not show a pattern of limited lenition either, as is shown in Table \ref{manawydanmath}:

\begin{table}[h]
  \centering
  \begin{tabular}{ll|l|ll}
    \textbf{col.} & \textbf{line} & \textbf{gloss} & \textbf{lenited} & \textbf{consonant} \\ \hline
    68  & 6  & \mw{llunẏei}   \mw{uanaỽẏdan}  & yes    & m     \\
    68  & 7  & \mw{gỽniei}   \mw{prẏderi}   & no    & p     \\
    69  & 10  & \mw{torrei}   \mw{gẏuarth}   & yes    & c     \\
    74  & 35  & \mw{bei}     \mw{lẏgoden}   & yes    & ll    \\
    80  & 2  & \mw{bei}     \mw{ueichaỽc}   & yes    & b     \\
    81  & 28  & \mw{bẏdei}    \mw{uẏỽ}     & yes    & b     \\
    82  & 1  & \mw{allei}    \mw{gẏlchu}    & yes    & c     \\
    85  & 12  & \mw{bẏdei}    \mw{gỽbẏl}    & yes    & c     \\
    92  & 23  & \mw{guelei}   \mw{bleid}    & no    & b     \\
    100 & 12  & \mw{allei}    \mw{llaỽer}    & no    & ll    \\
    101 & 23  & \mw{glẏỽei}   \mw{lef}     & yes    & ll    \\
    103 & 5  & \mw{oes}     \mw{gẏnghor}   & yes    & c     \\
    105 & 2  & \mw{gallei}   \mw{uot}     & yes    & b    
  \end{tabular}
  \caption{Lenition after \mw{oes} and verbs ending in \ei\ in the White Book recensions of \mw{Manawuydan Uab Llyr} and \mw{Math Uab Mathonwy}.}
  \label{manawydanmath}
\end{table}

Similarly to what Table \ref{deudwy} shows after \mw{deu, dwy} in \mw{Culhwch ac Olwen}, Table \ref{manawydanmath} shows that lenition is not completely exceptionless in these branches, but it is not conditioned by initial consonant of the following word either. These two branches are not typically held to be significantly younger than the others\todo{Check!}, so absence of limited postverbal lenition argues against the idea that it is a purely chronolectal phenomenon. Perhaps, then, it is a dialectal phenomenon. Elements in \mw{Culhwch ac Olwen} demonstrate `detailed knowledge of the geography of south Wales, from Pembrokeshire, to the Severn estuary'~\parencite[60]{rodway_where_2007}. \mw{Pwyll Pendeuic Dyuet} is a tale about the prince of Dyfed, so its place of origin may equally well be southerly. This contrasts with the remaining tree branches, which are likely to have been composed in Gwynedd~\parencite[58]{rodway_where_2007}. Perhaps, then, limited postverbal lenition was a dialectal feature of Southern Welsh. This hypothesis deserves some more digging into.
\section{Llyfr Du Caerfyrddin}
The Black Book of Carmarthen has some verbs ending in \ei. This manuscript is the earliest surviving manuscript written \todo{wholly? partly?} in Welsh. Beside this, it comes from the Southerly place of Carmarthen, hence its name. This makes the book a suitable place to look for limited postverbal lenition after \ei. Lenition is not written consistently in this manuscript, but alliteration patterns may elucidate upon the quality of consonants following \ei. All relevant instances will be discussed here individually. 

I follow Jarman's edition of the manuscript in the orthography of the examples given below, and order them in subsections according to Jarman's edition~\parencite{jarman_llyfr_1982}.

\subsection{Dadl y Corff a'r Enaid}
\mwcc[veigyuerkinan]{f. 9v. l.4}{Ac vei gyuerkinan}{And if it were a ?witty direction?}
\mw{Fai} causes lenition to a voiceless stop here. Isaac: \verb|ac %fai %gyfergynnan|. However, \mw{kinan} is spelt with a \mw{k}. If \mw{gyuer} is supposed to alliterate with \mw{kinan}, it may need to be emendated. Isaac emendates by considering the word a compound, but seeing as how it is a noun following an adjective, this is not necessary.

\mwcc[veipaup]{f. 9v. l.8}{Ac vei. vei. paup}{And it would, if it would be for all.}
Here, \mw{vei} does not cause lenition to \mw{paup}. Note, however, Isaac's emendation: \verb|ac %fai [i] %bawb|.

\mwcc[dyhaetei]{f. 12v. l.1}{n\y m dyhaetei alar}{Grief would not reach me.}
Isaac: \verb|ni 'm dyhaeddai \alar|. This seems to be a clear example of lenition. Note, however, how there is also the word \mw{alar} `boredom, sorrow'.
\subsection{Englynion y Beddau}
\mwcc[e27g]{Eng. 27}{ny bitei gur. y breinhin}{There would be no man to the king}
Isaac: \verb|ni byddai \w@^r i %free@"nin|. Isaac emendates \mw{y} as MoW \textit{i}, which causes lenition to \mw{breinhin}. This is not written, so \mw{gur} is not a reliable counterexample.

\mwcc[e27t]{Eng. 27}{divei ny ochelei. trin.}{faultless, would not shun battle}
Isaac: \verb|difai ni \ochelai %drin|. This is the same \mw{englyn} as \ref{e27g}, so it must be considered equally undependable.

\mwcc[e31]{Eng. 31}{ny bitei drimis heb drin}{There would be no three months without battle}
Isaac: \verb|ni byddai %drimis heb %drin|. \mw{Drimis} is a temporal adverb to \mw{bitei}. This type of lenition is different from postverbal lenition \parencite[12]{van_development14}

\mwcc[e33]{Eng. 33}{a lyviasei luossit}{that would have ruled hosts.}
Isaac: \verb|a %lywiasai %luosydd|. Note that this line is a fully-fledged \mw{cynghanedd groes}. Both spelling and alliteration with \mw{lyviasei} suggest that \mw{luossit} should be lenited.

\mwcc[e45]{Eng. 45}{Dyliei kynon yno y kiniav}{Cynon there would be entitled to dinner}
Isaac: \verb|dylyai EP%Gynon yno ei %giniaw|. Orthography does not show lenition, but this needs to be checked with alliteration with \mw{kiniav}, whose lenition depends on the value of \mw{y}. 

\mwcc[e55]{Eng. 55}{gur a digonei da ar y arwev}{[translation]}
Isaac: \verb|gw@^r a $ddigonai $dda ar ei arfau|. \mw{Digonei} is lenited without question, so alliteration would suggest lenited \mw{da}.

\mwcc[e59l]{Eng. 59}{ny dodei lew ar ladron}{He would not grant a voice to thieves.}
Isaac: \verb|ni dodai %lef ar %ladron|. Lenition is presented orthographically. Lenited \mw{ladron} confirms lenition of \mw{lew}.

\mwcc[e59g]{Eng. 59}{ny rotei gwir y alon}{He would not give truth to enemies}
Isaac: \verb|ni %roddai \wir i \alon|. Isaac emends to a lenited form of \mw{gwir} here. This makes sense given lenited \mw{alon}, but at the same time rises the question why the scribe wrote lenition differently twice in the same line.

\mwcc[e60]{Eng. 60}{ae clathei caffei but}{And he would bury it, would have spoils}
Isaac: \verb|a 'i clathai caffai %fudd|. Lenition is not written here. Alliteration patterns make no suggestion either way. Isaac nevertheless emends to the lenited form.

\mwcc[e66]{Eng. 66}{dygirchei tarv trin ino treis}{A bull would visit battle there [and] violence}
Isaac: \verb|dygyrchai tarw trin yno %drais|. The unlenited form is written here, and it is also preferable for alliteration with \mw{tarw}. However, \mw{trin} may also alliterate with \mw{treis}, and possibly with \mw{treis} only. This would be possible if lenition after inserted \mw{ino} did not operate yet.

\subsection{\textit{Gereint fil' Erbyn}}
\mwcc[crist]{f. 37r. ll.~8-9}{rotei crist a arched}{[translation]}
Isaac: \verb|rhoddai EPCrist a arched|. No sign either way from alliteration. Isaac does not add lenition. However, lenition is not written in the next line, for example: \mw{mirein prydein} `Beautiful Britain', where one would expect lenition following a preposed adjective.

\subsection{Pa \^{W}r Yw'r Porthor?}
\mwcc[guaed]{f. 47v. ll.~12-13}{maglei guaed ar guelld}{Spotted the grass with blood}
Isaac: \verb|maglai \waed ar \wellt|. The lenited form is preferable here, so that it may alliterate with \mw{guelld}, which is lenited following \mw{ar}.

\mwcc{f. 48r. ll.~1-2}{trae llathei pop tri.}{While he killed every third person.}
Isaac: \verb|tra 'u llathai %bob tri|. There is no way to check alliteration, and orthography is unreliable in this poem, as seen in example \ref{guaed}.

\mwcc{f. 48r. ll.~3}{As eirolei kei}{Cai, as long as he hewed down.}	
Isaac: \verb|as eiriolai EP%Gai|. \mw{Kei} may alliterate with the next line: \mw{hid trae kymynhei}.

\mwcc{f. 48v. l.8}{N\y\ bei duv ae digonhei}{There was no day that would satisfy him.}
Isaac: \verb|Ni bai Duw a 'i digonai|. \mw{Duv} alliterates with \mw{digonhei}. Infixed pronoun \mw{'y} does not typically cause lenition \parencite[§58]{evans_grammar_1964}. 

\subsection{Ymddiddan rhwng Gwyddneu Garanhir a Gwyn ap Nudd}
\mwcc{f. 50r. ll.~3-4}{m\y gedaul. kein a d\y gei treis.}{[translation]}
Isaac: \verb|mygedawl %gain a $ddygai %drais|. The manuscript gives the unlenited form, which alliterates with the line before: \mw{aessaur brihuid. torrhid eis.}

\mwcc{f. 50r. ll.~5-6}{kint \y\ sirthei kadoet rac carnetaur d\y\ ueirch}{[translation]}
Isaac: \verb|cynt y syrthiai cadoedd rhag carneddawr dy $feirch|. Unlenited \mw{kadoet} is confirmed by unlenited \mw{carnetaur}.

\subsection{Dau Ddarn o Chwedl Trystan}
\mwcc{f. 50v. ll.~10-11}{p\y r toei wanec carrec camhur}{[translation]}
Isaac: \verb|pyr toai \waneg %garreg camwr|. The lenited form is given, but emending \mw{g} would give alliteration with lenited (following feminine \mw{wanec}) \mw{carrec}.


\subsection{\textit{Enwev Meibon Llywarch Hen}}
\mwcc[icalch]{f. 54r. ll.~7-8}{briwei calch hen}{He used to fracture the armour of (Llywarch) Hen.}
Isaac: \verb|briwai %galch [mab EPLlywarch] Hen|. Lenition is not represented in the orthography here, but this may indeed be due to orthography only. Although lenition is written in this poem, lenition of voiceless stops is not written consistently. Compare, for example, \mw{ar perwit pren} `on the apple tree's wood' where lenition following \mw{ar} is not shown.
\section{Llyfr Blegywryd}

Another Southern Welsh text that may contain evidence is Llyfr Blegywryd, so the Cambridge Trinity College MS. O.7.1 redaction of this text was chosen to see if a pattern of limited postverbal lenition may be found here. The manuscript contains a copy of \mw{Cyfraith Hywel Dda}, in a version also called the `Demetian code'. It is a sensible assumption to think that a law text tailored for the Southern region of Dyfed also used a more southerly language. 

Besides, its scribe and place of composition is known, giving a fairly accurate date and place of composition of this Demetian code. It was written down by Gwilym Wasta (Was Da) in Dinefwr Castle, in present-day Carmarthenshire in the beginning of the fourteenth century \parencite[429]{owen_gwilym_1980}. 

\subsection{\ei}
Here, all seven instances of \ei\ followed by a lenitable consonant are given and discussed.

\mwcc{LlB~1v.~l.~14}{megys y caffei teir ran kymry. Gỽyned. Powys. Deheubarth.}{Similarly, there would be three parts of Wales, Gwynnedd, Powys, and Deheubarth.}

Here, \mw{teir} is not lenited following \mw{caffei}, but this instance may be disregarded because \mw{teir} is a numeral, and numerals are known not to be lenited anyway. \parencite[30-31]{van_development14}

\mwcc[dylyeitygubeikenedlawc]{LlB~27v.~l.~3}{namyn rodet e hunan y lỽ yg kyfeir pob dyn or a dylyei tygu gyt ac ef bei kenedlaỽc}{except the oath is to be given itself in the presence of every man of those who are entitled to swear with him and whoever may be noble.}

Note that the example above has two examples: \mw{dylyei tygu} amd \mw{bei kenedla\w c}. This leaves two examples of non-lenition of voiceless stops after \ei.

\mwcc[gaheigynt]{LlB~30r.~l.~16}{herwyd gỽyr y deheu hi a gahei gynt y hegỽedi yn hollaỽl.}{According to the men of the south, she would obtain her marriage gift fully earlier.}

Adverbial use of \mw{gynt} causes lenition here, or more precisely: \mw{gynt} is a temporal adverb to \mw{gahei}, and may be considered an instance of lenition of apposition, where the temporal adverb stands in apposition to the past tense found in \ei\ \parencite[11--13]{van_development14}. This example must be discounted as a counterexample to limited postverbal lenition.

\mwcc[ueiuarw]{LlB~37r.~l.~5}{gan y ofyn trỽy yr vn gyt etiued hỽnnỽ a uei uarỽ heb etiued idaỽ oe gorff}{By request through the  same co-heir who would die without heir to him for his capital.}

Here, lenition is found following \ei, to a consonant other than a voiceless stop. 

\mwcc{LlB~58r.~l.~14}{Eil yỽ y uot yn gyfrannaỽc ar yr hyn y bo y ohonaỽ pei gellit y hennill trỽy varn.}{The second [rule?] is that he is responsible for this that may be from him, if it may be won through judgment.}

Use of \mw{pei} as a conjunction rather than as an inflected verb invalidates it as an example for the use of postverbal lenition, as found by Van Sluis \parencite*[23]{van_development14}. 

\mwcc{LlB~59r.~l.~6}{vn yỽ pan diuarnho y kyfryỽ a varnassei gynt yn y gyffelyb achaỽs}{One is when he judges the same that he had judged before in a similar circumstance.}

In this example, \mw{gynt} is a temporal adverb that stands in apposition to \mw{varnassei}, and is therefore lenited. This example is therefore discounted, similarly to example \ref{gaheigynt}.

All in all, examples \ref{dylyeitygubeikenedlawc} and \ref{ueiuarw} give a total of three usable instances of postverbal lenition. These few examples do confirm the hypothesised pattern of limited postverbal lenition, but are limited in number.
\subsection{\oes}
Here, all eleven instances of \oes\ followed by a lenitable consonant are given:
\mwcc{LlB~3v.l.~22}{Gỽedy hỽnnỽ nyt oes le dilis or parth hỽnnỽ
}{After this there is no free place from this area.}

\mwcc{LlB~23r.~l.~22}{nyt oes werth kyfreith arnaỽ. 
}{It does not have legal value.}

\mwcc{LlB~36v.~l.~22}{Odyna nyt oes priaỽt ran ar tir
}{Thenceforth, there is no proper share of the land.}

\mwcc{LlB~37r.~l.~1}{nyt oes warthal gan dewis.}{There is no payment with choice.}

\mwcc{LlB~42r.~l.~5}{canyt oes kenydyl yr alltut y galler galanas arnunt
}{Because there is no tribe of foreigners on whom a blood-fine may be paid}

\mwcc{LlB~43v.~l.~22}{canyt oes tir vdunt. 
}{Because there is no land from them.}

\mwcc{LlB~49r.~l.~22}{Nyt oes werth kyfreith ar gnyỽ hỽch hyt ym pen y ulỽydyn
}{There is no legal value on a young boar until the end of one year.}

\mwcc{LlB~56v.~l.~19}{Ac nyt oes werth gossodedic yg kyfreith hywel ar aelaỽt a gỽaet dyn eglỽyssic.}{And there is no appointed value on the member and blood of a churchly man in the law of Hywel.}

\mwcc{LlB~58v.~l.~8}{kanyt oes werth kyfreith ar y tauaỽt}{Because there is no legal value on the tongue}

All examples of lenition following \oes\ are valid instances of postverbal lenition, and the expected pattern of limited postverbal lenition is followed throughout. 

\subsection{\ei\ and \oes\ together}

If one counts instances of \ei\ and \oes\ in Llyfr Blegywryd together, the following result is found:

\begin{table}[h]
\centering
\label{blegywryd}
\begin{tabular}{r|ll}
               & \ei & \oes \\ \hline
T\textsuperscript{-voice} & 0/2 & 0/4  \\
¬T\textsuperscript{-voice} & 1/1 & 7/7 
\end{tabular}
\caption{Lenition after \ei\ and \oes\ in \mw{Llyfr Blegywryd}, broken down by initial consonant type of the following word.}
\end{table}

It is essential here to consider both \ei\ and \oes\ in order to come to a valid conclusion on whether limited postverbal lenition operated as expected. On their own, instances of \ei\ are insufficient in number to provide a convincing case that lenition depended on consonant type. Limited lenition after \oes\ is found more abundantly, but the lack of phonological distinction between voiced and voiceless stops after \mw{s} might lead one to consider lack of lenition due to phonotactic reasons only. Together, however, they point in the same direction i.e.\ towards the expected pattern of limited postverbal lenition.

So far, then, the pattern is found in two manuscripts i.e.\ the White Book of Rhydderch and \mw{Llyfr Blegywryd}. These manuscripts contain texts of different genres i.e.\ narrative prose and legal prose, respectively. The fact that limited postverbal lenition is found across these different genres in these different places suggests that limited postverbal lenition was probably not just an orthographical peculiarity in the law texts.

\subsection{\mw{dros}}
By way of comparison, lenition after \mw{dros} in the same corpus is given here in Table \ref{drosblegywryd}. This table contains all instances of a lenitable consonant immediately following \mw{dros}, with the exception of numerals, and \mw{pob}, as these word are found to lenite or fail to lenite due to unrelated reasons \parencite[23--24,~30--31]{van_development14}

\mw{Dros} causes lenition, and it ends in \mw{s}. If limited postverbal lenition after \oes\ is a property of postverbal lenition only, then every consonant including voiceless stops may be expected to be lenited. If, by contrast, lenition after \oes\ is limited to consonants other than voiceless stops due to phonotactic properties, then the same pattern of lenition of consonants other than voiceless stops only may be expected. 

\begin{table}[h]
\centering
\begin{tabular}{ll|l|ll}
\textbf{folio} & \textbf{line} & \textbf{gloss}    & \textbf{lenited} & \textbf{consonant} \\ \hline
7r       & 5       & \mw{dros alanas}   & yes       & g     \\
7r       & 6       & \mw{dros alanas}   & yes       & g    \\
7r       & 10      & \mw{dros alanas}   & yes       & g        \\
13r      & 18      & \mw{dros talu}    & no        & t          \\
13v     & 21   & \mw{dros werth}  & yes    & g        \\
19v      & 23      & \mw{dros gyghellaỽr} & yes       & c             \\
22v      & 3       & \mw{dros ki}     & no        & c             \\
25r      & 24      & \mw{dros padell}   & no        & p             \\
25v      & 1       & \mw{dros wer}    & yes       & g             \\
25v      & 3       & \mw{dros uỽyt}    & yes       & b             \\
25v      & 4       & \mw{dros uỽyt}    & yes       & b             \\
25v      & 5       & \mw{dros uỽyt}    & yes       & b             \\
27r      & 8       & \mw{dros ỽdyf}    & yes       & g             \\
33r      & 2       & \mw{Dros ouyssyaỽ}  & yes       & g             \\
33r      & 3       & \mw{Dros cussan}   & no        & c             \\
34r      & 14      & \mw{dros westua}   & yes       & g             \\
42r      & 2       & \mw{dros alanas}   & yes       & g             \\
48r      & 13      & \mw{Dros loscỽrn}  & yes       & ll             \\
49v      & 10      & \mw{Dros uaed}    & yes       & b             
\end{tabular}
\caption{Lenition after \mw{dros} in the Cambridge Trinity College MS. O.7.1 redaction of \mw{Llyfr Blegywryd}}
\label{drosblegywryd}
\end{table}

Table \ref{drosblegywryd} shows a general pattern of lack of lenition of voiceless stops after \mw{dros}, but this pattern is not exceptionless, as f.~19v.~l.~23 contains lenited \mw{gyghellaỽr} after \mw{dros}. On the basis of this table, we may not safely state that lenition after \oes\ was limited by anything else than phonotactic considerations.

This does not detract from finding that limited postverbal lenition occurs after \ei\, however. Phonotactic properties cannot have played a role, because the consonant to be lenited after \ei\ naturally follows a vowel. Example \ref{ueiuarw} has \mw{oe gorff} `from his body', and lenition is applied here as normally following a vowel. 

\section{Llyfr Iorwerth}

The Northern Welsh law text, \mw{Llyfr Iorwerth}, serves as a valuable point of comparison with \mw{Llyfr Blegywryd}. It is more traditionally called the `Venedotian code', referring to the Northerly province of Gwynedd. The hypothesis is that limited postverbal lenition is a Southern Welsh feature, so one should expect lenition after \ei\ and \oes\ to occur in all instances.

For comparison, the NLW MS. Peniarth 35 redaction is chosen as the source for the data below. There is an older manuscript containing the same text, namely NLW MS. Peniarth 29, also called the Black Book of Chirk. This book, however, barely writes lenition of voiceless stops at all, and is discussed at \ref{chirk}. 

\subsection{\ei}
Table \ref{eiiorwerth} shows limited postverbal lenition after \ei, similarly to what is seen in Llyfr Blegywryd. This goes against the hypothesis that limited postverbal lenition was an exclusively Southern Welsh feature.
\begin{table}[H]
\centering
\begin{tabular}{ll|l|ll}
\textbf{folio} & \textbf{line} & \textbf{gloss}  & \textbf{lenited} & \textbf{consonant} \\ \hline
22r            & 18            & \mw{allei talu}      & no               & t                  \\
36v            & 16            & \mw{uei well}        & yes              & g                  \\
39r            & 8             & \mw{hadawei kyhyded} & no               & c                  \\
85r            & 13            & \mw{dylyei cadỽ}     & no               & c                 
\end{tabular}
\caption{Lenition following \ei\ in the NLW MS. Peniarth 35 redaction of \mw{Llyfr Iorwerth}}
\label{eiiorwerth}
\end{table}

\mwcc[hadaweikyhyded]{LlI~f39r.~l.~8}{Bei yn gystal y hadawei kyhyded oed a rannu deu hanner.}{If what he would leave would be equal, it would be an equal [partition], and two halves are shared.}

Although it is not immediately clear from the manuscript text, \mw{kyhyded} does not stand in the same clause as \mw{hadawei}, so example \ref{hadaweikyhyded} must be considered a research exception. This nevertheless leaves three examples showing that lenition was not applied to voiceless stops, but was applied to other consonants. 

\subsection{\oes}

As seen in table \ref{oesiorwerth}, lenition after \oes\ does follow a pattern of limited postverbal lenition, similarly to what is found in Llyfr Blegywryd.

\begin{table}[H]
\centering
\begin{tabular}{ll|l|ll}
\textbf{folio} & \textbf{line} & \textbf{gloss} & \textbf{lenited} & \textbf{consonant} \\ \hline
23r            & 24            & \mw{oes kynnogyn}   & no               & c                  \\
23r            & 25            & \mw{oes uach}       & yes              & m                  \\
89v            & 6             & \mw{oes kyflauan}   & no               & c                  \\
98r            & 17            & \mw{oes ureint}     & yes              & b                  \\
107r           & 24            & \mw{oes werth}      & yes              & g                  \\
109r           & 18            & \mw{oes claỽr}      & no               & c                  \\
114r           & 23            & \mw{oes kenedyl}    & no               & c                 
\end{tabular}
\caption{Lenition following \oes\ in the NLW MS. Peniarth 35 redaction of \mw{Llyfr Iorwerth}}
\label{oesiorwerth}
\end{table}
\subsection{\ei\ and \oes\ together}

Table \ref{iorwerth} summarises the finding that limited postverbal lenition is also found in \mw{Llyfr Iorwerth}, a Northern Welsh text. 

\begin{table}[h]
\centering
\begin{tabular}{r|ll}
               & \ei & \oes \\ \hline
T\textsuperscript{-voice} & 0/3 & 0/4  \\
¬T\textsuperscript{-voice} & 1/1 & 3/3
\end{tabular}
\caption{Lenition after \ei\ and \oes\ in \mw{Llyfr Iorwerth}, broken down by initial consonant type of the following word.}
\label{iorwerth}
\end{table}