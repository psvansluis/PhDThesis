\begin{welsh}
\chapter{Adroddiad uwchraddio i PhD Paulus van Sluis}
Fy mhwnc yw'r ffrwydrolion di-lais mewn Cymraeg Canol, a sut a than pryd y cedwid y ffrwydrolion di-lais treigledig (\lT) \^a’r ffrwydrolion lleisiol cysefin (\xD) ar wah\^an. Nid unodd y ddwy gyfres hyn tan y cyfnod hanesyddol~\autocite{koch_*cothairche_1990}. Mae system sy'n cadw \lT\ a \xD\ ar wah\^an yn dal i fodoli yn Llydaweg \autocite{falchun_systeme_1951}. Mae nifer o awduron wedi trafod sut y byddai'r system o ffrwydrolion cyn i \lT/\xD\ uno wedi gweithio o ran seineg~\autocite{koch_*cothairche_1990,harvey_aspects_1984,schrijver_old_2011}. Bydd rhagymadrodd fy nhraethawd yn trafod hanes y syniadau hyn. Nid oes  mwy o lenyddiaeth am y ffaith nad unodd y cyfresau hyn tan y cyfnod hanesyddol. Serch hynny, gallaf drafod erthyglau eraill am hanes y treiglad meddal~\autocite{martinet_celtic_1952,sims-williams_dating_1990}, \textit{gemination}~\autocite{greene_gemination_1956}, a'r cysylltiadau \^a datblygiad y treiglad llaes~\autocite{schrijver_spirantization_1999,isaac_chronology_2004}.

Wedyn, mae'r cwestiwn ymchwil yn rhannu'n naturiol yn  ddwy ran: sut yr unodd y ddwy gyfres yn yr iaith lafar, a sut yr unodd y ddwy gyfres yn y llawysgrifau. Cwestiwn seinegol a ffonolegol yw'r cwestiwn cyntaf; Cwestiwn orgraffyddol yw'r ail. Ceisiais ateb y cwestiwn cyntaf drwy ddadansoddi orgraff eithafol Hen Gymraeg, a thrwy batrymau a welir yn y gynghanedd gynnar. Ceisiais ateb yr ail gwestiwn drwy gasglu tystiolaeth: edrychais ar amrywiaeth o destunau i weld ble peidwyd \^a dangos treigladau. Bydd y traethawd ymchwil hefyd yn rhannu'n naturiol rhwng ffonoleg ac orgraff.

Bydd fy rhagymadrodd hefyd yn cynnwys llenyddiaeth i osod yr ail ran yn ei chyd-destun: mae yna gynseiliau sy'n dangos i ni sut y gallwn gasglu olion arloesi'r orgraff a'r iaith i greu offer i ddyddio a lleoli testunau. O ran orgraff, astudiwyd ymddangosiad sandhi \mw{h} yn fanwl gan \textcite{sims-williams_spread_2010}, ac astudwyd datblygiad orgraff \mw{m} treigledig gan \textcite{russell_rowynniauc_2003}, er enghraifft. O ran iaith, gallaf gyfeirio at waith gan Nurmio~\autocite*{nurmio_middle_2014,nurmio_studies_????} sy'n trafod datblygiad sawl ffurf luosog, neu at waith gan Rodway~\autocite*{rodway_datable_1998,rodway_absolute_2002,rodway_two_2003,rodway_dating_2013}, sy'n trafod olion arloesi yn y system ferfol. Cyfeiriaf yn sicr at \textcite{rodway_where_2007}, sy'n cyflwyno arolwg o'r dull hwn o ddyddio testunau, ac yn awgrymu ffiniau arno. Yn fy achos innau, mae'n aneglur fel arfer ble byddaf yn delio ag arloesiad ieithyddol, ynteu orgraffyddol. 


\section{Ffonoleg a seineg}
Mewn orgraff Hen Gymraeg, ni wahaniaethid rhwng cysefin a threigledig. Mae hynny’n wir ar gychwyn gair, ac ar ddiwedd gair. Yng nghanol gair, unodd \mw{b, d, g} cysefin gyda \mw{p, t, c} treigledig, ond nid ar gychwyn gair. Yng nghanol gair, ond nid ar gychwyn gair, y mae yna eithriadau i’r methiant i ysgrifennu treigladau. Fel arfer, y mae ffrwydrolion mewnol cyfagos at gytsain soniarus yn cael eu hysgrifennu fel \mw{b, d, g}. Tybiaf fod hynny oherwydd cyfatebiaeth i gytseiniaid dechreuol. Mae cytseiniaid soniarus yn ‘llyncu’ anadliad caled mewn ffrwydrolion. Felly, mae’r \mw{b, d, g} yma heb anadliad caled, ac mae eu horgraff ar sail \mw{b, d, g} cysefin dechreuol. 

Mae ffynhonnell arall yn awgrymu yr un peth: patrymau sandhi allanol y ffrwydrolion ym Marddoniaeth y Tywysogion. Hwn yw fy mhennod esiampl a ddaeth allan o wersi cynghanedd gan Dafydd. Ei chanlyniad yw bod \mw{b, d, g} cysefin yn hirach na \mw{p, t, c} treigledig, a bod gan \mw{p, t, c} treigledig anadliad caled.

Felly, credaf mai hyn fu'r sefyllfa seinegol tan i’r ffrwydrolion di-lais treigledig a’r ffrwydrolion lleisiol uno: yngenid y ffrwydrolion di-lais cysefin gydag anadliad caled a hyd hir, yngenid y ffrwydrolion di-lais treigledig gydag anadliad caled a hyd byr, ac yngenid y ffrwydrolion lleisiol heb anadliad caled, a chyda hyd hir. Anghytunaf yma gyda \textcite[\S30]{koch_*cothairche_1990}, sydd yn credu hyn: ‘lenited [b̥, g̊, d̥] was distinct from radical [b-, g-, d-] by being voiceless and from radical [pʰ-, kʰ-, tʰ-] by being unaspirated.’

Ar y cyfan, gallaf ysgrifennu tair pennod am y dimensiwn ffonolegol: yr un gyntaf ydy'r Hen Gymraeg, sy'n dadlau bod anadliad caled yn nodwedd o'r ffrwydrolion di-lais treigledig. Yr ail am y gynghanedd, ble dadlaf fod anadliad caled yn cyferbynnu'r ffrwydrolion di-lais a lleisiol, heb ystyried y treiglad, a bod hyd yn cyferbynnu'r ffrwydrolion cysefin a threigledig. Gall trydedd bennod gasgliadol wynebu'r dystiolaeth hon gyda llenyddiaeth eilradd sy'n arwain at yr un peth, gan gynnwys \textcite{falchun_systeme_1951}, sydd hefyd yn awgrymu hyd, a chyda thystiolaeth o Gymraeg Cyfoes, a'r ieithoedd Celtaidd sydd i gyd yn awgrymu y defnyddid anadliad caled yn hytrach na llais i gyferbynnu'r ddwy gyfres wreiddiol o ffrwydrolion. Trafodaf yma hefyd faterion sydd angen eu datrys cyn deall y system ffonolegol a gynigir, e.e.\ cytunir na fu cyferbyniad tair-ffordd yna ond ar gychwyn gair, ond beth yw `cychwyn gair'? Nid yw syniad o `gair' yn bodoli mewn ieithyddiaeth~\autocite{haspelmath_indeterminacy_2011}. 


\section{Orgraff}
Ni ddangosir treiglad meddal y ffrwydrolion di-lais yn orgraff amrywiaeth o destunau Cymraeg Canol. Mewn llawysgrifau cynnar, fel Llyfr Du Caerfyrddin a'r Llyfr Du o’r Waun, nid yw treiglad y ffrwydrolion di-lais yn digwydd bron o gwbl, er bod treiglad y cytseiniaid eraill yn weddol gyson (ac eithrio \mw{d, rh} treigledig, wrth gwrs). Nid achos unigryw yw hyn: roedd yna gyfnod yn ystod cyfnod Cymraeg Canol pan ysgrifennwyd treigladau, ond nid treiglad meddal \mw{p, t, c}. Am faint o amser parhaodd y cyfnod hwn? A sut ehangodd yr arloesi? Ni ddangosir treiglad meddal \mw{p, t, c} ym \mw{Marwnad Madog ap Maredudd} yn Llyfr Du Caerfyrddin (tua 1250), er bod Llawysgrif Hendregadredd (tua 1300) yn tueddi i ddangos y treigladau i gyd. Bu farw Madog ap Maredudd ym 1160. Felly, mae’n debyg y newidiodd pethau rhwng 1160 a 1300, ond pryd yn union? Dyna'r cwestiwn mawr. Bydd astudiaeth achos ar gerddi dyddiadwy yn Llyfr Du Caerfyrddin a'u hymddangosiad diweddarach yn ffurfio pennod yn fy nhraethawd, ac mae'r strwythur arfaethedig yn cynnig sawl astudiaeth achos tebyg i ateb y cwestiwn mawr hwn, ond nid oes lle yn yr adroddiad hwn i ddisgrifio pob un.

Yn fy nhraethawd meistr, darganfyddais y patrwm o beidio i ddangos treiglad \mw{p, t, c} ar ôl -\mw{ei} ac \mw{oes} ym mhedair cainc y Mabinogi a Chulhwch ac Olwen~\autocite{van_sluis_development_2014}, fel y gallwch weld yn Enghraifft~\ref{ei}. 
\mwcc[ei]{\acrshort{wbr}~2.2-4}{ac ual \y\ llathrei \w ynnet \y\ c\w n \y\  llathrei cochet \y\ clusteu}{A disgleiriai'r c\^wn mor wyn ag y disgleiriai yn goch eu clustiau}
Cyfansoddwyd y testunau hyn yn gynharach na'u llawysgrif (tua 1350). Credaf nawr mai orgraff gynharach yw'r diffyg ysgrifennu'r treiglad hwn. Tybiaf fod gwybodaeth o hynny’n offer hanfodol i ddyddio cynsail testunau fel hynny. Bydd orgraff chwedlau'r Mabinogion yn ffurfio pennod arall o'm traethawd.

Y cwestiwn nesaf yw pam methodd \ei\ ac \oes\ foderneiddio orgraff y treigladau, ond nid geiriau fel \mw{y} ‘ei’ (gwrywaidd), er enghraifft. Nid wyf wedi siartio’r ateb i hynny yn llwyr eto, ond nid oes gwahaniaeth o ran ystyr rhwng \oes\ a ddilynir gan dreiglad, ac \oes\ sydd heb dreiglad wedyn. Mae yna wahaniaeth rhwng \mw{y} heb a chyda threiglad, fodd bynnag. Ar wahân i hynny, sylweddolais nad yw y gair \mw{ar} yn Llyfr Aneirin fel arfer yn treiglo ble dylai treiglad fod. Credaf fod hynny oherwydd amwysedd sydd yn y gair hwn yn orgraff Cymraeg Canol. Mae \mw{ar} yn gallu cyfateb i \mw{ar} ac i \mw{a’r} mewn Cymraeg Cyfoes. Mae’r naill yn achosi treiglad, ond nid yw’r llall fel arfer. Wrth foderneiddio’r treigladau, mae’n hawdd iawn camddeall brawddegau sy'n cynnwys \mw{ar}. 

Rwy'n disgwyl bydd y rhan orgraffyddol o'm traethawd yn rhannu'n naturiol rhwng penodau am orgraff cyn-1300, ac \^ol-1300. Bydd y rhai cyntaf yn trafod nad ysgrifenwyd \mw{p, t, c} treigledig, a than pryd ac ym mha lefydd. Bydd yr ail rai yn casglu olion o'r hen system hon, ac yn dynodi amgylcheddau gramadegol o hynny.

Er mwyn casglu mwy o ddata, bydd rhaid i fi ddewis casgliad addas a hygyrch. Gan fod yn debyg y digwyddodd y newid orgraffyddol yn ystod y 13eg ganrif, gallaf ddefnyddio \textcite{isaac_rhyddiaith_2013}, casgliad o destunau o'r ganrif honno, i gael golwg ar y cyfnod cyntaf.  Yr hyn sydd angen wedyn ydyw dynodi pa lawysgrifau yn y casgliadau hynny sy'n ddyddiadwy ac yn lleoliadwy, ac yn cynnwys testunau sy'n ddealladwy a chyson. Dylai'r llawysgrifau gynnwys testunau a gyfansoddwyd yn fuan cyn iddynt ymddangos yn eu llawysgrif. Tynnaf ryw 200--300 o bwyntiau o ddata o bob llawysgrif, gan fy mod wedi darganfod bod hyn yn ddigon i gael goleuni ar system yr orgraff. Mae pob pwynt o ddata yn achos mewn testun ble dylai treiglad meddal fod. Ysgrifennid hyn weithiau, a pheidid weithiau. Felly, mae casglu data angen dealltwriaeth ddofn o ramadeg Cymraeg Canol i farnu ble mae ysgrifydd wedi methu treiglo, gan fod rhaid `cywiro' y treigladau ar sail eu gramadeg nhw. Gwn erbyn hyn beth i'w gynnwys mewn bas data fel hon, a dysgaf ddefnyddio Microsoft Access ar y 15ed o Chwefror i wneud hynny mewn system hygyrch.

Er mwyn deall sut goroesodd olion o'r hen system, gallaf ddefnyddio casgliad o lawysgrifau o'r cyfnod wedyn:  \textcite{luft_rhyddiaith_2013}. Gan fod y bas data hwn yn chwiliadwy, byddaf yn gallu edrych ar wahanol amgylcheddau i ddod o hyd i batrymau treiglo. Ar \^ol darganfod amgylcheddau gramadegol sy'n dueddol i gadw'r hen dreigladau, gwnaf astudiaethau achos ar lawysgrifau diweddarach i weld beth allaf ddweud amdanynt ar sail yr offeryn newydd hwn. Daw bas data arall allan o hynny: un sy'n cynnwys achosion ble dylid treiglad meddal fod, ond ble nad yw'n ymddangos fel arfer, ond nid yw'n cynnwys achosion fel \mw{y} `ei', sy'n treiglo'n rheolaidd. 

O ran asesiad risg o ohiriad, mae'r rhan am orgraff yn fwy peryglus na'r rhan am ffonoleg, ac nid yn unig oherwydd ei bod yn llai cyflawn. Mae'n anodd barnu cymhlethdod newidiad orgraff \mw{p, t, c} treigledig, a pha ddimensiynau sy'n berthnasol. Efallai bydd dimensiwn daearyddol i'r newidiad orgraff. Os felly, hoffwn ddysgu rhaglen GIS fel QGIS. 

\section{Casgliad}
Daw'r casgliad yn \^ol i'r dechrau: bydd hwn yn cynnig system sy'n cyferbynnu'r tri math o ffrwydrolion, ac yn gosod hynny yn ei le o fewn ymchwil ar ffonoleg hanesyddol yr ieithoedd Brythonig. Bwriad arall ydy ysgrifennu casgliad o'r rhan orgraffyddol fel llawlyfr bach ar gyfer ieithyddion a haneswyr sydd eisiau gwybod pa mor hen ydy eu testun. Byddaf yn dangos sut i ddefnyddio orgraff y treigladau i gael mwy o wybodaeth o hynny. Gallaf hefyd drafod perthynas y ffordd hon i ddyddio a'r dulliau sydd ar gael nawr, fel trafodir yn y rhagymadrodd.


\section{Strwythur arfaethedig}

Mae'r traethawd yn rhannu yn dwy ran yn naturiol: yr un cyntaf fydd ffonoleg a seineg y mater. Yr ail un fydd orgraff. Felly, hoffwn strwythuro'r traethawd yn fras fel a ganlyn:

\begin{english}\onesp{
\begin{itemize}
    \item Front matter
    \begin{enumerate}
        \item Title
        \item Foreword
        \item Table of contents
        \item List of Tables
        \item List of Figures
        \item Introduction
    \end{enumerate}
    \item {Part I: Phonology and phonetics}
    \begin{enumerate}
        \item Old Welsh
        \item Cynghanedd
        \item Phonological stop system
    \end{enumerate}
    \item Part II: Orthography
    \begin{enumerate}
        \item Early clear examples of non-representation of \lT, with case studies on:
        \begin{itemize}
            \item Datable poems in the Black Book of Carmarthen.
            \item The Black Book of Chirk and later law texts
            \item Several independent translations of \mw{Brut y Brenhinedd}
        \end{itemize}
        \item Traces of the old mutation system in younger manuscripts, with case studies on:
        \begin{itemize}
            \item The Mabinogion in the White Book of Rhydderch
            \item The orthography of Scribe X86
            \item The curious case of the Book of Aneirin
        \end{itemize}
        \item Implications for dating texts
    \end{enumerate}
    \item Back Matter
    \begin{enumerate}
        \item Conclusion
        \item Glossary
        \item List of Acronyms
        \item Bibliography
    \end{enumerate}
\end{itemize}
}\end{english}

Mae siart Gantt ar dudalen \pageref{gantt} yn awgrymu faint o waith sydd wedi ei wneud, mae'n dangos sut mae fy ngwahanol brosiectau yn arwain at y strwythur arfaethedig, ac mae'n cynnig y camau nesaf i gyflawni'r prosiect.

\end{welsh}

\begin{sidewaysfigure}
\begin{ganttchart}[%
                hgrid style/.style={-},
                time slot format=isodate,
                x unit=.1875mm,
                bar/.append style={fill=black!50, rounded corners=3.33pt},
                bar left shift=.15,
                bar right shift=-.15,
                bar top shift=.4,
                bar height=.2,
                link/.style={thin, ->, black!75, rounded corners=6.67pt},
                link bulge=15,
                link mid=.5,
                milestone right shift=10,
                bar progress label node/.append style={above left=.75pt},
                ]{2015-10-01}{2018-09-30}
\addfontfeature{RawFeature={-onum;+lnum}}

\newcommand{\writingup}{\ganttlinkedbar[%
inline,
bar inline label node/.append style={above=.75pt}, 
bar label font=\scriptsize,
bar/.append style={fill=black!25}
]{Writing up \& Revision}{2018-03-01}{2018-09-30}}

\gantttitlecalendar{year, month}\\

\ganttbar{{MPhil/PhD}}{2015-10-01}{2017-02-15}
\ganttmilestone[inline, milestone label font=\scriptsize,milestone inline label node/.append style={above right=1pt}]{Final research proposal}{2015-12-31}
\ganttbar[bar/.append style={fill=black!25}]{}{2017-02-16}{2018-09-30}\\[grid]

\ganttbar{Introduction}{2015-10-01}{2015-11-15}
\ganttbar[progress=50]{}{2016-04-01}{2016-06-15}
\ganttlinkedbar[bar/.append style={fill=black!25}]{}{2017-11-12}{2018-02-21}
\writingup\\[grid]

%%ffonoleg/seineg
\ganttbar[progress=70]{Old Welsh}{2016-01-01}{2016-06-30}
\ganttlinkedbar{}{2017-02-16}{2017-03-15}\writingup\\

\ganttbar[progress=100]{Cynghanedd lessons}{2015-10-01}{2016-02-01}\\
\ganttlinkedbar{Provection in \textit{ByT}}{2016-01-01}{2016-04-14}
\ganttlinkedbar[progress=100]{}{2016-12-24}{2017-01-15}\writingup\\

\ganttbar{Breton Summer School}{2016-06-20}{2016-07-02}\ganttnewline
\ganttlinkedbar[progress=10]{Phonological stop system}{2017-01-20}{2017-04-15}\writingup\\[grid]

%%orgraff
\ganttbar{Black Book of Chirk}{2016-01-01}{2016-04-01}\\
\ganttlinkedbar{Aneirin \& Taliesin}{2016-06-15}{2016-11-31}\\
\ganttlinkedbar[progress=50]{Early Orthography}{2017-03-01}{2017-09-31}\writingup\\

\ganttbar{Postverbal lenition}{2015-10-01}{2015-12-31}\\
\ganttlinkedbar{Epithets, Later traces}{2016-04-15}{2016-07-13}
\ganttlinkedbar[progress=40]{}{2017-05-01}{2018-01-31}\writingup\\[grid]
% \ganttbar{Case Studies}{}{}
% \ganttlinkedmilestone{}{2016-02-15}


\ganttbar[progress=0]{General Conclusion}{2018-01-15}{2018-06-15}\\
\ganttbar{Bibliography}{2015-10-01}{2018-09-30}\\
\ganttbar{Glossary, Acronyms}{2017-01-01}{2018-09-30}

\end{ganttchart}
\caption{Proposed writing plan for the PhD stage}
\label{gantt}
\end{sidewaysfigure}
\begin{english}
\chapter{Translation of Paulus van Sluis's upgrade report}
My topic is the voiceless stops of \gls{mw}, and how and until when lenited voiceless stops (\lT) were kept separate from radical voiced stops (\xD). These two series did not merge until the historical period~\autocite{koch_*cothairche_1990}. This system separating \lT\ from \xD\ still exists in Breton~\autocite{falchun_systeme_1951}. Several scholars have treated the question of how the stop system worked phonetically before the \lT/\xD\ merger~\autocite{koch_*cothairche_1990,harvey_aspects_1984,schrijver_old_2011}. The introduction of my thesis will discuss the history of these ideas. There is no further literature on the fact that these series did not merge until the historical period. Nevertheless, I can discuss articles about the general history of lenition~\autocite{martinet_celtic_1952,sims-williams_dating_1990}, gemination~\autocite{greene_gemination_1956}, and the connections with the development of the spirant mutation~\autocite{schrijver_spirantization_1999,isaac_chronology_2004}.

From here onwards, the research question naturally separates into two parts: how the two series merged in spoken Welsh, and how they did so in the manuscripts. The first question is a phonetic and phonological one; the second is a question on orthography. I attempted to answer the first question by analysing exceptions to the orthographical rules of \gls{ow}, and by analysing patterns in the early cynghanedd. I tried to answer the second question by collecting data: I looked at a variety of texts to see where lenition was not shown. My thesis will also naturally separate into a phonological and an orthographical part.

My introduction will also contain literature to give context to the second part: there are precedents that show how we can collect traces of orthographical and linguistic innovation as a tool to date and localise texts. For orthographical innovations, the appearance of sandhi \mw{h} has been studied closely by \textcite{sims-williams_spread_2010}, and the orthography of lenited \mw{m} has been studied by \textcite{russell_rowynniauc_2003}, for example. For linguistic innovations, I can discuss Nurmio~\autocite*{nurmio_middle_2014,nurmio_studies_????}, who discusses the development of several plural forms, or Rodway~\autocite*{rodway_datable_1998,rodway_absolute_2002,rodway_two_2003,rodway_dating_2013}, who treats the innovations within the verbal system. I will certainly refer to \textcite{rodway_where_2007}, who gives a summary of this method to date texts, and shows the limits of it. In my particular case, it is often unclear where I am dealing with linguistic innovation, or orthographical innovation.

\section{Phonology and phonetics}
\Gls{ow} did not typically write the difference between a radical and a mutated consonant. This is true at the beginning of a word, and at the end of a word. In the middle of a word, radical \mw{b, d, g} merged with lenited \mw{p, t, c}, but not at the beginning of a word. In the middle of a word, but not at the beginning of a word, there are exceptions to the rule that lenition is not written. Word-medial stops adjacent to a resonant are sometimes written as \mw{b, d, g}. I assume this was done in analogy with the phonology of initial consonants. Resonants may `swallow' aspiration of stops, so these stops written with \mw{b, d, g} do not have aspiration, and their orthography is based on initial \mw{b, d, g}.

Another source indicates the same thing: external sandhi patterns in the Poetry of the Princes. This is my example chapter, which is the result of lessons on the cynghanedd given by Dafydd. Its result is that radical \mw{b, d, g} were pronounced longer than lenited \mw{p, t, c}, and that lenited \mw{p, t, c} had aspiration.

I therefore believe that this was the phonetic situation before lenited voiceless stops and radical voiced stops merged: radical voiceless stops were pronounced long and aspirated, lenited voiceless stops short and aspirated, and radical voiced stops long and unaspirated. I disagree here with \textcite[\S30]{koch_*cothairche_1990}, who believes the following: ‘lenited [b̥, g̊, d̥] was distinct from radical [b-, g-, d-] by being voiceless and from radical [pʰ-, kʰ-, tʰ-] by being unaspirated.’

On the whole, I can write three chapters on the phonological dimension: the first is on \gls{ow}, where I argue that aspiration was a feature of lenited voiceless stops. The second is on the cynghanedd, where I argue that aspiration contrasted voiceless stops from voiced stops, irrespectively of the mutation, and that length contrasted radical from lenited stops. A third, concluding, chapter may put these findings in the light of secondary literature leading to the same conclusion, including \textcite{falchun_systeme_1951}, who also suggests length, and with evidence from \gls{mow} and the other Celtic languages which all suggest that aspiration rather than voice served to keep the original pre-apocope stop series apart. I will also use this chapter to treat some matters that need to be solved before one may understand the proposed phonological system, e.g.\ it is agreed that this three-way stop system only existed at the beginning of a word, but what does `beginning of a word' mean? The concept of `word' is not a linguistic one~\autocite{haspelmath_indeterminacy_2011}.

\section{Orthography}
A variety of \gls{mw} texts does not represent lenition of voiceless stops in its orthography. Within early manuscripts, such as the Black Book of Carmarthen and the Black Book of Chirk, we barely find any lenition of these consonants, even though the other consonants lenite fairly consistently (except for \mw{d, rh}, of course). These are not exceptional instances: there was as period within the \gls{mw} period when mutations were generally written, but not the lenition of \mw{p, t, c}. How long did this period last? And how did the innovation spread? The lenition of \mw{p, t, c} is not shown in \mw{Marwnad Madog ap Maredudd} in the Black Book of Carmarthen (about 1250), even though the Hendregadredd manuscripts (about 1300) does tend to show all of these lenitions. Madog ap Maredudd died in 1160, so it looks like things changed between 1160 and 1300, but when exactly? This is the big question. A case study on dateable texts in the Black Book of Carmarthen and their later appearance will form a chapter in my thesis, and the proposed thesis structure shows several similar case studies that may answer this question, but this report does not permit me to describe each one individually.

In my Master's thesis, I discovered a pattern of failing to show lenition of \mw{p, t, c} following \ei\ and \mw{oes} in the four branches of the Mabinogi, and in Culhwch ac Olwen~\autocite{van_sluis_development_2014}, as you can see in Example~\ref{eieng}.
\mwcc[eieng]{\acrshort{wbr}~2.2-4}{ac ual \y\ llathrei \w ynnet \y\ c\w n \y\  llathrei cochet \y\ clusteu}{And as white as the dogs shone, so red shone their ears.}
These texts were composed earlier than the manuscripts we find them in (ca.\ 1350). I believe now that the failure to write lenition here is a trace of older orthography. I hypothesise that knowledge of these mutations constitute an essential tool in dating the original composition of texts such as these. The orthography of several stories of the Mabinogion will form another chapter of my thesis.

The next question is why \ei\ and \oes\ failed to modernise the orthography of the mutations following, but why \mw{y} `his' did modernise. I have not fully formulated an answer to this question yet, but there is no difference in meaning between \oes\ following by a mutation, and \oes\ without a mutation following. By contrast, there is a difference between \mw{y} with and without mutation following. Aside from this, I noticed that \mw{ar} in the Book of Aneirin sometimes fails to lenite where there should be a mutation following. I believe that this is because of the ambiguity this word presents in \gls{mw} orthography: it may correspond to \mw{ar} `on' in \gls{mow}, and to \mw{a'r} `and the'. The former always causes lenition, while the latter does not typically do so. When modernising the orthography of lenition, it is very easy to misunderstand phrases containing \mw{ar}, 

I expect that the orthographical part of my thesis will naturally separate into chapters on the pre-1300 orthography, and the period after 1300. The first half will establish that lenited \mw{p, t, c} were not written, until when, and in what places. The second half will collect traces of this early system in later manuscripts, and will pinpoint the grammatical categories that may show these traces.

In order to collect more data, I will have to choose a suitable and accessible collection of texts. Since it seems likely that the orthographical change happened in the 13th century, I can use \textcite{isaac_rhyddiaith_2013}, a collection of texts from this century in order to gain insight into this first period. The next phase is to judge which manuscripts may be dated and localised, and that contain easily workable texts. Ideally, these manuscripts should contain texts originally composed shortly before their appearance in the manuscript. I shall take some 200--300 data points from each manuscripts, as I discovered that this tends to be sufficient to shed light on the orthographical system used. Every data point is an instance within a text where lenition should be found, irrespectively of whether it is actually written or not. Therefore, to collect these data requires a deep understanding of \gls{mw} grammar in order to judge where a scribe may have failed to show lenition, because it will be my job to `correct' his mutations on the basis of his grammar. I know by this point what to include in such a database, and I will learn to use Microsoft Access on the 15th of February in order to do this in a suitable system.

In order to understand how traces of the old system have survived, I can use a collection of manuscripts from the period immediately following: \textcite{luft_rhyddiaith_2013}. Since this database is searchable, I will be able to look at various grammatical environments to find patterns in the mutation system. After finding environments that typically keep the old mutation system, I will do case studies on later manuscripts in order to see what I may say about them on the basis of this new tool. This will produce a different database: one that contains instances where lenition should be, but where it is typically not shown, but not instances where lenition typically does occur, such as \mw{y} `his'.

In terms of risk of delay of completion, I consider the orthographical part more dangerous than the phonological part, and not solely on account of its lesser degree of completion. It is difficult to judge beforehand how complex the change of the orthography of lenited \mw{p, t, c} is, and what dimensions may be relevant. Perhaps there is a geographical dimension to the orthographical change. If so, I may want to learn to use a GIS program, such as QGIS.

\section{Conclusion}
The conclusion mirrors the start: it will propose a stop system that contrasts three types of stops, and puts this in the context of other research on historical phonology of Brittonic. I also mean to write the conclusion of the orthographical part as a little guidebook for linguists and historians wishing to know how old their texts are. I will demonstrate how to use the orthography of lenition to gain more knowledge of this. I can also treat this way of dating with the methods that are available at present, as is discussed in the introduction.

\section{Proposed structure}
The thesis naturally separates into two parts: the first will be on the phonology and phonetics of the matter, while the second will be on orthography. Therefore, I would like to structure the thesis roughly as follows:

\onesp{
\begin{itemize}
    \item Front matter
    \begin{enumerate}
        \item Title
        \item Foreword
        \item Table of contents
        \item List of Tables
        \item List of Figures
        \item Introduction
    \end{enumerate}
    \item {Part I: Phonology and phonetics}
    \begin{enumerate}
        \item Old Welsh
        \item Cynghanedd
        \item Phonological stop system
    \end{enumerate}
    \item Part II: Orthography
    \begin{enumerate}
        \item Early clear examples of non-representation of \lT, with case studies on:
        \begin{itemize}
            \item Datable poems in the Black Book of Carmarthen.
            \item The Black Book of Chirk and later law texts
            \item Several independent translations of \mw{Brut y Brenhinedd}
        \end{itemize}
        \item Traces of the old mutation system in younger manuscripts, with case studies on:
        \begin{itemize}
            \item The Mabinogion in the White Book of Rhydderch
            \item The orthography of Scribe X86
            \item The curious case of the Book of Aneirin
        \end{itemize}
        \item Implications for dating texts
    \end{enumerate}
    \item Back Matter
    \begin{enumerate}
        \item Conclusion
        \item Glossary
        \item List of Acronyms
        \item Bibliography
    \end{enumerate}
\end{itemize}
}

The Gantt chart on \pref{gantt} gives an indication of how much work has been done up until now, shows how my various projects lead to the proposed structure, and proposes the immediate next steps in order to complete the project.
\end{english}