\chapter{Phonetics of voiceless stops}
\section{Ball}
\subsection{Vowel length as a cue for voicedness}
Martin J. Ball notes that in purely phonetic terms, voiceless stops are distinguished from their voiced consonants in more than one way. One way is preceding vowel length: \tqt{{}[F]or each speaker there is a marked difference in vowel length that depends on syllabic environment. Vowels are longest in open syllables, shorter before lenis consonants, and shortest of all before fortis consonants.}{ball_phonetics_1984}{10--11} Ball avoids the terms `voiceless' and `voiced' here because, on a phonetic level, voice is not the only and not even the primary method by which voiceless and voiced phonemes are distinguished. Preceding vowel length apparently also plays a role in distinguishing the phonemes.

Ball also discusses the why of different vowel lengths preceding /p, t, k/ and /b, d, ɡ/: \tqt{These results should lead us to discuss the reasons for such vowel length differences. As mentioned above, they often play a major role in English as a perceptual cue to aid the discrimination of final fortis-lenis consonant pairs. 

However in Welsh, this feature is unlikely to be the only one available to aid discrimination in similar cases, because the quality or phonological length of the vowel often also gives information as to the following consonants, as noted previously.

Of course, this does not necessarily rule out its function as one of a set of perceptual cues for consonant identification. It can be argued that even in the case of English, there are usually more cues to aid in the identification of the final consonant than simply allophonic vowel length. For example, in final stops, the fortis stop will often have a glottal reinforcement, or be unreleased, or even released with slight aspiration. In the case of final fricatives, there can be some degree of voicing in the lenis example. Therefore we could claim that in Welsh allophonic vowel length is just one of a series of perceptual cues aiding the perception of final consonants.
}{ball_phonetics_1984}{12}
\subsection{Initial stops}
\tqt{Traditionally initial stops have been distinguished in terms of place, and of the two features voiced/voiceless and fortis/lenis. However, recent work […] has suggested that the feature fortis/lenis (at least in its traditional definition of greater versus lesser force of articulation) cannot be verified through experimental work. Even if the feature were validated, it would be redundant for the stop category, for […] no known language has stop categories that cannot be specified by the voiced/voiceless and aspirated/unaspirated features.

These features are directly related to the state of the glottis (i) during the closure stages of the stop (voiced/voiceless); and (ii) during and immediately following the release of the stop (aspirated/unaspirated).

[\dots\ T]hese features of the stops are predictable consequences of the timing of events at the glottis. With regard to aspiration he pointed out that one of the crucial factors was the state of the glottis immediately following the release of the stop, and the duration of the time taken for the onset of voice, during what is called the noise interval.

If the voice follows on immediately after the release of the stop (zero or very low voice onset time) there can be no emission of air which characterizes the aspirated stop because there is no air available once the glottis is closed. All the air that builds up in the oral cavity is expended in the plosion and frication which occurs at the instant of release of the stop. Therefore the noise interval of the unaspirated stop is of very short duration during which no emission of pulmonic air takes place. 

If however there is a delay before the onset of voice following the release of the stop, there is far more air available, for the glottis being open, pulmonic air flows freely through the oral cavities. The traditional name for this flow of pulmonic air is aspiration. Therefore the noise interval of the aspirated stop is of a  relatively long duration during which an emission of pulmonic air takes place.

[\dots\ A]spirated and on-aspirated stops are purely a matter of duration of the noise interval and open/closed glottis. It was therefore decided to investigate the two series of Welsh stops in initial position (/p, t, k/ vs /b, d, ɡ/) to determine the voice onset time (VOT) for each stop, and see whether this were a sufficient distinguishing feature between the series.

[…] These results clearly show that the voice onset time is a sufficient distinguishing feature between these two series of stops in initial position. Indeed, the voicing feature appears erdundant, with /p, t, k/ being voiceless aspirated, and /b, d, ɡ/ generally voiceless and unaspirated.}{ball_phonetics_1984}{14--15}
\subsection{Treatment of \mw{s} + stops}
\tqt{{}[I]t was decided also to investigate onset time in initial fricative plus stop clusters (i.e\ /sb/, /sd/ and /sɡ/). Orthographically, these are represented by <sb>, <st> and <sg> respectively, as if there were some confusion as to which series the stop element belonged to. However, the results […] show clearly that in terms of the aspiration feature the stop elements belongs to the so-called lenis or voiced set --- /b, d, ɡ/.}{ball_phonetics_1984}{15--16}
\subsection{Final stops}
\tqt{In this study of stops in word final position the relation between the duration of the holding or close phase of the stop, and the cessation of voicing has been examined. While it is true that aspiration can noccur in word-final stops, this was difficult to measure in terms of voice onset time, as these tokens were not embedded in natural speech, therefore there are unnatural gaps between them which would distort results. 

[\dots\ O]n average for the fortis stops voicing ceases 29 ms before the holding phase of the stop begins (which could mean a certain amount of pre-aspiration is present); and that for the lenis stops voicing ceases on average 38 ms after the holding stage begins, therefore lasting for about 58\% of the holding phase.

In terms of voicing then, the fortis stops are difinitely voiceless; and the lenis ones tend to be voiced for just over half the holding stage, but in no cases does the voicing last until the release stage. This suggests that the additional perceptual cues of phonological vowel quality and allophonic vowel length might indeed b necessary for through discrimination between the two series of stops.}{ball_phonetics_1984}{16--18}
\section{Ball \& M\"{u}ller}
\tqt{VOT is indeed a sufficient distinguishing feature between the fortis series and the lenis in initial position. Initial /s/+stop cluster lacks the fortis-lenis contrast, but as these clusters do not enter the mutation system, they are not discussed further here.}{ball_mutation_1992}{84}
Except: /s/ + voiceless stop is a phonotactic combination that has (or should have) fortis-lenis contrast, except not word-medially. Between words, however, lenition may also occur. If the first word of such a pair ends with an /s/, and the following word starts with a voiceless stop, then potentially lenition may occur, and therefore a difference in phonetic quality of the stop may reflect the difference between a lenited and an unlenited version of this stop. The treatment of lenited voiceless stops following /s/ may have differed from other when it followed other consonants, considering the Middle Welsh tendency not to lenite voiceless stops following \mw{oes} and \mw{dros}.
\subsection{Ball \& M\"{u}ller on Hamp}
\tqt{In comparing the phonetic changes caused by mutations in the Celtic languages, Hamp makes the claim that the product of a mutation in one of the Brythonic languages (Welsh, Breton, Cornish) often coincides with an acceptable radical (i.e.\ unmutated) phoneme (e.g.\ Welsh /p/ \textrightarrow\ /b/), whereas in the Goidelic languages (Irish, Gaelic, Manx) the product of the mutation is rarely itself an acceptable radical consonant. (However, for eclipsis in Irish, /p, t, k, b, d/ change to /b, d, g, m, n/, which are acceptable radicals, and only /ɡ/, which changes to /ŋ/, fits Hamp's claim. Also, with Irish lenition, several segments result in acceptable radical initials. It could be that, in this instance, Hamp is referring to spelling, rather than pronunciation.) Therefore, in Welsh and Breton the learner `is continually finding himself in the wrong part of the dictionary' (p.\ 236); in other words, this learner has to know not only what the mutations are, but which syntactic structures cause them.}{ball_mutation_1992}{30}
\section{Gemination}
\subsection{Greene}
\tqt{If we write V for any vowel and take k as an example of the stops, we may consider the cases of \textit{VkV, VkkV, VrkV} in primitive Goidelic; these will give us in OIr.\ \textit{Vχ(V), Vkk(V), Vrk(V)} (e.g.\ \textit{ech, macc, torc}). That is, the former opposition \textit{k : kk} has been transformed into an opposition \textit{χ : kk, k}, where the latter pair are variants of one phoneme; there cannot be any place in the system where \textit{kk} opposes itself to \textit{k}.}{greene_gemination_1956}{285}
\tqt{To sum up: there is no evidence that OIr possessed any geminated consonants other than \textit{ll, m(m), nn, rr} and these were long in all non-lenited positions. When a particle which did not cause lenition was written together with the following word, the non-lenition of the initial was sometimes indicated by doubling it, less often by the substitution of tenuis for media (\textit{ba-calar}); these devices are also used in word-interior to indicate non-lenition. \textit{l, m, n, r, s} remained unchanged after particles which caused nasalization of stops and the unmutated state was sometimes indicated by doubling here also. Thus we may return to the view propounded by Zeuss and give up the word ``gemination'', except in a purely orthographical sense.}{greene_gemination_1956}{288}

\subsection{Thurneysen}
\tqt{Gemination originally consisted in the doubling (lengthening) of an initial consonant caused by assimilation of the final of one word to the initial of the following. In our period, however, it is already in decline, being no longer shown after consonants (\S 143) , and only irregularly after unstressed vowels. Further, since scribes never double the initial of a separate word, the gemination can only be seen where the two words are written together. In the course of time the geminated form is superseded by the ordinary unlenited form.

The geminated and nasalized forms of \textit{s- r- l- m- n-} are identical cp.\ \S 236, 1

It is clear from Mid.\ and Mod.\ Ir.\ that, in the same conditions as above, \textit{h-} was prefixed to an initial stressed vowel where the previous word ended in a vowel; but in O.Ir.\ there was no means of representing the sound (\S 25, cp.\ \S 177). That at an earlier period this \textit{h} was also audible after consonants is shown by a few forms such as \textbf{int}, nom.\ sg.\ masc.\ of the article before vowels, < \textit{*ind-h} < \textit{*sindos} or \textit{*sindas} (\S 467), \textbf{nant} `that (it) is not' < \textit{*nand-h} \S 797), \textbf{arimp} `in order that it may be' < \textit{*arimb-h} (\S 804, cp. \S 787).}{thurneysen_grammar_1946}{\S 240}
\section{Jones}
\tqt{Welsh has a six-term system of plosives with three contrastive places of articulation: bilabial \textasciitilde\ alveolar (dental NW and some areas of west Wales) \textasciitilde\ velar. The series /p, t, k/ are aspirated and voiceless, whilst the series /b, d, g/ are unaspirated, and although frequently referred to as voiced, voicing is not a constant feature of their articulation. They are regularly unvoiced in word initial and final positions: [\bd is] `finger', [\dd ið] `day', [ki\gd] `meat' [mɑ\bd] `son' and frequently so in a medial fully voiced environment, e.g.\ intervocalically: [se\bd on] `soap', [blo\dd ɛ] `flowers', [de\gd ɛ] `tens'. They may be partially voiced in all these environments, but fully voiced occasionally in medial fully voiced environments only. Since aspiration/non-aspiration is the constant feature distinguishing the two series of stops, voicing may be considered redundant, and the contrast operates on the voice-onset axis, i.e.\ on the duration of the release stage of the plosives. The delay in the onset of voicing is consistently longer for /p, t, k/ than for /b, d, g/ ; in Welsh, therefore, as in other languages (see Lisker and Abramson, 1964) aspiration is a concomitant feature of the delay in voice-onset time.

The aspiration/non-aspiration contrast in stops is frequently claimed to have an accompanying difference in the force of articulation, /p, t, k/ being articulated with greater subglottal pressure (Chomsky and Halle, 1968, 326), or being `more strongly articulated' (O'Connor, 1973, 127) than /b, d, g/, hence the former series are termed fortis and the latter, lenis. Experimental investigations do not support such claims. There is evidence (Davidsen-Nielsen, 1969) suggesting that where voicing is not a marked and constant feature of the contrast between the two series of stops, and such is the case in Welsh, there may well be no significant difference in the intraoral air pressure during the closure stage of the stop from the one series to the other (see further Ladefoged, 1971, 95-6). Until shown to be otherwise, we may assumate that such is the case for Welsh. However, fortis and lenis are useful labels which will be adopted to include, in the case of the stops, differences in terms of aspiration and of voicing.

\textit{Articulation} For the complete articulation of each of these plosives there are three stages: (i) the closure stage, during which the articulatory organs move together to form an obstruction to the pulmonic air, the soft palate being raised, sealing off the nasal cavity; (ii) the hold stage, during which egressive air is compressed behind the closure; (iii) the release stage, during which the articulators part rapidly, allowing the compressed air to escape.

During the articulation of fortis stops there will be no vocal cord vibration and following the release stage, there will be a delay in the onset of voicing for the adjacent sound; during this delay period there will be a flow of egressive air through the open glottis and the oral cavities which constitutes the feature of aspiration. For the lenis series there may be vocal cord vibration during the whole/part of the closure stage, depending on whether the stop is fully/partially voiced, onset of voice, however, immediately follows their release, i.e.\ there is no delay during which aspiration may occur.

\textit{Distribution} The plosives may occur initially, medially and finally in citation forms, e.g.\ /pant/ `hollow', /tɑn/ `fire', /kɑθ/ `cat', /sʊp/ `heap', /at/ `to', /plɪk/ `a pluck', /kapɛ/ `caps', /atɛb/ `answer', /sɛki/ `to thrust', /bid/ `world', /dal/ `hold', /garð/ `garden', /kribɛ/ `combs', /pedol/ `horseshoe', brigɪn/ `twig'. Note that the occurrence of /p, t, k/ in a word final position is usually a mark of a borrowing from English (see J.\ Morris-Jones, 1913, 66). The fortis series are strongly aspirated initially and finally in the word, but have weaker aspiration medially. In secondary forms the lenis series alternate with the fortis series in SM, e.g.\ /bɛn, dɑd, gi/ are secondary forms of /pɛn/ `head', /tɑd/ `father', and /ki/ `dog', respectively.}{jones_distinctive_1984}{41--42}

Most words ending in /p, t, k/ in Welsh are English loanwords, so only after English loanwords entered the language, the existence of these phonemes word-finally became a significant part of Welsh phonology. This may have influenced the system of phonological oppositions somehow. I should find out more about when English loanwords entered Welsh.
\subsection{Jones on fricatives+plosives}
\tqt{In such [i.e.\ s+T, f+T, ʃ+T, ɬ+T] clusters the plosive is neither aspirated nor voiced, hence the cotnrast between the fortis and lenis is neutralized in this context and phonologically the stops could be assigned to either of the two. [\dots\ I]t would seem that historically the plosives in such clusters have been interpreted as /b, d, g/, for in the centuries old alliteration rules of \underline{cynghanedd} (metrical consonance in Welsh metrics) the stops in such clusters alliterated with those of the lenis series (Morris-Jones, 1925)}{jones_distinctive_1984}{43}
\section{Quotes on Breton}


 \subsection{Jackson}
From HPB, with emphasis added:
\tqt{The question of consonant length in Welsh has not been adequately investigated experimentally, but it is clear that most of the original geminate or long consonants listed above are still so after a stressed vowel in speech, in polysyllables at any rate. It must be remembered that the system described for Welsh and Breton must have applied in Pr.W.\ and Pr.B.\ to monosyllables and final syllables only, and that when the accent shift occurred in OW.\ and OB.\ the same principles came to apply to the new stressed syllable of polysyllables, the penultimate (except that no shift took place in Vannetais, which must therefore preserve the reflexes of the original system). \emph{The system of fortis consonants in absolute initial described by Falc'hun for le Bourg Blanc is likely to be very old indeed, and original, though nothing of the sort has yet been demonstrated in Welsh --- or indeed any Breton dialect other than Falc'hun's so far as I know.} It appears then, that on the whole Welsh and Breton agree quite closely, at any rate in principle in spite of some differences in detail, and the little that is known about quantity in Cornish does not conflict with this. }{jackson_historical_1967}{\S 132}
In his section on lenition, Jackson does think that lenited voiceless stops merged with unlenited voiced stops:
\tqt{A most important change came of this rather simple fortis : lenis system, porbably in the second half of the 5th century, though it made no difference to the existence of a systematic oppostion. This is the ``mutation'' known as \textit{lenition}. By this, whereas the fortes remained unchanged the lenis occlusives and \textit{m} underwent a weakening which took the corresponding form: --- lenis \textit{p, t, k} were voiced to the corresponding lenis \textit{b, d, g}; the original lenes \textit{b, d, g}, with lenis \textit{m} became the corresponding sprants bilabial \textit{β}, dental \textit{\dh}, velar \textit{ʒ}, and strongly nasal bilabial \textit{μ}. The result is that by the late 5th century the oppositions in respect of these sounds were what may be expressed in tabular form as follows:

    \begin{tabular}{lrrrrrrr}
    (Fortis) non-lenited: & \textit{P} & \textit{T} & \textit{K} & \textit{B} & \textit{D} & \textit{G} & \textit{M} \\
    (Lenis) lenited: & \textit{b} & \textit{d} & \textit{g} & \textit{β} & \textit{\dh} & \textit{ʒ} & \textit{μ} \\
    \end{tabular}%

The other opposed pairs inherited from Common Celtic remained unchanged at the time of ``lenition'', giving the following table:

    \begin{tabular}{lrrrrr}
    Fortis: & \textit{J} & \textit{W} & \textit{N} & \textit{L} & \textit{R} \\
    Lenis: & \textit{\ci} & \textit{w} & \textit{n} & \textit{l} & \textit{r} \\
    \end{tabular}%

In addition there was \textit{S} versus \textit{s = Σ}, the origin and function of which was however different, see \S 418.
}{jackson_historical_1967}{\S 420}

\tqt{At a stage which cannot be dated, the whole series of remaining single fortes apparently lost their fortis character in Welsh and became \textit{p, t, k, b, d, g, m, n, s} except that \textit{W, R, }and \textit{L} had special developments which betray their origin in fortes (On \textit{W} > \textit{gw} see below). The same happened in many Breton dialects, except in the case of \textit{L} and \textit{R} (and \textit{N}) which likewise became lenes (on \textit{W} > \textit{gw} see below)., but other dialects seem to have retained the single fortes in absolute initial to the present day; see on this \S\S 62 ff., 66ff., and 96ff. For Cornish, which is a dead language, there is no satisfactory evidence on the question of fortis and lenis.}{jackson_historical_1967}{\S 421}

No, no, no. Le Bourg Blanc Breton shows that the distinction between fortes \textit{P, T, K, B, D, G} and lenes \textit{p, t, k, b, d, g} is maintained at least in part, since a three-way distinction of stop sets is maintained out of the four-way distinction I wrote above (lenited voiced stops become fricatives), whereas Jackson's hypothetical reduction would reduce them to two sets immediately.

\tqt{An important consonant development belonging to this period is the treatment of certain geminates. In the middle or later part of the 6th century geminate \textit{PP, TT}, and \textit{KK}, by themselves and in the groups \textit{lPP, rPP, rTT, lKK,} and \textit{rKK}, underwent a change, probably via an intermediate stage of very strongly aspirate stops, which resulted in the voiceless spirants \textit{f, θ}, and \textit{x}. […] The Brit.\ geminates \textit{BB, DD, GG} were simplified and fell together with lenis \textit{b, d, g} (the products of lenition of older lenis \textit{p, t, c}); so Brit.\ \textit{*aBBeros} or \textit{*aBBer\=a}, ``river-mouth'', from older *\textit{adbero-}, still had \textit{BB} at the time of lenition which therefore did not undergo it (hence the Mod.B.\ is not \textit{*aver}), but the group was later simplified to \textit{b}, so that the Mod.B.\ is \textit{aber} [ˈ\textit{a:ber} with the same long stressed vowel and short \textit{b} as in e.g.\ \textit{ober} from Brit.-Lat.\ \textit{opera} in which lenis \textit{p} was lenited to lenis \textit{b}. This simplification of geminate voiced stops happened therefore later than the period of lenition […]. It is very likely to have been contemporary and perhaps structurally linked with the change of \textit{PP, TT, }and \textit{KK} to \textit{f, θ}, and \textit{x}.}{jackson_historical_1967}{\S 434}

\subsection{Schrijver}
\tqt{In modern Leon dialects (Falc'hun 1951), the phonological feature of consonant length (traditionally: fortis vs.\ lenis) plays a far more important role than that of voice. In other dialects, the situation is reversed, and in many dialects consonant length plays no phonological role whatsoever. In general, MoB orthography is of little help in deciding these subtle matters. For MB, our only sources are orthography and rhyme. As to the latter, short consonants may rhyme with their long counterparts, and voiceless consonants with their voiced counterparts (see 2), which disqualifies rhyme as a diagnostic. MB orthography distinguishes short and long consonants as well as voiced and voiceless consonants. Since the MB literary language is thought to have been based on northern dialects (eastern Lion, western Treger) and since MoLeon and MoTreger dialects generally utilize both paramaters, we may tentatively describe the CMB consonant system on the basis of northern MoB dialect data.

The opposition between long and short voiced stops, which is known from modern Leon dialects, where it occurs only in word-initial position, is not reflected in MB orthography (which does not say much).}{schrijver_middle_2011}{378--379}
\section{Cornish}
\tqt{}{williams_middle_2011}{}

Breton and Cornish have provection. Does this phenomenon have any relevance?

\section{Evidence from the Early British inscriptions}
\tqt{\textit{\ddag 17 Lenition; LHEB \S142}

British lenition involved weakening of consonants, especially intervocalically, and can be divided into two types: (1) Voiced spirantization, i.e.\ /b d g g\textsuperscript{w} m/ > /β δ γ γ\textsuperscript{w} μ/, and (2) Voicing, i.e.\ /p t k/ > /b d g/. Jackson regarded lenition as a single process, but more recent philologists have swung back to the view, which he rejected, that there was an interval between (1) and (2).

[17] Jackson notes simply that the `Dark Age inscriptions' provide no evidence for lenition, owing to the spelling system used. While basically correct, this can be qualified slightly.

}{sims-williams_celtic_2003}{}