\subsection{Martinet}

\Textcite{martinet_celtic_1952} connects the morphophonemic process of lenition in Celtic with the phonetic process in Romance: 
\tqt{There is thus an obvious connection between lenition as a phonetic process and `lenition' as a set of morphophonemic alternations. But the use of the same word to designate synchronically quite different phenomena is apt to confusion.}{martinet_celtic_1952}{193--194}

Martinet states that lenited voiceless stops and unlenited voiced stops merged, and that this was due to the variable of [voice] becoming a feature of lenited voiceless stops:
\tqt{\textit{-b-, -d-, -g-} were weakened to spirants \textit{-β-, -ð-, -ɣ-} [\dots]; \textit{-p-, -t-, -k-} were \textbf{voiced} to \textit{-b-, -d-, -g-} [\dots] the whole process is exactly what we must surmise for Western Romance.}{martinet_celtic_1952}{198}.
However, Martinet was unlikely to have been aware of \textcite{falchun_systeme_1951} which came out only a year before this article.

He does mention the variable of [long], albeit in the context of geminates rather than it being any distinguishing feature between lenited voiceless stops and unlenited voiced stops. Having said this, it should be remembered that a geminate stop may be considered simply an unlenited stop. It is only after spirantisation that a geminate stop and a word-initial unlenited voiceless stop separated.
\tqt{In the normal speech of less accurate speakers, geminates will usually become long consonants articulated entirely in the second syllable, instead of being equally divided between two successive syllables. [\dots] If, however, the complex of sociolinguistic factors does not favor the preservation of the status quo, the weakening geminates will exert a pressure on their simple intervocalic counterparts, so that the articulation of the latter will become weaker too and will tend to be assimilated to their context}{martinet_celtic_1952}{198--199}
Martinet thus predicts a length distinction in the stop system due to geminates. Although he considers spirantisation essentially a separate development from lenition, I do not consider this the case. Prior to apocope, length distinguished geminate voiceless stops and single voiceless stops. Apocope led to the phonemicization of length distinction between simple unlenited voiced stops and lenited voiced stops. 

Martinet also makes some comments on whether it was the variable of [voice] or [aspiration] which kept voiceless and voiced stops apart. He makes the following comment on the phonetic quality of voiceless geminates:
\tqt{the [t:] (which probably was there as an energetically articulated aspirate) is shifted in another direction.}{martinet_celtic_1952}{199}
However, he doubts that this was the phonetic component differentiating voiceless and voiced stops in earlier Celtic:
\tqt{Besides, there are cogent reasons for doubting the existence of a clear aspiration in Early Celtic: there is no trace of it in Gaulish, where (if \textit{p, t, k} had actually been aspirates) we would expect to find, for IE \textit{t} and \textit{k}, spellings with \textit{th} and \textit{ch} in the Latin and with \textit{θ} and \textit{χ} in the Greek alphabet---not only in intervocalic positions, where these are sporadically attested, but in strong positions as well.}{martinet_celtic_1952}{198--199}
This argument of Martinet's does not hold water: Ancient Greek has three grapheme series \textit{π, τ, κ;  β, δ, γ; φ, θ, χ} because it has three stop series, i.e.\ voiceless stops, voiced stops, voiced aspirates. Early Brythonic did not have a three-way stop distinction until apocope phonemicised the distinction between lenited and unlenited stops. Also, if languages that use [aspiration] rather than [voice] to distinguish voiceless stops from voiced stops would necessarily copy the Latin or Greek signs for aspirated stops, then most Germanic languages would use <ph> where we see <p>, and <p> where we see <b>, et cetera.
Martinet also gives the following reason why voice rather than aspiration would keep Early Celtic stops apart:
\tqt{Moreover, if Proto-Celtic already had the aspirated pronunciation of \textit{k} except after \textit{s} which later prevailed in Irish and Welsh (cf.\ Welsh spellings like \textit{ysgol} `school', Gaelic ones like \textit{sgamhan} `lungs') we should expect to find traces of a `lenition' of initial \textit{k-} to \textit{g-} after IE word-final \textit{-s}, so that `the head', say, would appear as OIr.\ *\textit{in genn}, MIr.\ *\textit{an geann} instead of \textit{in cenn, an ceann}. It is therefore likely that the aspirated pronunciation of voiceless stops, as we find it today in Celtic languages spoken in the British Isles, results from an insular innovation.}{martinet_celtic_1952}{198--199}
In this instance, Martinet fails to take into account that \textit{s} typically turned into \textit{h} by the time of apocope. A consonant followed by \textit{h} is not expected to lenite per se. This explains the lack of interword lenition following *\textit{s}.

Neither argument of his is convincing that Proto-Celtic did not distinguish voiceless stops from voiced stops on the basis of aspiration. However, this does not prove the opposite, and  the earliest variants of Celtic spoken on the continent may not have conformed to the later Insular Celtic languages. Nevertheless, there is no evidence that the Insular Celtic phonetic makeup was a recent innovation.

