\usepackage[%
            xindy,
            toc,
            % acronym
            ]{glossaries}
\usepackage{glossary-mcols}
\newglossary*{lang}{Language abbreviations}
\newglossary*{textl}{Text abbreviations}
\setglossarypreamble[\acronymtype]{Note: abbreviations denoting a language are sorted under `Language abbreviations'. Abbreviations denoting a text are sorted under `Text abbreviations'.}

\setacronymstyle{short-long}
\makeglossaries
\usepackage[xindy]{imakeidx}
\makeindex

%% TERMS

\newglossaryentry{allophone}%
{
  name=allophone,
  description={One of two or more phonetically distinct segments which can realise a single phoneme in varying circumstances~\autocite[s.v.~\emph{allophone}]{Tra_dictionary96}}
}

\newglossaryentry{phoneme}%
{
  name=phoneme,
  description={The smallest unit which can make a difference in meaning in a particular language~\autocite[s.v.~\emph{phoneme}]{Tra_dictionary96}; a set of sound segments which are felt as equivalent to each other in a language}
}

\newglossaryentry{sandhi}%
{
  name=sandhi,
  description={A phonological process that is applied when a sequence of segments is formed across morpheme boundaries or word boundaries~\autocite[s.v.~\emph{sandhi}]{Tra_dictionary96}}
}

\newglossaryentry{provcon}
{
    name=provecting consonant,
    description={A type of \gls{sandhi} whereby a \gls{D}, \mw{h}, or \mw{rh} that follows a \gls{D} is provected to \gls{T}. It has two sub-types: \gls{doubcon} and \gls{aspcon}}
}

\newglossaryentry{doubcon}
{
    name=doubling consonant,
    description={A type of \gls{sandhi} whereby a \gls{D} that follows a \gls{D} is provected to \gls{T}}
}

\newglossaryentry{aspcon}
{
    name=aspirating consonant,
    description={A type of \gls{sandhi} whereby a \mw{h} or \mw{rh} that follows a \gls{D} is provected to \gls{T}}
}

\newglossaryentry{provconclus}
{
    name=provecting consonant cluster,
    description={A cluster of \gls{D} followed by \gls{D}\mw{, h,} or \mw{rh} that forms a single \gls{T} for alliterative purposes. It has two sub-types: \gls{doubconclus} and \gls{aspconclus}}
}

\newglossaryentry{doubconclus}
{
    name=doubling consonant cluster,
    description={A cluster of \gls{D} followed by another \gls{D} that forms a single \gls{T} for alliterative purposes}
}

\newglossaryentry{aspconclus}
{
    name=aspirating consonant cluster,
    description={A cluster of \gls{D} followed by \mw{h} or \mw{rh} that forms a single \gls{T} for alliterative purposes}
}

\newglossaryentry{cyfl}
{
    name={\mw{cyf\-lythyr\-aeth}},
    description={The alliteration of radical and mutated forms of a single \gls{archphon}},
    sort={cyflythyraeth}
}

\newglossaryentry{archphon}
{
    name={archi\-phoneme},
    description={A group of phonemes that share the same unmutated phoneme. For example, the phonemes /t d θ n̥/ all belong to the archiphoneme /t/ in \gls{mow}}
}


\newglossaryentry{petr}
{
  name={petrified lenition},
  description={The result of reanalysis of a word's lenited form as its radical form, \eg \gmw[with]{gan} < \mw{can}. Contrasts with \gls{morphophonlen}. See: \pref{sec:petrification}}
}

\newglossaryentry{morphophonlen}
{
  name={morphophonemic lenition},
  description={Any type of lenition whereby the lenited form of a word stands in non-free variation with its radical form. Contrasts with \gls{petr}. See:~\pref{sec:morph-lenit}}
}

\newglossaryentry{freelen}
{
  name={free lenition},
  description={A type of lenition where neither contact with a preceding morpheme nor pre-apocope phonetic context may account for its occurrence. Contrasts with \gls{contlen}. See:~\pref{sec:free-lenition}}
}

\newglossaryentry{contlen}
{
  name={contact lenition},
  description={A type of lenition where lenition is applied to a word because the immediately preceding element causes lenition. Contrasts with \gls{freelen}. See~\pref{sec:contact-lenition}}
}

\newglossaryentry{apoc}
{
  name={apocope},
  description={The loss of final syllables which occurred in the mid-sixth century in the Brittonic languages~\autocite[§§~182, 210]{jackson_language_1953}. This loss led to the phonemicisation of lenition}
}

\newglossaryentry{gemin}
{
  name={gemination},
  description={The articulation of a consonant for longer than a single instance of the same consonant type because the consonant is doubled}
}

\newglossaryentry{rescon}
{
  name={resonant consonant},
  description={All nasal consonants, e.g.~/m n ŋ/, as well as the liquids, e.g.~/n l/. Consonants in this class are articulated with a continuous and non-turbulent airflow.}
}

\newglossaryentry{R}{name=R,description={A \gls{rescon}, i.e.\ \mw{m, n, ng, l, r} and their voiceless counterparts},sort={R1}}
\newglossaryentry{D}{name=D,description={A voiced stop, i.e.\ \mw{b, d, g}},sort={D1}}
\newglossaryentry{T}{name=T,description={A voiceless stop, i.e.\ \mw{p, t, k}},sort={T}}
\newglossaryentry{l}{name=\textsuperscript{l},description={Indicates that the following consonant is lenited},sort={l}}
\newglossaryentry{x}{name=\textsuperscript{x},description={Indicates that the following consonant is not lenited},sort={x}}
\newglossaryentry{C}{name=C,description={Any consonant},sort={C1}}
\newglossaryentry{V}{name=V,description={Any vowel},sort={V1}}
\newglossaryentry{hash}{name={\#}, description={Indicates a word boundary}}

%% MSS SIGLA
\newcommand{\mssig}[2]{\newglossaryentry{s#1}{type={textl},name=\textit{#1},description={#2}}}
\mssig{A}{\acrshort{nlw} Peniarth MS 29 (The Black Book of Chirk)}
\mssig{B}{\acrshort{bl} Cotton Titus MS D.\ ii}
\mssig{C}{\acrshort{bl} Cotton Caligula MS A.\ iii}
\mssig{D}{\acrshort{nlw} Peniarth MS 32 (\mow{Y Llyfr Teg})}
\mssig{E}{\acrshort{bl} Additional MS 14931}
\mssig{G}{\acrshort{nlw} Peniarth MS 35}
\mssig{J}{Oxford, Jesus College MS 57}
\mssig{K}{\acrshort{nlw} Peniarth MS 40 (\mow{Llyfr Calan})}
\mssig{V}{\acrshort{bl} Harley MS 4353}
%% \newacronym[description={\acrshort{bl} Harley 4353}]{h4353}{H4353}{BL Harley 4353}


%% ACRONYMS

%% Languages
\makeatletter
\newacronym[type=lang]{lat}{Lat.\@\xspace}{Latin}
\newacronym[type=lang]{vlat}{VLat.\@\xspace}{Vulgar Latin}
\newacronym[type=lang]{gr}{Gr.\@\xspace}{Greek}
\newacronym[type=lang]{mow}{MoW}{Modern Welsh}
\newacronym[type=lang]{mob}{MoB}{Modern Breton}
\newacronym[type=lang]{mb}{MB}{Middle Breton}
\newacronym[type=lang]{mco}{MCo.\@\xspace}{Middle Cornish}
\newacronym[type=lang]{oco}{OCo.\@\xspace}{Old Cornish}
\newacronym[type=lang]{ob}{OB}{Old Breton}
\newacronym[type=lang]{ow}{OW}{Old Welsh}
\newacronym[type=lang]{oir}{OIr.\@\xspace}{Old Irish}
\newacronym[type=lang]{mir}{MIr.\@\xspace}{Middle Irish}
\newacronym[type=lang]{moir}{MoIr.\@\xspace}{Modern Irish}
\newacronym[type=lang]{mw}{MW}{Middle Welsh}
\newacronym[type=lang]{pbr}{PBr.\@\xspace}{Proto-Brittonic}
\newacronym[type=lang]{pc}{PC}{Proto-Celtic}
\newacronym[type=lang]{pie}{PIE}{Proto-Indo-European}
\newacronym[type=lang]{pi}{PIr.\@\xspace}{Primitive Irish}
\newacronym[type=lang]{pic}{PIC}{Proto-Insular-Celtic}
\makeatother

%%Manuscripts/Texts
\newacronym[type=textl,description={\acrshort{nlw} Peniarth MS 1 (The Black Book  of Carmarthen)}]{bbc}{BBC}{Black Book of Carmarthen}
\newacronym[type=textl,description={\acrshort{nlw} Peniarth MS 29 (The Black Book of Chirk)}]{bbch}{BBCh}{Black Book of Chirk}
\newacronym[type=textl]{bl}{BL}{British Library, London}
\newacronym[type=textl,description={\acrshort{nlw} Peniarth MS 2 (The Book of Taliesin)}]{bt}{BT}{Book of Taliesin}
\newacronym[type=textl,description={\acrshort{bl} Cotton Cleopatra MS A xiv}]{cca14}{CCA14}{Cotton Cleopatra A xiv}
\newacronym[type=textl,description={\acrshort{nlw} Peniarth MSS 4 \&\ 5 (The White Book of Rhydderch)}]{wbr}{WBR}{White Book of Rhydderch}
\newacronym[type=textl,description={The \gls{ow} glosses in Martianus Capella, \lat{De Nuptiis Philologiae et Mercurii} (Cambridge, Corpus Christi College MS 153)}]{mc}{MC}{Martianus Capella}
\newacronym[type=textl]{nlw}{NLW}{National Library of Wales, Aberystwyth}
\newacronym[type=textl,description={\acrshort{nlw} MS 6680B (The Hendregadredd Manuscript)}]{h}{H}{Hendregadredd Manuscript}
\newacronym[type=textl,description={\lat{De Mensuris et Ponderibus}, Oxford, Bodleian Library, Auctarium MS F.~4.~32, ff.~22v--24r}]{mp}{MP}{\lat{De Mensuribis et Ponderibus}}
\newacronym[type=textl,description={\acrshort{nlw} Llanstephan MS 1}]{ll1}{Llan.\ 1}{Llanstephan 1}
\newacronym[type=textl,description={\acrshort{nlw} Peniarth MS 44}]{p44}{Pen.\ 44}{Peniarth 44}
\newacronym[type=textl,description={\acrshort{nlw} Peniarth MS 6}]{p6}{Pen.\ 6}{Peniarth 6}
\newacronym[type=textl,description={\acrshort{nlw} MS 5266 (\mw{Brut Dingestow})}]{bd}{BD}{\mw{Brut Dingestow}}
\newacronym[type=textl,description={\acrshort{bl} Cotton Cleopatra MS B v part i}]{bcc}{CCB5}{\acrshort{bl} Cotton Cleopatra B v part i}
\newacronym[type=textl,description={\acrshort{nlw} Llanstephan MS 27} (The Red Book of Talgarth)]{ll27}{Llan.\ 27}{Llanstephan 27}
\newacronym[type=textl,description={Oxford Jesus College MS 119 (The Book of the Anchorite of Llanddewi Brefi) }]{j119}{J119}{Jesus College 119}
\newacronym[type=textl,description={\acrshort{bl} Cotton Titus MS D xxii }]{ctd22}{CTD22}{Cotton Titus D xxii}
\newacronym[type=textl,description={\acrshort{nlw} MS 17110E (The Book of Llandaff)}]{bll}{BLl}{Book of Llandaff}
\newacronym[type=textl,description={The \gls{ow} glosses in Ovid's \lat{Ars Amatoria} (Oxford, Bodleian Library, Auctarium MS F.~4.~32, ff.~37a--42a)}]{ovid}{Ovid}{Ovid's \lat{Ars Amatoria}}
%% \newacronym[type=textl,description={\mw[]{Canu Taliesin} edited by~\textcite{williams_canu_1960}.}]{ct}{CT}{\mw[]{Canu Taliesin}}

%%Other
\newacronym{np}{NP}{nominal predicate}
\newacronym{obj}{obj.\@\xspace}{object}
\newacronym{len}{len.\@\xspace}{lenition}
\newacronym{adv}{adv.\@\xspace}{adverb}
\newacronym{phr}{phr.\@\xspace}{phrase}
\newacronym{p}{p.\@\xspace}{page}
\newacronym{f}{f.\@\xspace}{folio}
\newacronym{line}{l.\@\xspace}{line}
\newacronym{hgcs}{HGCS}{High German Consonant Shift}
\newacronym{vot}{VOT}{Voice Onset Time}
\newacronym{ipa}{IPA}{International Phonetic Alphabet}


\newacronym[type=textl]{cbt}{CBT}{\mw{Cyfres Beirdd y Tywysogion}}
\newacronym[type=textl]{gpc}{GPC}{\mw{Geiriadur Prifysgol Cymru}}

\newacronym[type=textl,description={\textcite{andrews_gwaith_1996}. \citetitle{andrews_gwaith_1996}}]{CBTVII}{CBT~VII}{\gls{cbt} Vol.\ VII}
\newacronym[type=textl,description={\textcite{costigan_gwaith_1995-1}. \citetitle{costigan_gwaith_1995-1}}]{CBTVI}{CBT~VI}{\gls{cbt} Vol.\ VI}
\newacronym[type=textl,description={\textcite{jones_gwaith_1991-1}. \citetitle{jones_gwaith_1991-1}}]{CBTV}{CBT~V}{\gls{cbt} Vol.\ V}
\newacronym[type=textl,description={\textcite{jones_gwaith_1995-1}. \citetitle{jones_gwaith_1995-1}, Vol.~II}]{CBTIV}{CBT~IV}{\gls{cbt} Vol.\ IV}
\newacronym[type=textl,description={\textcite{jones_gwaith_1991}. \citetitle{jones_gwaith_1991}, Vol.~I}]{CBTIII}{CBT~III}{\gls{cbt} Vol.\ III}
\newacronym[type=textl,description={\textcite{bramley_gwaith_1994}. \citetitle{bramley_gwaith_1994}}]{CBTII}{CBT~II}{\gls{cbt} Vol.\ II}
\newacronym[type=textl,description={\textcite{williams_gwaith_1994}. \citetitle{williams_gwaith_1994}}]{CBTI}{CBT~I}{\gls{cbt} Vol.\ I}

\newacronym[type=textl]{ppd}{PPD}{\mw{Pwyll Pendeuic Dyuet}}
\newacronym[type=textl]{bul}{BUL}{\mw{Branwen Uerch Lyr}}
\newacronym[type=textl]{mul}{MUL}{\mw{Manawydan Uab Llyr}}
\newacronym[type=textl]{mum}{MUM}{\mw{Math Uab Mathonwy}}
\newacronym[type=textl]{co}{CO}{\mw{Culhwch ac Olwen}}
\newacronym{RE}{RE}{Research Exception}

%%% Local Variables:
%%% mode: latex
%%% TeX-master: "main"
%%% End:
