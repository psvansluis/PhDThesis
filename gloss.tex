\usepackage[%
            xindy,
            toc,
            acronym
            ]{glossaries}
\usepackage{glossary-mcols}
\newglossary*{lang}{Language abbreviations}
\setglossarypreamble[\acronymtype]{Note: acronyms denoting a language are sorted under `Language abbreviations'.}

\setacronymstyle{short-long}
\makeglossaries
\usepackage[xindy]{imakeidx}
\makeindex

%% TERMS

\newglossaryentry{provcon}
{
    name=provecting consonant,
    description={A \gls{D},\mw{h}, or \mw{rh} that follows a \gls{D}, thereby provecting it to \gls{T}. It has two sub-types: \gls{doubcon} and \gls{aspcon}}
}

\newglossaryentry{doubcon}
{
    name=doubling consonant,
    description={A \gls{D} that follows a \gls{D}, thereby provecting it to \gls{T}}
}

\newglossaryentry{aspcon}
{
    name=aspirating consonant,
    description={A \mw{h} or \mw{rh} that follows a \gls{D}, thereby provecting it to \gls{T}}
}

\newglossaryentry{provconclus}
{
    name=provecting consonant cluster,
    description={A cluster of \gls{D} followed by \gls{D}\mw{, h,} or \mw{rh} that forms a single \gls{T} for alliterative purposes. It has two sub-types: \gls{doubconclus} and \gls{aspconclus}}
}

\newglossaryentry{doubconclus}
{
    name=doubling consonant cluster,
    description={A cluster of \gls{D} followed by another \gls{D} that forms a single \gls{T} for alliterative purposes}
}

\newglossaryentry{aspconclus}
{
    name=aspirating consonant cluster,
    description={A cluster of \gls{D} followed by \mw{h} or \mw{rh} that forms a single \gls{T} for alliterative purposes}
}

\newglossaryentry{cyfl}
{
    name={\mw{cyf\-lythyr\-iaeth}},
    description={The alliteration of radical and mutated forms of a single \gls{archphon}},
    sort={cyflythyriaeth}
}

\newglossaryentry{archphon}
{
    name={archi\-phoneme},
    description={A group of phonemes that share the same unmutated phoneme. For example, the phonemes /t, d, θ, n̥/ all belong to the archiphoneme /t/}
}


\newglossaryentry{petr}
{
  name={petrified lenition},
  description={The result of reanalysis of a word's lenited form as its radical form, \eg \gmw[with]{gan} < \mw{can}. Contrasts with \gls{morphophonlen}. See: \pref{sec:petrification}}
  see={petr}
}

\newglossaryentry{morphophonlen}
{
  name={morphophonemic lenition},
  description={Any type of lenition whereby the lenited form of a word stands in non-free variation with its radical form. Contrasts with \gls{petr}. See:~\pref{sec:morph-lenit}}
}

\newglossaryentry{freelen}
{
  name={free lenition},
  description={A type of lenition where neither contact with a preceding morpheme nor pre-apocope phonetic context may account for its occurrence. Contrasts with \gls{contlen}. See:~\pref{sec:free-lenition}}
}

\newglossaryentry{contlen}
{
  name={contact lenition},
  description={A type of lenition where lenition is applied to a word because the immediately preceding element causes lenition. Contrasts with \gls{freelen}. See~\pref{sec:contact-lenition}}
}

\newglossaryentry{D}{name=D,description={A voiced stop, i.e.\ \mw{b, d, g}},sort={D}}
\newglossaryentry{T}{name=T,description={A voiceless stop, i.e.\ \mw{p, t, k}},sort={T}}
\newglossaryentry{l}{name=\textsuperscript{l},description={Indicates that the following consonant is lenited},sort={l}}
\newglossaryentry{x}{name=\textsuperscript{x},description={Indicates that the following consonant is not lenited},sort={x}}
\newglossaryentry{C}{name=C,description={Any consonant},sort={C}}
\newglossaryentry{V}{name=V,description={Any vowel},sort={V}}

%% MSS SIGLA
\newcommand{\mssig}[2]{\newglossaryentry{s#1}{name=\textit{#1},description={#2}}}
\mssig{A}{NLW Peniarth MS.\ 29 (The Black Book of Chirk)}
\mssig{B}{BL Cotton Titus D II}
\mssig{C}{BL Cotton Caligula A III}
\mssig{D}{Peniarth 32}
\mssig{E}{BL Additional 14931}
\mssig{G}{Peniarth 35}


%% ACRONYMS

%%Languages
\makeatletter
\newacronym[type=lang]{lat}{Lat.\@\xspace}{Latin}
\newacronym[type=lang]{vlat}{VLat.\@\xspace}{Vulgar Latin}
\newacronym[type=lang]{gr}{Gr.\@\xspace}{Greek}
\newacronym[type=lang]{mow}{MoW}{Modern Welsh}
\newacronym[type=lang]{mob}{MoB}{Modern Breton}
\newacronym[type=lang]{mb}{MB}{Middle Breton}
\newacronym[type=lang]{mco}{MCo.\@\xspace}{Middle Cornish}
\newacronym[type=lang]{oco}{OCo.\@\xspace}{Old Cornish}
\newacronym[type=lang]{ob}{OB}{Old Breton}
\newacronym[type=lang]{ow}{OW}{Old Welsh}
\newacronym[type=lang]{oir}{OIr.\@\xspace}{Old Irish}
\newacronym[type=lang]{mir}{MIr.\@\xspace}{Middle Irish}
\newacronym[type=lang]{mw}{MW}{Middle Welsh}
\newacronym[type=lang]{pbr}{PBr.\@\xspace}{Proto-Brittonic}
\newacronym[type=lang]{pc}{PC}{Proto-Celtic}
\newacronym[type=lang]{pie}{PIE}{Proto-Indo-European}
\newacronym[type=lang]{pi}{PIr.\@\xspace}{Primitive Irish}
\newacronym[type=lang]{pic}{PIC}{Proto-Insular-Celtic}
\makeatother

%%Manuscripts
\newacronym[description={\acrshort{nlw} Peniarth MS 1 (The Black Book  of Carmarthen)}]{bbc}{BBC}{Black Book of Carmarthen}
\newacronym[description={\acrshort{nlw} Peniarth MS 29 (The Black Book of Chirk)}]{bbch}{BBCh}{Black Book of Chirk}
\newacronym{bl}{BL}{British Library, London}
\newacronym[description={\acrshort{nlw} Peniarth MS 2 (The Book of Taliesin)}]{bt}{BT}{Book of Taliesin}
\newacronym[description={\acrshort{nlw} Peniarth MSS 4 \&\ 5 (The White Book of Rhydderch)}]{wbr}{WBR}{White Book of Rhydderch}
\newacronym{nlw}{NLW}{National Library of Wales, Aberystwyth}
\newacronym[description={\acrshort{nlw} MS 6680B (The Hendregadredd Manuscript)}]{h}{H}{Hendregadredd Manuscript}
\newacronym[description={\lat{De Mensuribis et Ponderibus}, Oxford, Bodleian Library, Auctarium F.~4.~32}]{mp}{MP}{\lat{De Mensuribis et Ponderibus}}
\newacronym[description={\acrshort{nlw} Llanstephan MS 1}]{ll1}{Llan.\ 1}{Llanstephan MS 1}
\newacronym[description={\acrshort{nlw} Peniarth MS 44}]{p44}{Pen.\ 44}{Peniarth MS 44}
\newacronym[description={\acrshort{nlw} MS 5266 (\mw{Brut Dingestow})}]{bd}{BD}{\mw{Brut Dingestow}}
\newacronym{bcc}{Cleo.\ B V i}{\acrshort{bl} Cotton Cleopatra MS B V part i}


%%Other
\newacronym{np}{NP}{Nominal Predicate}

\newacronym{cbt}{CBT}{\mw{Cyfres Beirdd y Tywysogion}}
\newacronym{gpc}{GPC}{\mw{Geiriadur Prifysgol Cymru}}

\newacronym[description={\cite{andrews_gwaith_1996} \citetitle{andrews_gwaith_1996}}]{CBTVII}{CBT~VII}{\gls{cbt} Vol.\ VII}
\newacronym[description={\cite{costigan_gwaith_1995-1} \citetitle{costigan_gwaith_1995-1}}]{CBTVI}{CBT~VI}{\gls{cbt} Vol.\ VI}
\newacronym[description={\cite{jones_gwaith_1991-1} \citetitle{jones_gwaith_1991-1}}]{CBTV}{CBT~V}{\gls{cbt} Vol.\ V}
\newacronym[description={\cite{jones_gwaith_1995-1} \citetitle{jones_gwaith_1995-1}}]{CBTIV}{CBT~IV}{\gls{cbt} Vol.\ IV}
\newacronym[description={\cite{jones_gwaith_1991} \citetitle{jones_gwaith_1991}}]{CBTIII}{CBT~III}{\gls{cbt} Vol.\ III}
\newacronym[description={\cite{bramley_gwaith_1994} \citetitle{bramley_gwaith_1994}}]{CBTII}{CBT~II}{\gls{cbt} Vol.\ II}
\newacronym[description={\cite{williams_gwaith_1994} \citetitle{williams_gwaith_1994}}]{CBTI}{CBT~I}{\gls{cbt} Vol.\ I}

\newacronym{ppd}{PPD}{\mw{Pwyll Pendeuic Dyuet}}
\newacronym{bul}{BUL}{\mw{Branwen Uerch Lyr}}
\newacronym{mul}{MUL}{\mw{Manawydan Uab Llyr}}
\newacronym{mum}{MUM}{\mw{Math Uab Mathonwy}}
\newacronym{co}{CO}{\mw{Culhwch ac Olwen}}
%%% Local Variables:
%%% mode: latex
%%% TeX-master: "main"
%%% End:
