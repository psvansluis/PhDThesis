\chapter*{Abstract}%
\label{cha:abstract}

\todo[inline]{abstract max 300 words}

In Modern Welsh, /p t k/ lenited to /b d g/ sound exactly the same as radical /b d g/. However, lenited voiceless stops and radical voiced stops go back to different consonants in the Common Celtic stage, and some Breton dialects maintain a distinction between lenited voiceless stops and their radical counterparts based on length~\autocite{falchun_systeme_1951}. The existence of a similar three-way stop distinction in Welsh has been argued for by~\textcite{koch_*cothairche_1990}. However, Breton stop phonology has been influenced by Latin, meaning voiceless and voiced stops are distinguished by means of voicing, while aspiration distinguishes Welsh voiceless and voiced stops. Breton alone therefore gives insufficient evidence for reconstructing the phonology of the Early Welsh three-way stop distinction. 

A language distinguishing three series of stops requires at least two binary variables or one ternary variable to distinguish. So what variables distinguished these series in Late Brittonnic? Previously, it was suggested that either voice and aspiration were these two variables~\autocite{koch_*cothairche_1990}, or voice and length~\autocite{schrijver_old_2011}.

I argue that an Early Welsh three-way stop distinction was based on the following two phonological variables: length and aspiration. I introduce several methodologies by which this reconstruction may be achieved: irregularities in Old Welsh orthography, patterns of provection in the eleventh- to thirteenth-century cynghanedd, and of course comparative evidence. 

I also argue that lenited voiceless stops merged with radical voiced stops as late as the thirteenth century on the basis of developments in Middle Welsh orthography \todo{so not writing lenition of \mw{p, t c} where it is expected}. Knowledge of when and how these series merged may aid us in dating texts found in Middle Welsh manuscripts.


%%% Local Variables:
%%% mode: latex
%%% TeX-master: "../main"
%%% End:
