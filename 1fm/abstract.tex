\chapter*{Abstract}%
\label{cha:abstract}

Welsh has morphophonemic lenition, whereby grammatical triggers cause word-initial consonants to become more sonorous. The Modern Welsh voiceless stops /p~t~k/ are lenited to their voiced counterparts /b~d~ɡ/; these lenited voiceless stops are nowadays identical to the unlenited voiced stops /b~d~ɡ/, so these two series of consonants have merged. This thesis aims to establish exact pronunciation of the various stop series before this merger as well as the date of this merger. 

This thesis establishes which phonological variables served to maintain a three-way  distinction between the stop series. It presents a review of the literature on British Celtic historical phonology and Breton dialects that maintain a three-way stop distinction. This overview then hypothesises an early Welsh stop system where lenited consonants were originally short and their unlenited counterparts long, and where the voiceless stops were distinguished from voiced by means of aspiration. This hypothesis is tested with two methodologies: irregularities in Old Welsh orthography and patterns of provection in the eleventh- to thirteenth-century cynghanedd.

This thesis also establishes the date of the merger between lenited voiceless stops and unlenited voiced stops and its implications for dating texts. Independent Welsh translations of Geoffrey of Monmouth’s \lat{Historia Regum Britanniae} show that word-initial lenited /p~t~k/ start to be written with \mw{b~d~g} in the thirteenth century, starting with lenited /k/ as \mw{g}, and /p/ as \mw{b}, and then /t/ as \mw{d}; this new orthography must have followed the phonological merger closely. Analysis of several Iorwerth law manuscripts shows how early stop orthography was always incompletely copied into later manuscripts and the instances  where lenition was modernised were largely chosen randomly. These haphazard modernisations left unique patterns of lenition in manuscripts. Analysis of three manuscripts containing \mw{Buchedd Dewi} shows how these patterns may reveal the stemmatics of Middle Welsh texts.


%%% Local Variables:
%%% mode: latex
%%% TeX-master: "../main"
%%% End:
