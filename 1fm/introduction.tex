\chapter{Introduction}
This thesis treats the initially separate qualities of lenited voiceless stops and unlenited voiced stops in the Welsh language. This chapter will firstly introduce the history of the idea that \lT\ did not equal \xD, at least word-initially. Then, this idea will be framed within a wider context of the history of lenition, and the related processes of gemination, spirantization and nasalization. Then, Middle Welsh evidence of this non-merger will be discussed, and research on the phonetics of stops will  be presented.

The following Old British stop system has been reconstructed so far: 
\begin{table}[h]
    \centering
\begin{tabular}{lll}
\begin{tabular}[c]{@{}l@{}}{[}-voice{]}\\ {[}long{]}\\ {[}asp{]}\end{tabular} & \begin{tabular}[c]{@{}l@{}}{[}+voice{]}\\ {[}long{]}\end{tabular} & {[}short{]} \\\hline
p{[}\textsuperscript{h}ː{]} & bː & b \\
t{[}\textsuperscript{h}ː{]} & dː & d \\
k{[}\textsuperscript{h}ː{]} & gː & g
\end{tabular}
    \caption{The Old British stop system between the 7th and 11th centuries according to \textcite[33]{schrijver_old_2011}.}
    \label{oldbritishconsonantsystem}
\end{table}



% \begin{itemize}
%     \item Koch (1990)
%     \begin{itemize}
%         \item Consistency of OW spelling (but note that voiced geminates are spelled the same as voiceless non-geminates.)
%         \item MoB, also Falc'hun
%         \item Brittonnic placenames in Anglo-Saxon and Anglo-Norman
%         \item 
%     \end{itemize}
%     \item van Sluis (2014)
%     \begin{itemize}
%         \item Lenition of voiceless stos is not written after \ei\ and \oes, but it is written following verbal endings to denote the object or nominal predicate of other verbal endings.
%     \end{itemize}
%     \item Falch'un (1952): Breton keeps them apart using length. 
%     \begin{itemize}
%         \item Phonetic properties of distinction between voiced and voiceless stops have changed from aspirated/non-aspirated towards voiceless/voiced.  \lT\ and \xD\ are kept separate by length. 
%         \item \xD/\lT-distinction may have been analogically reintroduced, see \cite{harvey_aspects_1984}
%     \end{itemize}
% \end{itemize}



\section{Lenited voiceless stops and radical voiced stops}
In this section, I will discuss previous work that treats the differing phonetic properties of lenited voiceless stops and unlenited voiced stops in a roughly chronological order.
\subsection{Falc'hun}
\textcite[63]{falchun_systeme_1951} noted that in Le Bourg Blanc Breton, word-initial lenited voiceless stops and unlenited voiced stops are kept separate by length.
% \tqt{\begin{french}
% Elles [les occlusives sonores initiales] sont toujours fortes, \`{a} moins qu'elles no proviennent de la mutation de \textit{p, t, k}. Pour plus de clart\'{e}, nous les transcrirons par \textit{bb, dd, gg}, et r\'{e}serverons les signes \textit{b, d, g} pour les occlusives intervocaliques, ou initiales r\'{e}sultant de la mutation \textit{p, t, k}.
% \end{french}}{falchun_systeme_1951}{63}
In Falc'hun's notation, Breton therefore has \mob{bb, dd, gg} for word-initial non-lenited voiced stops, while word-medial voiced stops and lenited word-initial voiceless stops are represented by \mob{b, d, g}. In other words, word-medial lenited voiceless stops are not kept separate from word-medial voiced radicals, signalling that they indeed have merged. The consequence of non-merger in initial positions is that some minimal pairs may be discerned:
\tqt{\begin{french}
Prenons les deux suivantes, qui ont \'{e}t\'{e} exp\'{e}riment\'{e}es, avec bien d'autres du m\^{e}me genre, sur des auditoires bretonnants du L\'{e}on, du Tr\'{e}guier ou de la Cornouaille:

1. \textit{(t\^{o}rr\`{e}d \'{e}\'{o} \'{e} g\={\c{a}}r)}; 2. \textit{(t\^{o}rr\`{e}d \'{e}\'{o} \'{e} gg\={\c{a}}r)}. Les bretonnants traduisent sans h\'{e}sitation: %
\begin{enumerate}
    \item << Sa charette \`{a} lui est cass\'{e}e >>, \textit{torret eo e garr.}
    \item << Sa jambe \`{a} elle est cass\'{e}e >>, \textit{torret eo he gar.}
\end{enumerate}
\end{french}}{falchun_systeme_1951}{64}

The distinct pronunciation of these consonants is restricted to postvocalic contexts
\todo{This restriction is similar to how lenition of voiceless stops is sometimes written in consonant clusters in \gls{ow} orthography, but hardly ever intervocalically, see Chapter \ref{oldwelsh}.}:
\tqt{
  \begin{french}
    Toutefois, la distinction n'est possible qu'apr\`{e}s voyelle. Le \textit{d} initial de \textit{an dud}, << les gens >>, de \textit{tud} ne diff\`{e}re rien de celui de \textit{an dour}, << l'eau >>.  
    Et si une occlusive sonore provenant de la mutation de \textit{p, t, k}, se trouve \`{a} l'initiale absolue, elle se prononce forte, ainsi le \textit{b} de \textit{breman}, << \`{a} pr\'{e}sent >>, qui provient du \textit{p} de \textit{pred}, << moment >>.
   Le \textit{b} sera doux dans \textit{abred, (abr\c{ē}d}, << de bonne heure >>, ais fort dans \textit{ha breman (a bbr\`{\c{ē}}m\~{a})}, << et \`{a} pr\'{e}sent >>, parce que \textit{ha}, r\'{e}duction de \textit{hag}, n'adoucit pas la consonne suivante [\dots].
 \end{french}}{falchun_systeme_1951}{64} 


% In Breton, it seems, the distinct pronunciation of lenited voiceless stops and unlenited voiced stops word-initially is based on the phonotactic prerequisite that the relevant consonant should follow a vowel. Notably, this excludes the Breton definite article \textit{an}.

% This phonotactic constraint may give a way to test Harvey's position that length distinction in word-initial stops may disambiguate between lenited voiceless and unlenited voiced: if Early Middle Welsh similarly shows that these stop series are disambiguated after vowels, but not after consonants, then this might point to common ancestry, thereby ruling out analogical formation.
Falc'hun gives the following measurements on word-initial stop length:

\tqt{\begin{french}
Des enregistrements ont permis de v\'{e}rifier les dur\'ees des occlusives sonores \`a lint\'erieur de la phrase, mais au d\'ebut du mot: 

\begin{tabular}{rr}
    \textit{bb} 8,6 centisecondes (13 ex.) & \textit{b} 6,8 centisecondes (10 ex.) \\
    \textit{dd} 9,5 centisecondes (12 ex.) & \textit{b[d]} 5,6 centisecondes (11 ex.) \\
    \textit{gg} 8,5 centisecondes (10 ex.) & \textit{g} 5,2 centisecondes (14 ex.) \\
\end{tabular}%

A l'intervocalique dans le mot, apr\`es voyelle accentu\'ee, la dur\'ee moyenne des occlusives sourdes \textit{p, t, k} a \'et\'e de 10,80; celle des sonores \textit{b, d, g}, de 5,64.
\end{french}}{falchun_systeme_1951}{65}

% Question: duration of what exactly, in terms of phonetics? Answer: most likely, the duration the airway remains closed before the release of the air signifying the stop.

% Question: does this phenomenon of lengthening voiced stops occur purely as a measure to disambiguate length only where ambiguity might otherwise arise (most notably after \textit{e} `his' or `her', depending on the presence or absence of lenition), or is there a quantitative opposition between long and short voiced stops word-initially throughout Le Bourg Blanc Breton? The former situation would argue for Harvey's position that the opposition was analogically reintroduced, since using lenition exclusively to disambiguate echoes later grammatical innovations such as syntactic lenition more than it does inherited patterns. Falc'hun implies this position in his quote two paragraphs below.

% \tqt{\begin{french}Les occlusives sonores fortes ont tendance \`a s'assourdir: on entend fr\'equemment\textit{ (va t\={\c{u}}\'e) va Doue} << mon Dieu! >> pour\textit{ (va dd\={\c{u}}\'e)}. M\^eme quand elles sont enti\`erement sonores, leur explosion peut \^etre suivie d'un souffle sourd, qui dure jusqu'\`a 5 centisecondes (cf. infra p. 159). Leur hauteur explosive est toujours plus grande que celle de \textit{b, d, g}. \end{french}}{falchun_systeme_1951}{65}

% The above quote seems to imply a relationship between fortition and distinguishing unlenited voiced stops and lenited voiceless stops.
According to Falc'hun, the ability to distinguish lenited voiceless stops from unlenited voiced stops is dependent on understanding lenition as a system, which explains why it only occurs word-initially: \tqt{\begin{french}
Cette opposition entre deux s\'eries d'occlusives sonores, l'une forte et l'autre douce, ne joue dans la langue qu'un r\^ole n\'egligeable. Son existence fait cependant mieux comprendre la logique du m\'ecanisme des mutations tel qu'il sera d\'ecrit plus loin. 
% Sans elle, dans un syst\`eme consonantique o\`u toute consonne est forte ou douce, on ne saurait o\`u classer \textit{b, d, g,} qui sont des douces, puisque provenant de l'adoucissement de \textit{p, t, k,} et qui seraient en m\^eme temps des fortes, puisque s'adoucissant elles-m\^emes en \textit{v, z, h}.
\end{french}}{falchun_systeme_1951}{65}

\subsection{Carlyle}
\label{sec:carlyle}

More evidence on Breton phonology is given by \textcite{carlyle_syllabic_1988}, \todo[inline]{summarize her position, or at least note that she confirms Falc'hun}

\subsection{Jackson}
According to \textcite[§132]{jackson_language_1953}, evidence from Breton phonology suggests that the opposition between lenited and unlenited may originally have been one of quantity (\ie duration) rather than quality (\ie voiced vs voiceless, stop vs fricative): 
\tqt{Thus, for example, the lenited \textit{b} in \textit{e baz} ``his cough'', from \textit{paz} (W. \textit{ei bas}, from \textit{pas}), and the lenited \textit{l} in \textit{e leur,} ``his floor'', from \textit{leur} (W. \textit{ei lawr}, from \textit{llawr}), have approximately only half the articulatory duration of the non-lenited \textit{p} in \textit{paz} or the non-lenited \textit{b} in \textit{bac'h} ``hook'' (W. \textit{bach}), and the non-lenited \textit{l} in \textit{leur}.}{jackson_language_1953}{§132} On the basis of these findings in Breton, Jackson proposes that Common Celtic consonants may have been comparatively long in absolute initial position, while short when placed between vowels.

Schrijver repeats this view: 
\tqt{As a result of phonemic lenition, \textit{*p, *t, *k} became short \textit{*b, *d, *g}. These were contrasted with the reflexes of the unlenited voiced stops, which were now phonemically long (and, presumably, tense), merging with the rare old geminate voiced stops: \textit{*bː, *dː, *gː}. In word-initial position, the contrast between short and long voiced stops is maintained (but according to Harvey (1984) analogically reintroduced) in MoB dialects. The unlenited voiceless stops were undoubtedly long too, \textit{*pː, *tː, *kː}, but they did not contrast with short voiceless stops. MoW evidence shows that these were probably strongly aspirated (except after \textit{s}), the aspiration having been lost in Cornish and Breton under Late Latin influence. The lenited voiced stops became the voiced fricatives \textit{*β, *ð, *γ}.}{schrijver_old_2011}{31}
Schrijver dates this development to the Proto-British period. Schrijver assumes that aspiration was the original distinguishing feature between unlenited and lenited voiceless stops. He also states that phonemic lenition caused voiceless stops to have a lenited (short) counterpart which does not merge with unlenited voiced stops word-initially. 

\subsection{Harvey}\todo{I should move extensive discussion of Harvey elsewhere, and merely introduce his position in the introduction}
\label{harveylenition}
Harvey argues that \xD\ and \lT\ had already merged in early Brittonic. Features of Le Bourg Blanc Breton may be explained by analogy.
% \tqt{It is no longer possible, from synchronic considerations, to tell whether a given internal intervocalic voiced stop (\textit{g, b, d}) is the product of the lenition of an unvoiced simplex (/k, p, t/) or of the non-lenition of a voiced genimate (/gg, bb, dd/); hence the disagreement about whether Welsh, Cornish, and Breton \textit{ober} `work' derives from Latin \textit{opera} or from a native /obber-/ < /od+ber-/.}{harvey_aspects_1984}{96}
% What Harvey states here is correct for Modern Welsh and Middle Welsh: the \textit{b} in e.g.\ \mw{ebol}, which goes back to lenited \textit{p},  and \mw{aber}, which goes back to geminate \textit{bb}, had already merger in both the phonology and the orthography of \gls{ow}~\autocite[33]{schrijver_old_2011}. 

\tqt{This treatment [i.e.\ apocope] also destroys the distinction between the relevant elements of examples [/V+\textit{d}\"{u}ːd/] and [/V+\textit{d}uβr/], since if gemination has indeed disappeared, the lenition of an unvoiced stop gives the same phoneme as the non-lenition of the corresponding voiced one and both are now in identical environments in initial sandhi. But once again, we find that this prediction is generally borne out in practice: thus in Welsh `his stags' (\textit{ceirw}, plural of \textit{carw}) and `her waves' (\textit{geirw}) are indistinguishably \textit{ei geirw}. The only counter-evidence comes from the Breton of le Bourg Blanc, where an opposition based on gemination exists in this position (for example, the \textit{g} in \textit{torret eo he gar} `her leg is broken' is pronounced much longer than that in \textit{ torret eo e garr} (<\textit{karr}) `his cart is broken'). But this phenomenon is not attested in any other Modern Celtic dialect, and even in this one it is not found in other environments such as internal intervocalic, the relevant stops [\dots] having merged there as elsewhere. It thus seems likely that the alternation is not original but is the result of analogical extension of the mutation system 
\newline\textsc{radical} = [+gemination]\newline\textsc{lenition} = [-gemination]\newline
to the stops from the resonants \textit{l(l), n(n), and r(r)}, which, as we have seen, are its proper domain because in their case the character of the opposition was not altered at the time of lenition.}{harvey_aspects_1984}{96--97}
In short: Harvey believes that lenited voiceless stops and unlenited voiced stops merged in the immediate post-acope period. He provides three arguments why the dissimilarity in Breton \lT\ and \xD\ are not inherited. The first is that it is only attested in Le Bourg Blanc Breton, and in none of the other modern Celtic languages. The second is that it is only found word-initially, but not word-medially. The third argument is that the resonants in Le Bourg Blanc Breton provide a solid analogical base. 

The only Brittonic evidence of a non-merger is in the Breton dialect of le Bourg Blanc, which uses length to distinguish these consonants. The fact that no present-day Celtic languages except for some Breton dialects maintain a distinction between lenited voiceless stops and unlenited voiced stops does not mean that there was none. Absence of evidence is not evidence of absence. Moreover, \textcite{koch_*cothairche_1990} and this thesis  provide several strands of evidence showing that \lT\ and \xD\ were in fact kept separate in Welsh, even if not up until the present day.
 
The point that the distinction between \lT\ and \xD\ was only maintained word-initially does not by itself mean the system arose as a result of analogy. Pre-apocope lenition operated without respect to word boundaries pre-apocope, and the maintenance of this rule is still why Celtic languages have lenition. However, nasalisation and spirantisation do show difference in application word-internally and across word boundaries at times. Compare, for example, the development of the consonant cluster *\mw{ntr} in Welsh \mw[nail]{cethr} < \glat{centrum}, where this cluster becomes /θr/ as a result of spirantisation, and \mow{fy nhref} < \gpbr[my town]{*men treb\=a}, where /-n tr-/ becomes /-n̥r/ after nasalisation\footnote{For more information on the details of this process, see \textcite{schrijver_spirantization_1999,isaac_chronology_2004}.}. Additionally, the very fact that this three-way stop distinction is only found word-initially is shared between Le Bourg Blanc Breton and the earliest Middle Welsh. It therefore suggests that the phonology of the distinction between lenited voiceless and unlenited voiced stops developed separately word-initially and word-medially, and that they may have merged word-medially as early as the common Brittonic period.
 
 
The argument that resonants provided a solid analogical base in using length to distinguish fortis from lenis is true. However, a three-way stop distinction is cumbersome. This is hardly the kind of system one should expect from analogy, which is a force that simplifies things rather than complicating them. This same cumbersomeness is not found in these resonants themselves, which typically give phonemic contrast based on length, but do not also have a voiced-voiceless distinction to complicate the system into a three-way contrast. More generally, the precise fact that a three-way stop distinction is so counter-intuitive as a result of analogy means that it is simpler to assume that this cumbersome system arose as a result of a linguistic process which does not necessarily simplify matters: phonological change. Therefore, it is simpler to regard the distinction between \lT\ and \xD\ as a direct result of phonemicisation of post-apocope consonants, which was subsequently eroded in most of the Brittonic dialects.  


Harvey bases his argument that analogy drove the distinction between \lT\ and \xD\ on the sonorants \mw{r, l, }and \mw{n}. These consonants have two phonemes each: a long fortis, and a short lenis. Naturally then, it is length that distinguishes fortis from lenis here. This identification of length as a marker for fortis quality was then expanded to voiced stops. This matter is of relevance to Welsh, precisely because Welsh does not use length to distinguish fortis and lenis consonants for \mw{r} and \mw{l}. Rather, voice is the common factor in distinguishing \graph{rh} and \graph{r}, and \graph{ll} and \graph{l}, respectively, while \graph{n} does not distinguish between fortis and lenis word-initially\footnote{\gls{mw} does distinguish /nː/, /rː/, and /lː/ from their short counterparts. This distinction is lost before the end of the Middle Welsh period, see also: \textcite[127--128]{schumacher_mittel-_2011}. The relevance of this phonemic opposition for the opposition between lenited voiceless stops and unlenited voiced stops is doubtful, however, because sonorants only use length to distinguish fortis and lenis in non-initial position, whereas it is precisely in word-initial position that lenited voiceless stops and unlenited voiced stops are kept separate in the Early \gls{mw} orthography and in Le Bourg Blanc Breton.}. Resonants  \graph{ll} /ɬ/ and \graph{l} /l/ also differ in manner of articulation: the former sound is a lateral fricative while the latter is a lateral approximant. 

For Middle Welsh, Koch \parencite*{koch_*cothairche_1990} believes that the primary phonetic property distinguishing lenited voiceless stops and unlenited voiced stops was voice: lenited voiceless stops were voiceless (but unaspirated) and unlenited voiced stops were voiced. Early Middle Welsh and Le Bourg Blanc Breton apparently both distinguish \lT\ and \xD, but realize this distinguishing in a radically different way phonetically. For both Welsh and Breton, this distinction runs analogous to the distinction between fortis and lenis sonorants. Welsh sonorants have voiceless and voiced variants, and by the same mechanism are lenited voiceless stops and unlenited voiced stops kept separate. Breton sonorants have long and short variants, and lenited voiceless stops and unlenited voiced stops are similarly distinguished by length. If the Breton opposition between stops came into being as a result of analogy with sonorants, then the same would sensibly the case for Welsh. For reasons of economy, however, it is more sensible to regard the Welsh and Breton systems as having a common origin. The break-up of Brittonic into Welsh and Breton constitutes continuity in the phonemic oppositions here, and only phonetic innovation in one of the languages needs to be hypothesised\footnote{Loss of aspiration yielding length distinction in Breton is easily justifiable, given the absence of aspiration in the whole of continental western Europe and particularly France.}. 

\subsection{Koch}
\label{kochvoiceless}
\Textcite{koch_*cothairche_1990} gives several strands of evidence supporting the non-merger of word-initial lenited voiceless stops and radical voiced stops in Welsh. According to Koch, lenited voiceless stops did not merge with unlenited voiced stops until some time into the historical period, and were likely separate consonants up until part of the Early \gls{mw} period. The evidence for this is that these consonants did not merge up until after the split between Western Brittonic and Southwestern Brittonic, since some Breton dialects still distinguish between them: 
\tqt{As Falc'hun has shown, these series never completely converged in Breton, at least not in some dialects, where the historical voiced fortes still contrast with the historical voiceless lenes in minimal pairs by a clinically measurable difference in duration: e.g. [\c{e} g:a:r] `her leg' vs. [\c{e} ga:r] `his car'.}{koch_*cothairche_1990}{\S26} 
 
Because this merger had not yet taken place, it would not be sensible to write lenited voiceless stops as voiced stops. This in part explains why \gls{ow} orthography does not represent lenition. Koch notes that this non-representation of lenition of voiceless stops is completely consistent in \gls{ow} and \gls{ob}, although many errors would be expected if lenited voiceless stops and unlenited voiced stops had already merged by this point. Moreover, Koch states, the cynghanedd demonstrates that these sets only fell together by the end of the fourteenth century~\parencite[\S26]{koch_*cothairche_1990}. In terms of exact phonetic reality, Koch proposes the following three-way distinction in the immediate post-acope period: \tqt{lenited [\bd, \gd, \dd] was distinct from radical [b-, g-, d-] by being voiceless and from radical [p\textsuperscript{h}-, k\textsuperscript{h}-, t\textsuperscript{h}-] by being unaspirated.}{koch_*cothairche_1990}{\S30} This account is phonetically different from the one proposed by \textcite{falchun_systeme_1951} and \textcite{jackson_language_1953} in that Koch proposes voice as the variable distinguishing lenited voiceless stops from radical voiced stops, whereas the earlier authors propose length. 

In addition to the above, Koch states:
\tqt{it is not common in the languages of the world for the duration of consonant segments in absolute phrase-initial position to be phonologically significant. A Common Neo-Brittonic phonemic opposition of \textit{b-, g, d-} = /\textit{b:-, g:-, d:-}/ vs \textit{-p-, -c-, -t-} = /\textit{-b-, -g-, -d-}/ is accordingly unlikely, though a phonetic concomitant of duration was probably present. Anglo-Saxon borrowings of the \textit{Cerdic} type suggest that voicing in internal position was as old as the Migration Period, but it is likely that Brittonic speakers were still keeping the series distinct on the basis of \textbf{degree of voicing}.}{koch_*cothairche_1990}{\S29} 

I have three comments about this\todo{These comments should be part of a chapter discussing the phonetics of my thesis}. The first is that absolute initial positions do not exist for lenited consonants during the Common Neo-Brittonic period: lenition exists as a feature of the preceding word (and syntactically conditioned types of lenition are later developments), so a lenited voiceless stop always has a preceding phonological context of some sort in this stage.  We are therefore left with two possible scenario's: either the merger of \xD\ and \lT\ was a post-apocope development, or the merger of \xD\ and \lT\ was followed by a sound law disambiguating these consonants in word-initial contexts. The former scenario is postulated by Koch (and generally accepted), while the latter scenario is proposed by \textcite{harvey_aspects_1984}, but, crucially, both scenario's are implied to be post-apocope innovations.

The second is that what we call voiced stops in Modern Welsh are not in fact defined by voicedness, and are typically unvoiced. Their quality is determined by non-aspiration, which is exactly the phonetic value Koch had in mind for lenited voiceless stops. Accepting Koch's proposal concerning the phonetic difference between \lT\ and \xD\ implies that present-day word-initial \xD\ is phonetically more similar to \lT\ than to \xD. In other words, the phonological merger of \lT\ and \xD\ was the result of the phonetic shift in \xD\ towards \lT.

Thirdly, a quantitative distinction is exactly what we see in Modern Breton dialects. If we see a pattern of quantitative distinction in those Modern Brittonic dialects that distinguish lenited voiceless stops and radical voiced stops, then surely it is not such a stretch to assume the same for Common Neo-Brittonic. However, the Breton phonetic oppositions may have emerged under Latin influence~\autocite[31]{schrijver_old_2011}.
 

\section{A two-stage development of lenition}
Here, I want to discuss earlier theories on the differing treatment of lenition of voiceless stops, and whether the lenited form of voiceless stops developed at the same point in time as other consonants, or whether it developed later. Both theories are problematic. If they developed at the same time, it stands to reason that it developed in the common ancestor of Goidelic and Brittonic, but Irish voiceless stops lenite differently. If they developed at different points, one is left with an incomplete system of lenition that is not applied to all consonants. Lenition of non-voiceless stops would occur before the Brittonic-Goidelic split, but lenition of voiceless stops would occur afterwards. 

\subsection{Loth}
Loth proposes the second option on the basis that lenition of \mw{p, t, c} had to occur after other types of lenition. If it had occurred beforehand, lenition would have been applied twice to voiceless stops: \tqt{\textfrench{Les explosives sonores \textit{b, d, g}, entre deux voyelles ont d\^{u} \textit{commencer} leur mouvement vers les spirantes correspondantes, avant que les explosives sourdes, \textit{p, t, c} ne fussent devenues \textit{b, d, g}; autrement, celles-ci auraient eu le m\^{e}me sort qu'elles. Si le latin \textit{opera} avait donn\'{e} \textit{obera} au moment o\`{u} \textit{labore} \'{e}tait encore \textit{labure}, le \textit{b} d'\textit{ober} e\^{u}t \'{e}t\'{e} trait\'{e} comme celui de \textit{labur}; c'est-\`{a}-dire f\^{u}t devenu \textit{v}; on aurait aujourd'hui \textit{over, lavur} et non \textit{ober, lavur}. Des trois explosives sonores, \textit{g} para\^{i}t la premi\`{e}re \^{e}tre devenue spirante.}}{loth_les_1892}{87} Loth's explanation seems undesirable. Lenition is one grammatical phenomenon, so it stands to reason that a phonetic distinction between lenited and unlenited consonants was applied to all lenitable consonants. In addition, Loth's explanation does not consider chain shifts: it would be economical to assume that intervocalic /p/ and /b/ would shift at the same time to prevent a merger or an unnecessarily complex phonological distinction. What Loth demonstrates is that lenition of voiceless stops may not have occurred \emph{before} other types of lenition, but not necessarily that it must have occurred \emph{afterwards}. His objections do not preclude a hypothetical chain shift, where lenited voiceless stops gain a different phonetic value precisely because a shift in the phonetic value of lenited voiced stops is taking place. In addition, Loth silently assumes that a lenited voiceless stop automatically merges with an unlenited voiced stop, which I consider to be untrue word-initially.

\subsection{Sims-Williams}
Sims-Williams also argues for two separate stages of lenition:
\tqt{I shall argue that `lenition' --- the conventional name for the spirantization of [b d g m] > [β δ γ μ] and voicing of [p t k] > [b d g] --- occurred in two distinct stages: (1) spirantization, then (2) voicing (see III-IV below). This will seem heretical to Brittonicists, who instinctively regard [t] > [d], etc., and [d] > [δ], etc., as a single phenomenon because they constitute the morphophonemic alternation of `lenition' or `soft mutation'. They should recall, however, that the comparable alternation of `nasal mutation' in Welsh (and in Irish) undoubtedly came into being by two widely separated stages in the cases of voiced and voiceless plosives (LHEB 498-502 and 639).}{sims-williams_dating_1990}{221}

Sims-Williams then gives a detailed relative chronology, using words loaned from British into Irish as evidence:

\tqt{
Obviously British lenition of [p t k] cannot have \textit{preceded} that of [b d g] or these stops would have fallen together in British. On the other hand, the alternative, that [b d g] were lenited first, raises no such problems, and eliminates the problem of the lack of **\textit{Pádhraigh(e)} words. This opens the way for the simple and effective hypothesis that the `lenition' of voiceless stops in the two languages were processes as distinct chronologically as they were phonetically. To avoid the problem of the lack of **\textit{Páttraicc(e)} words, we must suppose that British `lenition' was, after all, completed before Irish `lenition'. This gives the following order:
\begin{enumerate}
\item  First Spirantization of \textit{voiced} stops and single [m], at an early stage (but by no means necessarily simultaneously in British and Irish);
\item Voicing of [p t k] in British (possibly related to the Vulgar Latin development; cf. III. above);
\item `Second Spirantization' of [k\textsuperscript{(w)} t k] in Irish.
\end{enumerate}
Starting after the First Spirantization of voiced stops and [m], the relevant single consonants develop as follows in `lenition position' (\eg intervocalically and in external sandhi):
\begin{table}[H]
\begin{tabular}{lp{0.4\linewidth}p{0.4\linewidth}}
 & BRITISH & IRISH \\
\multirow{4}{*}{(1)} & p t k & k\textsuperscript{w} t k (plus /p/ in foreign words?) \\
 & b d g ?g\textsuperscript{w} (< bb, d-b, ?g\textsuperscript{w}g\textsuperscript{w}, \etc; rare in Latin words). & b (< bb) g\textsuperscript{w} d g (nk\textsuperscript{w} nt nk and < g\textsuperscript{w}g\textsuperscript{w} dd gg) \\
 & β δ γ μ ?γ\textsuperscript{(w)} (\textit{by First Spir.}) & β δ γ μ ?γ\textsuperscript{(w)} ( \textit{by First Spir.}) \\
 & m (< mm) & m (< mm) \\\\
\multirow{4}{*}{(2)} & p t k (< pp tt kk) & k\textsuperscript{w} t k (plus /p/ in foreign words?) \\
 & b d g (< p t k \textit{by Voicing}) ?g\textsuperscript{w} & b g\textsuperscript{(w)} d g \\
 & β δ γ μ ?γ\textsuperscript{(w)} & β δ γ μ ?γ\textsuperscript{(w)} \\
 & m & m \\\\
\multirow{4}{*}{(3)} & p t k (< pp tt kk) & k\textsuperscript{(w)} t k (< kk\textsuperscript{(w)} tt kk) plus /p/ x\textsuperscript{(w)} þ x ?f (k\textsuperscript{(w)} t k p \textit{by Second Spirantization}) \\
 & b d g ?g\textsuperscript{w} & b g\textsuperscript{(w)} d g \\
 & β δ γ μ ?γ\textsuperscript{(w)} & β δ γ μ ?γ\textsuperscript{(w)} \\
 & m & m
\end{tabular}
\end{table}
% At STAGE (1) `Cothrige' or `first stratum' loans occurred, with British /k/ > Ir. /k/, and British /γ/ > Ir. /γ/, etc., very straightforwardly. At first sound-substitution occurred for initial /p/, as in \textit{puteus} > *\textit{k\textsuperscript{w}utyəs} (> \textit{cuithe}), but this was acquired before the end of this stage, as is shown by \textit{preðic(\={a}re)} > Pr. Ir. *\textit{preðik-} (> \textit{pridchid} `preaches'), \textit{par(o)ecia} > Pr. Ir. *\textit{par\={e}kia} (> \textit{pairche} `parish'), and possibly by \textit{peccatum} /pekka:t-/ > Pr. Ir. /pekka:t/ (>/pekaþ-/ > OIr. \textit{peccath, peccad} /p'ekəð/ `sin'). The voiced stop of any early loan-words of the shape of Latin \textit{agger} `rampart' (now [ager]) versus \textit{ager} [aγer] `field') or British *\textit{ab(b)eros} [aberos] `confluence'(< *\textit{adberos}) would naturally be identified with the similar voiced stops that had already arisen in Irish from nasal clusters like /nk/ and from the phonemic simplification of /gg/, /bb/, etc. to /g/, /b/, etc. when the First Spirantization occurred.

% At STAGE (2), with British Voicing (by which old /p t k/ fell together with /b d g/ < old /bb dd gg/) `Pádraig' loans begin (without any intermission), with British /γ/ > Irish /γ/, etc., just as before, but the new British voiced /g/ (< /k/) > Irish /g/, and so on. (Presumably any borrowings at this stage with British medial /p t k/ --- from geminates, and still probably realized quite long/tense but phonemicized as single/short by the Voicing of single /p t k/ --- would have to be supposed to have Ir. /pp(?) tt kk/ sound-substituted. But in fact there do not seem to be any loans with Latin -\textit{pp}- which \textit{have} to be dated before stage 3, and none of those with -\textit{tt}- and -\textit{cc}- (e. g. \textit{peccatum}) \textit{have} to belong at stage 2, rather than at stages 1 or 3.)

% At STAGE (3) `Pádraig' loans continue (without any intermission) after the Second Spirantization in Irish. Any borrowings with British medial /p t k/ (from geminates) would now have Irish /p t k/ (from geminates, but now phonemicized as single/short by the spirantization of old /p t k/), e. g. British Latin \textit{cippi} /kipi(:)/ `stumps' > OIr. \textit{cipp} /k'ip'/. The transition from stages (2) to (3) should be invisible in the loanword evidence, as indeed is the case; one cannot tell, for example, whether \textit{Pádraig} was borrowed before or after the earlier loan-type *\textit{K\textsuperscript{(w)}atrikiə(s)} would have been lenited to *\textit{K\textsuperscript{(w)}aþrixiəh}.

}{sims-williams_dating_1990}{}

Sims-Williams' account of lenition in Brittonic gives a probable phonological context within which lenition of all consonants except voiceless stops may have occurred. Brittonic lenition happened in two separate stages: the first was fricativization of consonants other than voiceless stops. This development occurred in both Brittonic and Goidelic, but this may not necessarily mean that it occurred before the Brittonic-Goidelic split. The second development, voicing of voiceless stops occurred separately. It is an anachronism, Sims-Williams states, to think that these developments occurred at the same time just because they are part of a single morphophonological process in later stages of Brittonic. This is an important point. Even though Loth's account of lenition in two stages seems cumbersome, Sims-Williams argues on the basis of Irish-British language contact that this must have been the case.

Sims-Williams' position is also taken up by McCone, who writes the following in a comment to \textcite{koch_*cothairche_1990}:
\tqt{Although Koch is basically right to assert that `prevocalically in absolute initial position, the voiceless stops are still aspirates [p\textsuperscript{h}, k\textsuperscript{h}, t\textsuperscript{h}] in all the living Celtic languages', the distributional dit between this feature and non-lenition or between non-aspiration and lenition remains far from perfect overall and one might anyway expect the aspirates to be more likely than the non-aspirates to develop into fricative \textit{θ, x\textsuperscript{(w)}} in Irish, after all, it is /p\textsuperscript{h}/, /t\textsuperscript{h}/ and /k\textsuperscript{h}/ that became /f/, /θ/, /x/ in later history of Greek while /p/, /t/, /k/ remained unchanged and a similar development probably occurred in the prehistoric Italic stop system [\dots]. Nearer home, the so-called `spirantisation'  of British Celtic [\dots] affected voiceless stops that were probably aspirated allophones for the most part. In short, it seems quite unlikely that arhuable Proto- and/or Insular Celtic allophonic variation between aspirated and unaspirated voiceless stops had any direct bearing upon the lenition of those stops in Irish and British.

One point that does not seem to have been made in the debate so far is simply this: as is clear from a number of other languages, it is by no means inevitable that lenition should affect voiced and voiceless stops simultaneously. For example, Ancient Greek /b/, /d/, /g/ have become /v/, /ð/, /γ/ in Modern Greek but /p/, /t/, /k/ underwent no parallel development to /f/, /θ/, /x/ [\dots]. More pertinent still is the second lenition of /b/, /d/, /g/ to /v/, /ð/, /γ/ after various vowels and sonorants in Spanish in the absence of a corresponding transformation of /p/, /t/, /k/ [\dots].}{mccone_towards_1996}{83--84}

\subsection{Isaac}
\textcite{isaac_chronology_2004} disagrees with Sims-Williams's line of reasoning on British loanwoards into Irish on the grounds that it conflates phonetics and phonology. Sims-Williams understands the non-existence of Irish \textit{**P\'attraicc(e)}-type words as evidence that Irish lenition of voiceless stops is at least as old as British lenition. However,  a \textit{**P\'attraicc(e)}-type word never existed in any stage of Brittonic phonology. Intervocalically, Proto-British voiceless stops were not aspirated, even though they were allophones of word-initial aspirated voiceless stops. Sims-Williams' confusion about which allophones went where and the interface of phonetics and phonology in borrowing caused his misapprehension:

\tqt{What the loans from British into Old Irish directly tell us about is the way the speakers of Irish interpreted, within the Irish \textit{phonological} system, the sounds that they heard being produced (be it Latin or the vernacular) by the speakers of British, the \textit{phonetics} of British. The borrowing process connects the \textit{phonetics of the source language} with the \textit{phonology of the target language.} As such it does not directly involve the phonology of the source language.}{isaac_chronology_2004}{73}
One may still derive useful snippets of knowledge from the existence of \goi{Cothraige} and \oi{Pátraic}, however. The former form is correctly loaned as an \textit{io-}stem noun, meaning it predates British apocope. By contrast, \oi{Pátraic} follows British apocope, and shows reinterpretation of /t/ in this context as something that could be interpreted as [d] in Irish. According to Isaac, this shows that the cognitive reorganisation of lenition as a morphophonemic process rather than a phonetic process in Brittonic was complete~\autocite[73]{isaac_chronology_2004}. 

\section{Phonetics and phonology of lenition in present-day Welsh} 
Unlike Breton, present-day Welsh does not preserve a three-way stop distinction either word-initially or elsewhere. Nevertheless, the way in which the contrast between voiced and voiceless stops is realised phonetically differs between dialects. In some Southeastern dialects of Welsh, the consonants themselves have phonetically merged, but they are kept separate by the length of neighboring vowels: e.g.\ \mw{ebol} /'eːpol/ `foal' v.s. \mw{capel} /'kapel/ `chapel'~\parencite[85]{awbery_phonotactic_1984}. One may wonder whether preceding vowel length similarly played a role in disambiguating the three Early Middle Welsh stop series as it does in disambiguating the two stop series remaining in Modern SE Welsh, and perhaps the modified survival of such a pattern in southeastern Welsh implies that this system survived for longer in these dialects. However, the idea that preceding vowel length may have served to disambiguate lenited voiceless stops from unlenited voiced stops has one drawback: the distinction between three different stop series was only maintained word-initially. The vowel lengthening would therefore have occurred in the word preceding the word to be lenited. The phonetic upshot of this would be (to use Koch's examples~\autocite*[\S 26]{koch_*cothairche_1990}): [\c{e}ː g(ː)ar] `her leg' vs. [\c{e} gar] `his car'. This supposition is problematic, because only stressed syllables distinguish between short and long vowels, and the posessive pronouns in the above examples are both unstressed. 
 
 

\section{Voiceless stops after verbs}

% This work deals with the system of postverbal lenition in Early Middle Welsh poetry. In particular, it deals with the type of lenition following verbs ending in \mw{–ei} and \mw{oes}\footnote{It should be noted that there is no known phonetic or phonological distinction between voiceless stops following \mw{s} and when their voiced counterparts follow \mw{s} in Welsh. Erratic behaviour of lenition following \oes\ is therefore problematic: if a scribe could not hear the distinction between lenited and unlenited consonants in this position, lack of lenition in writing is expected, but not exclusively following \oes. As such, the opposition in lenition between examples \ref{storch} and \ref{sbont} may be merely orthographical in nature.}. 
\Textcite{van_sluis_development_2014} uncovers a relevant pattern after two Middle Welsh verbal endings: third person singular \ei\ and existential verb \oes\ are followed by lenition, but words starting with a voiceless stop are not typically lenited. The following examples serve to illustrate the system:
\begin{mwl}
\mwc{\acrshort{wbr}~2.2-4}%
  {ac ual ẏ llathrei ỽynnet ẏ cỽn ẏ llathrei cochet ẏ clusteu}%
  {And as white as the dogs shone, so red shone their ears.}%
  \mwc{\acrshort{wbr}~481.22}%
  {canẏt oes lestẏr ẏn ẏ bẏt a dalhẏo ẏ llẏn cadarn hỽnnỽ namẏn hi.}%
  {Since there is no vessel in the world that may keep the strong drink except for this one.}%
  \mwc[storch]{\acrshort{wbr}~483.23}%
  {Nẏt oes torch ẏn ẏ bẏt a dalhẏo ẏ gẏnllẏuan namẏn torch canastẏr kanllaỽ}%
  {There is no collar in the world that may hold the leash except for the collar of Canastr Canllaw.}%
\end{mwl}
 So far, this pattern has been found in the White Book recensions of \mw{Culhwch ac Olwen} and \mw{Pwyll Pendeuic Dyuet}. 
What makes these two verbal endings special is that they are the only ones causing lenition as a contact lenition. Contact lenition may as a rule be considered an inherited grammatical feature that may be traced back to pre-apocope Brittonic, when lenition was not yet phonemicized. All other types of lenition, by contrast, occur depending on the grammatical relationship of the verb to the following word and may not be traced back to when lenition was allophonic. The latter type of lenition is typically found as lenition of the object or nominal predicate, and is not limited to consonants other than voiceless stops~\Autocite[70]{van_sluis_development_2014}.

This system of limited lenition following \ei\ and \oes\ had already broken down by the time the White Book of Rhydderch itself was written. In the White Book recension of \mw{Branwen Uerch Lyr}, lenition following these verbs is haphazard, and no regularity may be discerned in it~\Autocite[42]{van_sluis_development_2014}. Furthermore, the scribe departs from this system on a handful of occasions, which suggests that this feature was not a part of the scribe's writing, but merely copied this feature from an earlier text. These exceptions imply that this system where voiceless stops are not lenited disappeared at some point in the Middle Welsh period. Therefore, it must have disappeared at some point in time between the date of composition of these tales and when the White Book of Rhydderch was composed. The date of composition of these tales is still debated, e.g.\ by \textcite*{rodway_date_2005}, but the date the manuscript itself was written may be dated fairly safely to the middle of the fourteenth century~\autocite[228]{huws_medieval_2000}. 

The fact that postverbal contact lenition was not written with \graph{b, d, g} in \mw{Culhwch ac Olwen} and \mw{Pwyll Pendeuic Dyuet} in these fossilized cases implies that these letters could not be used to represent lenited voiceless stops, which are written with \graph{b, d, g} already. This, in turn, implies that lenited voiceless stops had not yet merged with unlenited voiced stops when these stories were written down. I will discuss the context and the implications of this issue in further detail in Chapter \ref{postverballenition}. 
% A description of the lenition found after these verbs is given by \textcite{morgan_y_1952}. He analyses these verbal endings as causing lenition unconditionally, irrespectively of consonant type or grammatical function of the following word. Exceptions to this rule are explained as the result of imperfect representation of lenition in the orthography. 
% \Textcite{van_sluis_development_2014}, however, finds that lenition following these verbs is not wholly unconditioned in the earliest Middle Welsh prose. Rather, these verbs only cause lenition to consonants other than voiceless stops. 




\section{Related issues}
\todo{To be expanded, obviously}
\subsection{The related issues of gemination and spirantization}
\cite{jackson_language_1953,martinet_celtic_1952,schrijver_spirantization_1999,isaac_old-_2004}

Greene on gemination, with comments in square brackets: 
\tqt{One of the difficulties about the spirant mutation is that the forms of it are not those normally given by \textit{-sk-, -sp-, -st-} in word-interior~; a second is that it affects only the tenues [=voiceless stops]. Jackson deals with the first point that a final lost Σ (from -s) caused a gemination, and that the resulting \textit{pp, tt, kk} later gaven \textit{f, th, ch,} a change well attested from word-interior. But the lack of a similar development for the mediae [=voiced stops] is left unexplained, except for the suggestion that ``all such geminate groups tended to become simiplified quite early in Pr.WCB, and in initial position no doubt much earlier than elsewhere''. Furthermore, the spirant mutation must be connected, not only with the reduction of internal \textit{-pp-, -tt-, -kk-} to \textit{f, th, ch}, but also with the change of ungeminated \textit{p, t, k} to the same sounds after \textit{r, l}. There is no doubt that these changes which took place after the lenition and the loss of final syllables ; Jackson places them in the ``mid or later sixth century'' in his chronology and gives convincing reasons.

I think the origin of these changes may be found in the fact that the Brythonic languages possessed after lenition two sets of mediae but only one set of tenues. One set of mediae was strong \textit{B, D, G,} representing original initial and geminated sounds ; the other weak, \textit{b, d, g,} representing the lenited forms of \textit{p, t, k}. This system still survives in Breton and has been described by Falc'hun. On the other hand, the tenues had only \textit{P, T, K,} representing original initial and geminated sounds and single unlenited \textit{p, t, k} after \textit{l} and \textit{r} ; in this system there could be no opposition of \textit{P} and \textit{p} and the oriiginal strong sound was weakened to \textit{p}. I take it that consonants in sandhi with final \textit{-s, -t, -k} in Welsh, as well as final \textit{-n} in Breton, were preserved strong, in all cases. After the loss of final syllables the tenues were weakened for lack of opposition and were then further weakened to \textit{f, th, ch} in all leniting positions---a weakening of the same type as the earlier lenition of the mediae. The same weakening occurred after particles which now ended in a vowel (final \textit{-s, -t, -k} and, in Breton, \textit{-n}, having dropped before consonants)---e.g.\ \textit{y} ``her'', \textit{tra, a}, and so on. [\dots] I take the development of this mutation as more ore less contemporary with the rise of provection. It was the latter process which provided a new series of tenues (\textit{p} from \textit{b + b, b + h}, etc.) in inlaut ; these are still geminated after the accent in \gls{mow}, and it is not impossible that it was their appearance which hastened the weakening of the old tenues---at least it is certain that the two series are never confused.

This Brythonic evidence justifies the statement that, once lenition had become phonemic in insular Celtic, the opposition \textit{geminated : single} was replaced by the opposition \textit{unlenited : lenited} and gemination ceased to have any phonemic function [except that the phonemic function of gemination was that it caused the morphophonemic rules governing spirantization, which may hardly be called `ceasing of phonemic function' PS].
}{greene_gemination_1956}{288--289}

\subsection{Using linguistic criteria to date texts}
E.g.\ \cite{rodway_dating_2013} and other works by Rodway.
\subsection{Sims-Williams on two-stage lenition and answers by Isaac}
%%% Local Variables:
%%% mode: latex
%%% TeX-master: "../main"
%%% End:
